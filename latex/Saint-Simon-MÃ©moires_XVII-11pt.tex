\PassOptionsToPackage{unicode=true}{hyperref} % options for packages loaded elsewhere
\PassOptionsToPackage{hyphens}{url}
%
\documentclass[oneside,11pt,french,]{extbook} % cjns1989 - 27112019 - added the oneside option: so that the text jumps left & right when reading on a tablet/ereader
\usepackage{lmodern}
\usepackage{amssymb,amsmath}
\usepackage{ifxetex,ifluatex}
\usepackage{fixltx2e} % provides \textsubscript
\ifnum 0\ifxetex 1\fi\ifluatex 1\fi=0 % if pdftex
  \usepackage[T1]{fontenc}
  \usepackage[utf8]{inputenc}
  \usepackage{textcomp} % provides euro and other symbols
\else % if luatex or xelatex
  \usepackage{unicode-math}
  \defaultfontfeatures{Ligatures=TeX,Scale=MatchLowercase}
%   \setmainfont[]{EBGaramond-Regular}
    \setmainfont[Numbers={OldStyle,Proportional}]{EBGaramond-Regular}      % cjns1989 - 20191129 - old style numbers 
\fi
% use upquote if available, for straight quotes in verbatim environments
\IfFileExists{upquote.sty}{\usepackage{upquote}}{}
% use microtype if available
\IfFileExists{microtype.sty}{%
\usepackage[]{microtype}
\UseMicrotypeSet[protrusion]{basicmath} % disable protrusion for tt fonts
}{}
\usepackage{hyperref}
\hypersetup{
            pdftitle={SAINT-SIMON},
            pdfauthor={Mémoires XVII},
            pdfborder={0 0 0},
            breaklinks=true}
\urlstyle{same}  % don't use monospace font for urls
\usepackage[papersize={4.80 in, 6.40  in},left=.5 in,right=.5 in]{geometry}
\setlength{\emergencystretch}{3em}  % prevent overfull lines
\providecommand{\tightlist}{%
  \setlength{\itemsep}{0pt}\setlength{\parskip}{0pt}}
\setcounter{secnumdepth}{0}

% set default figure placement to htbp
\makeatletter
\def\fps@figure{htbp}
\makeatother

\usepackage{ragged2e}
\usepackage{epigraph}
\renewcommand{\textflush}{flushepinormal}

\usepackage{indentfirst}
\usepackage{relsize}

\usepackage{fancyhdr}
\pagestyle{fancy}
\fancyhf{}
\fancyhead[R]{\thepage}
\renewcommand{\headrulewidth}{0pt}
\usepackage{quoting}
\usepackage{ragged2e}

\newlength\mylen
\settowidth\mylen{...................}

\usepackage{stackengine}
\usepackage{graphicx}
\def\asterism{\par\vspace{1em}{\centering\scalebox{.9}{%
  \stackon[-0.6pt]{\bfseries*~*}{\bfseries*}}\par}\vspace{.8em}\par}

\usepackage{titlesec}
\titleformat{\chapter}[display]
  {\normalfont\bfseries\filcenter}{}{0pt}{\Large}
\titleformat{\section}[display]
  {\normalfont\bfseries\filcenter}{}{0pt}{\Large}
\titleformat{\subsection}[display]
  {\normalfont\bfseries\filcenter}{}{0pt}{\Large}

\setcounter{secnumdepth}{1}
\ifnum 0\ifxetex 1\fi\ifluatex 1\fi=0 % if pdftex
  \usepackage[shorthands=off,main=french]{babel}
\else
  % load polyglossia as late as possible as it *could* call bidi if RTL lang (e.g. Hebrew or Arabic)
%   \usepackage{polyglossia}
%   \setmainlanguage[]{french}
%   \usepackage[french]{babel} % cjns1989 - 1.43 version of polyglossia on this system does not allow disabling the autospacing feature
\fi

\title{SAINT-SIMON}
\author{Mémoires XVII}
\date{}

\begin{document}
\maketitle

\hypertarget{chapitre-premier.}{%
\chapter{CHAPITRE PREMIER.}\label{chapitre-premier.}}

1718

~

{\textsc{Message étrange que M. le duc d'Orléans m'envoie par le marquis
de Biron, au sortir du lit de justice.}} {\textsc{- Dispute entre M. le
duc d'Orléans et moi, qui me force d'aller à Saint-Cloud annoncer à
M\textsuperscript{me} la duchesse d'Orléans la chute de son frère,
interrompue par les conjouissances de l'abbé Dubois et les nouvelles de
l'abattement du parlement.}} {\textsc{- La dispute fortement reprise
après\,; puis raisonnements et ordres sur ce voyage.}} {\textsc{- Ma
prudence confondue par celle d'un page.}} {\textsc{- Folie de
M\textsuperscript{me} la duchesse d'Orléans sur sa bâtardise.}}
{\textsc{- On ignore à Saint-Cloud tout ce qui s'est passé au lit de
justice.}} {\textsc{- J'entre chez M\textsuperscript{me} la duchesse
d'Orléans.}} {\textsc{- Je quitte M\textsuperscript{me} la duchesse
d'Orléans et vais chez Madame.}} {\textsc{- Menace folle et impudente de
la duchesse du Maine au régent, que j'apprends par Madame.}} {\textsc{-
M\textsuperscript{me} la duchesse d'Orléans m'envoie chercher chez
Madame, qui me prie de revenir après chez elle.}} {\textsc{- Lettre de
M\textsuperscript{me} la duchesse d'Orléans, écrite en partie de sa
main, en partie de la mienne (dictée par elle), singulièrement belle.}}
{\textsc{- J'achève avec Madame, que M\textsuperscript{me} la duchesse
d'Orléans envoie prier de descendre chez elle.}} {\textsc{- J'entretiens
la duchesse Sforze.}} {\textsc{- Je rends compte de mon voyage à M. le
duc d'Orléans.}} {\textsc{- Conversation sur l'imminente arrivée de
M\textsuperscript{me} la duchesse d'Orléans de Saint-Cloud.}} {\textsc{-
Entrevue de M. {[}le duc{]} et de M\textsuperscript{me} la duchesse
d'Orléans, arrivant de Saint-Cloud, et de M\textsuperscript{me} la
duchesse de Berry, après avoir vu ses frères qui l'attendaient chez
elle.}} {\textsc{- Force et but de M\textsuperscript{me} la duchesse
d'Orléans, qui sort après de toute mesure.}} {\textsc{- Misère de M. le
duc d'Orléans.}} {\textsc{- Je demeure brouillé de ce moment avec
M\textsuperscript{me} la duchesse d'Orléans, sans la revoir, depuis
Saint-Cloud.}} {\textsc{- Je vais à l'hôtel de Condé\,; tout m'y rit.}}
{\textsc{- M\textsuperscript{me} de L'Aigle me presse inutilement de
lier avec M\textsuperscript{me} la Duchesse.}}

~

J'oublie qu'un peu devant que nous sortissions du cabinet du conseil
pour le lit de justice, raisonnant à part, M. le duc d'Orléans, M. le
Duc et moi, ils convinrent de se trouver ensemble avec le garde des
sceaux au Palais-Royal au sortir du lit de justice, et me proposèrent
d'y aller. J'y résistai un peu\,; mais ils le voulurent pour raisonner
sur ce qui se serait passé. Comme je vis qu'il ne s'était rien ému ni
entrepris, je me crus libre de cette conférence, bien aise aussi de
n'ajouter pas cette preuve de plus que j'avais été d'un secret qui
n'était pas sans envieux. Entrant chez moi sur les deux heures et demie,
je trouvai au bas du degré le duc d'Humières, Louville et toute ma
famille jusqu'à ma mère, que la curiosité arrachait de sa chambre, d'où
elle n'était pas sortie depuis l'entrée de l'hiver. Nous demeurâmes en
bas dans mon appartement, où, en changeant d'habit et de chemise, je
répondais à leurs questions empressées, lorsqu'on vint m'annoncer M. de
Biron, qui força ma porte, que j'avais défendue pour me reposer un peu
en liberté. Biron mit la tête dans mon cabinet, et me pria qu'il me pût
dire un mot. Je passai demi rhabillé dans ma chambre avec lui. Il me dit
que M. le duc d'Orléans s'attendait que j'irais au Palais-Royal tout
droit des Tuileries, que je le lui avais promis, et qu'il avait été
surpris de ne m'y point voir\,; que néanmoins il n'y avait pas grand
mal, et qu'il n'avait été qu'un moment avec M. le Duc et le garde des
sceaux\,; que Son Altesse Royale lui avait ordonné de me venir dire
d'aller tout présentement au Palais-Royal pour quelque chose qu'elle
désirait que je fisse. Je demandai à Biron s'il savait de quoi il
s'agissait. Il me répondit que c'était pour aller à Saint-Cloud annoncer
de sa part la nouvelle à M\textsuperscript{me} la duchesse d'Orléans. Ce
fut pour moi un coup de foudre. Je disputai avec Biron, qui convint avec
moi de la douleur de cette commission, mais qui m'exhorta à ne pas
perdre de temps à aller au Palais-Royal où j'étais attendu avec
impatience. Il ajouta que c'était une confiance pénible, mais que M. le
duc d'Orléans lui avait dit ne pouvoir prendre qu'en moi, et le lui
avait dit de manière à ne lui pas laisser d'espérance de m'en excuser ni
de grâce à le faire avec trop d'obstination. Je rentrai avec lui dans
mon cabinet si changé, que M\textsuperscript{me} de Saint-Simon s'écria,
et crut qu'il était arrivé quelque chose de sinistre. Je leur dis ce que
je venais d'apprendre, et, après que Biron eut causé un moment et m'eut
encore pressé d'aller promptement et exhorté à l'obéissance, il s'en
alla dîner. Le nôtre était servi. Je demeurai un peu à me remettre du
premier étourdissement, et je conclus à ne pas opiniâtrer M. le duc
d'Orléans par ma lenteur à faire ce qu'il voudrait absolument, en même
temps à n'oublier rien pour détourner de moi un message si dur et si
pénible. J'avalai du potage et un oeuf, et m'en allai au Palais-Royal.

Je trouvai M. le duc d'Orléans seul dans son grand cabinet, qui
m'attendait avec impatience, et qui se promenait à grands pas. Dès que
je parus, il vint à moi et me demanda si je n'avais pas vu Biron. Je lui
dis que oui, et qu'aussitôt je venais recevoir ses ordres\,: il me
demanda si Biron ne m'avait pas dit ce qu'il me voulait\,; je lui dis
que oui\,; que, pour lui marquer mon obéissance, j'étais venu dans le
moment à six chevaux, pour être prêt à tout ce qu'il voudrait, mais que
je croyais qu'il n'y avait pas bien fait réflexion. Sur cela, l'abbé
Dubois entra, qui le félicita du succès de cette grande matinée, qui en
prit occasion de l'exhorter à fermeté et à se montrer maître\,; je me
joignis à ces deux parties de son discours\,; je louai Son Altesse
Royale de l'air dégagé et néanmoins appliqué et majestueux qu'il avait
fait paraître, de la netteté, de la justesse, de la précision de ses
discours au conseil, et de tout ce que je crus susceptible de louanges
véritables. Je voulais l'encourager pour les suites et le capter pour le
mettre bien à son aise avec moi, et m'en avantager pour rompre mon
détestable message. L'abbé Dubois s'étendit sur la frayeur du parlement,
sur le peu de satisfaction qu'il avait eu du peuple par les rues, où qui
que ce soit ne l'avait suivi, et où des boutiques il avait pu entendre
des propos très différents de ceux dont il s'était flatté\,; en effet,
cela était vrai, et la peur saisit tellement quelques membres de la
compagnie, que plusieurs n'osèrent aller jusqu'aux Tuileries, et que, ce
signalé séditieux de Blamont, président aux enquêtes, déserta sur le
degré des Tuileries, se jeta dans la chapelle, s'y trouva si faible et
si mal, qu'il fallut avoir recours au vin des messes à la sacristie, et
aux liqueurs spiritueuses.

Ces propos de conjouissance finis, l'abbé Dubois se retira et nous
reprîmes ceux qu'il avait interrompus. M. le duc d'Orléans me dit qu'il
comprenait bien que j'avais beaucoup de peine d'apprendre à me résoudre
à M\textsuperscript{me} la duchesse d'Orléans une nouvelle aussi
affligeante pour elle dans sa manière de penser, mais qu'il m'avouait
qu'il ne pouvait lui écrire\,; qu'ils n'étaient point ensemble sur le
tour de tendresse\,; que cette lettre serait gardée et montrée\,; qu'il
valait mieux ne s'y pas exposer\,; que j'avais toujours été le
conciliateur entre eux deux, avec une confiance égale là-dessus de part
et d'autre, et toujours avec succès\,; que cela, joint à l'amitié que
j'avais pour l'un et pour l'autre, le déterminait à me prier, pour
l'amour de tous les deux, à me charger de la commission.

Je lui répondis, après les compliments et les respects requis, que, de
tous les hommes du monde, aucun n'était moins propre que moi à cette
commission, même à titre singulier\,; que j'étais extrêmement sensible
et attaché aux droits de ma dignité\,; que le rang des bâtards m'avait
toujours été insupportable\,; que j'avais sans cesse et ardemment
soupiré après ce qu'il venait d'arriver\,; que je l'avais dit cent fois
à M\textsuperscript{me} la duchesse d'Orléans, et plusieurs fois à M. du
Maine, du vivant du feu roi et depuis sa mort, et une à
M\textsuperscript{me} la duchesse du Maine, à Paris, la seule fois que
je lui eusse parlé\,; diverses fois encore à M. le comte de Toulouse\,;
que M\textsuperscript{me} la duchesse d'Orléans ne pouvait donc ignorer
que je ne fusse aujourd'hui au comble de ma joie\,; que, dans cette
situation, c'était non pas seulement un grand manquement de respect,
mais encore une insulte à moi d'aller lui annoncer une nouvelle qui
faisait tout à la fois sa plus vive douleur, et ma joie connue d'elle
pour la plus sensible. «\,Vous avez tort, me répondit M. le duc
d'Orléans, et ce n'est pas là raisonner\,; c'est justement parce que
vous avez toujours parlé franchement là-dessus aux bâtards et à
M\textsuperscript{me} d'Orléans elle-même, et que vous vous êtes conduit
tête levée à cet égard, que vous êtes plus propre qu'un autre à ce que
je vous demande. Vous avez dit là-dessus votre sentiment et votre goût à
M\textsuperscript{me} d'Orléans\,; elle ne vous en a pas su mauvais
gré\,; au contraire, elle vous l'a su bon de votre franchise et de la
netteté de votre procédé, fâchée et très fâchée de la chose en soi, mais
non point contre vous. Elle a beaucoup d'amitié pour vous. Elle sait que
vous voulez la paix et l'union du ménage\,; il n'y a personne dont elle
le reçoive mieux que de vous, et il n'y a personne de plus propre que
vous à le bien faire, vous qui êtes dans tout l'intérieur de la famille,
et à qui elle et moi, chacun de notre côté, parlons à coeur ouvert les
uns sur les autres. Ne me refusez point cette marque-là d'amitié\,; je
sens parfaitement combien le message est désagréable\,; mais dans les
choses importantes, il ne faut pas refuser ses amis.\,»

Je contestai, je protestai\,; grands verbiages de part et d'autre\,;
bref, nul moyen de m'en défendre. J'eus beau lui dire que cela me
brouillerait avec elle\,; que le monde trouverait très étrange que je me
chargeasse de cette ambassade, point d'oreilles à tout cela, et
empressements si redoublés qu'il fallut céder.

Le voyage conclu, je lui demandai ses ordres. Il me dit que le tout ne
consistait qu'à lui dire le fait de sa part, et d'y ajouter précisément
que, sans des preuves bien fortes contre son frère, il ne se serait pas
porté à cette extrémité. Je lui dis qu'il devait s'attendre à tout de la
douleur de sa femme, et en trouver tout bon dans ces premiers jours\,;
lui laisser la liberté de Saint-Cloud, de Bagnolet, de Paris, de
Montmartre, de le voir ou de ne le point voir\,; se mettre en sa place
et adoucir un si grand coup par toutes les complaisances et les
attentions imaginables\,; donner lieu et plein champ aux caprices et aux
fantaisies, et ne craindre point d'aller trop loin là-dessus. Il y entra
avec amitié et compassion pour M\textsuperscript{me} la duchesse
d'Orléans, sentant, et revenant souvent au travers qu'elle avait si
avant sur sa bâtardise, moi rompant la mesure, et disant qu'il n'était
pas maintenant saison de le trouver mauvais. Je lui demandai aussi de ne
point trouver mauvais ni étrange si M\textsuperscript{me} la duchesse
d'Orléans, sachant ce que je lui portais, refusait de me voir. Il me
permit, en ce cas, de n'insister point, et me promit de ne s'en fâcher
pas contre elle. Après ces précautions, de la dernière desquelles je
méditais de faire usage, je le priai de me dire si, Madame étant à
Saint-Cloud, il me chargeait de la voir ou non. Il me remercia d'y avoir
pensé, et me pria de lui rendre compte de sa part de toute sa matinée,
et surtout me recommanda de revenir tout droit lui dire comment le tout
se serait passé. Je protestai encore de l'abus qu'il faisait de mon
obéissance, de ma juste répugnance, de mes raisons personnelles et
particulières de résistance, des propos du monde auxquels il
m'exposait\,; et finalement je le quittai comblé de ses amitiés et de
douleur de ce qu'il exigeait de la mienne.

Sortant d'avec lui, je trouvai un page de M\textsuperscript{me} la
duchesse d'Orléans, tout botté, qui arrivait de Saint-Cloud. Je le priai
d'y retourner sur-le-champ au galop, de dire en arrivant à la duchesse
Sforze que j'y arrivais de la part de M. le duc d'Orléans\,; que je la
suppliais que je la trouvasse en descendant de carrosse, et que je la
pusse entretenir en particulier avant que je visse M\textsuperscript{me}
la duchesse d'Orléans ni personne. Mon projet était de ne voir qu'elle,
de la charger du paquet, sous couleur de plus de respect pour
M\textsuperscript{me} la duchesse d'Orléans, de ne la point voir,
puisque je m'étais assuré que M. le duc d'Orléans ne trouverait pas
mauvais qu'elle refusât de me voir, et de lui faire trouver bon à mon
retour que j'en eusse usé de la sorte. Mais toute ma pauvre prudence fut
confondue par celle du page, qui n'en eut pas moins que moi. Il se garda
bien d'être porteur de telles nouvelles qu'il venait d'apprendre au
Palais-Royal, et qui étaient publiques partout. Il se contenta de dire
que j'arrivais, envoyé par M. le duc d'Orléans, ne sonna mot à
M\textsuperscript{me} Sforze, et disparut tout aussitôt. C'est ce que
j'appris par la suite, et ce que je vis presque aussi clairement en
arrivant à Saint-Cloud.

J'y étais allé au petit trot pour donner loisir au page d'arriver devant
moi, et à la duchesse Sforze de me recevoir. Pendant le chemin, je
m'applaudissais de mon adresse\,; mais je ne laissais pas d'appréhender
qu'il faudrait voir M\textsuperscript{me} la duchesse d'Orléans après
M\textsuperscript{me} Sforze. Je ne pouvais pas m'imaginer que
Saint-Cloud fût encore en ignorance des faits principaux de la matinée,
et néanmoins j'étais dans une angoisse qui ne se peut exprimer, et qui
redoublait à mesure que j'approchais du terme de ce triste voyage. Je me
représentais le désespoir d'une princesse folle de ses frères, au point
que, sans les aimer, surtout le duc du Maine, elle n'estimait sa propre
grandeur qu'en tant qu'elle relevait et protégeait la leur, avec
laquelle rien n'avait de proportion dans son esprit, et pour laquelle
rien n'était injuste\,; qui, accoutumée à une égalité de famille par les
intolérables préférences du feu roi pour ses bâtards sur ses enfants
légitimes, considérait son mariage comme pour le moins égal, et l'état
royal de ses frères comme un état naturel, simple, ordinaire, de droit,
sans la plus légère idée que cela pût être autrement, et qui regardait
avec compassion dans moi, et avec un mépris amer dans les autres,
quiconque imaginait quelque chose de différent à ce qu'elle pensait à
cet égard\,; qui verrait ce colosse monstrueux de grandeur présente et
future, solennellement abattu par son mari, et qui me verrait venir de
sa part sur cette nouvelle, moi qui étais dans sa confidence la plus
intime et la plus étroite sur toutes choses, moi dont elle ne pouvait
ignorer l'excès de ma joie de cela même qui ferait sa plus mortelle
douleur. S'il est rude d'annoncer de fâcheuses nouvelles aux plus
indifférents, combien plus à des personnes en qui l'estime et l'amitié
véritable et le respect du rang se trouvent réunis, et quel embarras de
plus dans une espèce si singulière\,!

Pénétré de ces sentiments douloureux, mon carrosse arrive au fond de la
grande cour de Saint-Cloud, et je vois tout le monde aux fenêtres et
accourir de toutes parts. Je mets pied à terre, et je demande au premier
que je trouve de me mener chez M\textsuperscript{me} Sforze, dont
j'ignorais le logement. On y court\,: on me dit qu'elle est au salut
avec M\textsuperscript{me} la duchesse d'Orléans, dont l'appartement
n'était séparé de la chapelle que par un vestibule, à l'entrée duquel
j'étais. Je me jette chez la maréchale de Rochefort, dont le logement
donnait aussi sur ce vestibule, et je prie qu'on m'y fasse venir
M\textsuperscript{me} Sforze. Un moment après, on me vint dire qu'on ne
savait ce qu'elle était devenue, et que M\textsuperscript{me} la
duchesse d'Orléans, sur mon arrivée, retournait m'attendre dans son
appartement. Un autre tout aussitôt me vint chercher de sa part\,; puis
un second coup sur coup. Je n'avais qu'un cri après la duchesse Sforze,
résolu de l'attendre, lorsque incontinent la maréchale de Rochefort
arriva, clopinant sur son bâton, que M\textsuperscript{me} la duchesse
d'Orléans envoyait elle-même pour m'amener chez elle. Grande dispute
avec elle, voulant toujours voir M\textsuperscript{me} Sforze, qui ne se
trouvait point. Je voulus aller chez elle pour m'éloigner et me donner
du temps\,; mais la maréchale inexorable me tirait par les bras, me
demandant toujours les nouvelles que j'apportais. À bout enfin, je lui
dis\,: «\,Celles que vous savez. --- Comment\,! reprit-elle, c'est que
nous ne savons chose au monde, si ce n'est qu'il y a eu un lit de
justice, et nous sommes sur les charbons de savoir pourquoi, et ce qui
s'y est passé.\,» Moi, dans un étonnement extrême, je me fis répéter à
quatre fois et jurer par elle qu'il était vrai qu'on ne savait rien dans
Saint-Cloud. Je lui dis de quoi il s'agissait, et à son tour elle pensa
tomber à la renverse. J'en fis effort pour n'aller point chez
M\textsuperscript{me} la duchesse d'Orléans\,; mais jusqu'à six ou sept
messages redoublés pendant cette dispute me forcèrent d'aller avec la
maréchale, qui me tenait par le poing, s'épouvantait du cas, et me
plaignait bien de la scène que j'allais voir ou plutôt faire.

J'entrai donc à la fin, mais glacé, dans cet appartement des goulottes
de M\textsuperscript{me} la duchesse d'Orléans, où ses gens assemblés me
regardèrent avec frayeur par celle qui était peinte sur mon visage. En
entrant dans la chambre à coucher la maréchale me laissa. On me dit que
Son Altesse Royale était dans un salon de marbre qui y tient et est plus
bas de trois marches. J'y tournai, et du plus loin que je la vis, je la
saluai d'un air tout différent de mon ordinaire. Elle ne s'en aperçut
pas d'abord, et me pria de m'approcher, d'un air gai et naturel. Me
voyant après arrêté au bas de ces marches\,: «\,Mon Dieu, monsieur,
s'écria-t-elle, quel visage vous avez\,! Que m'apportez-vous\,?» Voyant
que je demeurais sans bouger et sans répondre, elle s'émut davantage en
redoublant sa question. Je fis lentement quelques pas vers elle, et à sa
troisième question\,: «\,Madame, lui dis-je, est-ce que vous ne savez
rien\,? --- Non, monsieur, je ne sais quoi que ce soit au monde qu'un
lit de justice, et rien de ce qui s'est passé. --- Ah\,! madame,
interrompis-je en me détournant à demi, je suis donc encore bien plus
malheureux que je ne pensais l'être\,! --- Quoi donc, monsieur\,?
reprit-elle, dites vivement\,: qu'y a-t-il donc\,?» En se levant à son
séant d'un canapé sur lequel elle était couchée\,: «\,Approchez-vous
donc, asseyez-vous.\,» Je m'approchai, et lui dis que j'étais au
désespoir. Elle, de plus en plus émue, me dit\,: «\,Mais parlez donc\,;
il vaut mieux apprendre les mauvaises nouvelles par ses amis que par
d'autres.\,» Ce mot me perça le coeur et ne me rendit sensible qu'à la
douleur que je lui allais donner. Je m'avançai encore vers elle, et lui
dis enfin que M. le duc d'Orléans avait réduit M. le duc du Mairie au
rang unique d'ancienneté de sa pairie, et en même temps rétabli M. le
comte de Toulouse dans tous les honneurs dont il jouissait. Je fis en
cet endroit une pause d'un moment, puis j'ajoutai qu'il avait donné à M.
le Duc la surintendance de l'éducation du roi.

Les larmes commencèrent à couler avec abondance. Elle ne me répondit
point, ne s'écria point, mais pleura amèrement. Elle me montra un siège
et je m'assis, les yeux fichés à terre pendant quelques instants.
Ensuite je lui dis que M. le duc d'Orléans, qui m'avait plutôt forcé que
chargé d'une commission si triste, m'avait expressément ordonné de lui
dire qu'il avait des preuves en main très fortes contre M. du Maine\,;
que sa considération à elle l'avait retenu longtemps, mais qu'il n'avait
pu différer davantage. Elle me répondit avec douceur que son frère était
un malheureux, et peu après me demanda si je savais son crime et de
quelle espèce. Je lui dis que M. le duc d'Orléans ne m'en avait du tout
appris que ce que je venais de lui rendre\,; que je n'avais osé le
questionner sur une matière de cette nature, voyant qu'il ne m'en disait
pas plus.

Un moment après je lui dis que M. le duc d'Orléans m'avait expressément
chargé de lui témoigner la douleur très vive qu'il ressentait de la
sienne\,; à quoi j'ajoutai tout ce que le trouble où j'étais me put
permettre de m'aviser pour adoucir un compliment si terrible, et après
quelques interstices, je lui témoignai ma douleur particulière de la
sienne, toute la répugnance que j'avais eue à ce triste message, toute
la résistance que j'y avais apportée, à quoi elle ne me répondit
{[}que{]} par des signes et quelques mots obligeants entrecoupés de
sanglots. Je finis, suivant l'expresse permission que j'en avais de M.
le duc d'Orléans, par lui glisser que j'avais essayé de parer ce coup.
Sur quoi elle me dit que pour le présent je la voudrais bien dispenser
de la reconnaissance. Je repris qu'il était trop juste qu'elle ne pensât
qu'à sa douleur, et à chercher tout ce qui la pourrait soulager\,; que
tout ce qui y contribuerait serait bon à M. le duc d'Orléans\,: le voir,
ne le point voir que lorsqu'elle le désirerait\,; demeurer à
Saint-Cloud, aller à Bagnolet ou à Montmartre, d'y demeurer tant qu'il
plairait, en un mot tout ce qu'elle désirerait faire\,; que j'avais
charge expresse de la prier de ne se contraindre sur rien et de faire
tout ce qu'il lui conviendrait davantage. Là-dessus elle me demanda si
je ne savais point ce que M. le duc d'Orléans voudrait sur ses frères,
et qu'elle ne les verrait point si cela ne lui convenait pas. Je
répondis que, n'ayant nul ordre à cet égard, c'était une marque qu'il
trouverait fort bon qu'elle les vît\,; qu'à l'égard de M. le comte de
Toulouse, conservé en entier, il n'y pouvait avoir aucune matière à
difficulté, et que pour M. le duc du Maine, je n'y en croyais pas
davantage, que je hasarderais même de lui en répondre s'il en était
besoin. Elle me parla encore de celui-ci\,; qu'il fallait qu'il fût bien
criminel\,; qu'elle était réduite à le souhaiter. Un redoublement de
larmes suivit ces dernières paroles.

Je restai quelque temps sur mon siège, n'osant lever les yeux dans
l'état du monde le plus pénible, incertain de demeurer ou de m'en aller.
Enfin je lui dis mon embarras\,; que je croyais néanmoins qu'elle serait
bien aise d'être seule quelque temps avant de me donner ses ordres, mais
que le respect me tenait dans un égal suspens de rester ou de la
laisser. Après un peu de silence, elle témoigna qu'elle désirait ses
femmes. Je me levai, les lui envoyai et leur dis que, si Son Altesse
Royale me demandait, on me trouverait chez Madame, chez la duchesse
Sforze ou chez la maréchale de Rochefort. Je ne trouvai ni l'une ni
l'autre de ces deux dames, et je montai chez Madame.

Je vis bien en entrant qu'on s'y attendait à me voir et qu'on en avait
même impatience. Je fus environné du peu de monde qui était dans sa
chambre, à qui je ne m'ouvris de rien, tandis qu'on l'avertissait dans
son cabinet, où elle écrivait, comme elle faisait presque toujours, et
me fit entrer dans l'instant. Elle se leva dès que je parus, et me dit
avec empressement\,: «\,Hé bien\,! monsieur, voilà bien des
nouvelles\,!» En même temps ses dames sortirent, et je demeurai seul
avec elle. Je lui fis mes excuses de n'être pas venu d'abord chez elle
comme le devoir le voulait, sur ce que M. le duc d'Orléans m'avait
assuré qu'elle trouverait bon que je commençasse par
M\textsuperscript{me} la duchesse d'Orléans. Elle le trouva très bon en
effet, puis me demanda les nouvelles avec grand empressement. Ma
surprise fut extrême lorsque je connus enfin qu'elle n'en savait nulle
autre que le lit de justice et chose aucune de ce qui s'y était passé.
Je lui dis donc l'éducation du roi donnée à M. le Duc, la réduction des
bâtards au rang de leurs pairies, et le rétablissement du comte de
Toulouse. La joie se peignit sur son visage. Elle me répondit avec un
grand \emph{enfin} redoublé qu'il y avait longtemps que son fils aurait
dû l'avoir fait, mais qu'il était trop bon. Je la fis souvenir qu'elle
était debout\,; mais par politesse elle y voulut rester. Elle me dit que
c'était où la folie de M\textsuperscript{me} du Maine avait conduit son
mari, me parla du procès des princes du sang contre les bâtards, et me
conta l'extravagance de M\textsuperscript{me} du Maine, qui, après
l'arrêt intervenu entre eux, avait dit en face à M. le duc d'Orléans, en
lui montrant ses deux fils, qu'elle les élevait dans le souvenir et dans
le désir de venger le tort qu'il leur avait fait.

Après quelques propos de part et d'autre sur la haine, le discours, les
mauvais offices et pis encore du duc et de M\textsuperscript{me} la
duchesse du Maine contre M. le duc d'Orléans, Madame me pria de lui
conter de fil en aiguille (ce fut son terme) le détail de cette célèbre
matinée. Je la fis encore inutilement souvenir qu'elle était debout et
lui représentai que ce qu'elle désirait apprendre serait long à
raconter\,; mais son ardeur de le savoir était extrême. M. le duc
d'Orléans m'avait ordonné de lui tout dire, tant ce qui s'était passé au
conseil qu'au lit de justice. Je le fis donc à commencer dès le matin.
Au bout d'un quart d'heure Madame s'assit, mais avec la plus grande
politesse. Je fus près d'une heure avec elle à toujours parler et
quelquefois à répondre à quelques questions, elle ravie de l'humiliation
du parlement et de celle des bâtards, et que M. son fils eût enfin
montré de la fermeté.

La maréchale de Rochefort fit demander à entrer\,; et après des excuses
de M\textsuperscript{me} la duchesse d'Orléans à Madame, elle lui
demanda permission de m'emmener, parce que Son Altesse Royale me voulait
parler. Madame m'y envoya sur-le-champ, mais en me priant bien fort de
revenir chez elle dès que j'aurais fait avec M\textsuperscript{me} la
duchesse d'Orléans. Je descendis donc avec la maréchale. En entrant dans
l'appartement de Son Altesse royale, ses femmes et tous ses gens
m'environnèrent pour que je l'empêchasse d'aller à Montmartre, où elle
venait de dire qu'elle s'en allait. Je les assurai que mon message était
bien assez fâcheux sans que j'y ajoutasse de moi-même\,; que Son Altesse
Royale n'était point dans un état à la contraindre ni à la contredire\,;
que j'avais bien prévu qu'elle voudrait aller à Montmartre, et pris mes
précautions là-dessus\,; que M. le duc d'Orléans trouvait bon cela et
toute autre chose qui serait au soulagement et à la consolation de Son
Altesse Royale, et qu'ainsi je n'en dirais pas une parole.

J'avançai, toujours importuné là-dessus, et je trouvai
M\textsuperscript{me} la duchesse d'Orléans sur le même canapé où je
l'avais laissée, une écritoire sur ses genoux et la plume à la main. Dès
qu'elle me vit, elle me dit qu'elle s'en allait à Montmartre, puisque je
l'avais assurée que M. le duc d'Orléans le trouvait bon\,; qu'elle lui
écrivait pour lui en demander pourtant la permission, et me lut sa
lettre, commencée de six ou sept lignes de grande écriture sur de petit
papier\,; puis, me regardant avec un air de douceur et d'amitié\,:
«\,Les larmes me gagnent, me dit-elle\,; je vous ai prié de descendre
pour me rendre un office\,: la main ne va pas bien\,; je vous prie
d'achever d'écrire pour moi\,;» et me tendit l'écritoire et sa lettre
dessus. Je la pris, et elle m'en dicta le reste, que j'écrivis tout de
suite à ce qu'elle avait écrit.

Je fus frappé du dernier étonnement d'une lettre si concise, si
expressive, des sentiments les plus convenables, des termes si choisis,
tout enfin dans un ordre et une justesse qu'auraient à peine produits
dans le meilleur écrivain les réflexions les plus tranquilles, et cela
couler de source parmi le plus violent trouble, l'agitation la plus
subite et le plus grand mouvement de toutes les passions, à travers les
sanglots et un torrent de larmes. Elle finissait qu'elle allait pour
quelque temps à Montmartre pleurer le malheur de ses frères et prier
Dieu pour sa prospérité. J'aurai regret toute ma vie de ne l'avoir pas
transcrite. Tout y était si digne, si juste, si compassé que tout y
était également dans le vrai et dans le devoir, une lettre enfin si
parfaitement belle qu'encore que je me souvienne en gros de ce qu'elle
contenait, je n'ose l'écrire de peur de la défigurer. Quel profond
dommage que tant d'esprit, de sens, de justesse, qu'un esprit si capable
de se posséder dans les moments premiers si peu susceptibles de frein,
se soit rendu inutile à tout et pis encore, par cette fureur de
bâtardise qui perdit et consuma tout\,!

La lettre écrite, je la lui lus. Elle ne la voulut point fermer, et me
pria de la rendre. Je lui dis que je remontais chez Madame, et qu'avant
partir, je saurais de Son Altesse Royale si elle n'avait plus rien à
m'ordonner. Comme j'achevais avec Madame, la duchesse Sforze vint lui
parler de la part de M\textsuperscript{me} la duchesse d'Orléans sur son
voyage de Montmartre, pour la prier de garder avec elle
M\textsuperscript{lle} de Valois. La mère et la fille n'étaient pas trop
bien ensemble, et celle-ci haïssait souverainement les bâtards et leur
rang. Madame avec bonté approuva tout ce que voudrait
M\textsuperscript{me} la duchesse d'Orléans, plaignant sa douleur. Après
cette parenthèse, je repris mon narré.

Comme il finissait, la maréchale de Rochefort revint prier Madame de
vouloir bien descendre chez M\textsuperscript{me} la duchesse d'Orléans,
qui, en l'état où elle était, ne pouvait monter, et nous dit qu'elle
changeait d'avis pour Montmartre, et resterait à Saint-Cloud. La
maréchale sortie, je finis et je suivis Madame. Je ne voulus point
entrer avec elle chez M\textsuperscript{me} la duchesse d'Orléans pour
les laisser plus libres. M\textsuperscript{me} Sforze en sortit, qui me
dit que le voyage était encore changé, et qu'elle allait à Paris.
Là-dessus je la priai de rendre à Son Altesse Royale la lettre qu'elle
m'avait donnée pour M. le duc d'Orléans, et de savoir si elle n'avait
rien à m'ordonner.

M\textsuperscript{me} Sforze revint aussitôt, me mena chez elle, puis
prendre l'air au bord de ce beau bassin qui est devant le degré du
château. Nous nous assîmes du côté des goulottes, où il me fallut encore
bien conter. Je n'oubliai pas de me servir de la permission de M. le duc
d'Orléans pour lui dire ce que j'avais fait pour sauver le duc du
Maine\,; mais je voulus y ajouter que, voyant l'éducation sans
ressource, j'avais voulu la réduction au rang des pairies, et fait faire
en même temps le rétablissement du comte de Toulouse. J'appuyai sur ce
que j'avais toujours professé nettement à cet égard avec les bâtards,
même et surtout avec M\textsuperscript{me} la duchesse d'Orléans,
auxquels je ne tenais pas parole, puisque j'en sauvais un, n'ayant pu
empêcher la privation de l'éducation à l'autre contre mon plus sensible
intérêt. M\textsuperscript{me} Sforze, femme très sûre et fort mon amie,
qui avait ses raisons personnelles de n'aimer ni M. ni
M\textsuperscript{me} du Maine, et n'était fâchée que de la douleur de
M\textsuperscript{me} la duchesse d'Orléans, me dit qu'elle voulait
ignorer ce que j'avais fait pour obtenir la réduction du rang, mais
qu'elle ferait usage du reste. J'étais attaché d'amitié à
M\textsuperscript{me} la duchesse d'Orléans. Elle me témoignait toute
confiance. Elle me devait de la reconnaissance en toutes les façons
possibles. Je n'étais pas inutile entre elle et M. le duc d'Orléans. Je
désirais fort demeurer en état de contribuer à leur union et au bien
intérieur de la famille. Après de longs propos je la priai de se charger
auprès de M\textsuperscript{me} la duchesse d'Orléans de ce que je
n'attendais point que Madame fût sortie de chez elle pour la voir
encore, puisqu'elle allait à Paris, et je m'en allai droit au
Palais-Royal, où je trouvai M. le duc d'Orléans avec
M\textsuperscript{me} la duchesse de Berry. Il me vint trouver dans ce
même grand cabinet dès qu'il m'y sut, où je lui rendis compte de tout ce
qui s'était passé.

Il fut ravi de la joie, que Madame m'avait témoignée sur le duc du
Maine, et me dit que celle de M\textsuperscript{me} la duchesse de
Lorraine ne serait pas moindre. Il en venait de recevoir une lettre
toute là-dessus, pour l'en presser, et Madame me venait de dire qu'elle
en avait une d'elle, toute sur le même sujet. Mais il ne fut pas si
content de l'arrivée si prochaine de M\textsuperscript{me} la duchesse
d'Orléans, dont il me parut fort empêtré. Je lui dis, outre la vérité,
ce que je crus le plus propre à le toucher, et lui faire valoir son
respect, son obéissance, sa soumission à ses sentiments, et toute la
douceur et la soumission qu'elle avait fait paraître dès les premiers
moments. Je lui vantai surtout sa lettre, et je n'oubliai pas aussi ce
que je lui avais glissé par sa permission, et dit encore à
M\textsuperscript{me} Sforze, sur mon compte, à l'égard des bâtards. Il
me demanda conseil s'il la verrait en arrivant. Je lui dis que je
croyais qu'il devait descendre dans son cabinet au moment de son
arrivée\,; faire appeler M\textsuperscript{me} Sforze, la charger de
dire à M\textsuperscript{me} la duchesse d'Orléans qu'il était là pour
la voir ou ne la point voir, tout comme elle l'aimerait mieux, sans
nulle contrainte, savoir de ses nouvelles, et faire après tout ce
qu'elle voudrait là-dessus\,; que, s'il la voyait, il fallait lui faire
toutes les amitiés possibles\,; s'attendre à la froideur, peut-être aux
reproches, sûrement aux larmes et aux cris\,; mais qu'il était de
l'humanité, de plus, de son devoir d'honnête homme de souffrir tout
cela, en cette occasion, avec toute sorte de douceur et de patience, et,
quoi qu'elle pût dire ou faire, ne l'en traiter que mieux. Je lui
inculquai bien cela dans la tête, et, après m'être un peu vengé à lui
reprocher l'abus qu'il venait de faire de moi, je le laissai dans
l'attente de cette importune arrivée, et m'en allai me reposer, excédé
et poussé à bout, après une telle huitaine, d'une dernière journée si
complète en fatigue de corps et d'esprit, et j'entrai chez moi qu'il
était presque nuit.

Je sus après que M\textsuperscript{me} la duchesse d'Orléans était
arrivée au Palais-Royal une demi-heure après que j'en fus sorti. Ses
frères l'attendaient dans son appartement. Dès qu'elle les aperçut, elle
leur demanda s'ils avaient la permission de la voir, et, les yeux secs,
leur déclara qu'elle ne les verrait jamais si M. le duc d'Orléans le
désirait. Ensuite ils s'enfermèrent une heure ensemble. Dès qu'ils
furent sortis, M. le duc d'Orléans y descendit avec
M\textsuperscript{me} la duchesse de Berry, qui était restée pour le
soutenir dans cet assaut. Jamais tant de force ni de raison. Elle dit à
M. le duc d'Orléans qu'elle sentait trop l'extrême honneur qu'il lui
avait fait en l'épousant, pour que tout autre sentiment ne cédât pas à
celui-là. C'était la première fois depuis trente ans qu'elle lui parlait
de la sorte. Puis s'attendrissant, elle lui demanda pardon de pleurer le
malheur de son frère, qu'elle croyait très coupable, et qu'elle désirait
tel puisqu'il l'avait jugé digne d'un si grand châtiment. Là-dessus
pleurs, sanglots, cris de la femme, de la fille, du mari même, qui se
surpassèrent en cette occasion. Cette triste scène dura une heure.
Ensuite M\textsuperscript{me} la duchesse d'Orléans se mit au lit, et M,
le duc d'Orléans et M\textsuperscript{me} la duchesse {[}de Berry{]}
remontèrent le degré. Le soulagement alors fut grand de toutes parts.

Le lendemain et le jour suivant se passèrent en douceur, après lesquels
M\textsuperscript{me} la duchesse d'Orléans, succombant aux efforts
qu'elle s'était faits, commença d'aller au but qu'elle s'était proposé,
de savoir les crimes de son frère, puis de tâcher de lui ménager une
audience de son mari, espérant tout du face à face\,; enfin de proposer
la publication de ses méfaits ou son rétablissement. À mesure qu'elle ne
réussissait pas, chagrins, larmes, aigreur, emportements, fureurs, et
fureurs sans mesure. Elle s'enferma sans vouloir voir le jour ni son
fils même, qu'elle aimait avec passion, et porta les choses au delà de
toute sorte de mesure. Elle savait bien à qui elle avait affaire. Tout
autre que M. le duc d'Orléans, se voyant à bout de complaisance et
d'égards, lui eût demandé, une bonne fois et bien ferme, lequel elle
aimait le mieux et de préférence de lui ou de son frère\,: si lui,
qu'elle ne devait avoir d'autres intérêts que les siens, et ne lui
parler jamais de son frère ni de rien qui en approchât, ce qu'il lui
défendait très expressément, et ne pas troubler le repos et
l'intelligence de leur union par ce qui ne pouvait que la rompre\,; si
son frère, qu'elle pouvait se retirer au lieu qu'il lui marquerait et
avec la suite et les gens qu'il choisirait, et compter d'y passer sa vie
sans entendre jamais parler de ses frères, non plus que de lui ni de
leurs enfants (avec ce sage et nécessaire compliment, et une conduite
soutenue, M. le duc d'Orléans se serait bien épargné des scènes, des
chagrins, des dépits, des importunités, des malais ses et des misères,
et à M\textsuperscript{me} la duchesse d'Orléans aussi), et chasser
sur-le-champ M\textsuperscript{me} de Châtillon, les Saint-Pierre et
quelques bas domestiques qui faisaient leur cour à M\textsuperscript{me}
la duchesse d'Orléans de l'entretenir en cette humeur, et qui étaient
son conseil là-dessus, pour la gouverner dans tout le reste.

Ce n'était pas à moi à inspirer une si salutaire conduite à M. le duc
d'Orléans. Aussi me gardai-je très soigneusement de lui en laisser
apercevoir la plus petite lueur. Je fus d'autant plus réservé à ne lui
jamais parler de M\textsuperscript{me} la duchesse d'Orléans là-dessus,
et à laisser tomber tout discours quand il m'en faisait ses plaintes,
qu'ayant dit à M\textsuperscript{me} Sforze, à Saint-Cloud, que je la
priais de dire à M\textsuperscript{me} la duchesse d'Orléans que je
croyais plus respectueux de la laisser ces premiers jours sans
l'importuner peut-être, j'attendrais à avoir l'honneur de la voir
jusqu'à ce que Son Altesse Royale me fît dire par elle d'y aller. Le
lendemain j'allai seulement savoir de ses nouvelles sans entrer. Je vis
après M\textsuperscript{me} Sforze, qui me dit que Son Altesse Royale me
priait de ne pas trouver mauvais, si elle avait quelque peine à me voir
dans ces premiers jours. J'y entrai fort bien, et compris le contraste
que faisait en elle la joie, qu'elle ne pouvait douter que j'eusse, avec
sa douleur. Mais ces quelques jours n'ont point eu de fin, et de ce
moment je demeurai brouillé avec elle. J'aurai lieu d'en parler plus
d'une fois.

Rentrant chez moi, de Saint-Cloud, je pensai qu'il fallait aller à
l'hôtel de Condé, où j'appris que tout le monde était accouru aux
compliments. J'y trouvai M\textsuperscript{me} la Duchesse au lit, qui
avait pris médecine, dont le jour avait été mal choisi. Je fus reçu à
l'hôtel de Condé à peu près comme je l'avais été à Saint-Cloud le jour
de la déclaration du mariage de M\textsuperscript{me} la duchesse de
Berry. Telle est la vicissitude de ce monde. M. le Duc m'y prit en
particulier\,; chacun m'y arrêtait. Ceux que je fréquentais le moins,
les plus commensaux de la maison, m'y firent merveilles. Je ne savais
plus en quel lieu j'étais. J'y causai longtemps en particulier avec
d'Antin, puis avec Torcy, que j'exhortai à voir son ami Valincourt,
comme je comptais bien faire de mon côté, pour retenir le comte de
Toulouse. En sortant je fus pressé par M\textsuperscript{me} de L'Aigle
de lier avec M\textsuperscript{me} la Duchesse\,; mais je n'y voulus
point entendre, et je répondis nettement que je Pavais toujours trop été
avec M\textsuperscript{me} la duchesse d'Orléans, et les deux soeurs
trop mal ensemble. Bien que M\textsuperscript{me} la Duchesse n'eût rien
su ni voulu savoir de toute cette trame, et qu'elle eût mieux aimé que
son frère eût conservé un rang supérieur au nôtre, la haine de
M\textsuperscript{me} la duchesse d'Orléans redoubla pour elle et pour
tous les siens au point le plus public et le plus excessif.

\hypertarget{chapitre-ii}{%
\chapter{CHAPITRE II}\label{chapitre-ii}}

1718

~

{\textsc{Conduite des bâtards.}} {\textsc{- O et Hautefort détournent le
comte de Toulouse de suivre la fortune de son frère.}} {\textsc{-
Caractère et propos d'Hautefort à son maître.}} {\textsc{- Conversation
entre Valincourt et moi sur le comte de Toulouse et les bâtards.}}
{\textsc{- Il revient aussi me faire les remercîments du comte de
Toulouse et m'assurer qu'il s'en tiendra à sa conservation.}} {\textsc{-
Le comte de Toulouse voit le régent, vient au conseil.}} {\textsc{- Le
duc et la duchesse du Maine se retirent à Sceaux.}} {\textsc{- Le comte
de Toulouse et M\textsuperscript{me} Sforze blâment fortement et souvent
M\textsuperscript{me} la duchesse d'Orléans de ne me point voir.}}
{\textsc{- Elle est outrée qu'il n'ait pas suivi le duc du Maine, qui
est fort mal traité par sa femme.}} {\textsc{- Séditieux et clandestin
usage de feuilles volantes en registres secrets du parlement.}}
{\textsc{- Le premier président mandé et cruellement traité par la
duchesse du Maine.}} {\textsc{- Blamont, président aux enquêtes, et deux
conseillers enlevés et conduits en diverses îles du royaume.}}
{\textsc{- Mouvements inutiles du parlement.}} {\textsc{- Effet de ce
lit de justice au dehors et au dedans du royaume Raisons qui me
détournèrent de penser alors à l'affaire du bonnet.}} {\textsc{- M. le
Duc en possession de la surintendance de l'éducation du roi.}}
{\textsc{- Sage avis de M\textsuperscript{me} d'Aligre.}} {\textsc{-
Mauvaise sécurité du régent.}} {\textsc{- Création personnelle d'un
second lieutenant général des galères en faveur du chevalier de Rancé.}}
{\textsc{- Folie du duc de Mortemart, qui envoie au régent la démission
de sa charge pour la seconde fois.}} {\textsc{- Je la fais déchirer avec
peine, et j'obtiens après la survivance de sa charge pour son fils.}}
{\textsc{- Ma dédaigneuse franchise avec le duc de Mortemart.}}
{\textsc{- Survivances des gouvernements du duc de Charost à son fils\,;
de grand maître de la garde-robe\,; des gouvernements de Normandie et de
Limousin, aux fils des ducs de La Rochefoucauld, de Luxembourg et de
Berwick, et du pays de Foix au fils de Ségur, qui épouse une bâtarde,
non reconnue, de M. le duc d'Orléans.}} {\textsc{- La Fare, lieutenant
général de Languedoc, et l'abbé de Vauréal maître de l'oratoire.}}
{\textsc{- Gouvernement de Douai à d'Estaing.}} {\textsc{-
M\textsuperscript{me} la duchesse d'Orléans, qui s'était tenue enfermée
depuis le lit de justice, revoit le monde et joue.}}

~

Le duc du Maine et le comte de Toulouse, au sortir du cabinet du
conseil, descendirent dans l'appartement du duc du Maine, où ils
s'enfermèrent avec leurs plus confidents. Ils les surent si bien
choisir, que nul n'a su ce qu'il s'y passa. On peut, je crois, sans
jugement téméraire, imaginer qu'il s'y proposa bien des choses que la
sagesse du comte de Toulouse empêcha moins que le peu d'ordre et de
préparation de la cabale, et la prompte venue du parlement en trouble,
qui ne donna pas loisir d'y faire des pratiques. Le cardinal de Polignac
y fut toujours avec eux et leurs principaux amis en très petit nombre.
Je n'ai jamais compris comment ils ne tentèrent pas de se trouver au lit
de justice, pour y parler et y faire tous leurs efforts. La faiblesse
qu'ils connaissaient si bien dans le régent, surtout en face, les y
devait convier puissamment\,; mais la peur extrême, qui fut visible dans
le duc du Maine, ne lui permit pas sans doute d'y penser, encore moins
de se hasarder à rien. Il avait vu le régent si libre dans sa taille,
qu'il ne douta jamais qu'il ne fût bien préparé à tout\,; et moins un
grand coup, et si secrètement préparé était de son génie, plus il
redouta tout ce qu'il en ignorait. Quoi qu'il en soit, le comte de
Toulouse n'en sortit pour aller chez lui qu'après cinq heures du soir,
où il fit contenance de vouloir s'en aller à la suite de son frère. Ils
n'avaient rien su de précis qu'après le lit de justice, et ils avaient
eu trois heures à raisonner ensemble depuis.

La différence mise entre les deux frères combla la douleur de l'aîné et
le dépit de sa femme, et les remua plus que tout le reste à persuader au
comte de Toulouse de suivre leur fortune. Il témoigna chez lui son
penchant à le faire\,; mais d'O, qui avait conservé sur son esprit comme
dans sa maison une espèce de majordomat d'ancien gouverneur, l'en
détourna. Ce n'était pas qu'il ne fût fort attaché au duc du Maine\,;
mais il l'était plus encore à son intérêt, qui n'était pas d'anéantir
son maître et de le confiner à la campagne. On sut après que la
franchise avec laquelle le chevalier d'Hautefort lui avait parlé acheva
de lui faire prendre le bon parti. Le chevalier d'Hautefort était son
écuyer et lieutenant général de mer, frère du premier écuyer de
M\textsuperscript{me} la duchesse de Berry, de Surville, qui avait eu le
régiment du roi, si connu par ses disgrâces, et d'Hautefort, lieutenant
général, mort depuis chevalier de l'ordre, fort fâché avec raison de
n'être pas maréchal de France. Hautefort, {[}écuyer{]} du comte de
Toulouse, était un rustre qui, sans aucune vertu ni philosophie, s'était
persuadé d'affecter l'une et l'autre pour se faire admirer aux sots, et
sa place auprès du comte de Toulouse l'avait fait arriver à bon marché
dans la marine. Il lui dit nettement qu'il était la dupe de gens qui ne
l'avaient jamais aimé, qui avaient toujours tout fait sans lui, qui
s'étaient mis eux et leurs enfants sur sa tête, et dont les entreprises
folles les avaient conduits au point où ils se trouvaient\,; que,
quelque douloureuse que lui fût une chute, elle lui valait une
distinction inouïe et la plus flatteuse\,; que c'était à lui à peser
s'il voulait abandonner et perdre cette même distinction et toutes les
fonctions de ses charges, pour suivre une folle et un homme qui en
eux-mêmes s'en moqueraient de lui, et s'enterrer tout vif dans
Rambouillet avant quarante ans, où, après les premiers jours
d'admiration des sots, chacun le laisserait là et trouverait son choix
ridicule, dont il aurait tout le temps de s'ennuyer et de se repentir.
Que pour lui, il lui disait librement qu'ayant tant fait que d'être à
lui, il avait compté être avec un prince du sang, vrai ou d'apparence,
non à un particulier, et être avec un amiral auprès de qui il mènerait
dans son métier une vie agréable et considérée\,; qu'il serait ravi sur
ce pied-là de demeurer toute sa vie avec lui, mais que, pour s'enfouir
tout vivant dans Rambouillet, il le priait de n'y pas compter\,; que
tout ce qu'il y avait de bon chez lui pensait de même, et prendrait son
parti les uns après les autres\,; que pour lui, il aimait mieux le lui
dire tout d'un coup.

On assure que rien ne donna tant à penser au comte de Toulouse que cette
déclaration si prompte. Il se considéra tout seul à Rambouillet hors
d'état et de volonté de rien entreprendre, en risque d'être dégradé
comme son frère, pour son refus d'accepter le bénéfice de la déclaration
en sa faveur\,; tiraillé entre la reconnaissance qu'elle méritait, même
aux yeux du monde et la dépendance de la fortune et des caprices d'une
folle qu'il abhorrait, et d'un frère qu'il n'aimait ni n'estimait. Les
suites le firent trembler, et il prit son parti de conserver son rang et
son état ordinaire. Lui et son frère allèrent le soir au Palais-Royal
voir M\textsuperscript{me} la duchesse d'Orléans, comme je l'ai dit,
tandis que M\textsuperscript{me} du Maine et ses enfants se retirèrent à
l'hôtel de Toulouse, où ils les trouvèrent au retour. On peut juger de
la soirée\,; le maréchal de Villeroy, M. de Fréjus et très peu d'autres
les y virent. Le lendemain, samedi, M\textsuperscript{me} la duchesse
d'Orléans y alla\,; nouvelles douleurs, M\textsuperscript{me} du Maine
au lit, immobile comme une statue.

Ce même samedi, lendemain du lit de justice, j'envoyai prier Valincourt
de venir chez moi. Il y vint. Je lui parlai franchement sur le choix que
le comte de Toulouse avait à faire. Je ne lui dissimulai point ce que
j'avais voulu parer, et que n'ayant pu sauver l'éducation, ce que
j'avais obtenu sur le rang\,; que c'était moi qui avais imaginé,
proposé, et fait agréer la déclaration en faveur du comte de Toulouse.
Je le fis souvenir que je ne m'étais jamais caché sur le rang des
bâtards, et je le priai de parler si fortement à son maître, qu'il ne se
perdît pas pour son frère. Valincourt convint que j'avais raison, et me
pria qu'il pût dire au comte de Toulouse l'obligation qu'il m'avait.
C'était bien mon dessein\,; surtout je le pressai de faire que, dès le
lendemain dimanche, le comte de Toulouse se trouvât au conseil de
régence, et qu'il se défit de ses hôtes au plus tôt. Valincourt en était
déjà ennuyé\,: il revint peu après me faire les remercîments du comte de
Toulouse, et me dire que, malgré sa douleur et toutes les persécutions
de famille, il demeurerait et se trouverait le lendemain au conseil.
Cela me rafraîchit fort le sang, car j'en prévoyais l'affaiblissement et
la chute même du parti du duc et de la duchesse du Maine, et la division
prochaine des deux frères. Il me laissa entendre que le séjour de M. et
de M\textsuperscript{me} du Maine à l'hôtel de Toulouse pesait à tous,
et que le lendemain matin, dimanche, ils s'en iraient à Sceaux, où il
trouvait indécent qu'ils ne fussent pas encore\,; je priai Valincourt de
savoir du comte de Toulouse s'il voulait compliment ou silence de ma
part et de celle de M. le Duc qui en était en peine, qui mourait d'envie
de lui marquer son amitié personnelle, et qui s'était adressé à moi pour
savoir comment il en devait user à son égard. Valincourt me dit qu'il
croyait que le silence conviendrait mieux d'abord, mais qu'il le
demanderait franchement de ma part et de celle de M. le Duc, à M. le
comte de Toulouse, et qu'il me le ferait savoir. En effet, il m'écrivit
dans le soir même que M. le comte de Toulouse sentait moins sa
distinction que le malheur de son frère auquel même elle le rendait plus
sensible, et qu'il désirait que M. le Duc et moi ne lui dissions rien.
Je le fis savoir à M. Duc, et je rendis compte à M. le duc d'Orléans de
ce que j'avais fait avec Valincourt, qui fut très aise du parti que
prenait le comte de Toulouse, lequel alla voir le régent, le samedi au
soir. Cela se passa courtement, mais bien entre eux, à ce que me dit M.
le duc d'Orléans.

Le lendemain dimanche, M. et M\textsuperscript{me} du Maine s'en
allèrent à Sceaux. Après leur départ, le comte de Toulouse tint le
conseil de marine à l'ordinaire, et vint l'après-dînée au conseil de
régence avec un air froid, sérieux et concentré. Il y eut des gens
surpris et fâchés de l'y voir. Peu s'approchèrent de lui, et peu après
son arrivée, on se mit en place. Dès que je fus assis, je lui dis à
l'oreille qu'il était servi comme il l'avait désiré, que je ne lui
dirais qu'un seul mot dont je ne pouvais me passer\,: que c'était, ce
jour-là, la première fois que je m'asseyais au-dessous de lui avec
plaisir. Son remercîment tint de sa nature\,; il fut très froid\,; je ne
lui parlai plus de tout le conseil. Ce froid dura quelque temps. Je
pense aussi qu'il y crut de la bienséance, et je ne me pressai pas de le
réchauffer, mais peu à peu nous revînmes ensemble en notre premier état.
Je sus même, par la duchesse Sforze, qu'il blâmait fort
M\textsuperscript{me} la duchesse d'Orléans de ne me point voir, jusqu'à
l'en avoir bien fait pleurer, par tout ce que lui et
M\textsuperscript{me} Sforze lui avaient souvent dit là-dessus.
M\textsuperscript{me} la duchesse d'Orléans était outrée de ce qu'il
était demeuré, et n'avait rien oublié pour l'engager à suivre le sort de
son frère et servir la passion du duc du Maine et la rage de la duchesse
du Maine. Plusieurs se firent écrire à l'hôtel de Toulouse. M. le comte
de Toulouse, comme je l'ai dit, ne voulut recevoir de compliment de
personne, ni M. et M\textsuperscript{me} du plaine. J'étais quitte du
mien par Valincourt, et à l'égard du duc et de la duchesse du Maine, je
ne crus pas devoir leur donner aucun signe de vie. Je sus depuis qu'ils
se prirent fort à moi de ce qui leur était arrivé, quoique fort sobres
en discours. Je me contentai à leur égard d'avoir préféré le bien de
l'État à tout le reste, et satisfait de moi-même sur ce point principal,
je jouis dans toute son étendue du plaisir de notre triomphe, sans me
lâcher aussi en propos, et laissai M. du Maine en proie à ses perfidies,
et M\textsuperscript{me} du Maine à ses folies, tantôt immobile de
douleur, tantôt hurlante de rage, et son pauvre mari pleurant
journellement comme un veau des reproches sanglants et des injures
étranges qu'il avait sans cesse à essuyer de ses emportements contre
lui.

Le parlement, retourné à pied des Tuileries au palais, avec aussi peu de
satisfaction, par les rues, qu'il en avait eu en venant, y respira de la
frayeur et de la honte qu'il avait essuyées, et tâcha de s'en venger
clandestinement, en faisant écrire sur une feuille volante de registres
secrets et fugitifs, qu'il n'avait ni pu ni dû opiner au lit de justice,
et sa protestation contre tout ce qui s'y était fait.
M\textsuperscript{me} du Maine avait envoyé chercher le premier
président, sitôt qu'il fut rentré chez lui où on l'attendait de sa part.
Il n'osa désobéir, et s'y en alla. Il fut reçu avec un torrent d'injures
et de reproches, et traité comme le dernier valet qu'on eût surpris en
friponnerie\,; il n'eut jamais le temps de s'excuser ni de répondre.
Elle se prit à lui de n'avoir pas tout empêché et arrêté, et l'accabla
de mépris et de duretés les plus cruelles, en sorte qu'après une heure
de ce torrent d'horreurs, qu'il lui fallut essuyer, il s'en revint chez
lui avec ce surcroît de rage. Nous le sûmes dès le lendemain\,; on peut
juger si je le plaignis, et dans la vérité il leur était trop
indignement et abandonnément vendu pour être plaint de personne. Un
moins malhonnête homme que lui en serait crevé.

Le lendemain du lit de justice, lundi 29 août, vingt-sept mousquetaires,
commandés par leurs officiers, et partagés en trois détachements, avec
un maître des requêtes à chacun, allèrent, avant quatre heures du matin,
enlever de leur lit et de leurs maisons, Blamont, président aux
enquêtes, et les conseillers Saint-Martin et Feydeau de Calendes. Leur
frayeur fut mortelle, mais leur résistance nulle. Ils furent mis chacun
dans un carrosse, qu'on tenait tous prêts, et séparément conduits, le
premier aux îles d'Hyères, le second à Belle-Ile, le troisième dans
l'île d'Oléron, sans parler à personne sur la route ni dans le lieu de
la prison, et mortellement effrayés de se voir le Mississipi pour leur
plus prochaine terre. On ne trouva rien qui valût chez les deux
conseillers, mais infiniment chez Blamont, tant à Paris qu'en sa maison
de campagne, où un autre maître des requêtes s'était transporté en même
temps, en sorte qu'il y eut de quoi admirer l'imprudence ou la sécurité
d'un homme qui semblait chercher ce qui lui arriva par ses menées et par
l'éclat de sa conduite, et n'avoir pas eu plus de soin à mettre ses
papiers à couvert.

Cette capture, qui aurait pu se faire avec moins d'appareil, ne fut pas
plutôt sue au palais, que les chambres s'assemblèrent et résolurent une
députation aux femmes des exilés pour leur témoigner la part que la
compagnie prenait en leur détention, et une autre la plus nombreuse
qu'il se pourrait au roi et au régent, pour s'en plaindre. Ils furent
donc dès le dimanche matin au Palais-Royal, et l'après-dînée aux
Tuileries. Leur harangue, prononcée par le premier président, fut
pressante, mais en termes très mesurés et très respectueux. La réponse à
toutes les deux fut à peu près de même, grave et vague. Le lundi et le
mardi le palais fut fermé, et un avocat, ayant plaidé à la cour des
aides, pensa, être chassé de sa compagnie, qui avait résolu de cesser
ses fonctions\,; cependant cette grande résolution, qui allait à
suspendre tout cours de justice, qui tendait à soulever le monde et à
essayer un second tome du fameux Broussel, de la dernière minorité, ne
put se soutenir. Dès le mercredi le parlement reprit de lui-même ses
ordinaires fonctions\,; mais il ordonna aux gens du roi de se trouver
tous les matins au Palais-Royal, pour insister sur le rappel de leurs
membres. Ce manége, aussi ridicule qu'infructueux, dura jusqu'au 7
septembre. Comme les extrémités sont du goût des Français, il se débita
que, la cessation de l'exercice de la justice n'ayant pas réussi, le
parlement entreprendrait de ne se point séparer aux vacances, et de
continuer à s'assembler après la Notre-Dame de septembre. Néanmoins il
n'osa l'attenter. Il laissa seulement commission au président qui devait
tenir la chambre des vacations d'aller souvent solliciter auprès du
régent le retour de leurs membres. Ce président vit bien, par
l'éloignement des lieux, où on sut enfin qu'ils étaient arrivés et
détenus sans parler à personne, qu'ils n'étaient pas pour en sortir
sitôt, vit le régent deux ou trois fois, et lui épargna ensuite une
importunité inutile.

Ainsi finit cette grande affaire, et si importante que le repos de
l'État en dépendait, par le consolidement de l'autorité royale entre les
mains du régent, en empêchant un partage qui ne lui eût bientôt laissé
qu'une représentation vaine et vide, et qui eût attiré toutes sortes de
confusions, affaire compliquée dont le succès fut également dû à la
diligence et au profond secret, au peu d'arrangement de la cabale qui se
formait, et à la faiblesse de ses principales têtes.

L'honneur que cette exécution fit au régent dans les pays étrangers est
incroyable. On commença à s'y rassurer de la crainte de ne pouvoir
traiter solidement avec un prince qui, semblait se laisser arracher son
pouvoir par des légistes\,: c'est ainsi que le roi de Sicile s'en
expliqua en propres ternes à Turin, et que les autres puissances ne s'en
laissèrent pas moins clairement entendre.

La consternation du parlement ne fit pas un moindre effet dans le
royaume. Les autres parlements, qui tous avaient été sondés, et dont
quelques-uns n'avaient pas voulu se joindre à celui de Paris,
s'affermirent dans l'obéissance, et les provinces séduites par des
pratiques et depuis par l'exemple de l'indépendance, n'osèrent plus
montrer d'audace. La Bretagne, dont les états assemblés et le parlement
se tournaient ouvertement à la révolte, commença par ce coup à rentrer
peu à peu dans l'obéissance, et, s'il y eut nombre de particuliers
entraînés depuis par de folles espérances qui se précipitèrent dans la
rébellion, le nombre en fut si médiocre, l'espèce si méprisable, les
moyens si nuls, et la terreur et les cris si pitoyables dès qu'ils se
virent découverts, qu'il n'y eut qu'a les châtier par les voies
ordinaires de la justice, sans aucune sorte d'inconvénient ni de suites
à en craindre. Voilà comme la fermeté est le salut des États, et comme
une débonnaireté et une facilité qui dégénère en faiblesse, opère le
mépris et les attentats, précipite tout en dangers et en ruine, et ne se
peut relever que par des coups de force où le bonheur ne préside guère
moins que la conduite. J'avais tout appréhendé d'un coup double frappé à
la fois sur le parlement et sur le duc du Maine, et en effet tout en
était à craindre. Le besoin que, dans cette extrémité d'affaires, le
régent eut de l'union avec M. le Duc\,; l'opiniâtreté de M. le Duc à ne
plus laisser échapper la surintendance de l'éducation du roi et qui
sentit ses forces en cette occasion après tant de fois que M. le duc
d'Orléans lui avait donné et manqué de paroles les plus positives
là-dessus\,; ces intérêts divers, mais alors réunis de ces deux princes,
chacun pour son but, l'emportèrent sur les plus sages considérations. Le
favorable succès me combla de joie, et le délicieux fruit du rang que
j'en recueillis me fut d'autant plus précieux que ce grand objet ne me
séduisit ni l'esprit ni le coeur, et que je le pus goûter avec toute la
paix qu'une conscience pure répand dans l'âme d'un homme de bien qui a
sincèrement préféré l'État à soi-même.

Pour achever un morceau si curieux de l'histoire de cette régence, il
faut dire pourquoi je ne crus pas à propos de profiter de cette occasion
pour le bonnet. Je crus qu'il ne fallait pas surcharger la faiblesse du
régent de tant de choses à la fois et ne pas embarrasser l'affaire si
principale de la réduction des bâtards au rang de leurs pairies, dont il
fallait presque abandonner l'espérance, si nous ne l'obtenions pas à
l'occasion du changement de main de l'éducation\,; ne l'embarrasser pas,
dis-je, d'une autre affaire si inférieure à celle-là. Je pensai que le
bonnet était une affaire si ridicule en soi du côté des bonnets, et si
entamée, qu'il était impossible, que, près ou loin, une chose si juste
nous fût refusée, et qu'il était même peu décent pour nous de ne
l'obtenir que comme une vengeance du régent dont nous profiterions. Je
craignis que le parlement, outré de l'affront qu'il allait recevoir, uni
avec le duc du Maine enragé de sa chute, et que l'éclat commun
resserrait de plus en plus, se portât à des extrémités dont le monde ne
manquerait pas de nous charger, si notre intérêt devenait une des
amertumes de cette compagnie. Je sentis toute la différence pour la
solidité d'un avantage tel que la réduction des bâtards au rang de leurs
pairies, qui aurait M. le Duc pour garant qui, au lieu d'avoir le
parlement pour partie, était au contraire conforme à ses usages et à ses
règles, d'avec un avantage qui, portant directement sur les présidents à
mortier, et par leur intrigue sur le parlement, à qui ils le feraient
accroire, n'aurait de garantie que la durée de la colère et de la
fermeté d'un régent qui ne connaissait ni l'une ni l'autre, surtout pour
les intérêts d'autrui, et qui, suivant son goût, entendrait si
volontiers aux prétendus \emph{mezzo-termine}, rapatriages,
conciliations, qui lui pouvaient être opposés dans la suite, par
lesquels le régent et le parlement seraient peut-être ravis de sortir
d'affaire l'un d'avec l'autre à nos dépens. Ces considérations me firent
estimer que l'affaire du bonnet n'était pas de saison, et qu'il fallait
quelquefois savoir demeurer en souffrance. Je pensai enfin, mais sans
être déterminé par cette raison surabondante et assez peu apparente, que
le parlement, touché de cette modération de notre part, sentirait
peut-être enfin l'excès, la nouveauté, l'injustice si évidente de
l'usurpation de ses présidents à cet égard, et qui n'intéressait le
corps du parlement en nulle sorte, l'engagerait à y prendre peu de part
si cette affaire venait à être jugée, comme celle de la préopinion sur
les présidents et le premier président le fut en notre faveur en 1664,
peut-être même à se porter à nous faire justice comme le parti le plus
honorable sur un point si criant, et ôter le mur de séparation et de
division d'entre les pairs et le parlement par l'inconvénient duquel
cette compagnie n'avait cessé d'être continuellement flétrie, au lieu du
lustre peut-être excessif, où son union avec les pairs l'avait élevée et
établie avant ces usurpations.

Dès le lendemain du lit de justice, M. le Duc prit possession de la
surintendance de l'éducation du roi et en fit les fonctions. Il
s'établit peu de jours après dans l'appartement que le duc du Maine
occupait aux Tuileries. L'après-dînée du jour du lit de justice le
maréchal de Villeroy, accompagné de M. de Fréjus et de toute
l'éducation, alla piaffant, quoique enrageant, à l'hôtel de Condé, où
les souples respects d'une part, et les faux compliments de l'autre,
donnèrent une autre sorte de spectacle. Dès le lendemain, le roi s'alla
promener au Cours où M. le Duc l'accompagna, au lieu du duc du Maine, et
entra publiquement en fonction.

M\textsuperscript{me} d'Alègre ne tarda pas à me venir voir. Elle
m'avoua enfin, parmi toutes ses enveloppes ordinaires, ses phrases
suspendues et souvent coupées sans les achever, que ses avis si souvent
réitérés et si fort hiéroglyphiques, n'avaient tendu qu'à m'avertir, et
le régent par moi, de la dangereuse cabale qui se brassait de longue
main, qui se fortifiait tous les jours, et qu'il était grand temps
d'abattre par le grand coup qui venait d'être frappé\,; en même temps
elle m'avertit, pour le bien inculquer au régent, de ne se pas trop
reposer sur une exécution si importante\,; qu'elle connaissait les
allures des gens à qui elle avait affaire\,; que, quelque étourdis
qu'ils fussent d'un coup auquel ils ne s'attendaient pas de la conduite
et de la faiblesse du régent, ils n'en seraient que plus enragés et plus
unis\,; que ce coup même leur apprenait à changer leur sécurité, leur
lenteur, leur négligence en mesures plus justes, plus serrées, plus
fortes, pour atteindre au grand but qu'ils s'étaient proposé, de
profiter de plus en plus des dispositions de l'Espagne, irritée au
dernier point du dernier traité avec l'empereur et les puissances
maritimes, et du dépit général qui s'en répandait par toute la France.
Je ne manquai pas d'en rendre un compte exact à M. le duc d'Orléans, et
d'y ajouter mes réflexions. Je trouvai un homme si à son aise d'être au
lendemain de cette grande crise, si étouffé encore d'un tour de force
aussi contraire à son naturel, qu'il s'y était replongé tout à fait
comme un homme qui s'étend dans son lit en arrivant d'une grande course,
et qui ne veut pas ouïr parler d'autre chose que de repos. Il me chargea
de bien remercier M\textsuperscript{me} d'Alègre et m'assura en même
temps qu'après une telle touche il n'avait rien à craindre de personne,
sans que je le pusse jamais tirer pour lors d'un si dangereux préjugé.
Je fis à M\textsuperscript{me} d'Alègre plus de compliments que je n'en
étais chargé, et je ne craignis pas d'outrepasser ma commission, en la
priant fort de la part du régent d'avoir les yeux bien ouverts, et de
m'avertir de tout ce qu'elle pourrait soupçonner ou découvrir. J'y
joignis les louanges et les flatteries qui pouvaient le plus l'y
engager, et notre commerce demeura enseveli dans le même secret dans
lequel il l'avait toujours profondément été.

J'obtins en ce temps-ci deux grâces que je ne puis oublier, parce que je
n'en ai point reçu qui m'aient fait tant ni de si sensible plaisir. On a
pu voir, dans les commencements de ces Mémoires, que le saint et fameux
abbé de la Trappe avait été l'homme que j'avais le plus profondément
admiré et respecté, et le plus tendrement et réciproquement aimé\,: il
avait laissé un frère que je n'avais jamais vu, et avec qui je n'avais
jamais eu aucun commerce\,: il était de bien loin, et en tout genre, le
plus ancien officier de toutes les galères\,; il y avait acquis de la
réputation et l'affection du corps\,: il en était premier chef
d'escadre, commandant du port de Marseille depuis bien des années, et à
plus de quatre-vingt-quatre ou quatre-vingt-cinq ans il avait toute sa
tête et toute sa santé. La fantaisie le prit d'en profiter pour venir
faire un tour à Paris, où il n'était jamais venu de ma connaissance. Ce
fut M. de Troyes, dont il était cousin germain de son père, enfants des
deux frères, qui m'apprit son arrivée. Il s'appelait le chevalier de
Rancé. Je me hâtai de l'aller voir et de le convier à dîner\,: il
ressemblait tant à M. de la Trappe, que je dirai sans scandale que j'en
devins amoureux, et qu'on riait de voir que je ne pouvais cesser de le
regarder. Ses propos ne sentaient le vieillard que par leur sagesse,
avec tout l'air et la politesse du monde. Tout à coup j'imaginai de
faire pour lui la chose la plus singulière et la plus agréable\,: jamais
il n'y eut qu'un seul lieutenant général des galères, charge qui se vend
et qu'avait le marquis de Roye. Je résolus de demander au régent d'en
faire un second en la personne du chevalier de Rancé, à condition
qu'après lui sa place ne serait plus remplie, et que les choses à cet
égard reviendraient sur le pied où elles étaient auparavant. J'en parlai
à M. de Troyes, à l'insu duquel il n'aurait pas été honnête de
m'employer. Il fut charmé de ma pensée, et me promit de m'y seconder. En
même temps je le priai que le secret en demeurât entre nous deux pour ne
pas donner une espérance vaine et un chagrin sûr s'il y avait un refus
que nous ne pussions vaincre\,: l'amitié, quand elle est forte, rend
pathétique. Je représentai si bien à M. le duc d'Orléans les services,
le mérite, la qualité de frère de M. de la Trappe, le grand âge du
chevalier de Rancé, dont l'avancement extraordinaire ne pouvait faire
tort ni servir d'exemple à personne, qu'en présence de M. de Troyes, qui
m'appuya légèrement, peut-être parce que je ne lui en laissai pas trop
le loisir, j'emportai la création d'un second lieutenant général des
galères, sans pouvoir être remplie après le chevalier de Rancé, et dix
mille livres d'appointement en outre de ce qu'il en avait. Je fus
transporté de la plus vive joie qui, contre mon attente, s'augmenta
encore par celle du chevalier de Rancé, dont la surprise fut incroyable.
On peut juger que je pris soin que l'expédition fût bien libellée. Il
passa deux mois à Paris, beaucoup moins que je n'aurais désiré, et il
jouit encore de son nouvel état quelques années. Mais, comme les
exemples sont dangereux en France, l'âge, l'ancienneté, les services, la
naissance du chevalier de Roannais, premier chef d'escadre des galères,
crièrent tant à la mort du chevalier de Rancé, qu'il parvint enfin à
succéder à sa charge, qui, néanmoins, a fini avec lui. L'autre grâce,
voici quelle elle fut.

On a pu voir (t VII, p.~63), l'étrange trait du duc de Mortemart à mon
égard, à l'occasion de la mort de M\textsuperscript{me} de Soubise, ce
qui fut sur le point d'en arriver, et que M. de Beauvilliers lui ordonna
de sortir de chez lui dès que j'y entrerais, et de n'y jamais entrer
tant que j'y serais\,: ce qui a duré presque jusqu'à la fin de sa vie,
c'est-à-dire plusieurs années, qu'il me demanda de souffrir son gendre
chez lui. On a pu voir (t. IX, p.~56), l'autre trait qu'il me fit dans
le salon de Marly, sur notre requête contre d'Antin. Je ne le voyais
donc en aucune occasion, quoique ami intime de toute sa famille, même de
sa mère. Il s'était déjà pris une fois de bec avec le maréchal de
Villeroy sur les fonctions de leurs charges. On a vu (t. XV, p. 133),
que le service en manqua plusieurs jours, et qu'il voulut donner la
démission de sa charge. Cette disparate avait éloigné de lui M. le duc
d'Orléans. Un peu après l'affaire du chevalier de Rancé, il s'éleva une
autre dispute entre le duc de Mortemart et le maréchal de Villeroy, où
le premier poussa les choses d'autant plus loin qu'il avait plus de
tort, et le maréchal demeura d'autant plus sage qu'il se sentait toute
la raison de son côté. L'affaire portée au régent, il décida en faveur
du maréchal, et blâma d'autant plus l'autre, qu'il l'avait indisposé par
sa première dispute, par sa première démission et par d'autres disputes
moins importantes, mais fréquentes, pour des vétilles, avec les uns et
les autres. Mortemart, piqué d'avoir succombé après l'éclat qu'il avait
fait, peut-être autant d'avoir été tancé plus que M. le régent n'avait
accoutumé de faire, n'en fit pas à deux fois et lui envoya la démission
de sa charge de premier gentilhomme de la chambre, avec une lettre fort
peu ménagée.

Heureusement c'était un jour que je travaillais avec M. le duc
d'Orléans, et que j'arrivai comme il venait de la lire. Je trouvai ce
prince en furie, qui d'abordée me conta la chose, et conclut que, pour
cette fois, Mortemart serait pris au mot, et lui délivré de toutes ses
impertinences\,; tout de suite, en me regardant, il me fit entendre que
j'étais venu tout à propos. L'horreur que je sentis de la dépouille de
M. de Beauvilliers, et de m'en revêtir aux dépens de ses petits-fils,
m'inspira la plus nerveuse éloquence. Je représentai au régent que ce
n'était pas M. de Mortemart qu'il devait regarder, mais la mémoire de M.
de Beauvilliers, et les obligations étroites, importantes, continuelles,
qu'il lui avait à l'égard de Mgr le duc de Bourgogne, lorsqu'il allait
tout gouverner, puis à la mort de ce prince, et précédemment encore lors
du mariage de M\textsuperscript{me} la duchesse de Berry. Je m'espaçai
sur ces matières avec la dernière force, et je finis par lui dire qu'il
était fait et payé, tout régent qu'il était, pour souffrir toutes les
sottises et tous les égarements du gendre de M. de Beauvilliers. Il
disputa, me fit sentir encore que l'occasion était belle et unique. Mon
indignation redoubla, dont la fin fut que la démission fut sur-le-champ
mise en pièces.

Au sortir du Palais-Royal, j'allai dire à la duchesse de Mortemart la
folie que son fils venait de faire, la peine que j'avais eue à l'en
sauver, et le soin extrême qu'elle devait d'en empêcher une troisième
récidive, qui sûrement serait plus forte que moi, ou se brusquerait à
mon insu, puisque c'était le plus grand hasard du monde que celle-ci fût
arrivée le même jour et si peu de temps avant que je vinsse travailler
avec M. le duc d'Orléans. Le duc de Mortemart, revenu de sa fougue par
l'avoir satisfaite, sentit tout le péril où elle l'avait jeté, et se
trouva heureux de n'avoir pas perdu sa charge. Je fus très surpris,
trois jours après, de le voir entrer dans ma chambre, où il me fit de
grands remercîments. Je lui répondis froidement qu'il ne m'en devait
aucun, parce que je n'avais rien fait pour lui, mais tout par mon
tendre, fidèle et reconnaissant souvenir de M. le duc de Beauvilliers,
dont la famille me serait toujours infiniment chère, et pour conserver
sa charge à ses petits-fils, et je l'exhortai en peu de mots à ne se
plus jouer à mettre la patience de M. le duc d'Orléans à de pareilles
épreuves. On peut juger que la franchise d'une si sèche réponse abrégea
la visite, qui finit froidement, mais poliment, sans que depuis j'aie
ouï parler de lui. Le lendemain matin, sa femme, qu'il tenait
étrangement captive, dont la vertu, la piété, l'esprit et la conduite
méritaient un tout autre mari, vint chez moi me remercier avec la plus
grande effusion de cœur. Je l'assurai que j'étais tellement payé
d'avance par tout ce que j'avais reçu de son père, que je ne méritais
nul remercîment, mais d'être félicité d'avoir eu occasion de témoigner à
sa mémoire le plus tendre et le plus vif attachement, et de la tirer
elle-même de la peine de voir passer sa charge en d'autres mains. Je
n'ajouterai point ce qu'elle me dit sur l'occasion si aisée de la
prendre pour moi, ni ce que ses tantes m'en témoignèrent, car sa
belle-mère était sa tante aussi. Nous nous embrassâmes de bon coeur, qui
fut la fin de la visite et la dernière fois que je la vis\,; elle mourut
bientôt après, sans que son mari sentit une si grande perte.

J'achèverai tout de suite, pour n'avoir plus à y revenir. La sombre
folie du duc de Mortemart m'inquiétait toujours pour sa charge. On ne
pouvait se flatter qu'elle ne lui causât encore des querelles aussi mal
fondées que les dernières\,; qu'elles ne lui tournassent la tête comme
elles avaient déjà fait, et que M. le duc d'Orléans, excédé de lui, ne
pût être arrêté, pour s'en défaire, à la difficulté que j'y avais
éprouvée. Cela me revint si souvent dans l'esprit, qu'au bout de deux
mois je pris ma résolution, sans en parler à personne, de demander à M.
le duc d'Orléans la survivance de sa charge pour son fils qui n'avait
pas sept ans. Par là je ne craignais plus les frasques du père. Il ne
pouvait plus la vendre, et s'il s'avisait encore une fois de se piquer
et d'envoyer sa démission, il n'y avait plus à courir après, son fils
devenait le titulaire. Je pris donc cette résolution, et je l'exécutai
si bien que j'emportai la survivance. Comblé de joie d'avoir mis en
sûreté le petit-fils du duc de Beauvilliers pour sa charge, j'allai, au
sortir du Palais-Royal, l'apprendre aux duchesses de Beauvilliers, de
Mortemart et de Chevreuse, chacune chez elle, dont la surprise, la joie
et les expressions ne se peuvent rendre. Je dis aux deux premières qu'il
était très essentiel de bien constater la chose par leur remercîment
public. Dès le lendemain, quoiqu'elles n'allassent plus en aucun lieu,
depuis bien des années, au delà de leur famille et d'un très petit
nombre d'amis particuliers, je les accompagnai au Palais-Royal.
J'avertis M. le duc d'Orléans, dans son cabinet, qu'elles l'attendaient
pour lui faire leur remercîment. Il vint aussitôt les trouver\,; il se
passa le mieux du monde, et la survivance fut expédiée le lendemain. Ce
remerciement la rendit publique. Rien au monde ne m'a jamais tant fait
de plaisir, et toute cette famille n'a jamais oublié ce service.

Cette survivance en occasionna d'autres, que je mets tout de suite comme
elles furent données aussi. Le duc de Charost, mon ami, comme on l'a vu,
depuis bien des années, me pria de demander la survivance de sa charge
de capitaine des gardes du corps pour son fils\,; je lui dis que ce
n'était pas celle-là qu'il devait désirer pour lors, mais celle de ses
gouvernements de Calais et de Dourlens, et de sa seule lieutenance
générale de Picardie, qui est une grâce de quatre-vingt mille livres de
rente, et des emplois dont l'importance attirerait après très facilement
celle de sa charge. Il me crut, et je l'obtins deux jours après.
Là-dessus le duc de La Rochefoucauld eut celle de grand maître de la
garde-robe, pour son fils\,; le duc de Luxembourg, celle de gouverneur
de Normandie pour le sien\,; et le duc de Berwick, arrivé de son
commandement de Guyenne depuis deux jours, celle de son gouvernement de
Limousin pour son fils. La Fare acheta une lieutenance générale de
Languedoc du comte du Roure, qui obtint son gouvernement du
Pont-Saint-Esprit pour son fils en s'en démettant, et l'abbé de Vauréal
eut permission d'acheter de l'évêque de Saint-Omer la charge de maître
de l'oratoire, qui n'a point de fonctions, mais les entrées de la
chambre, et cinq ou six mille livres d'appointements.

Je ne ferais pas mention de cette dernière bagatelle, sans la singulière
et fort étrange fortune que ce Vauréal a faite depuis. C'est un grand
drôle, d'esprit et d'intrigue, d'effronterie sans pareil, grand et fort
bien fait, et qui en soit user avec peu de contrainte, riche et de la
lie du peuple, qui, à la faveur du petit collet, voulut s'accrocher à la
cour\,; son nom est Guérapin, et son état premier franc galopin. Ségur,
maître de la garde-robe de M. le duc d'Orléans, et qui depuis a bien
poussé sa fortune, épousa la bâtarde non reconnue de M. le duc d'Orléans
et de la comédienne Desmares\,; ce prince lui donna de l'argent, et la
survivance du gouvernement du pays de Foix qu'avait son père qui était
lieutenant général et grand'croix de Saint-Louis. Il avait acheté ce
gouvernement du maréchal de Tallard à qui le feu roi l'avait donné à
vendre. Il avait perdu une jambe à la guerre, et était encore, à près de
quatre-vingts ans, beau et bien fait. C'est ce mousquetaire qui jouait
si bien du luth dont on a vu en son lieu l'aventure avec l'abbesse de La
Joie, soeur du duc de Beauvilliers. Ces différentes grâces arrivées,
lors de la survivance du duc de Mortemart, m'ont emporté trop loin.
Rétrogradons maintenant deux bons mois\,; on y verra des choses plus
importantes.

Il y faut pourtant ajouter le gouvernement de Douai au marquis
d'Estaing, lieutenant général qui avait servi en Italie et en Espagne
sous M. le duc d'Orléans, et qu'il aimait et estimait fort avec raison,
qui vaquait par la mort du vieux Pomereu, lieutenant général, ancien
capitaine aux gardes, frère du feu conseiller d'État, et au conseil
royal des finances. Dernière bagatelle\,: M\textsuperscript{me} la
duchesse d'Orléans qui s'était tenue sous clef depuis le lit de justice,
s'en ennuya enfin, et rouvrit ses portes et son jeu à l'ordinaire.
Retournons maintenant sur nos pas.

\hypertarget{chapitre-iii.}{%
\chapter{CHAPITRE III.}\label{chapitre-iii.}}

1718

~

{\textsc{Efforts du duc du Maine, inutiles, pour obtenir de voir M. le
duc d'Orléans et se justifier.}} {\textsc{- Députation du parlement au
régent sur ses membres prisonniers.}} {\textsc{- Le parlement de
Bretagne écrit en leur faveur au régent.}} {\textsc{- Le parlement de
Bretagne écrit à celui de Paris, qui lui répond.}} {\textsc{- Le régent
demeure ferme.}} {\textsc{- Menées en Bretagne.}} {\textsc{- Le régent
entraîné maintient très mal à propos Montaran, trésorier des états de
Bretagne, qui le voulaient faire compter et lui ôter cet emploi.}}
{\textsc{- Le comte Stanhope passe trois semaines à Paris revenant
d'Espagne en Angleterre.}} {\textsc{- Riche flotte d'Amérique arrivée à
Cadix.}} {\textsc{- Les conseils sur leur fin, par l'intérêt de l'abbé
Dubois et de Law.}} {\textsc{- Appel du cardinal de Noailles, etc., de
la constitution Unigenitus.}} {\textsc{- Il se démet de sa place de chef
du conseil de conscience.}} {\textsc{- Tous les conseils particuliers
cassés.}} {\textsc{- L'abbé Dubois fait secrétaire d'État des affaires
étrangères, et Le Blanc secrétaire d'État de la guerre.}} {\textsc{-
Brancas et le premier écuyer conservent leurs départements\,; plusieurs
des conseils leurs appointements.}} {\textsc{- Canillac entre au conseil
de régence.}} {\textsc{- La Vrillière a la feuille des bénéfices.}}
{\textsc{- Le comte d'Évreux, Coigny, Biron, Asfeld, demeurent comme ils
étaient.}} {\textsc{- Admirable mandement publié par le cardinal de
Noailles sur son appel de la constitution.}} {\textsc{- Fêtes données à
Chantilly à M\textsuperscript{me} la duchesse de Berry.}} {\textsc{- Le
frère du roi de Portugal incognito à Paris.}} {\textsc{- Mariage du roi
Jacques d'Angleterre, dit le chevalier de Saint-Georges, avec une
Sobieska, qui, en allant le trouver avec la princesse sa mère, est
arrêtée à Inspruck par ordre de l'empereur.}} {\textsc{- Tyrannie
étendue à cet égard.}} {\textsc{- Faiblesse du régent pour le traitement
du duc du Maine.}} {\textsc{- Autres gens des conseils récompensés.}}
{\textsc{- Bonamour et sept membres du parlement de Bretagne exilés,
puis quatre autres encore.}} {\textsc{- M\textsuperscript{me} la
duchesse d'Orléans à l'Opéra.}} {\textsc{- Curiosité sur les tapis.}}
{\textsc{- Mort du maréchal-duc d'Harcourt et de l'abbé de Louvois.}}
{\textsc{- Conseillers d'État pointilleux et moqués.}} {\textsc{-
Koenigseck ambassadeur de l'empereur à Paris.}} {\textsc{- Époque
singulière de l'entier silence de tout ce qui eut trait à la
constitution au conseil de régence.}} {\textsc{- Retour des conseillers
du parlement de Paris exilés, non du président Blamont.}} {\textsc{-
Faux sauniers nombreux excités.}} {\textsc{- Mézières avec des troupes
est envoyé contre eux.}} {\textsc{- Le duc du Maine achète une maison à
Paris.}} {\textsc{- Meudon donné à M\textsuperscript{me} la duchesse de
Berry.}} {\textsc{- Rion en a d'elle le gouvernement.}} {\textsc{- Du
Mont, qui l'avait, en conserve les appointements.}} {\textsc{-
Chauvelin, longtemps garde des sceaux si puissant, et chassé, devient
président à mortier\,; Gilbert avocat général, et l'abbé Bignon
bibliothécaire du roi.}} {\textsc{- Nangis veut se défaire du régiment
du roi.}} {\textsc{- J'en obtiens l'agrément pour Pezé, et aussitôt
Nangis ne veut plus vendre.}} {\textsc{- Le duc de Saint-Aignan,
ambassadeur en Espagne, reçoit ordre du régent de revenir.}} {\textsc{-
Je lui assure à son insu une place en arrivant au conseil de régence.}}
{\textsc{- Berwick accepte de servir contre l'Espagne.}} {\textsc{-
Asfeld s'en excuse.}} {\textsc{- Six mille livres de pension à
M\textsuperscript{lle} d'Espinoy\,; autant à M\textsuperscript{lle} de
Melun\,; quatre mille livres à Meuse\,; autant à Béthune le Polonais.}}
{\textsc{- Six mille livres à Méliant, maître des requêtes, en mariant
sa fille unique au fils aîné du garde des sceaux.}} {\textsc{- Dix mille
livres au marquis de La Vère, frère du prince de Chimay.}} {\textsc{-
Huit mille livres à Vertamont, premier président du grand conseil.}}
{\textsc{- M\textsuperscript{me} la duchesse de Berry en reine à
l'Opéra, une seule fois.}} {\textsc{- Elle donne audience de cérémonie à
l'ambassadeur de Venise sur une estrade de trois marches.}} {\textsc{-
Force plaintes.}} {\textsc{- Elle n'y retourne plus.}}

~

La fermentation se cachait, mais subsistait toujours, M. du Maine, fort
abandonné à Sceaux, où il avait déclaré qu'il ne voulait voir personne,
protestait qu'il ne se sentait coupable de rien. M\textsuperscript{me}
la duchesse d'Orléans l'y allait voir. Ils faisaient tous leurs efforts
pour lui obtenir une audience de M. le duc d'Orléans, dans laquelle il
prétendait se justifier, et ces efforts furent inutiles.

Le parlement de Bretagne écrivit au régent pour lui demander la liberté
des trois prisonniers du parlement de Paris, et en même temps à ce
parlement pour lui rendre compte de cet office, et pour louer et
approuver toute la conduite du parlement de Paris. Celui-ci, en même
temps, députa au régent le premier président et huit conseillers pour
lui demander la liberté de leurs trois confrères. Il leur répondit que
la conduite qu'aurait désormais le parlement réglerait la sienne à
l'égard des prisonniers, et mortifia beaucoup par cette réponse des gens
qui s'étaient tout promis de cette démarche vers le régent sans aller au
roi, et de la facilité de M. le duc d'Orléans dont ils avaient tant et
si longuement abusé. Il ne fit aucune réponse au parlement de Bretagne,
et trouva son interposition et sa lettre au parlement de Paris fort
impertinente et séditieuse. Le parlement de Paris abattu de ce qui
s'était passé au lit de justice, de la prison de trois de ses membres et
de la réponse qu'il venait de recevoir sur leur liberté, n'osa répondre
au parlement de Bretagne qu'en termes fort mesurés et après avoir montré
sa réponse au régent. Ce qui s'était passé à Paris avait influé sur la
Bretagne. De plus, à ce qu'il s'y méditait, il n'était pas temps de rien
témoigner. Le feu s'y entretenait avec art et mesure. Ce fut pour cela
qu'il lui fallut donner quelque pâture par ces lettres du parlement de
Bretagne dont on vient de parler. Leur mauvais succès et la réponse du
parlement de Paris au parlement de Bretagne, qui se sentait si fort de
la touche que le parlement de Paris avait reçue, et dont il était encore
dans le premier étourdissement, fit sentir à la Bretagne la nécessité
d'amuser la cour par une déférence qui n'altérait point ses sourdes
mesures. Ainsi les états nommèrent les députés que la cour avait choisis
pour apporter à l'ordinaire leurs cahiers à Paris, en finissant leurs
séances\,; ils avaient précédemment obtenu qu'une députation par chaque
diocèse s'y pût assembler entre deux tenues d'États, pour l'exécution de
ce qui y était ordonné.

Cela était tout nouveau. Le spécieux de préparer et d'abréger les
matières pour les États suivants, avait surpris la facilité du régent.
L'occupation présentée n'était pas celle qu'on s'y proposait\,; le
dessein, comme les suites ne le firent que trop évidemment reconnaître,
était de s'organiser entre eux, d'accroître le nombre pour remuer,
embarquer, se fournir des moyens de soutenir des troubles, choisir les
chefs et les affidés de chaque diocèse, de concerter leurs mesures pour
conduire les états au but qu'ils se proposaient en fascinant la
multitude du bien public, de la restitution de leurs anciens privilèges,
de la facilité des conjonctures. C'est ce qui causa tant de bruit et
tant de prétentions aux états qui suivirent et qui, enfin reconnus,
porta le régent à supprimer ces nouvelles et si dangereuses députations
diocésaines qui s'assemblaient, tant qu'il leur plaisait, d'une tenue
d'États à l'autre et qui s'entre-communiquaient et s'entendaient
secrètement. Le coup frappé par le lit de justice opéra sans bruit cette
suppression et termina les États de même. Mais le mal que ces
députations diocésaines avaient fait, subsistant avec le dépit de ne
pouvoir user de la même et si grande et commode facilité pour le pousser
à leur gré. Ce fut à revenir par d'autres voies et plus couvertes qu'il
fallut travailler, et c'était l'embarras où se trouvèrent alors les
secrets conducteurs de ces sourdes pratiques. Ils les continuèrent donc
comme ils purent par les connaissances, les liaisons et les mesures que
ces députations diocésaines leur avaient donné lieu de prendre, et la
conjoncture présente qui demandait une surface soumise et paisible ne
leur permit pas d'agir autrement pour un temps.

Le gouvernement fit aussi une grande faute et pour des intérêts
particuliers, à laquelle je m'opposai vainement, par la déplorable
facilité et sécurité du régent. La province entière était mécontente de
Montaran, son trésorier, et le voulait ôter, et dans ce mécontentement
il n'entrait rien qui eût trait à aucune autre chose qu'à un détail
pécuniaire entièrement domestique et entièrement étranger aux intérêts
politiques ou pécuniaires du roi ni à aucune forme publique. Montaran,
qui était fort riche, regardait avec raison son emploi comme sa fortune
par les énormes profits qui y étaient ou attachés ou tirés. Sa
magnificence et son attention à obliger de sa bourse les gens de la cour
et beaucoup encore de son crédit, lui acquirent la protection des dames
et de beaucoup de gens considérables\,; il se trouvait de plus soutenu
par son frère, capitaine aux gardes, estimé dans son métier, fort gros
et fort honnête joueur, et par là mêlé depuis longtemps avec le meilleur
et le plus grand monde. Par ces appuis le trésorier se maintint contre
les cris de toute la province, qui alléguait avec raison qu'il était
inouï, chez les particuliers, que, par autorité supérieure, un trésorier
empêchât son maître de le faire compter avec lui et de le renvoyer quand
il le voulait\,; que cette liberté commune à tout le monde était la
moindre chose qu'elle pût espérer en faveur, du moins, de ce qu'elle
payait au roi sans murmure, qui ne tendait qu'à voir clair en ses
affaires et en pouvoir charger qui bon lui semblerait. Ces raisons
étaient vraiment sans réplique, mais le crédit de Montaran l'emporta. Il
n'est pas croyable à quel point la province en fut aigrie et l'usage
qu'en surent tirer les instruments des menées, même envers les plus
éloignés d'avoir connaissance ni part encore moins à ce qui se tramait.

Milord Stanhope arriva de Madrid à Paris au commencement de septembre,
peu content, comme on l'a pu voir, du voyage qu'un ministre d'Angleterre
aussi accrédité que lui avait pris la peine d'y faire, {[}au moment{]}
où la flotte d'Amérique, très richement chargée, venait d'arriver à
Cadix. Ce ministre demeura trois semaines à Paris, où, conduit par
l'abbé Dubois vendu à l'Angleterre, il vit souvent M. le duc d'Orléans,
et s'en retourna reprendre sa place dans le conseil secret du roi son
maître.

Cet abbé, plus puissant que jamais auprès du sien, n'y perdait pas son
temps pour sa fortune. {[}Être{]} conseiller d'État et entré dans le
conseil des affaires étrangères, dont il ne lui laissait que la plus
grossière écorce, ne le satisfaisait pas. Cette légère écorce le
gênait\,; il lui importait, pour son but du chapeau, que l'Angleterre et
l'empereur le vissent maître unique, et sans fantômes de compagnons, de
toutes les affaires étrangères. Law ne se trouvait guère moins gêné du
conseil des finances. Celui de la guerre était devenu une pétaudière, et
dès qu'il était intérieurement résolu de laisser de plus en plus tomber
le peu qu'il restait de marine, le conseil qui en portait le nom était
fort vide et très inutile\,; celui des affaires du dedans du royaume ne
tenait qu'à un bouton par sa matière et par le peu de compte que M. le
duc d'Orléans faisait de d'Antin. Enfin, celui de conscience ne pouvait
plus subsister, comme on le verra tout à l'heure. En général, ces
conseils avaient été fort mal arrangés dès le commencement, par les
menées du duc de Noailles qui n'oublia rien pour confondre et mêler
leurs fonctions, et les commettre ensemble pour les rendre ridicules et
importuns, pour les détruire et se faire premier ministre. S'il ne
réussit pas à le devenir, il réussit du moins à énerver les conseils et
à frayer le chemin à l'abbé Dubois pour s'en défaire et arriver ainsi au
but qu'il s'était proposé vainement pour lui-même.

M. le duc d'Orléans m'en parla avec dégoût, et me témoigna qu'il les
voulait casser. Dubois et Law y avaient trop d'intérêt et le tenaient de
trop près et de trop court pour espérer de l'empêcher. Je me contentai
de lui dire que faire et défaire était un grand inconvénient dans le
gouvernement et qui n'attirait pas le respect ni la confiance du dedans
ni du dehors, et je lui reprochai en détail les fautes qu'il avait voulu
faire dans la manière de leur établissement, et celles où, à leur égard,
il s'était sans cesse laissé entraîner depuis. Je lui représentai le
dégoût qu'il allait gratuitement donner à ceux qui les composaient, et
la considération de les avoir lui-même proposés et fait passer au
parlement le jour qu'il y prit la solennelle possession de la régence.
Enfin, je le priai de réfléchir sur tout ce qu'il avait eu la faiblesse
de fourrer dans le conseil de régence, où par conséquent il ne se
pouvait plus rien traiter d'important, et que, dénué de ce nombre de
conseils dont les affaires s'y référaient, excepté l'important étranger
et certains coups de finance, et destitué de ce groupe de personnes de
tous états qui les composaient, celui de régence tomberait dans un vide
qui mécontenterait tout le monde, et dans un mépris qui montrerait trop
à découvert qu'il voulait gouverner tout seul de son cabinet. Le défaut
des personnes faciles et faibles est de tout craindre et tout ménager au
point de se laisser acculer, et, sortis du danger, se croire
invulnérables et tomber tout à coup dans l'autre extrémité, si l'intérêt
de ceux à qui cette même faiblesse les livre le demande et les y pousse.
C'est ce qui arriva au régent, que Dubois et Law, d'intelligence
ensemble, entraînèrent.

Le cardinal de Noailles, arrêté par le P. de La Tour, général de
l'Oratoire, qui eut après tout lieu de se repentir d'une prudence dont
les vues étaient droites, mais trop courtes avec tout son bon esprit, le
cardinal de Noailles, dis-je, avait fait, malgré ses vrais amis, ceux de
la vérité, et qui voyaient le plus clair, la faute capitale de n'avoir
pas déclaré son appel de la constitution \emph{Unigenitus}, lors de
celui des quatre célèbres évêques en pleine Sorbonne avec elle, et en ce
même temps que tant d'universités et de grands corps réguliers et
séculiers firent publiquement le leur. Je lui exposai chez moi toutes
les raisons importantes, pressantes, évidentes de déclarer son appel en
si bonne compagnie qui l'aurait augmentée encore d'un grand nombre, à
l'appui de son nom, et qui, selon les apparences, eût emporté celui du
parlement de Paris et de quelques autres\,; mes exhortations furent
vaines, et ceux qui aimaient l'Église et l'État, et qui voyaient les
suites d'un délai si pernicieux, en gémirent. Il faut, ici se souvenir
de la conversation que j'eus là-dessus alors avec M. le duc d'Orléans,
dans sa petite loge de l'Opéra, enfermés tête-à-tête, lieu étrange à
traiter d'affaires pareilles, qui est rapporté tome XIV, page 267.
L'intérêt de l'abbé Dubois, pour son chapeau, l'avait changé, et son
maître, qui ne traitait cette affaire qu'en politique, se laissa
entraîner à la sienne et à la cabale intérieure que les chefs de la
constitution avaient su se faire auprès de lui, et plus que par elle,
par le duc de Noailles qui vendit son oncle à sa fortune, je ne dirai
pas ses sentiments premiers, l'Église et l'État\,; il a fait toutes ses
preuves qu'il ne se soucie guère ni de l'une ni de l'autre. Toutes ces
choses ont été expliquées au même lieu indiqué. Les affaires s'étant
depuis continuellement aigries par l'intérêt des chefs de la
constitution en France, malgré Rome qui leur résistait, le cardinal de
Noailles sentit enfin la faute énorme qu'il avait faite, et crut ne
pouvoir plus trouver d'abri que par la déclaration de son appel. Il en
rendit compte au régent, bien résolu à cette fois de ne se plus laisser
gagner, et se démit en même temps de sa place de chef du conseil de
conscience qui, de ce moment, ne s'assembla plus à l'archevêché, mais
chez l'archevêque de Bordeaux qui y était en second. L'appel du cardinal
de Noailles fut donc rendu public, dès le lendemain, 23 septembre. Il
fut incontinent suivi de celui du chapitre de Notre-Dame, de presque
tous les curés de Paris et du grand nombre du reste du diocèse, de
plusieurs communautés séculières et régulières, et d'une foule immense
d'ecclésiastiques particuliers, aux acclamations générales et publiques,
avec tout le bruit et le fracas qu'on peut se représenter.

Cet éclat donna le dernier coup aux conseils. Celui de conscience ne
s'assembla qu'une fois chez l'archevêque de Bordeaux, et fut cassé. Sa
chute précipita celle des autres\,; le régent envoya à chacun de leurs
chefs une lettre du roi pour les remercier, et fit en même temps l'abbé
Dubois secrétaire d'État des affaires étrangères, et Leblanc secrétaire
d'État de la guerre\,; j'eus grande part au choix de ce dernier, qui
était du conseil de guerre dès son établissement, à la mort du roi, en
sorte que la forme du gouvernement de ce prince, que le régent avait
voulu détruire à sa mort, dut, trois ans après, son rétablissement au
même régent, tant il est vrai qu'il n'est en ce monde que bas et petit
intérêt particulier, et que tout est cercle et période\,; il y eut
pourtant des gens qui, tout d'abord, se sauvèrent du naufrage. Le
premier écuyer demeura chargé des ponts, chaussées, grands chemins,
pavés de Paris, et y acquit toujours beaucoup d'honneur, et le marquis
de Brancas, des haras qu'il laissa achever de ruiner. Ils conservèrent
leurs appointements avec quelque augmentation. Ils étaient du conseil du
dedans du royaume. Asfeld demeura de même chargé des fortifications et
des ingénieurs, et le détail de la cavalerie et des dragons fut laissé
au comte d'Évreux et à Coigny, leurs colonels généraux. On laissa à
plusieurs conseillers réformés des conseils leurs appointements.
Canillac refusa les siens. Il voulait mieux et l'obtint bientôt\,; il
conduisit M. le duc d'Orléans à le prier de vouloir bien entrer dans le
conseil de régence\,; et Canillac, pour cette fois, voulut bien être
complaisant.

Le cardinal de Noailles publia un mandement sur son appel, qui fut
applaudi comme un chef-d'oeuvre en tout genre. Quoique fort gros, il
n'était que la première partie du total en attendant la seconde. Je n'en
dirai pas davantage pour ne pas enfreindre la loi que je me suis faite
de ne point entrer ici dans l'affaire de la constitution par les raisons
que j'en ai alléguées. Il fit grand bruit et grand effet. Ce cardinal
vit toujours M. le duc d'Orléans.

M. le Duc, qui voulait plaire à M. le duc d'Orléans, dont il était
extrêmement content depuis le dernier lit de justice, voulut donner une
fête à M\textsuperscript{me} la duchesse de Berry, qu'il convia d'aller
passer quelques jours à Chantilly. Ce voyage dura dix jours, et chaque
jour eut différentes fêtes. La profusion, le bon goût, la galanterie, la
magnificence, les inventions, l'art, l'agrément des diverses surprises
s'y disputèrent à l'envi. M\textsuperscript{me} la duchesse de Berry y
fut accompagnée de toute sa cour. Elle ne fit pas grâce d'une ligne de
toute sa grandeur, qui eut lieu d'être satisfaite de tous les honneurs
et de tous les respects qu'elle y reçut. Elle y eut, sans y déroger en
rien, toute sorte de politesse pour M. le Duc et pour
M\textsuperscript{me} la Duchesse douairière. À l'égard de l'épouse de
M. le Duc, elle affecta une hauteur dédaigneuse, et partit de Chantilly
sans lui avoir dit un seul mot. Elle ne lui pardonna jamais d'avoir fait
rompre le mariage du prince de Conti avec M\textsuperscript{lle} sa
sueur, comme je l'ai raconté, tome X, page 413 et suiv. Lassai, qui
depuis bien des années était chez M\textsuperscript{me} la Duchesse la
mère ce que Rion était devenu chez M\textsuperscript{me} la duchesse de
Berry, fut chargé de lui faire particulièrement les honneurs de
Chantilly. Il tenait une table particulière pour lui\,; il y avait une
calèche et des relais pour eux deux, et cette attention fut marquée
jusqu'au plus plaisant ridicule.

Il pensa y arriver une aventure tragique au milieu de tant de somptueux
plaisirs. M. le Duc avait de l'autre côté du canal une très belle
ménagerie, remplie en très grande quantité des oiseaux et des bêtes les
plus rares. Un grand et fort beau tigre s'échappa et courut les jardins
de ce même côté de la ménagerie, tandis que les musiciens et les
comédiens, hommes et femmes, s'y promenaient. On peut juger de leur
effroi et de l'inquiétude de toute cette cour rassemblée. Le maître du
tigre accourut, le rapprocha et le remena adroitement dans sa loge, sans
qu'il eût fait aucun autre mal à personne que la plus grande peur.

Pendant ces superbes fêtes, et qui eurent tout le gracieux qui leur
manque si ordinairement, arriva de Hollande à Paris, incognito, le frère
du roi de Portugal, qui avait fait avec réputation les deux dernières
campagnes en Hongrie, et descendit chez l'ambassadeur du roi son frère.
L'accueil qu'on lui lit fut nul jusqu'au scandale. Aussi séjourna-t-il
ici le moins qu'il put, quoique mal avec le roi de Portugal, auprès
duquel il ne voulut pas retourner. Cette raison fit que le régent ne se
soucia pas de s'en contraindre ni d'en importuner le roi. Paris, les
étrangers, le Portugal même, ne laissèrent pas d'en être fort choqués\,;
mais le prince ni l'ambassadeur n'en témoignèrent pas la moindre chose,
je crois par un air de mépris et de grandeur qui fut fort approuvé.

Le chevalier de Saint-Georges, pressé enfin de se marier pour avoir
postérité, et maintenir par là l'espérance du parti qui lui restait en
Angleterre, et son malheureux sort l'empêchant de trouver une alliance
proportionnée à ce qu'il aurait dû être en effet comme il l'était de
droit, conclut son mariage avec la fille du prince Jacques Sobieski et
de la sœur de l'impératrice épouse de l'empereur Léopold, de la duchesse
de Parme mère de la reine d'Espagne, et de l'électeur palatin. Le prince
Jacques était fils aîné du fameux Jean Sobieski, roi de Pologne, et de
{[}Marie-Casimire{]} de La Grange, fille du cardinal
d'Arquien\footnote{Nous avons reproduit exactement le texte du manuscrit
  qui avait été modifié dans les précédentes éditions. Antoine de La
  Grange, marquis d'Arquien, père de la reine de Pologne,
  Marie-Casimire, avait été nommé cardinal le 12 novembre 1695.}. Il
était chevalier de la Toison d'or et gouverneur de Styrie, et demeurait
à Olaw, en Silésie, où il avait de grands biens. Il donna six cent mille
livres de dot, et le pape neuf cent mille livres, avec quatre-vingt
mille livres de pension, et des meubles. L'épouse, mariée par procureur,
partit d'Olaw le 12 septembre, accompagnée de sa mère, pour aller à
Rome\,; mais arrivées à Inspruch, elles furent arrêtées toutes deux par
ordre de l'empereur, qui, pour mieux et plus bassement faire sa cour au
roi Georges, ôta en même temps au prince Jacques la pension qu'il lui
donnait, lui envoya ordre de sortir de ses États, et défendit au duc de
Modène d'accomplir le mariage signé entre le prince de Modène son fils
et une autre fille du prince Jacques Sobieski. C'était pousser la
persécution bien loin et d'une manière que toute l'Europe, même en
Angleterre, trouva bien peu honorable, pour en parler modestement, et
dont le pape fut indigné.

L'évêque de Viviers, député des états de Languedoc, n'avait point fait
sa harangue au prince de Dombes, gouverneur de cette province en
survivance, qui avait été absent. Viviers était frère de Chambonnas, qui
était à M. du Maine, et sa femme dame d'honneur de M\textsuperscript{me}
du Maine. Embarrassé du traitement depuis leur chute au dernier lit de
justice, il demanda au régent comment il lui plaisait qu'il en usât. Le
régent lui dit d'en user à l'ordinaire\,: tellement que le prélat le
traita d'Altesse Sérénissime. MM. le duc d'Orléans, parfaitement sans
fiel comme la colombe, croyait que les autres étaient comme lui. Il ne
tenait pourtant qu'à lui de bien savoir à quoi s'en tenir sur le duc du
Maine\,: mais il ne pouvait ni faire de mal à ceux qu'il savait être le
plus ses ennemis, ni soutenir celui qu'il n'avait pu s'empêcher de leur
faire. Sa nature, de plus, n'était pas d'être conséquent en rien. Il se
flattait de regagner, et, par cette faiblesse, il augmentait le courage
et l'audace, et ne réussissait qu'à perdre davantage avec amis et
ennemis, sans qu'aucune expérience pût l'en corriger.

Canillac avait gagné huit mille livres de rente en refusant ses
appointements du conseil des affaires étrangères, et obtenu une place
dans la régence. Sur cet exemple, tous les gens de quelque considération
qui avaient eu des places dans les conseils en tirèrent pied ou aile.
L'archevêque de Bordeaux eut les économats et conserva ses
appointements. Bonrepos garda aussi les siens et un brevet de conseiller
d'État d'épée. Biron continua à se mêler du détail de l'infanterie, avec
dix mille livres d'appointements pour cela, outre ceux du conseil de
guerre qu'on lui laissa. Cheverny entra au conseil des parties comme
conseiller d'État d'épée surnuméraire, en attendant vacance, et eut les
appointements de ce conseil, outre ceux qu'il avait pour celui des
affaires étrangères, et La Vrillière eut l'expédition de tous les
bénéfices, qui, sous le feu roi, s'expédiaient par le secrétaire d'État
qui se trouvait en mois. Je ne parle point des diverses formes que
prirent ceux du conseil des finances.

Bonamour, gentilhomme de Bretagne, qui avait été exilé, puis rappelé,
fut exilé de nouveau avec sept membres du parlement de la même province,
dont les menées ne purent être si cachées qu'elles ne fussent
découvertes\,; quatre autres le furent encore bientôt après.

M\textsuperscript{me} la duchesse d'Orléans, malgré sa douleur sur
l'état du duc du Maine, alla à l'Opéra dans la petite loge de M. le duc
d'Orléans, parce qu'elle n'allait jamais dans la grande loge qu'avec
Madame. La raison en est que Madame y a un tapis et que
M\textsuperscript{me} la duchesse d'Orléans n'y en peut avoir. On voit
donc que jusqu'alors le tapis était réservé aux seuls fils de France.
Les princesses du sang en ont depuis franchi le saut à leurs tribunes
dans les églises de Paris, mais elles n'ont encore osé en mettre à leurs
loges aux spectacles. On n'en comprend pas bien la différence, si ce
n'est qu'elles vont seules aux églises, et qu'au spectacle elles mènent
les dames qui seraient avec elles sur le tapis, à moins que les
princesses du sang fussent seules sur le banc de devant, ce qu'elles
n'ont encore osé faire\,; mais l'expédient qu'elles y ont trouvé est de
n'aller plus aux loges ordinaires, et d'en louer à l'année de petites,
reculées sur le théâtre, où elles ne paraissent point en spectacle.
Ainsi, tapis et non tapis est évité, et c'est la solution de
l'enlèvement que fit M\textsuperscript{me} la Duchesse la mère, avec la
violence qu'on a vue en son lieu, de la petite loge qu'avait la
maréchale d'Estrées.

Le maréchal d'Harcourt mourut enfin le 19 octobre, n'ayant que
cinquante-cinq ans. Plusieurs apoplexies redoublées l'avaient réduit à
ne pouvoir articuler une syllabe, à marquer avec une baguette les
lettres d'un grand alphabet placé devant lui, qu'un secrétaire, toujours
au guet, écrivait à mesure et réduisait en mots, et à toutes les
impatiences et les désespoirs imaginables. Il ne voyait plus depuis
longtemps que sa plus étroite famille et deux ou trois amis intimes.
Telle fut la terrible fin d'un homme si fait exprès pour les affaires et
les premières places par son esprit et sa capacité, et autant encore par
son art, et si propre encore par la délicatesse, la douceur et
l'agrément de son esprit et de ses manières à faire les délices de la
société. Il a été si souvent mention de lui dans ces mémoires, que je
n'en dirai pas davantage. Il laissa peu de bien et tirait du roi plus de
soixante mille livres de rente, dont rien de susceptible de passer à son
fils aîné, et il avait plusieurs enfants. L'abbé de Louvois le suivit de
fort près. Il mourut de la taille. Ce fut dommage\,: un homme d'esprit,
savant, aimable, que les jésuites empêchèrent d'être placé, et qui eût
été un très digne évêque, et qui aurait honoré et paré l'épiscopat.

Les conseillers d'État, de jour en jour devenus plus pointilleux par la
tolérance de leurs prétentions, dont on n'avait jamais ouï parler avant
la difficulté que fit La Houssaye d'être en troisième après le comte du
Luc au traité de Bade, qui mit le dernier sceau à la paix d'Utrecht, se
plaignirent amèrement de ce que deux conseillers d'État commissaires
généraux des finances depuis l'extinction des conseils, venus rapporter
en manteau court des affaires de finances au conseil de régence, y
avaient eu place au bout de la table, et y avaient opiné les derniers.
M. le duc d'Orléans les amusa et s'amusa d'eux, et ces messieurs n'y
gagnèrent rien que de faire rire.

Le comte de Koenigseck, ambassadeur de l'empereur, fit une entrée
magnifique. Il se mêla fort avec la bonne compagnie, fit belle, mais
sage dépense, et tant par la manière de traiter les affaires, que par sa
conduite dans le monde, et l'agrément de la société, il se fit fort
estimer et compter. Il n'a pas moins acquis de réputation à la tète des
armées impériales.

Je ne rapporterais pas la bagatelle suivante, si elle n'était l'époque
du silence entier, qui fut depuis elle religieusement gardé au conseil
de régence, sur l'affaire de la constitution, dont on y parlait souvent
par rapport aux querelles des évêques constitutionnaires dans leurs
diocèses et avec les parlements, et dont on ne dit plus un seul mot
depuis\,; car du fond de l'affaire, il y avait longtemps qu'elle ne se
traitait plus que dans le cabinet du régent. Les chefs de la
constitution avaient raison d'éviter le grand jour dans une matière
devenue toute de manége et de la plus étrange tyrannie de leur part, où
leur fortune et l'amour de la domination en avait tant, et la religion
nulle, qui n'en était que le voile, jusque-là que Rome, contente de
l'obéissance qu'elle avait emportée, était outrée de tout ce qui se
passait en France, qui, à son égard, n'était plus bon qu'à des
éclaircissements de ses entreprises, des lois de l'Église, des pratiques
de tous les temps, et à ventiler et rendre odieuse la puissance
arbitraire et infaillible que cette cour se voulait arroger. J'ai parlé
en son lieu d'Aubigny, parent factice de M\textsuperscript{me} de
Maintenon\,; de sa découverte par Godet, évêque de Chartres\,; de sa
promotion à l'évêché de Noyon, puis à l'archevêché de Rouen\,; homme
sincèrement de bien et d'honneur, mais ignorantissime, grossier, entêté,
excrément de séminaire, fanatique sur la constitution, et accoutumé par
l'autorité de M\textsuperscript{me} de Maintenon à toutes sortes de
violences dans son diocèse, qu'il n'avait cessé de désoler, farci
d'ailleurs de toutes les plus misérables minuties de Saint-Sulpice, la
moindre contravention desquelles était à son égard crime sans rémission.
La mort du roi et la chute de l'autorité, qui lui donnait celle de faire
tout ce qu'il voulait, ne put le rendre plus traitable, et ne fit que
lui procurer des dégoûts sans le corriger dans ses entreprises. Il en
fit une très violente contre des curés fort estimés, qu'il poursuivit à
son officialité, par laquelle il les fit interdire. Ils se pourvurent à
la chambre des vacations du parlement de Rouen, qui cassa
l'interdiction, et les renvoya à leurs fonctions. Elle tança l'official
et mit l'archevêque en furie. Il accourut à Paris pour faire casser
l'arrêt et réprimander la chambre des vacations qui l'avait rendu. Le
garde des sceaux, plein de son ancien chrême et aussi ardent que lui sur
la matière, quoique bien mesuré, parce qu'il avait bien de l'esprit, lui
promit tout et ne douta pas d'emporter l'affaire d'emblée.

J'ignorais parfaitement l'affaire, lorsque, arrivant au Palais-Royal, le
mardi 23 octobre, pour travailler avec M. le duc d'Orléans avant le
conseil de régence qui se devait tenir immédiatement après, je trouvai
en descendant de carrosse l'archevêque de Rouen, qui attendait le sien,
tout agité et tout bouffi, si occupé qu'il ne me dit mot, à moi qui
étais fort de sa connaissance, et bien avec lui depuis qu'il avait été
mon évêque à Noyon. Je passai mon chemin après l'avoir salué assez
inutilement, dans la distraction où il était. Cela me fit soupçonner
qu'il avait quelque affaire pressante, dont il venait apparemment de
parler au régent, et conséquemment qu'il s'agissait de quelque vexation
sur la constitution.

Je contai, en arrivant, ma rencontre à M. le duc d'Orléans, et lui
demandai si ce prélat l'avait vu, et s'il savait ce qui l'occupait si
fort. Il me dit qu'il sortait d'avec lui\,; qu'il était en effet fort en
colère contre la chambre des vacations du parlement de Rouen, qui avait
reçu l'appel comme d'abus d'une interdiction de curés qu'elle avait
cassée\,; que l'archevêque en demandait justice, et qu'on en allait
parler tout à l'heure au conseil de régence. À la façon, quoiqu'en deux
mots, dont M. le duc d'Orléans m'en parla, je le vis prévenu pour
l'archevêque\,; que le garde des sceaux l'en avait entretenu, et que la
cassation de l'arrêt, et la réprimande à la chambre qui l'avait rendu,
allaient passer d'emblée. Je ne dis mot, mais j'abrégeai mon travail et
m'en allai du Palais-Royal descendre chez M. le Duc aux Tuileries, à qui
je dis ce que je venais de voir et d'apprendre, et qu'il ne fallait pas
laisser passer cette affaire sans y voir clair. Il fut du même
sentiment, et me dit qu'il en parlerait à quelques-uns du conseil, avant
qu'on prît place.

Je montai où il se tenait pour les voir arriver. Je parlai au comte de
Toulouse qui pensa de même, et à plusieurs autres que je mis de mon
côté. Le duc de La Force, grand constitutionnaire de politique et de
parti, voulut me résister. Je lui parlai ferme et net, et lui dis que,
ne voulant que voir clair dans une affaire, et empêcher qu'elle ne fût
étranglée, sans demander qu'on fût pour une partie ou pour l'autre,
j'avais droit, justice et raison d'exiger qu'il fût de cet avis. Il eut
peur de moi, et me promit d'en être.

M. le duc d'Orléans et tout le monde arrivé et en place, il dit à la
compagnie qu'avant d'entamer aucune affaire, M. le garde des sceaux
avait à rendre compte d'une qui était provisoire, et qui regardait M.
l'archevêque de Rouen, et tout de suite se tournant au garde des sceaux,
lui fit signe de parler. Argenson rapporta l'affaire avec tout l'art et
toute la force qu'il y put mettre, pour l'archevêque, sans dire un seul
mot des raisons des curés, et conclut, comme je l'avais prévu, à la
cassation de l'arrêt, confirmation de la sentence de l'official de
Rouen, tancement au moins des curés, et réprimande à la chambre qui
avait rendu l'arrêt. Dès qu'il eut cessé de parler, M. le duc d'Orléans
dit\,: «\,Monsieur de Canillac,\,» qui voulut opiner, et qui était le
dernier du conseil. Je l'interrompis à l'instant, et me tournant au
régent, je lui dis que M. le garde des sceaux avait parfaitement
rapporté toutes les raisons de M. l'archevêque de Rouen. Je m'étendis un
peu en louange sur la netteté et l'éloquence du rapport, mais j'ajoutai
qu'étant aussi parfaitement instruits des raisons de l'archevêque, nous
ne l'étions point du tout de celles des curés, par conséquent de celles
de l'arrêt dont il s'agissait, dont M. le garde des sceaux ne nous avait
pas dit un mot\,; que, bonnes ou mauvaises, il fallait bien que la
chambre des vacations du parlement de Rouen en eût eu pour rendre
l'arrêt dont la plainte nous était portée\,; qu'instruits d'un côté,
point du tout de l'autre, nous n'étions pas en état de porter un
jugement\,; que par cette raison il me semblait que ce n'était pas sur
l'arrêt, dont nous ignorions les raisons, que nous pouvions opiner\,;
mais seulement si Son Altesse Royale l'avait agréable, s'il était à
propos, comme je le croyais, de demander à la chambre des vacations du
parlement de Rouen les motifs qu'elle avait eus de le rendre, pour nous
mettre en état, par cette instruction, d'opiner en connaissance de cause
sur la cassation ou la manutention de cet arrêt. Je vis tout le conseil
dresser les oreilles tandis que je parlais, et le garde des sceaux se
secouer comme un homme fort mécontent.

Mon avis frappa M. le duc d'Orléans si bien qu'il dit que j'avais raison
et qu'il n'y avait qu'à opiner là-dessus. Il demanda l'avis à Canillac,
puis aux autres\,: tous furent de mon avis, jusqu'à d'Effiat et à M. de
Troyes, qui n'osèrent montrer la corde, voyant bien que cela passerait
tout de suite. Le garde des sceaux même se contenta de faire le plongeon
au lieu d'opiner. Quand ce fut à M. le duc d'Orléans\,: «\,Cela passe,
dit-il, de toutes les voix.\,» Puis, se tournant au garde des sceaux\,:
«\,Monsieur, lui dit-il, demandez les motifs de son arrêt à la chambre
des vacations du parlement de Rouen.\,» Au lieu de répondre, Argenson
fit une pirouette sur son siège, puis dit tout bas au duc de La Force,
qui me le rendit après\,: «\,Monsieur, il n'y a plus moyen de parler ici
de rien qui touche à la constitution\,; aussi vous promets-je bien qu'on
n'y en parlera plus.\,» Il tint exactement parole, et oncques depuis il
n'y en a été parlé, pas même de cette affaire commencée. Mais, assez
longtemps après, Pontcarré, premier président du parlement de Rouen, qui
était de mes amis, m'apprit, à ma grande surprise, qu'ils savaient tous
dans leur compagnie qu'ils m'avaient l'obligation d'avoir sauvé leur
arrêt\,; qu'il avait tenu et qu'il avait fait mettre dans leurs
registres ce que j'avais fait pour eux au conseil de régence.

M. le duc d'Orléans accorda la liberté de revenir aux deux conseillers
du parlement de Paris, mais il ne voulut pas ouïr parler du président
Blamont, qui s'était distingué en sédition. Il s'en fomentait beaucoup
dans le royaume par le moyen de faux sauniers. Ces gens, qui ne
songeaient qu'à leur profit dans ce dangereux négoce, grossirent peu à
peu. Il y avait longtemps que ceux qui méditaient des troubles les
avaient pratiqués\,; mais ces espèces de troupes se grossirent et se
disciplinèrent à tel point qu'on {[}ne{]} put enfin se fermer assez les
yeux pour n'y pas apercevoir des troupes qui se rendaient redoutables
par leur valeur et par leur conduite, qui s'attiraient les peuples en ne
prenant rien sur eux, qui en étaient favorisés par l'utilité d'acheter
d'eux du sel à bon marché, qui s'en irritaient encore plus contre la
gabelle et les autres impôts, enfin, que ces faux sauniers, répandus par
tout le royaume et marchant souvent en grosses troupes qui battaient
tout ce qui s'opposait à eux, étaient des gens devenus dangereux, qui
avaient des chefs avec eux et des conducteurs inconnus, qui, par ces
chefs, les faisaient mouvoir, animaient les peuples et leur présentaient
une protection toute prête. Le mépris d'eux, qu'on n'avait pu ôter au
régent, se changea enfin en inquiétude trop juste, mais trop tardive, et
l'obligea à prendre des mesures pour arrêter un désordre fomenté par des
vues fort criminelles. Il y avait plus de cinq mille de ces faux
sauniers qui faisaient le faux saunage haut à la main, en Champagne et
en Picardie. Mezières, lieutenant général et gouverneur d'Amiens, fut
envoyé contre eux avec des troupes pour les dissiper.

Quoique le duc du Maine n'eût rien moins qu'aucune des qualités du
fameux amiral de Coligny, qui, trois jours avant l'affaire de Meaux, fut
trouvé, par celui que la cour envoya chez lui examiner ce qu'il s'y
passait, seul et sans armes, dans sa maison de Châtillon-sur-Loing,
taillant ses arbres dans son jardin\,; M. du Maine, dis-je, prit ce
temps précisément pour faire le marché d'une maison que
M\textsuperscript{me} la princesse de Conti avait fait bâtir et de deux
ou trois voisines qu'il acheta six cent mille livres avec ce qu'il y
fallut ajouter, dont il fit l'hôtel du Maine, au bout de la rue de
Bourbon, l'Arsenal n'ayant paru à M\textsuperscript{me} la duchesse du
Maine qu'une maison propre à y aller seulement faire quelques soupers.

Le roi étant fort jeune et avec beaucoup de belles maisons, et
M\textsuperscript{me} la duchesse de Berry, veuve et sans enfants, elle
eut envie d'avoir Meudon, et l'obtint de M. le duc d'Orléans en échange
du château d'Amboise qu'elle avait pour habitation par son contrat de
mariage. Cette espèce de présent ne laissa pas de faire du bruit\,; elle
en donna le gouvernement à Rion, et du Mont qui l'avait, ne laissa pas
de conserver les mêmes appointements qu'il en avait.

Chauvelin, avocat général depuis la mort de son frère aîné, acheta la
charge de président à mortier de Le Bailleul qui ne la faisait point, et
qui d'ailleurs la déshonorait par sa vie et sa conduite, et vendit la
sienne à Gilbert de Voisins, maître des requêtes du conseil des
finances. Je ne marquerais pas cette bagatelle, si ce même Chauvelin
n'était devenu depuis le jouet de la fortune, qui, après l'avoir élevé
tout à coup au plus haut point, le précipita au plus bas. Gilbert déjà
fort estimé, acquit une grande réputation dans la place d'avocat
général. L'abbé Bignon eut la bibliothèque du roi qu'avait l'abbé de
Louvois, avec le même brevet de retenue de douze mille livres.

Pezé, parent du maréchal de Tessé, et fort proche de la feue maréchale
de La Mothe, rapidement devenu capitaine aux gardes et gentilhomme de la
manche du roi, était un homme de beaucoup d'esprit et de talents. Il
savait cheminer, et avait une grande ambition. Le roi paraissait avoir
pour lui une bonté particulière qu'il savait grossir et faire valoir. Il
sut que Nangis à qui le régiment du roi ne donnait plus le même crédit,
ni les mêmes privances sous un roi enfant, en avait traité avec le duc
de Richelieu, et que le marché s'était rompu. Pezé qui comptait bien
faire grand usage de ce régiment quand le roi aurait plus d'âge, employa
le duc d'Humières auprès de moi pour en avoir l'agrément. Je l'obtins\,;
mais quand Pezé voulut traiter avec Nangis, il trouva un homme de
travers qui se fâcha qu'il en eût demandé l'agrément, avant d'avoir
commencé par savoir s'il le voulait vendre, et n'en voulut jamais ouïr
parler, disant qu'il voulait garder le régiment. Ce procédé parut tout à
fait ridicule. Pezé outré, me pria de le représenter à M. le duc
d'Orléans\,; je le fis, mais le régent n'eut pas la force d'imposer, et
Nangis ne me l'a jamais pardonné, dont je ne me souciai guère. La suite
fera voir que la mauvaise humeur de Nangis ne tendait qu'à rançonner le
régent dans cette affaire.

Tout tournait à la rupture avec l'Espagne, le duc de Saint-Aignan y
était devenu odieux au cardinal Albéroni, et y était sur un pied fort
triste. Il eut ordre de revenir. Comme ce n'était pas par sa faute que
les affaires s'y brouillaient, j'obtins de M. le duc d'Orléans de le
faire entrer en arrivant au conseil de régence, sans que M. de
Saint-Aignan y eût songé. Le duc de Berwick, en retournant à son
commandement de Guyenne, s'engagea au régent, d'accepter le commandement
de l'armée qui devait agir contre le roi d'Espagne sur cette frontière
en cas de rupture. Il avait la grandesse et la Toison\,; son fils aîné
établi avec l'une et l'autre en Espagne, y avait épousé la sueur du duc
de Veraguas non marié et sans enfants\,; elle était dame du palais de la
reine, et lui gentilhomme de la chambre du roi\,; son père lui avait
cédé les duchés de Liria et de Quiriça dont il avait eu le don avec la
grandesse, après la bataille qu'il gagna contre les Impériaux et les
Anglais à Almanza. On fut étonné qu'avec tant de liens qui devaient
l'attacher au roi d'Espagne, il eût accepté un emploi pour lequel il
n'était pas l'unique, et qui lui attira l'indignation de Leurs Majestés
Catholiques, dont, pour toujours, quoi qu'on ait pu faire depuis, elles
n'ont jamais du revenir, et qui nuisit fort pendant assez longtemps au
duc de Liria son fils, quoiqu'il servît dans l'armée d'Espagne opposée à
celle de son père. M. le duc d'Orléans aussi n'oublia jamais ce service
du duc de Berwick. Il estimait fort Asfeld, et Berwick qui l'estimait et
l'aimait beaucoup aussi, le désirait dans son armée. Le duc d'Orléans en
parla à Asfeld, dont la délicatesse fut plus grande. «\, Monseigneur,
répondit-il au régent, je suis Français, je vous dois tout, je n'attends
rien que de vous\,;» mais prenant sa Toison dans sa main et la lui
montrant\,: «\,Que voulez-vous que je fasse de ceci que je tiens du roi
d'Espagne, avec la permission du roi, si je sers contre l'Espagne, et
qui est le plus grand honneur que j'aie pu recevoir\,?» Il paraphrasa si
bien sa répugnance, et l'adoucit de tant d'attachement pour M. le duc
d'Orléans, qu'il fut dispensé de servir contre l'Espagne, en promettant
d'aller à Bordeaux avant que le maréchal en partît pour l'armée, si la
rupture arrivait, et de s'y tenir pour avoir soin d'amasser et de faire
voiturer à l'armée tout ce qu'il serait nécessaire, sans néanmoins de sa
personne sortir de Bordeaux. Cela fut par la suite exécuté de la sorte.
Asfeld y servit très utilement, et sa délicatesse fut généralement
applaudie en France et en Espagne\,; le régent ne l'en aima pas moins et
l'en estima davantage, et le roi d'Espagne lui en sut beaucoup de gré.

Je voyais ces dispositions avec regret, et j'en parlais souvent à M. le
duc d'Orléans, qui tâchait de me persuader que ce n'était que des
semblants pour amener l'Espagne à entrer enfin dans les propositions de
paix qui lui étaient faites, et lui-même se le figura ainsi fort
longtemps. Nancré arriva d'Espagne en admiration d'Albéroni\,: aussi ne
valaient-ils pas mieux l'un que l'autre.

M\textsuperscript{lle} d'Espinoy et M\textsuperscript{lle} de Melun, sa
soeur, qui étaient pauvres, obtinrent chacune six mille livres de
pension du roi. Meuse en eut quatre mille, et Béthune, fils de la sœur
de la feue reine de Pologne, autant\,: c'étaient deux hommes de grande
qualité, aussi fort mal dans leurs affaires\,; et le marquis de La Vire
qui était officier général de beaucoup de réputation, en Espagne, dont
il avait quitté le service, à l'occasion de l'affaire du régiment des
gardes wallonnes, dont il a été parlé en son temps, eut aussi une
pension de dix mille livres. Il avait été fait lieutenant général en
arrivant il était frère du prince de Chimay, lequel était grand
d'Espagne et chevalier de la Toison d'or, et qui depuis a été mon
gendre. Méliant, depuis conseiller d'État, à mon instante prière, eut
aussi six mille livres de pension, en mariant sa fille unique, très
riche, au fils aîné du garde des sceaux. Vertamont, premier président du
grand conseil, fort riche, en obtint une de huit mille livres contre
laquelle on cria fort, et non sans raison.

La banque de Law fut déclarée royale le 4 décembre, pour lui donner plus
de crédit et d'autorité. Le dernier, sans doute\,; pour le crédit, elle
y en perdit.

M\textsuperscript{me} la duchesse de Berry hasarda une chose jusqu'alors
sans exemple, et qui fut si mal reçue, qu'elle n'osa plus la réitérer.
Elle fut à l'Opéra dans l'amphithéâtre, dont on ôta plusieurs bancs.
Elle s'y plaça sur une estrade, dans un fauteuil, au milieu de sa maison
et de trente dames, dont les places étaient séparées du reste de
l'amphithéâtre, par une barrière. Ce qui parut de plus étonnant, c'est
qu'elle y parut autorisée parla présence de Madame et de M. le duc
d'Orléans, qui étaient en public dans la grande loge du Palais-Royal. Le
roi, dans Paris, fit paraître l'entreprise encore plus hardie.

Elle en fit une autre qui ne le fut pas moins, mais qui fit tant de
bruit, ainsi que la précédente, qu'elle n'osa y retourner. Elle s'avisa
de donner audience publique de cérémonie à un ambassadeur de Venise,
dans un fauteuil, placé sur une estrade de trois marches, quoi que
M\textsuperscript{me} de Saint-Simon pût lui représenter. La surprise
des dames assises et debout, venues à cette audience, fut extrême et
telle, que plusieurs voulaient s'en retourner, qu'on eut peine à
retenir. L'ambassadeur, étonné, s'arrêta à cette vue étrange, et demeura
quelques moments incertain. Il approcha néanmoins, comme prenant son
audience, pour éviter l'éclat\,; mais, après sa dernière révérence et
quelques moments de silence, il tourna le dos et s'en alla sans avoir
fait son compliment. Au sortir de Luxembourg, il fit grand bruit, et, le
jour même, tous les ambassadeurs protestèrent contre cette entreprise et
protestèrent encore qu'aucun ambassadeur ne se présenterait plus chez
M\textsuperscript{me} la duchesse de Berry qu'ils ne fussent assurés,
avec certitude, que cette entreprise ne se réitérerait plus. Ils
s'abstinrent tous de la voir, et ne s'apaisèrent qu'avec peine et au
bout d'assez longtemps sur les assurances les plus fortes qu'on pût leur
donner que pareille chose n'arriverait jamais. On remarquera, en
passant, que jamais reine de France n'a donné d'audience en cérémonie,
sur une estrade, pas même sur un simple tapis de pied.

\hypertarget{chapitre-iv.}{%
\chapter{CHAPITRE IV.}\label{chapitre-iv.}}

1718

~

{\textsc{Conversation entre M. le duc d'Orléans {[}et moi{]}, sur ses
subsides secrets contre l'Espagne, qui la voulut avoir enfermé seul avec
moi dans sa petite loge à l'Opéra.}} {\textsc{- Conversation forte entre
M. le duc d'Orléans et moi, dans son cabinet, tête à tête, sur la
rupture avec l'Espagne.}} {\textsc{- Faiblesse étrange du régent, qui
rompt avec l'Espagne, contre sa persuasion et sa résolution.}}
{\textsc{- Launay gouverneur de la Bastille.}} {\textsc{- Projet
d'Albéroni et travail de Cellamare contre le régent.}} {\textsc{-
Précautions de Cellamare pour pouvoir parler clairement à Madrid, et
prendre les dernières mesures.}} {\textsc{- Je suis mal instruit de la
grande affaire dont je vais parler.}} {\textsc{- Cause étrange de cette
ignorance.}} {\textsc{- Les dépêches de Cellamare, envoyées avec tant de
précautions, arrêtées à Poitiers et apportées à l'abbé Dubois, qui, dans
cette affaire surtout, en fait un pernicieux usage\,; et le secret de
tout enfoui.}} {\textsc{- Résultat bien reconnu des ténèbres de cette
affaire.}} {\textsc{- Instruments de la conjuration pitoyables.}}
{\textsc{- Cellamare arrêté\,; sa conduite.}} {\textsc{- J'apprends de
M. le duc d'Orléans ce qui vient d'être raconté de Cellamare, du duc et
de la duchesse du Maine, et du projet vaguement.}} {\textsc{- Conseil de
régence sur l'arrêt de l'ambassadeur d'Espagne, où deux de ses lettres
au cardinal Albéroni sont lues.}} {\textsc{- Pompadour et Saint-Geniez
mis à la Bastille.}} {\textsc{- Députation du parlement au régent,
inutile, en faveur du président de Blamont.}} {\textsc{- Abbé Brigault à
la Bastille.}} {\textsc{- D'Aydie et Magny en fuite.}} {\textsc{- La
charge du dernier donnée à vendre à son père.}} {\textsc{- Tous les
ministres étrangers, au Palais-Royal, sans aucune plainte.}} {\textsc{-
On leur donne à tous des copies des deux lettres de Cellamare à
Albéroni, qui avaient été lues au conseil de régence.}}

~

J'étais inquiet de voir que tout se préparait à rompre avec l'Espagne.
L'intérêt de l'abbé Dubois y était tout entier\,; on a vu, dans ce que
j'ai donné de M. de Torcy, quelle fut sa conduite en Angleterre. Il
n'avait osé y conduire son maître que par degrés, et ce fut à ce premier
degré, dont je prévis l'entraînement et les suites, que je crus devoir
m'opposer à temps. Il n'était alors question que de subsides de la
France à l'Angleterre, se déclarant contre l'Espagne, conjointement avec
l'empereur, et ces subsides devaient être secrets. Après avoir effleuré
cette matière avec M. le duc d'Orléans, nous convînmes, lui et moi, de
la traiter à fond. Il en usa pour cette affaire comme il avait fait pour
celle des appels, et me traîna, malgré tout ce que je lui pus
représenter, dans sa petite loge de l'Opéra. Il en ferma la porte après
avoir défendu qu'on y frappât, et là, tête à tête, nous ne songeâmes à
rien moins qu'à l'opéra. Je lui représentai le danger d'élever
l'empereur, à l'abaissement duquel et de sa maison la France avait sans
cesse travaillé depuis les grands coups que le cardinal de Richelieu lui
avait su porter, toutes les fois que l'État n'avait pas été trahi par
l'intérêt et l'autorité des reines mères italiennes ou espagnoles\,; de
l'empereur qui, de plus, ne pardonnerait jamais à la France d'avoir
enlevé l'Espagne et les Indes à sa maison et à lui-même, de l'empereur
enfin qui avait mis la France à deux doigts de sa perte, et qui, lorsque
la reine Anne la sauva, fit l'impossible contre elle, et fut le dernier
de tous les alliés à signer la paix\,; que l'agrandissement de
l'Angleterre et du roi Georges n'était pas moins redoutable, qui, sous
les trompeuses apparences d'une feinte amitié, étaient nos plus anciens
et plus naturels ennemis, que l'épreuve de cette vérité était de tous
les siècles, si on en excepte des instants comme entre Henri IV et
Élisabeth, et les moments d'autorité de Charles II et du changement du
conseil de la reine Anne\,; que leur double intérêt revenait au même\,:
celui du roi Georges, de tout faire pour l'empereur, par la raison de
ses États d'Allemagne, et par l'investiture de Brême et de Verden, après
laquelle il soupirait depuis si longtemps, et que l'empereur lui faisait
attendre pour le tenir en ses mains et s'en servir sûrement dans toutes
ses vues\,; de la nation, qui n'avait d'objet que le commerce, que de
ruiner celui d'Espagne et le nôtre en même temps, peu inquiets de celui
du Portugal où ils étaient les maîtres, de celui de Hollande qu'ils
avaient à demi ruiné et dont ils dominaient la république, et que nous
avions grand intérêt de ne pas laisser achever de ruiner, parce qu'il ne
pouvait nous être contraire au point où il se trouvait réduit. J'ajoutai
l'intérêt commun de toute l'Europe, de brouiller sans cesse et
irrémédiablement, si elle le pouvait, les deux branches de la maison de
France\,; dont la jalousie était telle, depuis que la couronne d'Espagne
y était entrée, qu'il n'était efforts qu'elle n'eût faits pour l'en
arracher, et depuis ne l'avoir pu par les armes, pour brouiller les deux
couronnes et y semer sans cesse la zizanie depuis la mort du roi\,; que
cet objet était si grand pour l'empereur et pour l'Angleterre, qu'il ne
fallait pas croire que nulle difficulté pût les rebuter, et d'autre part
aussi tellement visible que tous leurs artifices ne pouvaient qu'être
grossiers\,; que l'intérêt si grand, si évident, si naturel de notre
union avec l'Espagne, nous était appris par leur acharnement à tout
tenter pour la rompre, quand nous ne sentirions pas jusqu'à quel point
il était capital à la France d'entretenir une union indissoluble avec
l'Espagne, d'avoir mêmes amis et mêmes ennemis, et, comme je le lui
avais si souvent représenté dans son cabinet et en plein conseil,
d'imiter l'union des deux branches de la maison d'Autriche, qui avait
mis le sceau à sa grandeur, et dont l'identité continuelle tant que
celle d'Espagne avait duré l'avait conservée.

Je lui fis remarquer avec détail que l'empereur et l'Angleterre ne
pouvaient être que de faux amis, et encore de moments, parce que ces
deux puissances avaient et auraient toujours des intérêts directement
contraires à ceux de la France, au lieu qu'outre le même sang et la
proximité, nul intérêt essentiel ne pouvait jamais aliéner la France de
l'Espagne, depuis qu'elle n'obéissait plus à un roi de la maison
d'Autriche, ni l'Espagne de la France. Je lui touchai après son intérêt
personnel, de ne se pas mettre au hasard de rompre avec l'Espagne, après
tout ce qui s'était passé vers la fin du feu roi sur son compte avec
l'Espagne. Ensuite je lui fis sentir la grossièreté du piége qu'on lui
tendait\,; que des subsides secrets étaient un engagement qui
l'entraînerait à la rupture, qu'on n'osait lui proposer d'abord, et où
on l'amènerait par degrés\,; qu'il était honteux et très nuisible à la
France, de payer les ennemis de l'Espagne pour lui faire la guerre, et
plus honteux à lui personnellement, après ce qui s'était passé de
personnel, qu'à tout autre qui aurait le timon de l'État\,; que
l'intérêt, le but, les vues de l'entraîner à la rupture étaient trop
grands et trop évidents pour qu'il dût espérer que l'empereur et
l'Angleterre ne trahissent pas le prétendu secret des subsides qu'il
donnerait, et qu'il devait compter qu'eux-mêmes auraient grand soin de
faire revenir à l'Espagne qu'il leur en fournissait\,; que dès lors il
devait s'attendre aux plus vifs reproches, aux emportements de la reine,
à tout le venin d'Albéroni, dont l'abbé Dubois saurait bien profiter
pour l'aigrir, pour emporter ainsi ce qu'il n'ose proposer encore\,;
qu'alors, Son Altesse Royale donnerait beau jeu aux brouillons qui ne
cherchaient qu'à ranimer les haines amorties de l'Espagne contre sa
personne pour s'en avantager à l'abri de la naissance et de la puissance
du roi d'Espagne, et faire payer bien cher la complaisance pour l'abbé
Dubois, qui, n'osant aller directement où il aspire, ne songeait, pour y
parvenir, qu'à servir si utilement nos ennemis naturels contre des amis
que tout nous doit faire à jamais considérer comme des frères, et
j'ajoutai avec feu\,: «\,Qu'il obtienne donc la pourpre par le crédit de
l'empereur qui peut maintenant tout à Rome, et par celui du roi Georges,
qui peut infiniment sur l'empereur\,!»

M. le duc d'Orléans, qui jusque-là m'avait écouté attentivement et
tranquillement, excepté quelques applaudissements sur ne pas rompre avec
l'Espagne, s'écria que voilà comme j'étais, suivant toujours mes idées
aussi loin qu'elles pouvaient aller\,; que Dubois était un plaisant
petit drôle pour imaginer de se faire cardinal\,; qu'il n'était pas
assez fou pour que cette chimère lui entrât dans la tête, ni lui, si
elle y entrait jamais, pour le souffrir\,; que pour son intérêt
personnel, il ne risquerait rien, parce qu'il ne s'agissait que de
subsides secrets qui seraient toujours ignorés de l'Espagne, et qu'à
l'égard de celui de l'État, il se garderait bien de lâcher aux Anglais
ni à l'empereur la courroie assez longue pour que la puissance de
l'empereur pût s'augmenter, ni le commerce des Anglais s'accroître. Je
ne me payai point de ces raisons\,; j'assurai le régent qu'en de telles
liaisons on était toujours mené plus loin qu'on ne pensait et qu'on ne
voulait, et pour le secret de ses subsides, je lui maintins que
l'intérêt de ces deux puissances était si capital de le brouiller avec
l'Espagne, qu'elles se garderaient bien de ne le pas publier comme le
moyen le plus court et le plus certain d'arriver à leur but principal,
qui était de le forcer à la rupture ouverte, et par là même à une
liaison avec elles de nécessité et de dépendance.

Tout cela agité, approfondi, discuté et disputé entre nous deux, tant
que l'opéra dura sans le voir ni l'entendre, nous laissa chacun dans sa
persuasion. M. le duc d'Orléans, qu'il demeurerait très sûrement maître
de son secret et de son aiguière, et que, par cette complaisance, il
s'assurerait d'autant plus d'être le modérateur de l'Europe\,; moi, au
contraire, que le secret et l'aiguière lui échapperaient l'un et
l'autre, et bientôt, et qu'il se trouverait dans un embarquement dont il
aurait tout lieu et tout le temps de se bien repentir. En effet, de là à
la rupture, il s'écoula peu de mois. Il arriva, comme je l'avais prévu,
que l'Espagne fut promptement informée de l'engagement que le régent
avait pris avec l'empereur et l'Angleterre, et qu'elle redoubla tout
aussitôt ses soins à donner à M. le duc d'Orléans tant d'affaires
domestiques, qu'il ne fut plus à craindre pour celles du dehors, dont on
verra bientôt les effets, mais qui heureusement ne firent que montrer
l'étendue des projets et de ses ressorts.

La rupture s'approchait par les ruses de l'abbé Dubois, qui n'en
laissait voir à personne que ce qu'il ne pouvait empêcher, par
l'extérieur de mesures qui ne se qualifiaient que de simples
précautions\,; et il avait fermé la bouche là-dessus à M. le duc
d'Orléans, jusque avec le très petit nombre de ceux avec qui il
s'ouvrait le plus sur différentes affaires\,; car nul n'eut jamais sa
confiance sur toutes que l'abbé Dubois, depuis qu'il s'y fut tout à fait
abandonné.

Dubois ne put pourtant si bien faire que le secret m'en fût gardé
jusqu'au bout. Une après-dînée que j'allai au Palais-Royal pour mon
travail ordinaire, tête à tête, comme j'avais accoutumé un jour au moins
de chaque semaine, et que je commençais à en mettre les papiers sur le
bureau de M. le duc d'Orléans, il me dit qu'avant de commencer, il avait
chose bien plus importante à me dire, sur laquelle il voulait raisonner
à fond avec moi\,; et, tout de suite, m'expliqua la situation en
laquelle il se trouvait avec l'empereur, l'Angleterre et l'Espagne, et
combien il était vivement pressé de se déclarer ouvertement et par les
armes contre la dernière.

Après avoir bien écouté tout son récit, je le fis souvenir de ce que je
lui avais dit et prédit à l'Opéra, quand, tête à tête, nous y agitâmes,
dans sa petite Loge, l'affaire des subsides secrets, et je lui rappelai
fort en détail tout ce que je lui avais allégué alors contre la rupture
avec l'Espagne dont il avait été si bien convaincu, qu'il n'avait
persisté à donner les subsides contre mon avis que dans la prétendue
certitude du secret et de nul danger d'engagement plus fort, ni que les
choses pussent aller trop loin de la part de l'empereur et de
l'Angleterre contre l'Espagne, choses que je lui avais toujours
fortement contestées. La rupture à laquelle il était violemment poussé
par l'abbé Dubois fut longuement et fortement discutée.

Le régent ne trouva point de réponse valable à mes raisons\,; mais il
était embarrassé de l'empereur, enchanté par l'Angleterre, plus que tout
entraîné par sa faiblesse pour l'abbé Dubois, qui comptait la fortune
après laquelle il soupirait avec de si vifs élans indissolublement
attachée à la rupture. Voyant donc le régent convaincu, mais pourtant
point persuadé, et gémissant intérieurement des chaînes dans lesquelles
il se sentait entravé, j'imaginai tout à coup de les lui faire rompre
par quelque chose d'extraordinaire. Je lui dis donc avec feu que je le
suppliais de vouloir bien ne se pas effaroucher d'une supposition
impossible, de m'écouter tout du long et de suivre mon raisonnement\,:
«\,S'il vous était aussi évident, lui dis-je, qu'il y eût quelque part à
portée de vous un devin ou un prophète qui sût clairement l'avenir, et
qui fût en pouvoir et en volonté de répondre à vos consultations, comme
il est évident que cela n'est pas, n'est-il pas vrai qu'il y aurait de
la folie d'entreprendre une guerre sans avoir su de lui auparavant quel
en serait le succès\,? Si ce prophète ne vous annonçait que places et
batailles perdues, n'est-il pas vrai encore que vous n'entreprendriez
pas cette guerre, et que rien ne vous y pourrait entraîner\,? Et moi je
vous dis que sur celle dont il s'agit votre résolution devrait être
aussi fermement la même, si cet homme merveilleux ne vous promettait que
victoires et que succès, et en voici mes raisons\,: dans l'un et dans
l'autre cas, vous affaiblissez l'État, vous en agrandissez d'autant les
ennemis naturels par qui vous vous laissez entraîner à la guerre\,; vous
tentez toute une nation, accoutumée depuis qu'elle existe dans le pays
où elle est, à l'aînesse dans la maison de ses rois\,; vous hasardez un
pouvoir précaire et vous donnez lieu de publier que vous ne l'employez
que pour votre intérêt personnel, et pour acheter aux dépens de l'État,
de son plus naturel intérêt et de tout le sang et les trésors répandus
depuis la mort du feu roi d'Espagne, pour acheter, dis-je, un appui
étranger contre les droits de Philippe V sur la France, dont par là vous
avouez toute la force et toute votre crainte. Et au cas d'heureux
succès, que ces mêmes puissances vous forcent à pousser plus loin que
vous ne voudrez, où en seriez-vous si le roi d'Espagne, à bout de
moyens, et de dépit, vous laissait faire, entrait en France désarmé,
publiait qu'il vient se livrer à ces mêmes Français qui l'ont mis et qui
l'ont maintenu sur le trône, qui sont les sujets de ses pères et de son
propre neveu paternel\,; qu'il ne vient que pour le secourir et en
prendre la régence que sa naissance lui donne, sitôt que son absence ne
l'en exclut plus, et l'arracher lui, sa nation et son héritage à un
gouvernement tel qu'il lui plaira de le représenter\,? Je ne sais,
ajoutai-je, quelle en pourrait être la révolution\,; mais je vous
confesse, monsieur, à vous tout seul, que pour moi, qui n'ai jamais été
connu du roi d'Espagne que pour avoir joué aux barres avec lui et à des
jeux de cet âge, qui n'en ai pas ouï parler depuis qu'il est en Espagne,
ni lui beaucoup moins de moi, et qui n'y connais qui que ce soit\,; moi,
qui suis à vous dès l'enfance, et qui savez à quel point j'y suis\,; qui
ai tout à attendre de vous, et quoi que ce soit de nul autre, je vous
confesse, dis-je, que, si les choses venaient à ce point, je prendrais
congé de vous avec larmes, j'irais trouver le roi d'Espagne, je le
tiendrais pour le vrai régent et le dépositaire légitime de l'autorité
et de la puissance du roi mineur\,; que si moi, tel que je suis pour
vous, pense et sens de la sorte, qu'espéreriez-vous de tous les autres
vrais Français\,?»

La sincérité, la vérité, la force de ce discours accabla le régent, et
le tint assez longtemps en silence, la tête et le visage entre ses deux
mains, les coudes sur son bureau, comme il se mettait toujours quand il
était fort en peine\,; puis il avoua sans détour que j'avais raison, et
que je lui rendais un grand service de lui parler de la sorte.

Là-dessus, M. le Duc entra. Le régent le mena d'abord dans la galerie,
et je demeurai dans le grand salon à me promener, où, assis et le bureau
entre deux, la conversation s'était passée. La visite de M. le Duc fut
très courte, et M. le duc d'Orléans et moi nous remîmes aussitôt à son
bureau. J'y voulus déployer les papiers que j'y avais mis, mais il ne me
le permit pas, et me dit qu'il fallait continuer notre raisonnement qui
roulait sur des choses bien plus importantes. Il se leva, et nous nous
promenâmes dans le salon et dans la galerie.

Je lui dis que je n'avais point de nouveau raisonnement à faire, que je
lui avais tout dit, que redire ne serait que répéter et rebattre, mais
que je croyais aussi en avoir assez dit pour avoir dû le persuader et
l'empêcher de tomber dans le précipice par les pièges de l'ambition de
l'abbé Dubois, qui, de l'un à l'autre, l'engageait où il ne devait
jamais se laisser aller. Le régent me protesta qu'il le ferait mettre
dans un cachot, s'il osait jamais faire un pas vers la pourpre, et
convint avec moi de ne point rompre avec l'Espagne. Je tâchai de l'y
affermir de plus en plus\,; puis je lui dis\,: «\,Vous voilà donc bien
persuadé et bien convaincu, mais je ne serai pas sorti d'ici que l'abbé
Dubois vous reprendra et vous retournera, verra que c'est depuis que je
vous ai entretenu que vous ne voulez plus vous déclarer contre
l'Espagne, fera si bien qu'il vous changera et vous tiendra de si près
qu'il viendra à bout de ce qu'il s'est mis dans la tête, et vous fera
déclarer contre l'Espagne.\,» Le régent m'assura que sa résolution de
n'en rien faire était si bien prise, que rien ne la lui ferait changer,
et toutefois au bout de huit jours, la guerre à l'Espagne fut déclarée.

Pendant ces huit jours, je fis ce que je n'ai jamais fait pendant toute
la régence\,: j'allai trois ou quatre fois chez M. le duc d'Orléans, et,
ce qui ne m'est jamais arrivé qu'alors, jamais je ne le pus voir.
L'inquiétude de la guerre, qui m'y avait conduit, augmenta par cette
clôture, où je vis bien que Dubois le tenait enfermé pour moi. Je lui
écrivis pour demander à le voir\,: point de réponse\,; je récrivis de
nouveau\,: il me fit dire verbalement que, dès qu'il me pourrait voir,
il me le manderait. Alors je jugeai la chose désespérée, et je ne me
trompai pas.

Le jour que la nouvelle éclata, il me manda qu'il me verrait quand je
voudrais. J'allai au Palais-Royal, je trouvai un homme embarrassé, la
tête basse, qui de honte n'osait me regarder. Mon abord fut froid, aussi
le silence dura assez longtemps. Il le rompit enfin d'une voix basse par
un «\,Que dirons-nous\,? --- Rien du tout, lui répondis-je, parce qu'aux
choses faites il n'y a plus à parler, il n'y a qu'à souhaiter que vous
vous en trouviez bien. Du reste, je vous supplie de croire que, pour
quelque intérêt particulier ou personnel que ce pût être, je ne vous
aurais pas pourchassé comme j'ai fait inutilement depuis huit jours.
Vous savez que mon goût ni ma coutume n'est pas de vouloir forcer les
portes\,; mais j'ai cru que mon attachement pour vous et mon devoir à
l'égard du bien de l'État me devaient faire sortir de mon naturel et de
toutes bornes. Vous n'avez pas jugé à propos de me voir, je m'en lave
les mains\,; parlons maintenant d'autre chose\,;» et tout de suite je
tire des papiers de mes poches et je les étends sur son bureau. Il en
fit le tour pour s'y aller asseoir sans dire une parole, et tant que je
fus avec lui je ne vis qu'embarras, souplesses et caresses\,; de mon
côté, je ne montrai point d'humeur. Il fallut après du temps, pour en
parler à la régence et pour dresser et lui montrer la déclaration de
guerre, ce qui se fit en même temps. J'y reviendrai ensuite, parce que
j'ai prévenu le temps de ma conversation du Palais-Royal, comme j'ai
retardé celle de l'Opéra, parce que j'ai voulu les mettre tout de suite
toutes les deux, quoique séparées d'un long intervalle pour mettre tout
à la fois sous les yeux ce qui se passa entre M. le duc d'Orléans et moi
sur la guerre d'Espagne. Retournons maintenant un peu sur nos pas.

Le colonel Stanhope, depuis longtemps envoyé d'Angleterre en Espagne,
arriva à Paris, retournant en Angleterre.

Barnaville, qui, de lieutenant de roi de Vincennes, avec la charge de
confiance des prisonniers, avait passé au gouvernement de la Bastille,
venait de mourir. Launay, qui en était lieutenant de roi, eut ce
gouvernement, et ce fut un très bon choix. J'en parle ici, parce qu'il y
fut mis dans un temps important et critique.

Cellamare, ambassadeur d'Espagne, de beaucoup de sens et d'esprit,
s'employait depuis longtemps à préparer bien des brouilleries, comme on
le voit par ce que j'ai donné des extraits des lettres de la poste faits
par M. de Torcy. On y voit combien le cardinal Albéroni avait cette
affaire dans la tête, et avec quel empressement Cellamare y répondait
pour lui plaire. Le projet n'était pas de moins que de révolter tout le
royaume contre le gouvernement de M. le duc d'Orléans, et, sans avoir vu
clair à ce qu'ils comptaient faire de sa personne, ils voulaient mettre
le roi d'Espagne à la tête des affaires de France, avec un conseil et
des ministres nommés par lui et un lieutenant sous lui de la régence qui
aurait été le véritable régent, et qui n'était autre que le duc du
Maine. Ils comptaient sur les parlements, à l'exemple de celui de
Paris\,; sur les chefs et les principaux moteurs de la constitution, sur
la Bretagne entière, sur toute l'ancienne cour accoutumée au joug des
bâtards et de M\textsuperscript{me} de Maintenon, et depuis longtemps
ils ne cessaient d'attacher tous ceux qu'ils pouvaient à l'Espagne par
toutes sortes de prestiges, de promesses et d'espérances. On verra que
leurs mesures répondirent mal à l'importance de ce projet. Il est vrai
qu'ils ne purent pas attendre sa maturité. La rupture de la France avec
l'Espagne était imminente, il en fallait arrêter les suites au plus tôt
et différer la révolte tout le moins qu'il leur serait possible. Ils
furent découverts comme ils prenaient leurs dernières mesures\,; mais le
régent et l'État y furent étrangement trahis, et M. le duc d'Orléans y
montra une incroyable faiblesse.

Les choses étant à ce point du côté de l'Espagne et de ceux qui
s'étaient dévoués à leur vengeance ou à leurs propres espérances, il
fallut parler clair à Madrid sur l'état des choses et sur les noms.
Cellamare, trop sage pour confier à pas un de ses gens un paquet de
cette conséquence, voulut que le courrier fût choisi à Madrid, et que ce
fût quelqu'un au-dessus d'un courrier, qui eût en même temps dans sa
personne et dans sa qualité de quoi ôter toute défiance. Pour mieux
cacher un secret si important, ils choisirent à Madrid un jeune
ecclésiastique qui s'appelait ou se fit appeler l'abbé Portocarrero, à
qui ils donnèrent pour adjoint le fils de Monteléon. Rien de mieux
imaginé que deux jeunes gens que le hasard semblait faire rencontrer à
Paris, l'un venant de Madrid, l'autre de la Haye, et se joindre après
pour retourner de compagnie en Espagne. Le nom de Portocarrero imprimait
et, depuis le fameux cardinal Portocarrero, portait avec soi sa faveur
de la France. L'autre était le fils de l'ambassadeur d'Espagne, depuis
longtemps en Angleterre, qui avait été assez longtemps en France et y
avait laissé des amis considérables. Il était déclaré de tout temps pour
la France, et pour que l'Espagne ne s'en séparât jamais\,; on le
savait\,; l'abbé Dubois en avait été souvent témoin à Londres, et que
cet attachement lui avait mal réussi auprès d'Albéroni. On a vu, par ces
extraits de lettres de la poste de M. de Torcy, que Monteléon fut
là-dessus inébranlable. Monteléon, sorti d'Angleterre par la rupture et
les actions de la flotte anglaise contre l'Espagne dans la Méditerranée,
était allé à la Haye attendre ce que sa cour voudrait faire de lui, et
il paraissait qu'il envoyait son fils en Espagne pour cette affaire
particulière. Deux jeunes gens de noms agréables à la France et qui
semblaient si bien se rencontrer de pur hasard à Paris, l'un venant de
Madrid, l'autre de la Haye, et qu'il était si naturel qu'ils s'en
retournassent ensemble, avaient tout ce qu'il fallait pour ôter tout
soupçon qu'ils pussent être chargés d'aucun paquet de conséquence par
l'ambassadeur, qui avait ses propres courriers et le renvoi de ceux
qu'il recevait d'Espagne. On peut juger aussi que ces jeunes gens
eux-mêmes ignoraient parfaitement ce dont ils étaient chargés, et il
était tout simple que, s'en allant en Espagne, l'ambassadeur les
chargeât de quelque paquet par occasion.

Ils partirent donc, munis de passeports du roi, à cause de la
conjoncture de rupture prochaine, les premiers jours de décembre, avec
un banquier espagnol établi en Angleterre, qui y venait de faire une
fort grande banqueroute, et que les Anglais avaient obtenu du régent de
pouvoir faire arrêter partout où ils pourvoient en France. On me
trouvera bien mal instruit dans tout le cours de cette grande affaire,
mais je ne puis ni ne veux dire que ce que j'en ai su, et du reste je
donnerai mes conjectures\footnote{Comme Saint-Simon avoue qu'il n'a
  connu qu'imparfaitement les détails de cette affaire, il ne sera pas
  inutile de chercher à compléter son récit par le témoignage d'autres
  écrivains. Duclos, dans ses \emph{Mémoires secrets} (année 1718),
  donne des renseignements précis sur la manière dont le complot fut
  découvert\,: «\,Il y avait alors à Paris une femme, nommée la Fillon,
  célèbre appareilleuse, par conséquent très connue de l'abbé Dubois.
  Elle paraissait même quelquefois aux audiences du régent, et n'y était
  pas plus mal reçue que d'autres. Un ton de plaisanterie couvrait
  toutes les indécences au Palais-Royal, et cela s'est conservé dans le
  grand monde. Un des secrétaires de Cellamare avait un rendez-vous avec
  une des filles de la Fillon, le jour que partait l'abbé Portocarrero.
  Il y vint fort tard et s'excusa sur ce qu'il avait été occupé à des
  expéditions de lettres, dont il fallait charger nos voyageurs. La
  Fillon laissa les amants ensemble, et alla sur-le-champ en rendre
  compte à l'abbé Dubois. Aussitôt on expédia un courrier muni des
  ordres nécessaires pour avoir main-forte. Il joignit les voyageurs à
  Poitiers, les fit arrêter\,; tous leurs papiers furent saisis, et
  rapportés à Paris le jeudi 8 décembre. Ce courrier arriva chez l'abbé
  Dubois précisément à l'heure où le régent entrait à l'Opéra.\,» Le
  reste du récit n'est qu'un résumé des \emph{Mémoires} de Saint-Simon.
  Voy. aussi Lemontey, \emph{Histoire de la régence}, édit. de 1832, t.
  I, p.~216. D'après Lemontey, un copiste nommé Buvat, que les
  conspirateurs avaient employé, dénonça le complot à l'abbé Dubois.}.
L'abbé Dubois, de plus en plus maître de M. le duc d'Orléans, le voulait
être du secret de tout, pour n'avoir ni contradicteur ni même de
compagnon, et M. le duc d'Orléans lui fut fidèle en obéissance.
Lui-même, comme on le verra, n'en sut que ce qu'il plut ou ce qu'il
convint à l'abbé Dubois.

Soit que l'arrivée de l'abbé Portocarrero, et le peu de jours qu'il
demeura à Paris fût suspect à l'abbé Dubois et à ses émissaires, soit
qu'il eût corrompu quelqu'un de principal auprès de l'ambassadeur
d'Espagne, par qui il fut averti que ces jeunes gens étaient chargés
d'un paquet important, soit qu'il n'y eût pas d'autre mystère que la
mauvaise compagnie du banqueroutier parti avec eux, et l'attention de
l'abbé Dubois à obliger les Anglais en le faisant arrêter, et qu'il eût
ordonné de les arrêter tous trois, et d'enlever tous leurs papiers, de
peur que le banqueroutier ne leur eût donné les siens pour ne les pas
perdre s'il venait à être pris\,; quoi qu'il en soit, l'abbé Dubois fit
courre après eux, et ils furent arrêtés à Poitiers, tous leurs papiers
enlevés et apportés à l'abbé Dubois par le courrier qui, aussitôt après
leur capture, fut dépêché de Poitiers pour lui en apporter la nouvelle.
Les hasards font souvent de grandes choses. Le courrier de Poitiers
entra chez l'abbé Dubois comme M. le duc d'Orléans entrait à l'Opéra.
Dubois parcourut les papiers\footnote{Lemontey (\emph{ibid}., p.~219)
  donne l'indication des papiers qui avaient été saisis à Poitiers.}, et
dit la nouvelle de la capture à M. le duc d'Orléans comme il sortait de
sa loge. Ce prince, qui avait accoutumé de s'enfermer alors tout de
suite avec ses roués, en usa de même ce jour-là, sous prétexte que
l'abbé Dubois n'avait pas eu le temps d'examiner les papiers, avec une
incurie à laquelle tout cédait. Les premières heures de ses matinées
étaient peu libres. Sa tête, offusquée encore des fumées du vin et de la
digestion des viandes du souper, n'était pas en état de comprendre, et
les secrétaires d'État m'ont souvent dit que c'était un temps où il ne
tenait qu'à eux de lui faire signer tout ce qu'ils auraient voulu. Ce
temps fut pris par l'abbé Dubois pour lui rendre compte des papiers
arrivés de Poitiers, tel qu'il jugea à propos. Il n'en dit et n'en
montra que ce qu'il voulut, et ne se dessaisit jamais d'aucun entre les
mains du régent, aussi peu de pas un autre. La confiance aveugle, et la
négligence abandonnée de ce prince en cette occasion fut
incompréhensible\,; et ce qui l'est encore plus, c'est que l'une et
l'autre régna dans toute la suite de cette affaire et dans toutes ses
parties, et rendit l'abbé Dubois le maître unique des preuves, des
soupçons, de la conviction, de l'absolution, de la punition.

Il n'admit dans cette affaire que le garde des sceaux et Le Blanc, parce
qu'il ne put s'en passer, mais sans leur dire qu'autant et si peu qu'il
lui convenait. Le premier était dans son intimité et dans son entière et
absolue dépendance\,; le second n'était que dans la même dépendance, et
se flattait mal à propos de l'intimité\,; tous deux, dans la stupeur de
sa conduite dans cette affaire, et dans la frayeur de lui faire la
moindre question et d'outrepasser ses ordres d'une ligne. C'était de sa
seule volonté que leurs places dépendaient\,; il le leur faisait sentir
tous les jours. Ils comptaient donc le maître pour rien et le valet pour
tout. Leurs démarches, leurs interrogatoires, les comptes qu'ils
rendirent au régent dans tout le cours de cette affaire, ce qu'ils
poussèrent, ce qu'ils firent semblant de pousser, ce qu'ils laissèrent
échapper ou tomber, ce qu'ils favorisèrent, ce qu'ils dirent au régent
et ce qu'ils lui turent, en un mot toute leur conduite, leurs démarches,
jusqu'à leurs paroles, et tout cela jusque dans le dernier détail et
dans la précision la plus exacte, fut à chaque pas réglé par Dubois. Cet
abbé fut le seul, l'unique, le suprême conducteur et modérateur, avec un
empire et une jalousie que rien ne troubla, et qui ne trouva que
soumission aveugle la plus exacte dans la frayeur et le tremblement de
ces deux hommes, qui reçurent dans cette servile disposition les ordres
qu'ils en attendaient à chaque instant, et jusque pour chaque minutie,
uniquement occupés d'une obéissance littérale et aveugle, à laquelle ce
maître terrible ne leur laissa pas ignorer que leur fortune était
singulièrement attachée. Ainsi la connaissance entière et effective de
cette profonde affaire et de toutes ses différentes parties demeura
uniquement à l'abbé Dubois tout seul, qui ne s'y servit aussi que de ces
deux seuls hommes, auxquels il ne communiqua que par mesure et que ce
qu'il lui convint de leur communiquer. Il ne traita pas M. le duc
d'Orléans avec plus de confiance, à qui le garde des sceaux et Le Blanc
n'osèrent jamais rien rendre que les leçons précises, et bien
exactement, qu'ils recevaient pour cela de l'abbé Dubois, et au temps,
au ton et à la mesure qu'il leur prescrivait à chaque fois. Par cette
conduite, je ne puis assez le répéter, Dubois demeura seul instruit et
maître absolu du fond de tout le secret de l'affaire, du degré et du
sort des coupables, d'en augmenter et d'en diminuer le nombre et le
poids à sa volonté, sans crainte de pouvoir être démenti, ni même
contredit, ni traversé en la moindre chose. On arrêtait les gens et on
les relâchait sur les ordres du roi donnés par le régent, dont l'abbé
Dubois disposait seul et absolument, sans que jamais il y ait eu de
démarches ni de procédures juridiques, parce qu'elles n'auraient pas pu
être également dans sa main.

Le garde des sceaux, qui avait le plus de part en la confiance de l'abbé
Dubois et qui en a toujours espéré et été ménagé pendant sa disgrâce,
est mort avant lui dans ces dispositions et a emporté avec lui ce qu'il
savait de ce secret. Le Blanc, déjà poussé et chassé par Dubois avant sa
mort, et tombé au bord de l'abîme, dont il essuya depuis toutes les
horreurs, avait beaucoup moins su de tout cela que le garde des sceaux,
qui était le seul dont Dubois pût prendre quelque conseil dans la
nécessité\,; et Le Blanc, de retour enfin au monde et à la fortune sur
une terre nouvelle et sous d'autres cieux, s'est bien gardé de dire
{[}rien{]} de ce qu'il pouvait savoir d'une affaire dont les principaux
et les plus grands coupables étaient, non seulement sortis de prison et
de toute inquiétude dès avant sa plus profonde chute, mais rétablis en
leur premier état, grandeur et splendeur, ainsi que tous les autres
accusés et soupçonnés.

Soit que M. le régent en ait plus su qu'il n'a voulu le montrer, et que
la crainte du nombre et du nom, des établissements et de la
considération de ceux qui ont trempé dans cette affaire, lui ait fait
prendre le parti qu'il y a pris\,: soit que sa négligence continuelle et
son prodigieux asservissement sous le joug que l'abbé Dubois avait su
lui imposer, l'eût laissé, comme je l'ai cru, dans l'ignorance du vrai
fond et des circonstances importantes de l'affaire et de la plupart des
gens considérables qui y étaient entrés, ou pour ménager la faiblesse du
prince qu'il connaissait si parfaitement, ou pour se faire peu à peu, en
temps et lieu, un mérite auprès de ceux dont il avait tu les noms, ni
moi ni personne n'avons rien pu tirer de M. le duc d'Orléans au delà du
récit ténébreux que je vais faire. Mais toujours, d'une obscurité si
étrangement profonde résulte bien certainement un complot de M. et de
M\textsuperscript{me} du Maine, laquelle y travailla longtemps avant le
dernier lit de justice et dès l'entrée de la régence par l'ameutement de
la prétendue noblesse, du parlement, de la Bretagne, et tout ce qu'elle
sut mettre en oeuvre pour tenir ce qu'on a vu (t. XI, p.~421) qu'elle
avait déclaré si nettement aux ducs de La Force et d'Aumont lorsqu'ils
furent forcés de la voir à Sceaux, sur l'affaire du bonnet\,: «\,Que
quand on avait une fois acquis, comme que ce fût, la qualité de prince
du sang et l'habilité de succéder à la couronne, il fallait bouleverser
l'État et mettre tout en feu plutôt que se les laisser arracher.\,»

Ces ameutements, en apparence contre les ducs, ou le gros des ameutés
furent les premiers trompés, ne furent en effet pratiqués que pour se
fortifier contre les princes du sang depuis que l'aigreur se fut mise
entre eux par le procès de la succession de M. le Prince, et empêcher le
régent de juger la demande formée contre eux par les princes du sang et
d'en rayer la qualité avec le prétendu droit d'habilité factice de
succéder à la couronne. Aussi réussirent-ils à lui faire une telle peur
qu'il en éluda le jugement contre ses paroles souvent données, contre
toute justice, raison et bienséance, et qu'il ne céda, après tant de
délais, de subterfuges, de tours de souplesse, qu'aux cris et à une
véritable obsession des princes du sang, qui se relevèrent à rie le pas
laisser respirer. C'est ce qui parut mieux encore par la démarche
beaucoup plus que hardie, à laquelle se porta le duc du Maine,
d'invoquer avec éclat la majorité du roi et les états généraux comme
seuls compétents d'un jugement de cette nature\,; qui n'était pas moins
faire qu'anéantir les lois autant qu'il était en lui, l'autorité du roi
mineur et celle du régent du royaume, et en donner à tous les sujets le
dangereux et très coupable exemple. Enfin ce qui se peut appeler le
premier tocsin de l'éclat dont nous allons parler, fut la fameuse
requête signée de cette prétendue noblesse, dont M. le duc d'Orléans
avait été si longtemps et si volontiers la dupe, ainsi qu'elle-même en
gros, par rapport aux ducs, et présentée au parlement par six seigneurs,
desquels six la plupart portaient sur le front l'attachement au duc du
Maine\,; tocsin, dis-je, de ce qui se tramait, si le régent passait
outre au jugement par lequel le duc et la duchesse du Maine sentaient
bien que la qualité de princes du sang et l'habilité donnée aux bâtards
par le feu roi de succéder à la couronne ne pouvait manquer d'être
anéantie. Depuis le moment de l'arrêt qui prononça cet anéantissement,
et son enregistrement, le Rubicon fut intérieurement passé, et tout
montra sans cesse depuis qu'il ne s'agissait plus que de mettre la main
à l'oeuvre. Et quelle était cette oeuvre\,? La vengeance contre les
juges et les parties, c'est-à-dire contre tout le sang royal légitime
qui était en France\,; détruire le régent\,; revêtir le roi d'Espagne et
le duc du Maine, sous lui, de la régence\,; abolir les renonciations\,;
réveiller les cendres du procès de la branche éteinte de Soissons contre
l'état de celle de Condé, dont M. le Duc m'a souvent dit que
M\textsuperscript{me} du Maine ne s'était point cachée, et dont j'ai
très bien su d'ailleurs qu'elle avait parlé plus d'une fois comme d'une
pièce dont elle prétendait bien s'aider, et qui, à son compte, ne
laissait devant son mari que les infants d'Espagne\,; réussir à tout
cela pour le soulèvement de la noblesse, des parlements, par les
ressorts constitutionnaires\,; introduire les forces d'Espagne, en
soulevant tout le royaume, au moins par mer\,; sûrs de la Bretagne par
l'idée flatteuse d'états généraux, d'union des parlements, et des autres
tribunaux par les cris excités contre l'administration des finances et
contre les moeurs du régent, et en dernier lieu en tirant tous les
avantages possibles de sa mésintelligence avec l'Espagne, et tout cela
fortifié de plusieurs gens considérables, de l'affectation si follement
et si publiquement marquée du maréchal de Villeroy par ses éclatantes
précautions contre le poison, de tout craindre et sans cesse pour la vie
du roi, par un premier président du parlement de Paris, tout à eux et
parfaitement sans âme, et par l'affolement de sa compagnie, de se
prétendre les tuteurs des rois, irritée contre le régent, et brûlante de
domination et de vengeance, par la Bretagne, infatuée du rétablissement
de ses anciens privilèges et de l'honneur de rendre la liberté à toute
la France en recevant les troupes d'Espagne dans ses ports, et leur
servant de places d'armes, d'entrepôt et de magasins\,; mais les
instruments à faire réussir de si beaux projets ne répondirent pas à
leur importance ni à leur étendue. L'étonnement fut grand quand on vit
des chefs d'entreprise si risibles, et les personnages du complot si
dignes de mépris.

Le lendemain de l'arrivée du courrier de Poitiers à l'abbé Dubois, le
prince de Cellamare, averti de son côté d'un événement fâcheux, mais qui
se flattait encore que la compagnie du banquier banqueroutier avait pu
être la cause de l'arrêt des deux jeunes voyageurs et de l'enlèvement de
leurs papiers, cacha son inquiétude sous une apparence fort tranquille,
et alla à une heure après-midi chez M. Le Blanc redemander un paquet de
lettres qu'il leur avait donné par l'occasion de leur retour en Espagne
et munis de passeports du roi. Le Blanc, qui avait sa leçon faite de
plus d'une façon par l'abbé Dubois qu'il avait vu le matin chez lui, et
après de M. le duc d'Orléans, qu'ils avaient vu ensemble, sur la
conduite à tenir dans les divers cas qui étaient possibles à l'égard de
l'ambassadeur, lui répondit que le paquet avait été vu, qu'il y avait
des choses importantes, et que, loin de lui être rendu, il avait ordre
de le remener lui-même en son hôtel avec M. l'abbé Dubois, qui, averti à
l'instant de l'arrivée de Cellamare chez Le Blanc, y était promptement
accouru. Ils le firent donc monter dans le carrosse de M. Le Blanc, et y
entrèrent avec lui. L'ambassadeur, qui sentit bien qu'un pareil
compliment ne se hasardait pas sans s'être précautionné sur l'exécution,
ne fit aucune difficulté, et ne perdit pas un moment de sang-froid et
d'air de tranquillité, pendant les trois heures au moins qu'ils
passèrent chez lui à fouiller tous ses bureaux et ses cassettes et
séparer les papiers qu'ils voulurent, en homme qui ne craint rien et qui
est assuré dans sa conduite. Il traita toujours M. Le Blanc fort
civilement\,; pour l'abbé Dubois, avec qui il sentit bien qu'il n'avait
rien à ménager, et que tout son complot était découvert, il affecta de
le traiter avec le dernier mépris, jusque-là que, Le Blanc se mettant
après une petite cassette\,: «\,Monsieur Le Blanc, monsieur Le Blanc,
laissez cela, lui dit-il, cela n'est pas pour vous\,; cela est bon pour
l'abbé Dubois,\,» qui était là présent\,; puis, en le regardant, il
ajouta\,: «\,Il a été maquereau toute sa vie, ce ne sont là dedans que
lettres de femmes.\,» L'abbé se mit à rire, n'osant pas se fâcher. Ce
fut apparemment un bon mot que Cellamare voulut lâcher. Il était vieux
déjà, il le paraissait encore plus que son âge. Il avait beaucoup
d'esprit, de savoir et de capacité, et tout cela tourné au solide, nulle
sorte de débauche, et toute sa galanterie n'était que pour le commerce
du grand monde, pénétrer ce qu'il voulait savoir, faire et entretenir
des partisans au roi d'Espagne et semer sans imprudence le
mécontentement du régent\,; c'était donc là uniquement ce qui
l'engageait à se mêler avec choix dans les meilleures compagnies. Du
reste, fort retiré chez lui à lire ou à travailler. Au moment de son
arrivée chez lui avec ses deux acolytes, un détachement de mousquetaires
s'empara des portes et de la maison.

Quand tout fut visité, le scellé du roi et le cachet de l'ambassadeur
furent mis sur tous les bureaux et les cassettes qui renfermaient des
papiers\footnote{Voy. dans l'ouvrage de Lemontey (\emph{ibid}.,
  p.~220-221) la liste des pièces qui furent saisies chez Cellamare\,;
  elle a été dressée d'après les documents conservés aux archives des
  affaires étrangères.}. L'abbé Dubois et Le Blanc s'en allèrent
ensemble rendre compte au régent, et laissèrent auprès de l'ambassadeur
les mousquetaires pour le garder lui et ses domestiques, et du Libois,
un des gentilshommes ordinaires du roi, comme il se pratique toujours
d'en laisser un auprès des ambassadeurs dans les fâcheuses occasions.
Celui-ci avait beaucoup d'esprit et d'entendement, et avait presque
toujours été choisi pour ces tristes commissions.

J'appris ce matin chez moi la capture de Poitiers, sans avoir rien su de
ceux qui y furent arrêtés. Comme j'étais à table, il vint un garçon
rouge me dire de la part de M. le duc d'Orléans de me trouver à quatre
heures aux Tuileries pour le conseil de régence. Comme ce n'était pas
jour d'en tenir, je lui demandai ce qu'il y avait donc de nouveau. À son
tour, il fut surpris de mon ignorance, et m'apprit que l'ambassadeur
d'Espagne était arrêté. Dès que j'eus mangé un morceau, je quittai la
compagnie, et m'en allai au Palais-Royal, où j'appris de M. le duc
d'Orléans tout ce que je viens de raconter. Je lui parlai des papiers\,;
il me dit que l'abbé Dubois les avait\,; qu'il n'avait pas eu le temps
encore de les examiner, ni de lui en rendre compte\,; qu'il allait
seulement montrer quelque chose au conseil de régence, qu'il avait voulu
instruire lui-même sur cet éclat. Ces propos et divers autres aussi
vagues gagnèrent le temps, et je m'en allai l'attendre aux Tuileries.
J'y trouvai de l'étonnement sur plusieurs visages, quelques petits
pelotons de deux, de trois et de quatre ensemble\,; en général, des gens
frappés de l'éclat de l'arrêt d'un ambassadeur d'Espagne, et peu enclins
à l'approuver.

M. le duc d'Orléans arriva peu après. Il avait, mieux qu'homme que j'aie
connu, le talent de la parole, et, sans avoir besoin d'aucune
préparation, il disait ce qu'il voulait, ni plus ni moins\,; les termes
étaient justes et précis, une grâce naturelle les accompagnait, avec
l'air de ce qu'il était, toujours mêlé d'un air de politesse. Il ouvrit
le conseil par un discours sur les personnes et les papiers arrêtés à
Poitiers, qui avaient découvert une conspiration fort dangereuse contre
l'État, prête à éclater, dont l'ambassadeur d'Espagne était le principal
promoteur. Son Altesse Royale allégua les raisons pressantes qu'il avait
eues de s'assurer de la personne de cet ambassadeur, de faire visiter
ses papiers, de le faire garder par du Libois et par les mousquetaires.
Il s'étendit à montrer que la protection du droit des gens ne s'étendait
pas jusqu'aux conspirations\,; que les ambassadeurs s'en rendaient
indignes quand ils entraient, encore plus quand ils excitaient des
complots contre l'État où ils résidaient. Il cita plusieurs exemples
d'ambassadeurs arrêtés pour moins. Il expliqua les ordres qu'il avait
donnés pour informer de sa part tous les ministres étrangers qui étaient
à Paris, de cette affaire, et il ordonna à l'abbé Dubois de rendre
compte au conseil de ce qu'il avait fait chez Cellamare, de quelle façon
cela s'était passé avec cet ambassadeur, et de lire ensuite au conseil
deux lettres de ce ministre au cardinal Albéroni, trouvées dans les
papiers apportés de Poitiers.

L'abbé Dubois balbutia un récit court et mal en ordre de ce qu'il avait
fait chez l'ambassadeur, et s'étendit davantage sur l'importance de la
découverte et sur celle de ce qu'on voyait déjà de la conspiration. Les
deux lettres qu'il lut ne laissèrent point douter que Cellamare ne fût à
la tête de cette affaire, et qu'Albéroni n'y entrât aussi avant que lui.
On fut aussi très scandalisé des expressions de ces lettres sur M. le
duc d'Orléans, qui n'étaient ménagées ni en choses ni en termes.

Ce prince reprit la parole pour témoigner avec beaucoup de modération
qu'il ne soupçonnait point le roi ni la reine d'Espagne d'entrer dans
une affaire de cette nature\,; qu'il ne l'attribuait qu'à la passion
d'Albéroni et à celle de l'ambassadeur pour lui plaire, et qu'il en
demanderait justice à Leurs Majestés Catholiques. Il remontra ensuite
l'importance de ne rien négliger pour l'entier éclaircissement d'une
affaire si capitale au repos et à la tranquillité du royaume, et finit
par dire que, jusqu'à ce qu'il en sût davantage, il ne voulait nommer
personne de ceux qui pouvaient y être entrés. Tout ce discours fut fort
applaudi, et je crois qu'il s'en trouva dans la compagnie qui se
sentirent bien à leur aise quand ils entendirent que le régent ne
voulait nommer ni laisser répandre de soupçons sur personne jusqu'à ce
qu'il fût plus éclairci.

Néanmoins, dès le lendemain matin, samedi dix décembre, Pompadour fut
arrêté à huit heures, comme il se levait, et conduit à la Bastille.
M\textsuperscript{me} de Pompadour et M\textsuperscript{me} de
Courcillon, sa fille, et belle-fille de Dangeau, allèrent au
Palais-Royal. M. le duc d'Orléans leur fit faire excuse de ce qu'il ne
pouvait leur parler par le maréchal de Villeroy, qui était avec lui,
avec des compliments vagues qui ne signifiaient rien.

Pompadour était un grand homme, triste et froid, qui avait passé avec sa
femme, fille du maréchal de Navailles, la plus grande partie de sa vie
sans cour et sans servir, dans une grande obscurité à Paris, où il
n'avait pas laissé de se ruiner, et qui n'avait reparu dans le monde que
par le mariage de sa fille, qui était une beauté et fort jeune, avec
Courcillon, qui y trouvait une alliance qui l'honorait fort, et des
biens à venir, dont le père et la mère n'avaient pu dissiper les fonds.
Par ce mariage, ils entrèrent à la cour. Dangeau donna à Pompadour sa
place de menin de Monseigneur, qui ne lui servait à rien, et Pompadour
vécut à la cour sans être de rien et sans considération aucune. Il avait
de l'esprit et de la lecture\,; mais il n'en sut jamais rien faire. Ses
conseils et son crédit ne pouvaient fortifier un parti, et chacun rit et
s'étonna qu'il fût entré dans celui-ci. Sa femme avait le petit manége.
À l'appui de M\textsuperscript{me} de Dangeau et de la décoration de la
duchesse d'Elboeuf, sa soeur, elle fit une cour basse à
M\textsuperscript{me} de Maintenon, et à M\textsuperscript{me} des
Ursins quand elle fit ici ce voyage triomphant dont il a été parlé, et
se fit ainsi gouvernante des enfants de lime la duchesse de Berry. Sa
fonction ne fut que d'un moment, et la mort du roi la fit retomber et
son mari dans le néant, dont le mariage de leur fille les avait tirés.

Ce même samedi 10 décembre, Saint-Geniez fut aussi arrêté et conduit à
la Bastille. Saint-Geniez était une espèce d'aventurier, bâtard de
Saint-Geniez, mort en 1685, lieutenant général, gouverneur de
Saint-Omer, et frère du maréchal de Navailles, mort en 1684. Il avait eu
deux fils d'A. Drouart, morte en 1671, qu'il fit légitimer, en 1678, par
lettres patentes du roi enregistrées, et que par son testament il
appelle ses enfants naturels et légitimés. Le cadet eut une abbaye\,; je
ne sais ce qu'il est devenu. Celui dont il s'agit ici servit toute sa
vie avec beaucoup de valeur, et s'attacha fort au maréchal de Villeroy,
qui lui fit donner un brevet de colonel de dragons en 1704. Il épousa,
en 1695, une fille de Rolland, fermier général, manière d'aventurière
aussi et grande danseuse. En 1717, il s'avisa de vouloir être légitime,
et demanda, par un placet au roi et au régent, que les enregistrements
de ses lettres de légitimation, obtenues par son père, fussent rayés. On
se moqua de lui. C'était un bon garçon, sans cervelle, uniquement propre
à un coup de main. Il n'eut que deux filles. Je ne sais ce que tout cela
est devenu depuis la fin de l'affaire qui me fait parler de lui.

Le même jour, les députés du parlement vinrent au Palais-Royal demander
la liberté du président Blamont. Le régent leur répondit qu'il avait
fait arrêter l'ambassadeur d'Espagne pour une conspiration, qu'il le
renvoyait à Madrid, et qu'il en demandait justice au roi d'Espagne,
qu'il voulait être éclairci sur ceux qui y étaient entrés, et que, pour
le présent, il ne pouvait répondre à ce qu'ils demandaient. Le moment de
cette députation fut trouvé mal choisi.

D'Aydie, veuf de la soeur de Rion, et de même nom que lui, et qui
logeait à Luxembourg, disparut. Un abbé Brigault, fort dans le bas
étage, qui était en fuite, fut pris à Nemours et conduit à la Bastille.
Magny, introducteur des ambassadeurs, prit aussi la fuite. Sa charge fut
donnée à vendre à Foucault, son père, conseiller d'État, chef du conseil
de Madame. On a vu ailleurs que ce Magny n'était qu'un misérable fou.
Ces trois hommes n'étaient pas pour fortifier beaucoup un parti. À la
naissance près d'Aydie, on ne comprenait pas ce qu'un parti en pouvait
faire.

Le mardi 13 décembre, jour que tous les ministres étrangers allaient au
Palais-Royal, et qui était le premier mardi d'après la détention de
Cellamare, ils y furent tous, ambassadeurs et autres. Aucun ne fit de
plaintes de ce qui était arrivé\,; on leur donna à tous la copie des
deux lettres qui avaient été lues au conseil. L'après-dînée, on fit
monter l'ambassadeur d'Espagne dans un carrosse avec du Libois, un
capitaine de cavalerie et un capitaine de dragons, choisis pour le
conduire à Blois, et y rester auprès de lui jusqu'à ce qu'on eût
nouvelle de l'arrivée du duc de Saint-Aignan en France. Quelques jours
après, Sandraski, brigadier de cavalerie et colonel de hussards, Seret,
autre colonel de hussards, et quelques autres moindres officiers, furent
conduits à la Bastille.

\hypertarget{chapitre-v.}{%
\chapter{CHAPITRE V.}\label{chapitre-v.}}

1718

~

{\textsc{Évêques et cardinaux en débat sur les carreaux à la chapelle du
roi, pour le sacre de Massillon, évêque de Clermont, qui s'y fit devant
le roi, qui lui donna trente mille livres de gratification, en attendant
une abbaye.}} {\textsc{- Le parlement refuse d'enregistrer la banque
royale.}} {\textsc{- Le régent s'en passe, le méprise, la publie et
l'établit.}} {\textsc{- Menille à la Bastille.}} {\textsc{- Cellamare
écrit très inutilement aux ministres étrangers résidant à Paris.}}
{\textsc{- Conseil secret au Palais-Royal, qui se réduit après à M. le
Duc et à moi, à qui le régent confie que le duc et la duchesse du Maine
sont des plus avant dans la conspiration, et qui délibère avec nous ce
qu'il doit faire.}} {\textsc{- Nous concluons tous trois à les faire
arrêter\,; conduire M. du Maine à Dourlens, et M\textsuperscript{me} du
Maine au château de Dijon, bien gardés et resserrés.}} {\textsc{- M. le
Duc dispute un peu sur Dijon et se rend.}} {\textsc{- M. et
M\textsuperscript{me} du Maine et leurs affidés ont tout le temps de
mettre leurs papiers à couvert et en profitent.}} {\textsc{- Perfidie de
l'abbé Dubois.}} {\textsc{- Conseil secret entre M. le duc d'Orléans, M.
le Duc, l'abbé Dubois, Le Blanc et moi, où tout est résolu pour le
lendemain.}} {\textsc{- Le duc du Maine arrêté à Sceaux par La
Billarderie, lieutenant des gardes du corps, et conduit dans la
citadelle de Dourlens.}} {\textsc{- M\textsuperscript{me} la duchesse du
Maine arrêtée par le duc d'Ancenis, capitaine des gardes du corps, et
conduite au château de Dijon.}} {\textsc{- Enfants du duc du Maine
exilés.}} {\textsc{- Cardinal de Polignac exilé à Anchin.}} {\textsc{-
Un gentilhomme ordinaire du roi est mis auprès de lui.}} {\textsc{-
Davisard et autres gens attachés ou domestiques du duc et de la duchesse
du Maine, mis à la Bastille.}} {\textsc{- Excellente et nette conduite
du comte de Toulouse.}} {\textsc{- Le duc de Saint-Aignan se retire
habilement d'Espagne, où on voulait le retenir.}} {\textsc{- Mort du
comte de Solre, sans nulle prétention toute sa vie.}} {\textsc{- Son
fils et sa belle-fille s'en figurent de toutes nouvelles et inutiles.}}
{\textsc{- Mort de Nointel, conseiller d'État, et du vieux Heudicourt.}}
{\textsc{- Belle-Ile\,; sa famille\,; son île.}} {\textsc{- Caractère de
Belle-Ile.}} {\textsc{- Caractère du chevalier de Belle-Ile.}}
{\textsc{- Union des deux frères Belle-Ile\,; leur conduite
domestique\,; leur liaison avec moi.}} {\textsc{- L'aîné commence à
pointer et fait avec le roi l'échange de Belle-Ile.}} {\textsc{- Raison
de s'être étendu sur les deux frères Belle-Ile.}}

~

Deux incidents arrivés le vendredi 16 décembre méritent d'être
rapportés, et n'interrompront pas longtemps l'affaire de la
conspiration. Le premier fut ecclésiastique\,: le P. Massillon, de
l'Oratoire, excellent prédicateur, avait reçu ses bulles pour l'évêché
de Clermont, auquel le roi l'avait nommé. Il avait fort plu à la cour
par des sermons à la portée de l'âge et de l'état du roi, qu'il avait
précédemment prêchés à la chapelle. Le roi eut curiosité de voir son
sacre. Il fut dit que, pour sa commodité, il se ferait dans la chapelle.
Les évêques, toujours très attentifs à usurper, tirèrent sur le temps et
déclarèrent que pas un n'assisterait à ce sacre s'il s'y trouvait des
cardinaux. Il n'y avait point d'exemple de sacre dans la chapelle du
roi, très peu ailleurs, où le roi ou la reine eussent été\,; et lorsque
cela était arrivé, c'était dans des tribunes. La difficulté des évêques
était qu'ils n'osaient prétendre des carreaux dans la chapelle, et que,
n'en ayant point, ils n'en voulaient pas voir aux cardinaux. Mais la
difficulté était ridicule. Les évêques se trouvent continuellement à la
messe du roi et à celle de la reine, et à toutes les cérémonies et
offices qui se font à la chapelle en présence de Leurs Majestés. Ils n'y
ont jamais eu ni prétendu de carreau, et y en ont toujours vu aux
cardinaux, sans parler des ducs et des duchesses. Quelle différence donc
d'un sacre dans la chapelle, ou de la simple messe du roi, ou d'une
autre cérémonie\,? C'est qu'ils sentaient leurs forces, la faiblesse du
régent, la situation actuelle des cardinaux, et qu'ils cherchaient à se
fabriquer un titre de leur ridicule difficulté.

Le cardinal de Noailles était éreinté par l'appel qu'il venait de
publier, et le grand aumônier lui disputait de faire porter sa croix
devant lui dans la chapelle\,; il ne pouvait donc songer à y aller.
Polignac était encore moins en état d'y paraître et de disputer, comme
on le verra incontinent\,; Rohan et Bissy en étaient à faire leur cour
aux évêques pour les attirer à faire tous les pas de fureur qui leur
convenaient dans la circonstance toute fraîche de la déclaration de
l'appel du cardinal de Noailles et de plusieurs évêques et corps, etc.,
en même temps. Bissy, dans la foule qu'il travaillait à exciter, et qui
n'espérait de succès à Rome que par celui qu'il opérerait ici par les
évêques de France, se trouva, heureusement pour leur prétention, le seul
des cardinaux qui pût se trouver à ce sacre.

Il le leur sacrifia d'autant plus volontiers, que cette complaisance de
ne s'y point trouver n'altérait point la possession des cardinaux, et ne
donnait aucun titre aux évêques, qui, contents de ne point voir de
carreaux dans la chapelle, parce que le roi, pour voir mieux, y devait
être dans sa tribune, qui contient aisément toute sa suite, ne purent
trouver mauvais qu'il y eût là des carreaux, qui ne se pouvaient voir
d'en bas, et par conséquent que le cardinal y eût le sien auprès du roi,
comme le grand chambellan, le premier gentilhomme de la chambre, le
gouverneur du roi et son capitaine des gardes, tous ducs, etc.\,; pour
le cardinal de Gesvres, c'était avec de l'esprit, du savoir et une rage
d'être cardinal, qui avait occupé toute sa vie un hypocondriaque de sa
santé, qui, dès qu'il fut parvenu à la pourpre, se renferma presque
aussitôt et ne se trouva plus à rien. Mais je préviens sa promotion, qui
n'arriva que dans l'année suivante. Ainsi, le P. Massillon fut sacré
dans la chapelle par M. de Fréjus, précepteur du roi, assisté des
évêques de Nantes, premier aumônier de M. le duc d'Orléans, et de
Vannes. Le roi était dans sa tribune, accompagné de sa suite, parmi
laquelle était le cardinal de Rohan, douze ou quinze évêques en bas, et
point de cardinaux\,; la cérémonie s'en fit le 21 décembre. Le nouvel
évêque eut dix mille écus de gratification en attendant une abbaye.

L'autre incident fut d'une autre espèce. Quelque abattu que fût le
parlement du dernier lit de justice, il était encore plus irrité et
commençait à reprendre ses esprits. Le fracas de l'arrêt de
l'ambassadeur d'Espagne, le mouvement des emprisonnements qui suivirent
de si près, lui donnèrent du courage pour résister à l'enregistrement de
la banque royale, d'autant plus qu'elle était fort mal reçue du public.
Le premier président alla donc rendre compte au régent de la difficulté
que sa compagnie apportait à cet enregistrement. M. le duc d'Orléans
méprisa l'un et l'autre, et, à peu de jours de là, fit publier la banque
royale, l'établit très bien, et montra au parlement qu'il savait se
passer de son enregistrement.

Le samedi 17, le garde des sceaux alla à la Bastille\,; il y dîna même
et y demeura longtemps. Le soir, Menille y fut conduit. Il était ami
particulier de l'abbé Brigault, et avait été longtemps gentilhomme
servant du feu roi. Son esprit ni sa société n'était pas au-dessus de sa
charge. On haussait les épaules de pareils conjurés. Cellamare, avant
partir, avait écrit aux ambassadeurs et autres ministres étrangers, pour
les intéresser dans sa détention. Ses lettres leur furent rendues\,; pas
un d'eux ne s'en émut ni ne fit le moindre pas en conséquence, pas même
ce boute-feu de Bentivoglio, trop occupé des mines à charger sous les
pieds du cardinal de Noailles et de tous ceux qui venaient d'appeler en
même temps.

Le dimanche 25 décembre, jour de Noël, M. le duc d'Orléans me manda de
me trouver l'après-dînée chez lui, sur les quatre heures. M. le Duc, le
duc d'Antin, le garde des sceaux, Torcy et l'abbé Dubois s'y trouvèrent.
On y discuta plusieurs choses sur Cellamare et son voyage, sur les
mesures pour éviter les plaintes des ministres étrangers, qui n'en
avaient aucune envie\,; sur la manière de demander au roi d'Espagne une
justice qu'on n'en espérait point\,; enfin sur la manière de passer à
côté de l'enregistrement du parlement, et d'établir sûrement sans cela
la banque royale. Tout cela s'agita avec une tranquillité et une liberté
d'esprit de la part du régent, qui ne me laissa pas soupçonner qu'il se
pût agir d'autre chose. Ce petit conseil dura assez longtemps. Quand il
fut fini, chacun s'en alla. Comme je m'ébranlais pour sortir comme les
autres, M. le duc d'Orléans m'appela\,; cependant les autres sortirent,
et je me trouvai seul avec M le duc d'Orléans et M. le Duc. Nous nous
rassîmes. C'était dans le petit cabinet d'hiver, au bout de la petite
galerie. Après un moment de silence, il me dit de regarder s'il n'était
demeuré personne dans cette petite galerie, et si la porte du bout, par
où on y entrait de l'appartement où il couchait, était fermée. J'y allai
voir\,; elle l'était, et personne dans la galerie.

Cela constaté, M. le duc d'Orléans nous dit que nous ne serions pas
surpris d'apprendre que M. et M\textsuperscript{me} du Maine se
trouvaient tout de leur long dans l'affaire de l'ambassadeur d'Espagne,
qu'il en avait les preuves par écrit, qu'il ne s'agissait pas de moins
dans leur projet que de ce que j'en ai expliqué plus haut. Il ajouta
qu'il avait bien défendu au garde des sceaux, à l'abbé Dubois et à Le
Blanc, qui seuls le savaient, de faire le plus léger semblant de cette
connaissance, nous recommanda à tous deux le même secret et la même
précaution, et ajouta qu'il avait voulu, avant de se déterminer à rien,
consulter avec M. le Duc et moi seuls le parti qu'il avait à prendre. Je
pensai bien en moi-même que, puisque ces trois autres hommes savaient la
chose, il n'était pas sans en avoir raisonné avec eux, et peut-être déjà
pris son parti avec l'abbé Dubois\,; qu'il voulait flatter M. le Duc de
la confiance et le mettre de moitié de tout ce qu'il ferait là-dessus\,;
à mon égard, débattre réellement avec moi ce qu'il y avait à faire pour
ne s'en pas tenir à ces trois autres seuls, et parce qu'il avait
toujours accoutumé, comme on l'a toujours vu ici, de me faire part des
choses secrètes les plus importantes qui demandaient des partis instants
à prendre et qui l'embarrassaient le plus. M. le Duc sur-le-champ alla
droit au fait, et dit qu'il fallait les arrêter tous deux et les mettre
en lieu dont on ne pût rien craindre. J'appuyai cet avis et les
périlleux inconvénients de ne le pas exécuter incessamment, tant pour
étourdir et mettre en confusion tout le complot en lui ôtant ses chefs,
tels que ces deux-là et Cellamare déjà arrêté et parti, et se parer des
coups précipités et de désespoir qu'il y avait lieu de craindre de gens
si appuyés qui se voyaient découverts, et qui, en quelque état que
fussent leurs mesures, sentaient qu'on en arrêterait et qu'on {[}en{]}
découvrirait tous les jours, et que conséquemment ils n'avaient pas un
instant à perdre pour exécuter tout ce qui pouvait être en leur
possibilité, et tenter même l'impossible qui réussit quelquefois et
qu'il faut toujours hasarder dans des cas désespérés, tels que celui
dans lequel ils se rencontraient.

M. le duc d'Orléans trouva que ce serait en effet tout le meilleur
parti, mais il insista sur la qualité de M\textsuperscript{me} du Maine,
moins je pense en effet, que pour faire parler le fils de son frère. Ce
doute réussit fort bien par la haine qu'il portait personnellement à sa
tante et à son mari, et qu'il faut avouer que tous deux avaient
largement méritée, et par la nature aussi de l'affaire qui allait à
bouleverser l'État, et les renonciations qui délivraient sa branche à
son tour de l'aînesse de celle d'Espagne. M. le Duc répondit à
l'objection proposée que ce serait à lui à la faire, mais, que loin de
trouver qu'elle dût arrêter, c'était une raison de plus pour se hâter
d'exécuter\,; que ce ne serait pas la première ni peut-être la vingtième
fois qu'on eût arrêté des princes du sang\,; que plus ils étaient grands
et naturellement attachés à l'État par leur naissance, plus ils étaient
coupables quand ils s'en prévalaient pour le troubler, et qu'il n'y
avait à son sens rien de plus pressé que d'étourdir leurs complices par
un coup de cet éclat, et les priver subitement de toutes les machines
que la rage et l'esprit du mari et de la femme savaient remuer. Je louai
fort la droiture, l'attachement et le grand sens de l'avis de M. le
Duc\,; je l'étendis\,; j'insistai sur le courage et la fermeté que le
régent devait montrer dans une occasion si critique, et où on en voulait
à lui si personnellement, et sur la nécessité d'effrayer par là toute
cette pernicieuse cabale, de leur ôter leur grand appui et de nom et
d'intrigue et de moyens, et les rendre par ce grand coup pour ainsi dire
orphelins, sans chefs et sans point de réunion ni de subordination,
avant qu'ils eussent le temps d'aviser aux remèdes, si ce mal leur
arrivait comme ils le devaient désormais craindre continuellement. M. le
duc d'Orléans regarda M. le Duc qui reprit la parole et insista de
nouveau sur son avis et le mien. Le régent alors se rendit et n'y eut
pas de peine.

Après quelques propos sur cette résolution, on agita où on les gîterait.
La Bastille et Vincennes ne parurent pas convenables, il fallait éviter
tentation si prochaine aux partisans qu'ils avaient dans Paris, aux
humeurs du parlement, aux manéges qu'y ferait le premier président. On
discuta des places, car les arrêter, et les séparer l'un de l'autre, fut
résolu tout à la fois\,; il s'agit d'abord du gîte du duc du Maine.
Entre les lieux agités, M. le duc d'Orléans parla de Dourlens. Je saisis
ce nom, j'alléguai que Charost et son fils en étaient gouverneurs,
qu'ils l'étaient de Calais, place peu éloignée de l'autre et avaient
l'unique lieutenance générale de Picardie, que c'étaient des hommes
d'une race fidèle, et personnellement d'une probité, d'une vertu, d'un
attachement à l'État dont je ne craignais pas de répondre, et Charost de
tout temps mon ami particulier. Sur ce propos, il fut convenu d'envoyer
le duc du Maine à Dourlens, et de l'y tenir serré, et bien étroitement
gardé.

Ensuite on passa au gîte de M\textsuperscript{me} du Maine. Je
représentai que celui-là était bien plus délicat à choisir par la
qualité, le sexe et l'humeur de celle dont il s'agissait, propre à tout
entreprendre pour se sauver et pour faire rage sans crainte, et par son
courage et sa fougue naturelle, et par ne rien craindre pour elle-même
par son sexe et sa naissance, au lieu que son mari, si dangereux en
dessous, si méprisable à découvert, tomberait dans le dernier abattement
et ne branlerait pas dans sa prison où il tremblerait de tout son corps
dans la frayeur continuelle de l'échafaud. Divers lieux discutés, M. le
duc d'Orléans se mit à sourire, à regarder M. le Duc et à lui dire qu'il
fallait bien qu'il l'aidât, qu'il se prêtât de son côté, que c'était
l'affaire de l'État et guère moins la sienne que celle de lui régent, et
tout de suite lui proposa le château de Dijon. M. le Duc trouva la
proposition étrange, convint qu'il fallait mettre M\textsuperscript{me}
du Maine en lieu extrêmement sûr, mais que de le faire geôlier de sa
tante, cela ne se pouvait accepter. Toutefois il le dit aussi en
souriant, et, par sa contenance, donna lieu au régent d'insister. M. le
Duc se défendit, je ne disais mot, et je regardais de tous mes yeux. À
la fin M. le Duc me demanda s'il n'avait pas raison. Je me mis à sourire
aussi et je répondis que je ne pouvais nier qu'il n'eût raison ni moins
encore que M. le duc d'Orléans ne l'eût et plus grande et meilleure.
J'avais fort pensé et pesé pendant la petite dispute, et je trouvai un
grand avantage pour M. le duc d'Orléans de rendre M. le Duc son
compersonnier\footnote{Vieux mot employé plusieurs fois par Saint-Simon
  dans le sens d'associé.} dans le fait de la prison de
M\textsuperscript{me} du Maine, et par conséquent du duc du Maine aussi,
et elle en lieu plus sûr et plus sans espérance de fuite et de ressource
qu'aucun, dans le milieu du gouvernement de M. le Duc, et dans une place
de son entière dépendance\,; je ne dissimulerai pas, non plus, un peu de
nature, et de trouver la rocambole plaisante après tous les élans du
procès, tant de la succession de M. le Prince que pour la qualité de
prince du sang et pour l'habilité de succéder à la couronne, de voir
cette femme qui avait tant osé assurer qu'elle renverserait l'État et
mettrait le feu partout pour conserver ces avantages si étrangement
acquis, de la voir, dis-je, rager entre quatre murailles de la dition de
M. le Duc\footnote{Dans un lieu soumis à l'autorité de M. le Duc.}. Il
hésita longtemps à tout ce que M. le duc d'Orléans et moi pûmes lui
dire, à quoi la bienséance eut plus de part après tout ce qui s'était
passé entre eux, que la vraie répugnance. Aussi se laissa-t-il vaincre à
la fin, et consentit à l'étroite prison de sa chère tante dans la prison
de Dijon\,; tout cela résolu, et pour l'exécuter en bref, nous nous
séparâmes.

Le lundi et mardi suivants, 26 et 27 décembre, se passèrent à prendre
les mesures et donner les ordres nécessaires, avec tout le secret qu'il
se put\,; mais M. et M\textsuperscript{me} du Maine, qui voyaient
l'ambassadeur d'Espagne conduit à Blois, ses paquets pris, ses papiers
visités et bien des gens arrêtés, n'étaient pas sans appréhension de
l'être, et avaient eu tout le loisir de donner à leurs papiers tout
l'ordre qu'ils jugèrent à propos. Avec cette précaution leur crainte
diminua, quoi qu'il pût arriver. L'abbé en savait autant sur leur compte
lorsqu'il reçut les papiers de Blois qu'il montra en avoir appris depuis
par l'examen de ces mêmes papiers, et s'il avait été droit en besogne il
n'eût pas différé de les montrer au régent ni d'arrêter M. et
M\textsuperscript{me} du Maine au même instant que l'ambassadeur
d'Espagne au plus tard, et par cette diligence il eût prévenu la leur et
eût saisi leurs papiers importants\,; mais ce n'était pas son intérêt
particulier de servir si bien l'État ni son maître, et le scélérat ne
songea jamais qu'à soi.

Le mercredi 28 décembre, je fus mandé au Palais-Royal, pour
l'après-dînée, par M. le duc d'Orléans, avec M. le Duc, l'abbé Dubois et
Le Blanc, dans le petit cabinet d'hiver. C'était pour prendre les
dernières mesures et résumer toutes celles qui avaient été prises.
Pendant que nous conférions, le duc du Maine vint de Sceaux voir
M\textsuperscript{me} la duchesse d'Orléans au Palais-Royal, et, au bout
d'une heure de conversation avec elle, s'en retourna à Sceaux.
M\textsuperscript{me} du Maine était demeurée depuis quelques jours à
Paris, dans une maison assez médiocre de la rue Saint-Honoré, qu'ils
avaient louée. C'était le centre de Paris. Elle était là aux aguets et
le bureau d'adresse des siens, à quoi le peureux époux n'avait osé se
confier. La conférence chez M. le duc d'Orléans fut assez longue. Tout y
fut compassé et définitivement réglé pour l'exécution du lendemain. Tous
les cas possibles prévus et les ordres convenus jusque sur les
bagatelles, il arriva pourtant que les ordres donnés au régiment des
gardes et aux deux compagnies des mousquetaires, qui pourtant ne
branlèrent pas de leurs quartiers ni de leurs hôtels, ne laissèrent pas
de transpirer sur le soir, et de faire juger à ce qui en fut instruit
qu'il se méditait quelque chose de considérable. En sortant du cabinet,
je convins avec Le Blanc qu'aussitôt que le coup serait fait, il
enverrait simplement un laquais savoir de mes nouvelles.

Le lendemain, sur les dix heures du matin, ayant fait filer des gardes
du corps tout à l'entour de Sceaux sans bruit et sans paraître, La
Billarderie, lieutenant des gardes du corps, y alla et arrêta le duc du
Maine, comme il sortait d'entendre la messe dans sa chapelle, et fort
respectueusement le pria de ne pas rentrer chez lui, et de monter tout
de suite dans un carrosse qui l'avait amené. M. du Maine, qui avait mis
bon ordre qu'on ne trouvât rien chez lui ni chez pas un de ses gens, et
qui était seul à Sceaux avec ses domestiques, ne fit pas la moindre
résistance. Il répondit qu'il s'attendait depuis quelques jours à ce
compliment, et monta sur-le-champ dans le carrosse. La Billarderie s'y
mit à côté de lui, et sur le devant un exempt des gardes du corps et
Favancourt, brigadier dans la première compagnie des mousquetaires,
destiné à le garder dans sa prison.

Comme ils ne parurent devant le duc du Maine que dans le moment qu'ils
montèrent en carrosse, le duc du Maine parut surpris et ému de voir
Favancourt. Il ne l'aurait pas été de l'exempt des gardes\,; mais la vue
de l'autre l'abattit. Il demanda à La Billarderie ce que cela voulait
dire, qui alors ne put lui dissimuler que Favancourt avait ordre de
l'accompagner et de rester avec lui dans le lieu où ils allaient.
Favancourt prit ce moment pour faire son compliment comme il put, auquel
le duc du Maine ne répondit presque rien, mais d'une manière civile et
craintive. Ces propos les conduisirent au bout de l'avenue de Sceaux, où
les gardes du corps parurent. L'aspect en fit changer de couleur au duc
du Maine.

Le silence fut peu interrompu dans le carrosse. Par-ci, par-là M. du
Maine disait qu'il était très innocent des soupçons qu'on avait contre
lui, qu'il était très attaché au roi, qu'il ne l'était pas moins à M. le
duc d'Orléans, qui ne pourrait s'empêcher de le reconnaître, et qu'il
était bien malheureux que Son Altesse Royale donnât créance à ses
ennemis, mais sans jamais nommer personne\,: tout cela par hoquets et
parmi force soupirs, de temps en temps des signes de croix et de
marmottages bas comme de prières, et des plongeons de sa part à chaque
église ou à chaque croix par où ils passaient. Il mangea avec eux dans
le carrosse assez peu, tout seul le soir, force précautions à la
couchée. Il ne sut que le lendemain qu'il allait à Dourlens. Il ne
témoigna rien là-dessus. J'ai su toutes ces circonstances et celles de
sa prison après qu'il en fut sorti, par ce même Favancourt que je
connaissais fort, parce que c'était lui qui m'avait appris l'exercice,
et qui était sous-brigadier de la brigade de Crenay, dans la première
compagnie des mousquetaires, dans le temps que j'y étais dans cette même
brigade, et qui m'avait toujours courtisé depuis. M. du Maine eut deux
valets avec lui et fut presque toujours gardé à vue.

Au même instant qu'il fut arrêté, Ancenis, qui venait d'avoir la
survivance de la charge de capitaine des gardes du corps du duc de
Charost, son père, alla arrêter la duchesse du Maine dans sa maison, rue
Saint-Honoré. Un lieutenant et un exempt des gardes du corps à pied, et
une troupe de gardes du corps parurent en même temps et se saisirent de
la maison et des portes. Le compliment du duc d'Ancenis fut aigrement
reçu. M\textsuperscript{me} du Maine voulut prendre des cassettes.
Ancenis s'y opposa. Elle réclama au moins ses pierreries\,: altercation
fort haute d'une part, fort modeste de l'autre\,; mais il fallut céder.
Elle s'emporta contre la violence faite à une personne de son rang, sans
rien dire de trop désobligeant à M. d'Ancenis et sans nommer personne.
Elle différa de partir tant qu'elle put, malgré les instances d'Ancenis,
qui à la fin lui présenta la main, et lui dit poliment, mais fermement,
qu'il fallait partir. Elle trouva à sa porte deux carrosses de remise,
tous deux à six chevaux, dont la vue la scandalisa fort. Il fallut
pourtant y monter. Ancenis se mit à côté d'elle, le lieutenant et
l'exempt des gardes sur le devant, deux femmes de chambre, qu'elle
choisit, avec ses hardes, qu'on visita, dans l'autre carrosse. On prit
le rempart\,; on évita les grandes rues\,: qui que ce soit n'y branla,
dont elle ne put s'empêcher de marquer sa surprise et son dépit, ne jeta
pas une larme, et déclama en général par hoquets contre la violence qui
lui était faite. Elle se plaignit souvent de la rudesse et de
l'indignité de la voiture, et demanda de fois à autre où on la menait.
On se contenta de lui dire qu'elle coucherait à Essonne, sans lui rien
dire de plus. Ses trois gardiens gardèrent un profond silence. On prit à
la couchée toutes les précautions nécessaires. Lorsqu'elle partit le
lendemain, le duc d'Ancenis prit congé d'elle, et la laissa au
lieutenant et à l'exempt des gardes du corps avec des gardes du corps
pour la conduire. Elle lui demanda où on la menait\,: il répondit
simplement\,: «\,à Fontainebleau,\,» et vint rendre compte au régent.
L'inquiétude de M\textsuperscript{me} du Maine augmenta à mesure qu'elle
s'éloignait de Paris\,; mais, quand elle {[}se{]} vit en Bourgogne, et
qu'elle sut enfin qu'on la menait à Dijon, elle déclama beaucoup.

Ce fut bien pis quand il fallut entrer dans le château, et qu'elle s'y
vit prisonnière sous la clef de M. le Duc. La fureur la suffoqua. Elle
dit rage de son neveu, et de l'horreur du choix de ce lieu. Néanmoins,
après ces premiers transports, elle revint à elle, et à comprendre
qu'elle n'était ni en lieu ni en situation de faire tant de l'enragée.
Sa rage extrême se renferma en elle-même, elle n'affecta plus que de
l'indifférence pour tout et une dédaigneuse sécurité. Le lieutenant de
roi du château, absolument à M. le Duc, la tint fort serrée, et la
veilla et ses deux femmes de chambre de fort près. Le prince de Dombes
et le comte d'Eu furent en même temps exilés à Eu, où ils eurent un
gentilhomme ordinaire toujours auprès d'eux, et M\textsuperscript{lle}
du Maine envoyée à Maubuisson.

Son bon ami, le cardinal de Polignac, qu'on crut être de tout avec elle,
eut ordre le même matin de partir sur-le-champ pour son abbaye d'Anchin,
accompagné d'un des gentilshommes ordinaires du roi, qui demeura auprès
de lui tant qu'il fut en Flandre\,; le cardinal partit sur la fin de la
matinée même. Dans le même moment, Davisard, avocat général du parlement
de Toulouse, qui s'était signalé par ses factums pour le duc du Maine
contre les princes du sang\,; deux fameux avocats de Paris, dont l'un se
nommait Bargetton, qui y avaient fort travaillé avec lui\,; une
M\textsuperscript{lle} de Montauban, attachée à M\textsuperscript{me} du
Maine en manière de fille d'honneur, et une principale femme de chambre,
favorite, confidente, et sur le pied de bel esprit, avec quelques autres
domestiques de M. et de M\textsuperscript{me} du Maine, furent aussi
menés à la Bastille. Il fut résolu d'envoyer M\textsuperscript{lle} du
Maine à l'abbaye de Maubuisson, et ses deux frères à Eu, avec un
gentilhomme ordinaire du roi auprès d'eux.

Le Blanc me tint parole. J'étais chez moi à huis clos, inquiet de
l'exécution, et n'osant pas ouvrir la bouche, me promenant dans mon
cabinet et regardant à tous moments ma pendule, lorsqu'un laquais vint
de sa part savoir simplement de mes nouvelles. Je fus fort soulagé,
quoique dans l'ignorance comment tout se serait passé. Mon carrosse
était tout attelé. Je ne fis que monter dedans pour aller chez M. le duc
d'Orléans. Je le trouvai seul aussi, qui se promenait dans sa galerie.
Il était près de onze heures, Le Blanc et l'abbé Dubois sortaient d'avec
lui. Je le trouvai fort empêché de son entrevue avec
M\textsuperscript{me} la duchesse d'Orléans, et moi bien à mon aise de
n'être plus à portée avec elle qu'il pût me charger du paquet. Je
l'encourageai de mon mieux, et, au bout d'une demi-heure, je m'en allai
sur l'annonce du comte de Toulouse.

Je sus après de M. le duc d'Orléans qu'il lui avait parlé à merveille,
protesté qu'il ne savait pas un mot de cette affaire, et que Son Altesse
Royale ne le trouverait jamais mêlé en rien contre son service ni contre
la tranquillité de l'État, qu'il ne pouvait n'être pas sensible au
malheur de M. et de M\textsuperscript{me} du Maine\,; qu'il ne pouvait
se persuader, non plus, que Son Altesse Royale ne les crût fort
coupables, puisqu'elle en était venue à cette extrémité avec eux\,; que,
pour lui, il n'osait demander d'éclaircissement\,; qu'il craignait bien
quelque imprudence de M. du Maine, mais qu'il ne se résoudrait jamais à
croire son frère coupable qu'il n'en eût bien vu les preuves\,; qu'en
attendant il se tiendrait dans un silence exact, et ne ferait aucune
démarche que de l'agrément de Son Altesse Royale. Le régent fut content
au dernier point de ce discours d'un homme sur la vérité et la probité
duquel on pouvait compter avec certitude. Il lui dit tout ce qu'il crut
de plus honnête en général, et en particulier pour lui, sans entrer en
rien sur l'affaire, lui fit beaucoup d'amitiés, et se séparèrent très
bien ensemble. La conduite du comte de Toulouse répondit exactement à
son discours. Madame était à Paris, ainsi M. le duc d'Orléans lui parla
lui-même. Pour M\textsuperscript{me} la duchesse d'Orléans on peut
juger, à l'état où elle fut à la chute de son frère au dernier lit de
justice, de celui où cette nouvelle la mit.

Le duc de Saint-Aignan était, comme on le peut juger, très
désagréablement à Madrid, par la situation où les deux cours étaient
ensemble, et par la haine qu'Albéroni s'était fait un principe
d'entretenir en Espagne contre M. le duc d'Orléans, de décrier toutes
ses actions, son gouvernement, sa conduite personnelle, ses actions les
plus innocentes, et d'empoisonner jusqu'à ses démarches les plus
favorables à l'Espagne, et qui tendaient le plus à se la rapprocher. Ce
premier ministre ne gardait plus même depuis longtemps aucunes mesures
avec le duc de Saint-Aignan, jusqu'au scandale de toute la cour de
Madrid, même des moins bien disposés pour la France. Son ambassadeur ne
se maintenait que par la sagesse de sa conduite, et fut ravi des ordres
qui le rappelaient. Il demanda donc son audience de congé, et le prit,
en attendant, de tous ses amis et de toute la cour. Albéroni, qui
attendait à tous moments des nouvelles de Cellamare décisives sur la
conspiration, voulait demeurer maître de la personne de l'ambassadeur de
France, pour, en cas d'accident, mettre à couvert celle de l'ambassadeur
d'Espagne de ce qui lui pouvait arriver. Il différa donc cette audience
de congé sous différents prétextes. À la fin Saint-Aignan, pressé par
ses ordres réitérés, et d'autant plus positifs qu'on commençait à se
douter qu'il pourrait arriver dans peu un éclat sur Cellamare, parla
ferme au cardinal, et déclara que, si on ne voulait pas lui accorder son
audience de congé, il saurait s'en passer. Là-dessus, le cardinal en
colère lui répondit en le menaçant qu'il saurait bien l'en empêcher.
Saint-Aignan fut sage et se contint\,; mais voyant à quel homme il était
exposé, et jugeant avec raison du mystère à le retenir à Madrid, il prit
si bien et si secrètement ses mesures, qu'il partit la nuit même, et
gagna pays avec son plus nécessaire équipage, et qu'il arriva au pied
des Pyrénées avant qu'on eût pu le joindre et l'arrêter, comme il se
doutait bien qu'Albéroni, qui était un homme sans mesure, ne manquerait
pas d'envoyer après lui pour l'arrêter.

Saint-Aignan, déjà si heureusement avancé, ne jugea pas à propos de s'y
exposer plus longtemps, et dans l'embarras des voitures parmi ces
montagnes. Lui et la duchesse sa femme, suivis d'une femme de chambre et
de trois valets, avec un guide bien assuré, se mirent tous sur des mules
pour gagner Saint-Jean-Pied-de-Port sans s'arrêter en chemin que des
moments nécessaires pour repaître. Il ordonna à son équipage d'aller à
Pampelune à leur aise, et mit dans son carrosse un valet de chambre et
une femme de chambre intelligents, avec ordre de se faire passer pour
l'ambassadeur et l'ambassadrice, au cas qu'on les vînt arrêter, et de
crier bien haut. La chose ne manqua pas d'arriver. Les gens qu'Albéroni
avait détachés après eux rejoignirent l'équipage fort tôt après. Les
prétendus ambassadeur et ambassadrice jouèrent très bien leur
personnage, et ceux qui les arrêtèrent ne doutèrent pas d'avoir fait
leur capture, dont ils dépêchèrent l'avis à Madrid, et la gardèrent bien
dans Pampelune où ils l'avaient fait rebrousser.

Cette tromperie sauva M. et M\textsuperscript{me} de Saint-Aignan et
leur donna moyen d'arriver à Saint-Jean-Pied-de-Port. Dès qu'ils y
furent ils envoyèrent chercher du secours et des voitures à Bayonne, où
ils se rendirent en sûreté et s'y reposèrent de leurs fatigues. Le duc
de Saint-Aignan en donna avis à M. le duc d'Orléans par un courrier, et
envoya dire son arrivée à Bayonne au gouverneur de Pampelune et le prier
de lui renvoyer ses équipages\,: on y fut bien honteux d'avoir été
dupés. Les équipages furent renvoyés à Bayonne. Mais Albéroni, lorsqu'il
le sut, entra dans un emportement furieux et fit rudement châtier la
méprise.

Le comte de Solre, lieutenant général et gouverneur de Péronne, mourut à
soixante-dix-sept ans. C'était un fort petit homme de corps et d'esprit.
La valeur, la probité, la fidélité, la naissance et le service de toute
sa vie y suppléaient. Il était de la maison de Croï et sa femme de celle
de Bournonville, la maréchale de Noailles et elle filles des deux
frères. Elle était souvent à la cour, debout parmi les dames de qualité,
aux soupers du roi et aux toilettes de M\textsuperscript{me} la Dauphine
sans aucune prétention ni son mari non plus, qui fut reçu chevalier de
l'ordre, le cinquante-neuvième dans la promotion du dernier décembre
1688, et y marcha sans difficulté depuis dans toutes les fêtes de
l'ordre parmi les gentilshommes. Longtemps après le mari et la femme se
brouillèrent, et pour ne point donner de scène en se séparant, la
comtesse de Solre prit l'occasion du mariage de sa fille avec le prince
de Robecque, qui s'était attaché à l'Espagne, où il avait obtenu la
grandesse et la Toison. Elle lui mena sa fille, qu'elle aimait fort, qui
en arrivant fut dame du palais de la reine, et toutes deux ont passé le
reste de leur vie en Espagne, où je les ai beaucoup vues. Le fils aîné
du comte de Solre, qui était maréchal de camp, quitta le service après
la mort de son père, se fit appeler le prince de Croï, ne quitta plus la
Flandre, où il avait beaucoup de terres, y épousa M\textsuperscript{lle}
de Mylandon, riche héritière, et firent les princes chez eux. Cette
dame, devenue veuve, vint avec son fils à Paris pour le mettre dans le
service, et tâcha d'éblouir le cardinal Fleury de ses prétentions. Elle
n'y réussit que pour obtenir plus tôt l'agrément d'un régiment pour son
fils, et ses prétentions l'ont exclue de la cour\,; elle est restée à
Paris, toujours princesse, mais uniquement pour ses valets, et son fils
pareillement.

Nointel, conseiller d'État, mourut aussi. Il était fils de Bechameil,
surintendant de feu Monsieur, beau-père de Louville et beau-frère du feu
duc de Brissac, père de celui-ci, et de Desmarets, qui avait été
contrôleur général et ministre. Ce conseiller d'État était un bon homme
et un fort homme d'honneur.

Le vieux Heudicourt, qui avait été grand louvetier et mari de cette
M\textsuperscript{me} d'Heudicourt dont il a été parlé quelquefois ici,
que j'appelais le mauvais ange de M\textsuperscript{me} de Maintenon,
mourut chez lui à sa campagne. C'était un vieux débauché, gros et vilain
joueur, dont personne ne fit jamais le moindre cas. Son fils, dont il a
été parlé aussi, ne valut pas mieux, mais bien plus dangereux par son
esprit, ses saillies et sa méchanceté.

Il a été quelquefois mention ici, en diverses occasions, de Belle-Ile.
Il est temps de commencer à faire connaître un homme qui, de naissance
plébéienne, et de plus disgraciée de tous points, est parvenu à tout par
des fortunes si étranges, qu'il se peut dire à la lettre que sa vie est
un roman. Ces Fouquet sont Bretons, et les père et grand-père du fameux
surintendant étaient conseillers au parlement de Bretagne. On sait qu'il
y a des charges de conseillers qu'on appelle \emph{bretonnes}, dont les
titulaires ont été longtemps et doivent être toujours gentilshommes de
noms et d'armes. Souvent il y a eu parmi eux des gens de qualité
distinguée de la province. Il y a aussi des charges qu'on appelle
\emph{angevines}, toujours possédées comme le sont les mêmes charges de
conseillers dans tous les parlements. Cela fait en Bretagne une grande
différence entre les charges et leurs titulaires, quoiqu'il n'y en ait
aucune entre eux pour le rang, le service et les fonctions. Je n'ai pas
recherché si les charges de ces conseillers Fouquet étaient
\emph{bretonnes} ou \emph{angevines}. La fortune, la chute et les
malheurs du surintendant Fouquet sont trop connus pour s'y arrêter
ici\,; mais il faut expliquer comment il eut Belle-Ile et comment
Belle-Ile est venue à son petit-fils, duquel il s'agit ici.

Cette île qui a six lieues de long sur deux de large, séparée par six
lieues de mer des côtes de Vannes, appartenait à l'abbaye de
Sainte-Croix de Quimper. Charles IX la lui ôta et s'en empara, comme il
est arrivé souvent à nos rois de faire de ces démembrements en des lieux
dangereux et suspects comme l'est cette île par rapport à l'Angleterre,
et dans des temps de troubles, de guerres civiles et de religion, comme
du temps de Charles IX. Le comte de Retz, en grande faveur auprès du roi
et de Catherine de Médicis, sa mère, et depuis maréchal de France, et
enfin duc et pair, obtint d'eux Belle-Ile, partie en don, partie en
payant, et la fit ériger en marquisat. La position de cette île a
souvent donné envie aux rois successeurs de l'acquérir, et il y a eu en
divers temps des échanges projetés et même fort avancés, qui n'ont point
eu d'exécution. Fouquet, devenu surintendant des finances, en fit
l'acquisition de la maison de Retz. À sa disgrâce, Belle-Ile fut adjugée
à sa femme pour ses reprises.

Le père du surintendant, de conseiller de Bretagne s'était fait maître
des requêtes et devint conseiller d'État. Sa femme, mère du
surintendant, était Maupeou, dont le père était intendant des finances.
La vertu, le courage, la singulière piété de cette dame, mère des
pauvres, et dont le nom vit encore, fut inébranlable à la fortune et aux
malheurs de son fils, dont la première dura huit ans et les autres
dix-huit. Il mourut dans sa prison de Pignerol en mars 1680, à
soixante-cinq ans, et sa vertueuse mère, et qui avait aussi beaucoup
d'esprit, le survécut un an et en avait quatre-vingt-onze. Il avait
épousé une héritière de Bretagne, qui s'appelait Fourché, dont il n'eut
qu'une fille, mariée en 1657 au comte de Charost, mort duc et pair,
etc., dont elle eut le duc de Charost, gouverneur du roi d'aujourd'hui,
à la disgrâce du maréchal de Villeroy.

Le surintendant se remaria à la fille unique de Castille, président aux
requêtes du palais, et c'est elle à qui Belle-Ile fut adjugée pour ses
reprises. Elle eut d'elle Nicolas Fouquet, qui servit quelque temps sous
le nom de comte de Vaux, qui était considéré pour son mérite, mais qui,
par le malheur de son père, n'ayant pu avancer, quitta de bonne heure,
et est mort en 1705 sans enfants de la fille de la fameuse
M\textsuperscript{me} Guyon, laquelle fille est morte longtemps depuis
duchesse de Sully, sans enfants. Ce fut un mariage d'amour, longtemps
secret, déclaré enfin après que, de cadet et pauvre, le chevalier de
Sully eut recueilli la dignité et les biens de son frère. Le second fils
du surintendant, célèbre père de l'Oratoire et fort riche, légua tout
son bien au neveu dont il s'agit ici. Le troisième fut un homme de
beaucoup d'esprit et de savoir, que les malheurs de sa famille exclurent
de toute sorte d'emploi, qui n'avait rien et qui a été obscur et sauvage
au dernier point de toute sa vie. L'amour, et plus tôt satisfait que de
raison, lui valut une grande alliance. Le marquis de Lévi, grand-père du
duc de Lévi, n'eut d'autre parti à prendre que de lui laisser épouser sa
fille, de la chasser de chez lui et de ne vouloir jamais entendre parler
d'eux. Ils furent donc réduits à suivre le pot et les exils de l'évêque
d'Agde, frère du surintendant, et de vivre après de celui de sa mère,
retirée aux dehors du Val-de-Grâce, qui a élevé ses deux fils Belle-Ile,
dont il s'agit ici, et le chevalier son frère.

J'ai parlé en son temps de l'application de Belle-Ile au service, à
plaire, à capter, à se rendre utile aux généraux\,; comment il eut un
régiment de dragons\,; combien il se distingua dans Lille\,; comment il
devint mestre de camp général des dragons. J'ai parlé aussi de deux
mariages, le premier sans enfants, l'autre à une Béthune, fille du fils
de la sueur de la reine de Pologne (Arquien), et de la sueur du
maréchal-duc d'Harcourt. Ainsi Belle-Ile se trouva cousin germain des
ducs de Charost et de Lévi, et neveu du maréchal-duc d'Harcourt, cousin
issu de germain des électeurs de Cologne et de Bavière, fils de la fille
de la reine de Pologne (Arquien), et au même degré du roi Jacques
d'Angleterre, et du duc de Bouillon\,; très proche encore du roi de
Pologne, père de la reine par les Jablonowski, du duc Ossolinski, du
prince de Talmont et de beaucoup des plus grands seigneurs de Pologne,
et il sut tirer un grand parti de ces singulières et si proches
alliances. La soeur de son père avait épousé un Crussol-Monsalez, dont
il y a des enfants.

La mort du vieux marquis de Lévi et le temps qui amène tout, avait
réconcilié son fils le marquis de Charlus avec sa soeur et son mari
Belle-Ile. C'était une femme qui n'avait jamais eu d'autre inclination
que celle qui fit son mariage et qui vécut avec son mari comme un ange,
toute sa vie dans la pauvreté et dans la disgrâce. Revenue après bien
des années à Paris, et raccommodée avec sa famille, elle chercha à en
profiter. Elle avait de l'esprit et de la piété. Les malheurs dans
lesquels elle avait vécu l'avaient accoutumé à la dépendance, aux
besoins, à ne point sortir de l'état où son mariage l'avait mise. Son
caractère était la douceur et l'insinuation. Aimée et fort considérée
dans la famille de son mari, et seulement soufferte dans la sienne, elle
fit si bien qu'elle s'en fit enfin aimer. Elle comprit l'utilité qu'elle
pouvait espérer pour ses enfants de la situation de
M\textsuperscript{me} de Lévi à la cour, qui était fille du duc de
Chevreuse, et qui, en épousant son neveu fils de son frère, avait été
faite dame du palais. À la considération où étaient M. et
M\textsuperscript{me} de Chevreuse et M. et M\textsuperscript{me} de
Beauvilliers qui n'étaient qu'un, succéda la considération personnelle
de M\textsuperscript{me} de Lévi par l'amitié que M\textsuperscript{me}
de Maintenon et le roi prirent pour elle et les fréquentes parties
particulières dont elle fut toujours avec eux jusqu'à la mort du roi, et
la fortune voulut encore qu'elle fût après l'amie intime du cardinal
Fleury, avec M\textsuperscript{me} de Dangeau son amie et sa compagne
dans sa place de dame du palais et dans les continuelles privances de
M\textsuperscript{me} de Maintenon et du roi. M\textsuperscript{me} de
Lévi, avec infiniment d'esprit et beaucoup de piété solide, avait le
défaut de l'entêtement\,; et le sien était toujours poussé sans bornes.
Avec cela une vivacité de salpêtre. Prise d'affection et pour l'avouer
franchement de compassion pour sa tante de Belle-Ile, cette femme
adroite qui lui faisait sa cour, introduisit ses enfants en son amitié.
Bientôt elle les aima aussi pour eux-mêmes, se prit de leur mérite et de
leurs talents, et l'entêtement n'eut tôt après plus de bornes et n'en a
jamais eu depuis jusqu'à sa mort. Aussi cultivèrent-ils bien
soigneusement une affection si capitale et du mari et surtout de la
femme. Leur bonheur voulut qu'ils n'affolèrent pas moins le duc de
Charost et son fils. Mais le pouvoir de ceux-là ne fut pas tel que celui
de M\textsuperscript{me} de Lévi.

Il faut maintenant venir au caractère des deux frères. L'aîné, grand,
bien fait, poli, respectueux, entrant, insinuant et aussi honnête homme
que le peut permettre l'ambition quand elle est effrénée, et telle était
la sienne, avait précisément la sorte d'esprit dont il avait besoin pour
la servir. Il n'en voulait point montrer, il ne lui en paraissait que
pour plaire, jamais pour embarrasser, encore moins pour effrayer\,; un
fonds naturel de douceur et de complaisance, une juste mesure entre
l'aisance dans toutes ses manières et la retenue, un art infini, mais
toujours caché dans ses propos et ses démarches, une insinuation
délicate et rarement aperçue\,; une attention et une précaution
continuelle dans tous ses pas et dans ses discours, jusqu'au langage des
femmes et au badinage léger, lui ouvrirent une infinité de portes. Il ne
négligea ni les cochères, ni les carrées, ni les rondes. Il voulait
plaire au maître et aux valets, à la bourgeoise et au prêtre de paroisse
ou de séminaire quand le hasard lui en faisait rencontrer, à plus forte
raison au général et à son écuyer, aux ministres et aux derniers commis.
Une accortise qui coulait de source, un langage toujours tout prêt et
des langages de toutes les sortes, mais tous parés d'une naturelle
simplicité, affable aux officiers, essentiellement officieux, mais avec
choix et relativement à soi, et beaucoup de valeur sans aucune
ostentation\,: tel fut Belle-Ile tant qu'il demeura \emph{in
minoribus\,;} sans se démentir en rien de ce caractère, il se développa
davantage à mesure que la fortune l'éleva. C'est où nous n'en sommes pas
encore\,; ce qu'il pratiqua dans tous les temps de sa vie fut une
application infatigable à discerner ceux dont il pouvait avoir besoin, à
ne rien oublier pour les gagner, et après les infatuer de lui avec les
plus simples et les plus doux contours, à en tirer tous les avantages
qu'il put, et à ne jamais faire un pas, une visite, même une partie ou
un voyage de plaisir que par choix réfléchi, pour l'avancement de ses
vues et de sa fortune, et chemin faisant, appliqué sans cesse à
s'instruire de tout sans qu'il y parût le moins du monde.

Le chevalier de Belle-Ile avait bien des conformités avec son frère, et
encore plus de dissemblances. Sa figure n'était pas si bien, et l'air
ouvert et naturellement simple et libre dans l'aîné, manquait au cadet.
Il avait toutefois l'entrant et l'insinuant de son frère, mais qui ne
s'annonçait pas à son maintien comme l'aîné. Il fallait qu'il commençât
à parler pour le sentir, encore lorsqu'il s'agissait ou d'affaires ou de
gens à qui il importait de ne pas déplaire, car pour le gros, il était
naturellement cynique, peu complaisant, contredisant, mordant\,; mais
avec ceux qu'il croyait devoir ménager, et il savait en ménager
beaucoup, il était aussi maniable et aussi complaisant et mesuré que son
frère, sans toutefois que cela parût couler de source, ni aussi naturel
qu'à l'aîné\,; beaucoup plus d'esprit et d'étendue que lui, peut-être
aussi l'esprit et les vues plus indigestes et nulle douceur dans les
moeurs que forcée, et on l'apercevait\,; plus de justesse néanmoins et
de discernement que son frère et incomparablement plus difficile à
tromper, peut-être aussi moins parfaitement honnête homme, mais beaucoup
plus capable et intelligent en toutes sortes d'affaires, et rancunier
implacable, ce que le frère n'avait pas. Le chevalier avait aussi le
jargon des femmes, mais point de liant, quoique plus de tour et
d'adresse à découvrir ce qu'il voulait savoir et toute l'application
possible à s'instruire et de toutes et des différentes parties de la
guerre\,; il voulait que rien ne lui échappât, et comme son frère, ni
pas ni discours qui n'eût sa vue particulière, et toutes vues tournées à
une ambition plus vaste, et, s'il était possible, plus effrénée que
celle de son frère, et tous deux d'une suite que rien ne dérangeait et
d'un courage d'esprit invincible. Celui-ci avait plus de ruse et de
profondeur que l'autre, et moins capable que lui encore de se rebuter et
de démordre. Il avait un froid de glace, mais qui en dedans cachait une
disposition toute contraire, et un air compassé et de sagesse arrangée
qui n'attirait pas. Avec autant de valeur que son frère et possédant
comme lui tous les détails militaires et de subsistances et de dépôts,
il le surpassait peut-être en celui de toute espèce d'arrangements\,;
personne n'a eu comme eux l'art imperceptible d'amener de loin et de
près les hommes et les choses à leurs fins, et de savoir profiter de
tout. Le cadet, avec un flegme plus obstiné que son frère, était bien
plus propre que lui à gouverner et à régler les dépenses et l'économie
domestique, à dresser des mémoires d'affaires d'intérêt, à conduire dans
les tribunaux celles qu'il y fallait porter, et à leur donner le tour et
la subtilité dont elles pouvaient avoir besoin\,; enfin la présence
d'esprit et la souplesse à l'attaque et à la défense judiciaire, avec le
style éloquent, coulant et net. Tous deux enfin sans cesse occupés et
parmi cette application continuelle, vivement et continuellement les
yeux ouverts à se faire des protecteurs, des amis et des créatures avec
choix, et très mesurés dans leurs paroles et ne se lâchant jamais dans
les entretiens qu'avec une grande mesure et un grand choix.

L'union de ces deux frères ne fit des deux qu'un coeur et une âme, sans
la plus légère lacune, et dans la plus parfaite indivisibilité et tout
commun entre eux, biens, secrets, conseils, sans partage ni réserve,
même volonté en tout, même autorité domestique sans partage, toute leur
vie. Le cadet, moins à portée que l'aîné, ne songea qu'à sa fortune, et
s'occupa principalement du domestique et des affaires de la maison, et
l'aîné du dehors\,; mais tout se référa toujours de l'un à l'autre, et
tout fut conduit comme par un seul. On ne saurait ajouter au respect, à
l'amitié, aux soins, à l'attachement qu'ils eurent toujours pour leur
père, et à la confiance qu'ils eurent pour leur mère, qui trouvèrent
enfin leur bonheur par eux. L'aîné, fort sobre\,; la cadet aimait à
souper et à boire le petit coup, mais sans excès et sans préjudice aux
occupations sérieuses auxquelles il avait toujours l'esprit bandé.

M\textsuperscript{me} de Lévi, et par sa plus intime famille et
personnellement notre amie intime, les initia peu à peu avec
M\textsuperscript{me} de Saint-Simon et avec moi. Le duc de Charost y
contribua aussi\,; ils nous cultivèrent fort. J'y trouvai beaucoup de ce
qu'on ne trouvait plus, et ils devinrent enfin nos amis. Ils me furent
souvent utiles à m'apprendre bien des choses, et j'eus souvent le
plaisir de leur rendre des services. Nous étions sur ce pied-là dans le
temps duquel j'écris, et l'amitié entre nous s'est toujours depuis
conservée la même. Belle-Ile avait fait en Flandre connaissance avec Le
Blanc, qui se tourna en la plus intime amitié et confiance. Le Blanc
l'introduisit auprès de l'abbé Dubois chez lequel il fut bientôt en
privance et en apparence de confiance. Il fut bien aussi avec le garde
des sceaux et peu à peu avec beaucoup d'autres. M. le Duc le prit en
grande amitié, tellement que Belle-Ile profita de cette situation pour
réveiller les anciens projets de l'échange de Belle-Ile. Avant de rien
proposer là-dessus, il s'était assuré de Law par l'abbé Dubois et Le
Blanc, et du garde des sceaux par les mêmes. Il pouvait compter sur M.
le Duc et sur le comte de Toulouse, qui fut toujours de ses amis
déclarés. Il se saisit de Fagon qui avait une autorité dans les
finances, qui alla toujours en croissant, et qui toute sa vie lui fut
totalement dévoué\,; il s'assura encore de plusieurs autres. Il pointait
dès lors assez pour attirer les yeux, et il se trouva gens du plus haut
parage qui trouvèrent qu'il croissait trop vite, qui voulurent l'arrêter
de bonne heure, et que ses hommages ne purent émousser. Je ne sais par
où la vieille cour l'avaient pris en grippe de si bonne heure, et si
loin de pouvoir même espérer d'offusquer. Les maréchaux de Villeroy,
Villars et Huxelles furent les principaux à le traverser, quoique la
maréchale de Villars émoussât quelquefois son mari sur cet éloignement
sans cause. Néanmoins l'échange parut utile au roi, et Belle-Ile fit si
bien, qu'il se le rendit prodigieusement avantageux. Il eut le comté de
Gisors, Vernon et tous les domaines du roi qui en dépendent, en sorte
qu'il eut pour le moins autant de terres que M. de Bouillon en avait par
les comtés d'Évreux et de Beaumont, mais avec un revenu beaucoup
moindre, parce que les forêts d'Évreux, etc., avaient été données à M.
de Bouillon, et que Belle-Ile n'eut pas celles de ce qui lui fut cédé\,;
ce fut pour quelque sorte de compensation qu'on lui donna beaucoup de
domaines en Languedoc et de grand revenu.

Cet échange ne se conclut pas tout d'une voix des commissaires chargés
de le régler. Les difficultés que quelques-uns firent, arrêtèrent\,; le
monde cria qu'on lui donnait de vrais États pour une île comme déserte
et inutile au roi qui y avait un gouverneur, un état-major et une
garnison. Il ne fallut pas peu de temps, de patience et d'adresse pour
vaincre ces difficultés. Une autre s'éleva encore par les mouvements que
se donnèrent un grand nombre de gens distingués de la noblesse et de la
robe qui relevaient du roi, et qui se trouvèrent très offensés d'avoir à
relever désormais de Belle-Ile qui exercerait sur eux tous les droits du
roi, et avec une rigueur en usage entre particuliers en tout genre
utile, de chasse et honorifique, qui sont peu perceptibles avec le roi.
Ces nouveaux cris arrêtèrent encore\,; on trouvait Belle-Ile bien léger
pour être seigneur d'un domaine aussi étendu, aussi brillant, aussi
noble, et pour l'exercer en plein sur tant et de tels vassaux. Le
détroit fut encore long et difficile à passer. Mais l'adresse des
Belle-Ile en vint encore à bout sans le plus léger retranchement ni
modification.

La chose passée vint au conseil de régence. Les maréchaux, soutenus du
duc de Noailles et de Canillac, s'élevèrent\,; le prince de Conti les
appuya. Quoique les contradicteurs fissent le moindre nombre, leur poids
arrêta M. le duc d'Orléans\,: il dit qu'il fallait remettre la décision
à une autre fois. Belle-Ile, en homme avisé, ne voulut pas presser
l'affaire, pour laisser refroidir les esprits\,; mais six semaines
après, en entrant au conseil de régence, et auparavant averti par
Belle-Ile, M. le Duc me donna le mot, et je le donnai tout bas au comte
de Toulouse pendant le conseil. On n'y dit pas un mot de l'affaire.
Comme il se levait, M. le Duc dit à M. le duc d'Orléans, déjà debout,
s'il ne voulait pas finir l'échange de Belle-Ile\,; et, me regardant,
ajouta\,: «\,Les commissaires en sont d'avis, presque tout le monde en a
été d'avis ici.\,» Je répondis que ce n'était pas la peine de se
rasseoir, puisque la chose avait passé ici déjà à la pluralité. Le comte
de Toulouse ajouta\,: «\,Mais cela est vrai.\,» M. le Duc reprit, en
regardant en riant M. le duc d'Orléans\,: «\,Monsieur, vous voulez aller
à l'Opéra et moi aussi. Il est plus de cinq heures\,; prononcez donc, et
allons-nous-en.\,» Tout cela se fit debout, à la surprise de tout le
monde, sans que les contradicteurs dans l'autre conseil eussent le temps
de reprendre leurs esprits, ou osassent se prendre de bec avec M. le Duc
et le comte de Toulouse, et croyant peut-être que cela se faisait de
concert avec M. le duc d'Orléans, qui n'en savait pas un mot, et qui
dans sa surprise se laissa entraîner\,: «\,Oui, dit-il, il me semble que
cela a passé\,;» regarda le conseil tout autour, qui ne souffla pas,
puis ordonna à La Vrillière d'écrire sur le registre du conseil que cela
passait, et de faire expédier l'échange et s'en alla. M. le Duc et moi
en rîmes en sortant du conseil\,; j'en avais déjà ri avec le comte de
Toulouse. Un jugement si leste ne plut à personne du conseil, moins
encore aux contradicteurs, qui grommelèrent, et dirent que c'était une
moquerie.

Belle-Ile fut aussi bien servi dans la promptitude de l'expédition. Il
s'était fait des amis au parlement qui ne laissa pas de se rendre
difficile à l'enregistrement pur et simple\,; mais il le fit sans trop
de délais. La chambre des comptes fut plus épineuse et plus longue\,;
mais Belle-Ile à la fin en vint à bout\,: toutefois, il était bien loin
d'être au bout de ses peines, malgré cette consommation.

C'est s'être bien étendu sur deux particuliers alors si peu
considérables\,; mais ils le devinrent tellement dans leur suite par
leurs malheurs et les genres de périls qu'ils coururent, par la manière
dont ils en sortirent, par les effets prodigieux de la plus singulière
fortune, et qui devint enfin la plus haute en tous genres, dont ils ont
été les seuls artisans, que j'ai cru devoir bien faire connaître, et de
bonne heure, deux hommes si rares qui, devenus des personnages en
France, même en Europe, ont été les plus extraordinaires de leur siècle,
de quelque côté qu'on puisse les envisager.

\hypertarget{chapitre-vi.}{%
\chapter{CHAPITRE VI.}\label{chapitre-vi.}}

1719

~

{\textsc{1719. Conduite du duc du Maine.}} {\textsc{- Conduite de
M\textsuperscript{me} du Maine.}} {\textsc{- M\textsuperscript{me} la
Princesse obtient quelques adoucissements à M\textsuperscript{me} du
Maine, et à M\textsuperscript{me} de Chambonnas, sa dame d'honneur, de
s'aller enfermer avec elle\,; puis son médecin.}} {\textsc{- Commotion
de la découverte de la conspiration.}} {\textsc{- Conduite du duc de
Noailles.}} {\textsc{- Netteté de discours et de procédé du comte de
Toulouse.}} {\textsc{- Faux sauniers soumis d'eux-mêmes.}} {\textsc{-
Adresse de l'abbé Dubois.}} {\textsc{- Il fait faire par Fontenelle le
manifeste contre l'Espagne.}} {\textsc{- Il est examiné dans un conseil
secret au Palais-Royal, passé après en celui de régence, et suivi
aussitôt de la publication de la quadruple alliance imprimée, et de la
déclaration de guerre contre l'Espagne.}} {\textsc{- Le tout très mal
reçu du public.}} {\textsc{- Pièces répandues contre le régent sous le
faux nom du roi d'Espagne, très faiblement tancées par le parlement.}}
{\textsc{- Incendie du château de Lunéville.}} {\textsc{- Conspiration
contre le czar découverte.}} {\textsc{- Le roi de Suède tué.}}
{\textsc{- Prétendants à cette couronne, qui redevient élective, et la
soeur du feu roi élue reine avec peu de pouvoir, qui obtient après
l'association au trône du prince de Hesse, son époux, mais avec force
entraves contre l'hérédité et le pouvoir.}} {\textsc{- Baron de Goertz
est décapité, et le baron Van der Nath mis en prison perpétuelle.}}
{\textsc{- M. le duc de Chartres a voix au conseil de régence, où il
entrait depuis quelque temps.}} {\textsc{- Saint-Nectaire ambassadeur en
Angleterre.}} {\textsc{- Rareté de son instruction et de celle des
autres ministres de France au dehors.}} {\textsc{- Maligne plaisanterie
du duc de Lauzun fait cinq ans après le vieux Broglio maréchal de
France.}} {\textsc{- Officiers généraux et particuliers nommés pour
l'armée du maréchal de Berwick.}} {\textsc{- M. le prince de Conti
obtient d'y servir de lieutenant général et de commandant de la
cavalerie, et de monstrueuses gratifications.}} {\textsc{- Prodigalités
immenses aux princes et princesses du sang, excepté aux enfants du
régent.}} {\textsc{- Prodigalités au grand prieur.}} {\textsc{- Il veut
inutilement entrer au conseil de régence\,; mais ce fut quelque temps
après être revenu d'exil\,; et cela avait été oublié ici en son temps.}}
{\textsc{- L'infant de Portugal retourne de Paris à Vienne.}} {\textsc{-
Le duc de Saint-Aignan entre en arrivant au conseil de régence.}}
{\textsc{- Mort et caractère de Saint-Germain Beaupré.}} {\textsc{- Mort
du prince d'Harcourt.}} {\textsc{- Mort et aventure de
M\textsuperscript{me} de Charlus.}} {\textsc{- Mort de M. de Charlus.}}
{\textsc{- Jeux de hasard défendus.}} {\textsc{- Blamont, président aux
enquêtes revient de son exil en une de ses terres.}} {\textsc{- Le grand
prévôt obtient la survivance de sa charge pour son fils qui a six ans.}}
{\textsc{- Milice levée.}}

~

Le duc du Maine, outre l'aîné La Billarderie, lieutenant des gardes du
corps qui l'avait arrêté, fut conduit et gardé à Dourlens par
Favancourt, maréchal des logis des mousquetaires gris et qui était
sous-brigadier de mon temps dans la brigade où j'étais\,; il m'avait
toujours vu depuis de temps en temps, et néanmoins il fut chargé de ce
triste emploi sans que je le susse et sans même que j'eusse pensé à
personne pour cela. Je n'eus aussi aucun commerce avec lui direct ni
indirect pendant tout le temps qu'il le garda, et il fut auprès de lui
jusqu'à sa sortie. Quoique gentilhomme de Picardie, il était fin et
désinvolte à merveilles, et s'acquitta si bien de son emploi qu'il
satisfit ceux qui l'y avaient mis et en même temps le duc du Maine, qui
a depuis particulièrement protégé sa famille.

Au retour de Favancourt, je fus curieux de l'entretenir à fond. Il me
conta que la mort était peinte sur le visage du duc du Maine pendant
tout le voyage depuis Sceaux jusqu'à Dourlens\,; qu'il ne lui échappa ni
plainte, ni discours, ni question, mais force soupirs. Il ne parla point
du tout les premières cinq ou six heures et fort peu le reste du voyage,
et dans ce peu presque toujours des choses qui s'offraient aux yeux en
passant. À chaque église devant quoi on passait, il joignait les mains,
s'inclinait profondément et faisait force signes de croix, et par-ci,
par-là, marmottait tout bas des prières avec des signes de croix. Jamais
il ne nomma personne, ni M\textsuperscript{me} la duchesse du Maine, ni
ses enfants, ni pas un de ses domestiques, ni qui que ce soit. À
Dourlens il faisait ou montrait faire de longues prières, se prosternait
souvent, était petit et dépendant de Favancourt comme un très jeune
écolier devant son maître, avait trois valets avec lui avec qui il
s'amusait, quelques livres, point de quoi écrire\,; il en demanda fort
rarement, et donnait à lire et à cacheter à Favancourt ce qu'il avait
écrit. Au moindre bruit, au plus léger mouvement extraordinaire, il
pâlissait et se croyait mort. Il sentait bien ce qu'il avait mérité et
jugeait par lui-même de ce qu'il avait lieu de craindre d'un prince
qu'il avait pourtant dû avoir reconnu plus d'une fois être si
prodigieusement différent de lui. Pendant le voyage et à Dourlens il
mangea toujours seul.

M\textsuperscript{me} la duchesse du Maine, conduite par le cadet La
Billarderie, aussi lieutenant des gardes du corps, trouva en lui de la
complaisance. Elle en abusa et M. le duc d'Orléans le souffrit avec
cette débonnaireté si accoutumée. On eût dit, pendant la route, que
c'était une fille de France qu'une haine sans cause et sans droit
traitait avec la dernière indignité. L'héroïne de roman, farcie des
pièces de théâtre qu'elle jouait elle-même à Sceaux depuis plus de vingt
ans, ne parlait que leur langage, où les plus fortes épithètes ne
suffisaient pas à son gré à la prétendue justice de ses plaintes. Elles
redoublèrent en éclats les plus violents quand, à la troisième journée,
elle apprit enfin qu'on la conduisait à Dijon. Ses projets connus et
renversés, l'insolence qu'elle disait éprouver d'être arrêtée, tous les
insupportables accompagnements de sa captivité dont elle n'avait cessé
de se plaindre en furie, ne furent rien en comparaison de se voir mener
dans la forteresse de la capitale du gouvernement de M. le Duc, où il
était parfaitement le maître\,; elle vomit contre lui tout ce que la
rage soutenue d'esprit peut imaginer de plus injurieux\,; elle oublia
qu'elle était soeur de M. son père\,; elle n'épargna pas leur origine
commune, et triompha de bien dire sur l'enfant de treize mois. Elle fit
la malade, changea de voiture, s'arrêta à Auxerre et partout où elle
put, dans l'espérance que M\textsuperscript{me} la Princesse pourrait
obtenir un changement de lieu, peut-être dans celle de faire peur de ses
transports. En effet, M\textsuperscript{me} sa mère importuna tant M. le
duc d'Orléans, qu'on lui envoya trois femmes de chambre et que
M\textsuperscript{me} de Chambonnas, sa dame d'honneur, obtint la
permission de s'aller enfermer avec elle, puis son médecin et une autre
fille à elle\,; mais ce fut dans le château de Dijon, sur lequel tout
changement fut refusé. Ces égards étaient du bien perdu. M. le duc
d'Orléans ne pouvait l'ignorer, mais telle était sa déplorable
faiblesse.

Plusieurs gens, mais de peu, furent successivement arrêtés et mis à la
Bastille et à Vincennes. La commotion de la prison de M. et de
M\textsuperscript{me} du Maine fut grande\,; elle allongea bien des
visages de gens que le lit de justice des Tuileries avait déjà bien
abattus. Le premier président et d'Effiat, qui de concert avaient ourdi
tant de trames et tenu si longtemps le régent dans leurs filets\,; le
maréchal de Villeroy, qui en lui parlant se figurait toujours de parler
à M. le duc de Chartres, du temps de feu Monsieur, et qui se persuadait
être le duc de Beaufort de cette régence\,; le maréchal de Villars, qui
piaffait en conquérant\,; le maréchal d'Huxelles, tout important dans
son lourd silence, tout du Maine, tout premier président, et qui, lié
aux autres par ces mêmes liens, se persuadait être le Mentor de la
cabale et en sûreté avec ces personnages\,; Tallard, qui avec tout son
esprit ne fut jamais que le frère au chapeau du maréchal de Villeroy et
le valet des Rohan\,; M\textsuperscript{me} de Ventadour, transie par
son vieil galant et bien d'autres en sous-ordre, pas un n'osait dire un
seul mot\,; ils évitaient de se rencontrer\,; leur frayeur peinte sur
leurs mornes visages les décelait. Ils ne sortaient de chez eux que par
nécessité. L'importunité qu'ils recevaient de ce qui allait les voir se
montrait malgré eux. La morgue était déposée\,; ils étaient devenus
polis, caressants, ils mangeaient dans la main, et, par ce changement
subit et l'embarras qui le perçait, ils se trahissaient eux-mêmes.

Je ne puis dire de quelle livrée fut le duc de Noailles, mais il se
soutint mieux que les autres, quoique avec un embarras marqué, malgré
son masque ordinaire, et il s'aida fort à propos de son enfermerie à
laquelle tout le monde était accoutumé. S'il était ou n'était pas de
l'intrigue, je n'ai pu le démêler\,; mais ce qui fut visible, c'est
qu'il fut fort fâché de la découverte. La perte des finances, le
triomphe de Law n'avaient pu être compensés par toutes les grâces dont
le régent l'accabla. Il fut outré de plus de n'avoir été de rien sur le
lit de justice, ni sur l'arrêt de M. et M\textsuperscript{me} du Maine,
et je crois qu'il aurait voulu jouir de l'embarras du régent par quelque
succès de la conspiration. D'un autre côté, il était trop connu et trop
méprisé des principaux personnages pour que je me puisse persuader
qu'ils lui eussent fait part de leurs secrets.

Le comte de Toulouse, toujours le même, vint, aussitôt l'arrêt du duc et
de la duchesse du Maine, trouver M. le duc d'Orléans. Il lui dit
nettement qu'il regardait le roi, le régent et l'État comme une seule et
même chose\,; qu'il l'assurait sans crainte et sans détour qu'on ne le
trouverait jamais en rien de contraire au service et à la fidélité qu'il
leur devait, ni en cabale ni intrigue\,; qu'il était bien fâché de ce
qui arrivait à son frère, mais duquel, il ajouta tout de suite, il ne
répondait pas. Le régent me le redit le jour même, et me parut, avec
raison, charmé de cette droiture et de cette franchise. J'ai touché plus
haut cette conversation.

Ce coup frappé sur M. et M\textsuperscript{me} du Maine acheva
d'éparpiller cette prétendue noblesse dont ils s'étaient joués et servis
avec tant d'art, de succès et de profondeur\,; le gros ouvrit enfin les
yeux sans que personne en prit la peine\,; le petit nombre des
confidents, et qui servaient à mener et aveugler les autres, tombèrent
dans la consternation et l'effroi. De ce moment, les faux sauniers, qui
s'étaient peu à peu mis en troupes, et qui avaient souvent battu celles
qu'on leur avait opposées, mirent partout armes bas, et demandèrent et
obtinrent pardon. Cette promptitude mit tout à fait au clair qui les
employait et ce qu'on en prétendait faire. Je l'avais inutilement dit,
il y avait longtemps, à M. le duc d'Orléans, qui de lui-même m'avoua
alors que j'avais eu raison\,; mais malheureusement je l'avais trop
souvent et trop inutilement avec lui.

Pendant toute cette commotion, l'affaire du traité contre l'Espagne
était publique. Stairs, Koenigseck et l'abbé Dubois avaient pris soin de
la répandre dès que la résolution en fut prise, afin qu'il n'y eût plus
à en revenir, de forcer le régent à une prompte déclaration de guerre,
et à agir aussitôt après en conséquence. Dubois, qui se servait toujours
de la plume de Fontenelle, si connu par son esprit, la pureté de son
langage et ses ouvrages académiques, le chargea de la composition du
manifeste qui devait précéder immédiatement la déclaration de guerre.
Avant que le montrer au conseil de régence, M. le duc d'Orléans assembla
dans son cabinet M. le Duc, le garde des sceaux, l'abbé Dubois, Le Blanc
et moi, pour l'examiner. Je fus surpris de l'ordre qu'il m'en donna
après tout ce que je lui avais si fortement représenté contre cette
guerre. M. le Duc, si étroitement lié avec le régent depuis le lit de
justice, était là pour la forme, et Argenson et Le Blanc, comme les deux
acolytes de l'abbé Dubois. Je ne compris donc point ce qui m'y faisait
admettre en cinquième, à moins que Dubois n'ait voulu orner son triomphe
d'un captif qu'il n'osait et ne pouvait mépriser, et montrer à son
maître qu'il n'était point blessé contre ceux qui n'étoient point de son
avis, ou que le régent, honteux avec moi, m'eût voulu faire cette petite
civilité, et peut-être s'appuyer de moi pour adoucir des termes trop
forts du manifeste.

Le Blanc fit posément la lecture de la pièce. On voulut l'interrompre
pour y faire quelque changement. Je proposai qu'on l'entendît tout de
suite pour en prendre le total et le sens, faire chacun à part soi ses
remarques, et à la seconde lecture interrompre et dire ce qu'on jugerait
à propos\,: cela fut exécuté de la sorte. Cette pièce fut ce qu'elle
devait être, c'est-à-dire masquée, fardée, mais pitoyable jusqu'à
montrer la corde, parce que nul art ne pouvait couvrir le fond ni
produire au public rien de plausible\,; du reste, écrite aussi bien
qu'il était possible, parce que Fontenelle ne pouvait mal écrire. On
raisonna assez, on conclut peu, on y fit peu de changements. Ce beau
manifeste fut porté deux jours après au conseil de régence. Il y passa
tout d'une voix, comme tout ce que le régent y présentait. Le public ne
fut pas si docile. Il le fut encore moins à la déclaration de la guerre,
qui suivit de près le manifeste contre l'Espagne. Cela ne servit qu'à
montrer quelle était la disposition de la nation\,; mais comme rien
n'était organisé, et que ceux qui auraient voulu brouiller se trouvaient
étourdis et effrayés du lit de justice des Tuileries et du coup de
tonnerre tombé tôt après sur le duc et la duchesse du Maine et sur
l'ambassadeur d'Espagne, tout se borna à une fermentation qui ne put
faire peur au gouvernement. Le traité de la quadruple alliance fut
imprimé bientôt après, qui ne trouva point d'approbateurs. L'Angleterre
déclara en même temps la guerre à l'Espagne, et la Hollande ne tarda pas
à accéder à la quadruple alliance, c'est-à-dire de la France,
l'empereur, l'Angleterre et la Hollande. Il ne laissa pas de paraître
une lettre du roi d'Espagne, fabriquée à Paris, très offensante pour M.
le duc d'Orléans, et qui tout aussitôt se trouva fort répandue à Paris
et dans les provinces, tandis que le roi d'Espagne ignorait ce que
c'était, ainsi que toute l'Espagne. Elle fut incontinent après suivie
d'une autre pièce, faite dans quelque grenier de Paris, pour essayer
d'exciter des troubles à l'occasion de la guerre contre l'Espagne, de
l'indisposition générale contre l'administration des finances, et des
partis pour et contre la constitution, où les moeurs et la conduite du
régent n'étaient pas épargnées. Elle portait le faux nom de Déclaration
du roi catholique du 25 décembre 1718. Le parlement, qui se souvenait
amèrement du dernier lit de justice, et qui en même temps en tremblait
encore, n'osa demeurer dans le silence sur ce second libelle, comme il
avait fait sur le premier, mais aussi se contenta-t-il de supprimer
comme séditieuse et fausse une pièce qui méritait les plus grandes
rigueurs de la justice. M. le duc d'Orléans méprisa également la pièce
et le jugement du parlement\,; aussi ne fit-elle aucune fortune.

Il y eut un grand incendie à Lunéville. Le duc de Lorraine y avait bâti
un beau et grand château qu'il avait bien meublé et fort orné. Presque
tout le château et tous les meubles furent brûlés.

Le czar découvrit une grande conspiration contre lui et contre toute sa
famille. Il y eut force personnes arrêtées, quelques-unes punies de
mort, plusieurs reléguées en Sibérie, d'autres confinées en diverses
prisons.

Charles XII, roi de Suède, de la maison palatine, dont les exploits et
les merveilles avaient étonné et effrayé l'Europe et ruiné radicalement
ses États, fut tué la nuit du 11 au 12 décembre devant Frédéricshall en
Norvège, appartenant au roi de Danemark, dont il faisait opiniâtrement
le siège à la tête de dix-huit cents à deux mille hommes. Il était allé
la nuit aux travaux avec un aide de camp et un page pour toute suite, et
regardant, au clair de la lune, entre deux gabions, un boulet perdu lui
fracassa le menton et l'épaule, et le tua roide. Il n'avait que
trente-sept ans et n'avait point été marié.

Ce funeste accident enleva un héros à l'Europe et à la Suède un
fléau\footnote{Voyez la Note publiée à la fin du t. XIV.}. Le roi son
père en avait été un obscur, qui avait désolé son royaume, ruiné les
lois, abattu le sénat, anéanti l'ancienne noblesse avec tout l'artifice
et l'acharnement des tyrans les plus détestés. Aussi mourut-il jeune et
empoisonné dans de longues et cruelles douleurs. La fin du roi, son fils
parut aux Suédois une autre délivrance, dont ils surent profiter pour se
relever de leur dégradation domestique, en attendant que les années et
la suite des temps d'un gouvernement plus sage prit relever les affaires
du dehors, qui pour le présent paraissaient sans ressource. Ils
commencèrent par se remettre en possession de leur droit d'élire leurs
rois qu'ils avaient perdu d'effet, il y avait près d'un siècle, et
depuis par une renonciation expresse que le père du roi qui venait de
mourir leur avait extorquée.

Charles XII, unique mâle de sa branche, avait eu deux soeurs. L'aînée,
qui était morte veuve du duc de Holstein, tué en une des premières
batailles du roi du Suède, avait laissé des enfants, dont l'aîné duc de
Holstein était au siège de Frédéricshall. Ulrique, l'autre soeur, avait
épousé le fils du landgrave de Hesse qui était aussi à ce siège. C'est
le même qui servit longtemps dans les troupes de Hollande, qui fit
contre la France toute la guerre qui a fini par la paix d'Utrecht, qui
perdit en Italie un grand combat contre Médavy quelques jours après la
bataille de Turin, et qui commandait l'armée que le maréchal de Tallard
battit à Spire. Cette mort du roi de Suède combla la grandeur naissante
de la Russie. Le duc de Holstein, comme fils de la soeur aînée,
prétendait succéder à la couronne de Suède\,; le prince de Hesse aussi,
comme mari de l'unique soeur vivante. Tous deux avaient leur parti, mais
la jeunesse du duc de Holstein et la mort de sa mère lui portèrent un
grand préjudice, peut-être encore plus l'ancienne haine des deux
couronnes du nord. Il était de même maison que le roi de Danemark, mais
de deux branches presque toujours brouillées sur l'administration dés
États qu'elles avaient en commun.

Cette source de division entre elles ne put rassurer les Suédois, dont
l'armée voulut proclamer le prince de Hesse. Il brusqua sur-le-champ une
trêve avec les Danois, et se rendit au plus vite à Stockholm où peu de
jours après l'élection fut rétablie, et la princesse Ulrique élue reine,
sans faire mention du prince de Hesse son époux. En même temps le
pouvoir de la reine fut tellement limité qu'il ne lui en resta que
l'ombre. Tout l'exercice et l'autorité en fut transmis au sénat, et aux
quatre ordres des états généraux de la nation plus entièrement et avec
beaucoup plus de précautions qu'autrefois. Il est vrai, pour le dire ici
tout de suite, qu'ils accordèrent quelque temps après aux prières de
leur reine de lui associer son époux, mais ils ne le firent qu'avec les
mêmes précautions contre son autorité et contre la succession, et ils se
sont depuis si bien soutenus dans cette sage jalousie qu'il n'est roi de
Pologne, ni doge plus entravé qu'il l'est demeuré.

Trois mois après l'élection de la reine de Suède, le baron de Gœrtz,
dont il a été assez parlé ci-devant sur les affaires étrangères, paya
chèrement l'entière confiance que le roi de Suède avait en lui depuis
plusieurs années. La haine que la ruine de la Suède y avait allumée
contre le gouvernement du feu roi de Suède tomba sur son principal
ministre, dont la fortune, les biens, les hauteurs avaient excité
l'envie. Il fut accusé de malversations bien ou mal fondées\,; il fut
arrêté, son procès lui fut fait, et il eut la tête coupée\,; et le baron
Van der {[}Nath{]}, impliqué dans la même affaire, fut condamné et mis
en prison perpétuelle.

M. le duc d'Orléans, qui avait fait entrer depuis quelque temps M. le
duc de Chartres au conseil de régence et au conseil de guerre sans voix,
la lui donna. Il parut qu'il s'en repentit, en l'entendant opiner, bien
des fois. Saint-Nectaire fut nommé ambassadeur en Angleterre et pressé
de se rendre à Hanovre où était le roi Georges. Quand il demanda ses
instructions, l'abbé Dubois lui répondit sans détour de n'en point
attendre de lui, mais de les prendre des ministres du roi Georges, et
d'être bien exact à s'y conformer. Ainsi les Anglais nous gouvernaient
sans voile, et par l'abbé Dubois le régent leur était aveuglément
soumis. En Hollande, Morville avait le même ordre. Tous deux s'y
conformèrent très exactement\,; les autres ministres au dehors eurent
les mêmes ordres.

Broglio, qui n'avait pas servi depuis la défaite du maréchal de Créqui à
Consarbruck, et que le crédit de Bâville, son beau-frère, avait fait
lieutenant général et commandant en Languedoc pour y être, lui-même
Bâville, le maître absolu et sans contradiction, comme il le fut bien
des années, s'avisa de demander, sur les bruits de guerre, le bâton de
maréchal de France à M. le duc d'Orléans, sous le beau prétexte qu'il
était le plus ancien lieutenant général. Le régent se mit à rire, et lui
dit que M. de Lauzun l'était avant lui. Une plaisanterie de M. de Lauzun
avait donné lieu à cette demande qui fut alors très justement et très
unanimement moquée, mais qui, toute ridicule qu'elle fût, eut son effet
dans la suite. La guerre donna lieu à des bruits d'une promotion de
maréchaux de France, parce que le duc de Berwick était le seul d'entre
ceux qui l'étaient, en état de servir. Le monde en nomma à son gré de
toutes les sortes et plusieurs fort étranges. Cela donna lieu au duc de
Lauzun, toujours prêt aux malices, de les désarçonner tous par un
sarcasme, bien plus dangereux en ces occasions-là que les plus mauvais
offices. Il alla donc trouver le régent, et, de ce ton bas, modeste et
doux, qu'il avait si bien fait sien, il lui représenta qu'au cas qu'il y
eût une promotion de maréchaux de France comme le voulait le public, et
qu'il en fît d'inutiles, de vouloir bien se souvenir qu'il était depuis
bien des années le premier des lieutenants généraux. M. le duc
d'Orléans, qui était l'homme du monde qui sentait le mieux le sel et la
malignité, se mit à éclater de rire, et lui promit, qu'au cas qu'il
exposait il ne serait pas oublié. Il en fit après le conte à tout le
monde, dont les prétendus candidats se trouvèrent bien fâchés, et
Broglio affublé de tout le ridicule que M. de Lauzun avait prétendu
donner. Mais le rare est que ce qui lui attira la déraison publique
alors le fit maréchal de France cinq ans après\,; il est vrai que la
dérision fut pareille, mais il le fut.

En Languedoc, où le crédit et l'intérêt de Bâville l'avait mis et
soutenu après une longue oisiveté, on était fort las de lui. Le mépris
s'y joignit, les sottises qu'il fit au passage du prince royal de
Danemark le pensèrent perdre, comme on l'a vu en son lieu. Enfin, le
crédit de la jadis belle duchesse de Roquelaure, et l'embarras que faire
de son mari après sa triste déconfiture des lignes de Flandre, avaient
fait rappeler Broglio et mettre Roquelaure en Languedoc. De retour à
Paris, il y languit dans l'obscurité et arriva à une longue et saine
vieillesse, lorsque son second fils, qui fut depuis maréchal de France
et bien pis encore, se trouva assez à portée de M. le Duc, premier
ministre, et de ce qui le gouvernait, pour faire valoir la primauté de
lieutenant général de son père, et leur faire accroire que c'était
obliger tous les officiers généraux que le faire maréchal de France.

Par cette qualité, Broglio voulut comme que ce fût illustrer sa famille
dans l'avenir, laquelle, en effet, en avait grand besoin, tandis que son
frère aîné, pétri d'envie et de haine, déplorait, disait-il, cette
sottise et un ridicule dont son pauvre père se serait bien passé. En
effet, il fut complet de tous points, et, pour qu'il n'y en manquât
aucun, il fut remarqué que La Feuillade, qui avait très peu servi avant
Turin et point du tout depuis, et le duc de Grammont, qui furent tous
deux maréchaux de France en la même promotion, n'étaient entrés tous
deux dans le service qu'au siège de Philippsbourg, fait par Monseigneur
en 1688, c'est-à-dire treize ans complets depuis que Broglio l'eut
quitté, c'est-à-dire cessa d'être employé, n'étant que maréchal de camp.

Beaucoup de régiments de gens distingués et plusieurs officiers généraux
eurent ordre de se rendre à Bayonne pour servir contre l'Espagne sous
Berwick, à qui le roi d'Espagne ne pardonna jamais. M. le prince de
Conti obtint d'être fait lieutenant général, de servir dans l'armée du
duc de Berwick et d'y commander la cavalerie. Il s'y montra étrangement
dissemblable à M. son père et au sang de Bourbon, jusque-là que toutes
les troupes, jusqu'aux soldats n'en purent retenir leur scandale. Sa
conduite d'ailleurs ne répara rien, et jusqu'à beaucoup d'esprit qu'il
avait lui tourna à malheur. Il eut cent cinquante mille livres de
gratification et beaucoup de vaisselle d'argent en présent. Il se fit
encore payer ses postes, qu'il courut avec une petite partie de sa suite
aux dépens du roi, tant en allant qu'en revenant. Ce n'est pas que le
roi n'eût acheté et payé pour lui gouvernement et régiment, et qu'il ne
se fût fait lourdement partager d'actions de la banque de Law qui ne lui
coûtèrent rien. On rit un peu de l'invention de se faire payer les
postes et de la dispute là-dessus qui retarda son départ de dix ou douze
jours. À la fin son opiniâtreté l'emporta. Gouvernements et régiments
{[}furent{]} achetés par le roi pour les princes du sang, les
appointements de ces gouvernements triplés pour eux, pensions énormes et
gratifications pareilles, sans nombre et sans mesure\,; des monts d'or
au Mississipi, dont tout le fonds donné et payé par le roi\,; les
princesses du sang, femmes et filles, traitées pareillement, excepté les
seuls enfants de M. le duc d'Orléans, Madame et M\textsuperscript{me} sa
femme, laquelle pourtant sur la fin en tira quelque parti, mais pour
elle seule.

Un mois ou six semaines après cette rafle de M. le prince de Conti,
M\textsuperscript{lle} de Charolais eut une augmentation de pension de
quarante mille livres, et M\textsuperscript{me} de Bourbon, sa soeur,
religieuse à Fontevrault, une de dix mille francs.

Le grand prieur, pour qui M. le duc d'Orléans avait un faible, même un
respect fort singulier, comme l'impie et le débauché le plus constant et
le plus insigne qu'il eût jamais vu, après la tolérance de plusieurs
entreprises de princes du sang qui furent enfin tout à fait arrêtées,
fut au moins traité en prince du sang quant aux libéralités. J'ai oublié
de dire que, environ un an ou quinze mois après son retour, il voulut
entrer au conseil de régence, et j'eus vent que M. le duc d'Orléans y
consentirait. Je lui en parlai, et son embarras me montra que l'avis que
j'avais eu était bon. Je lui montrai l'infamie d'admettre au conseil de
régence un homme sans moeurs, sans honneur, sans principe, sans
religion, qui depuis trente ans ne s'était couché qu'ivre, qui ne voyait
que des brigands, des débauchés comme lui, des gens sans aveu et sans
nom\,; un homme déshonoré sur le courage et le pillage, qui avait volé
son frère, et capable de prendre dans les poches\,; enfin un homme que
ses infamies avaient tenu exilé une partie de sa vie, et nouvellement
les dix dernières années du feu roi. M. le duc d'Orléans ne put
disconvenir de pas un de ces articles, y ajouta même, voulut tourner la
chose en plaisanterie, puis me dit que je prenais l'alarme chaude, parce
que le grand prieur voudrait me précéder au conseil. Je lui répondis que
le grand prieur était bien assez insolent pour le prétendre, et lui
régent assez faible pour le souffrir, mais, comme que ce fût, qu'il
pouvait s'assurer que ni moi ni pas un autre due ne céderions au grand
prieur. Le régent, au lieu de se fâcher, se remit à plaisanter, mais en
évitant toujours d'articuler rien de certain.

L'objet de cette façon de répondre était premièrement de ne se point
engager contre ce qu'il voulait faire, puis de me donner à croire que ce
qu'il me répondait n'était que pour se divertir à m'impatienter, comme
il lui arrivait quelquefois\,; mais je le connaissais trop pour m'y
méprendre. Je sentis que le parti était pris, mais que l'embarras de
l'exécution la différait. Je profitai du temps, et tout de suite
j'informai de cette conversation et de ce que je pressentais les
maréchaux de Villeroy, Harcourt et Villars, et d'Antin, parce que ces
deux derniers venaient rapporter à la régence les affaires de leurs
conseils. Je n'eus pas de peine à les exciter. Nous convînmes qu'ils
parleraient tous quatre séparément au régent en même sens que j'avais
fait, et qu'ils finiraient par lui déclarer que, dans le moment que le
grand prieur entrevoit dans le cabinet du conseil pour y prendre place,
nous en sortirions tous, et lui remettrions les nôtres. Ils exécutèrent
très bien et très fortement ce qui avait été résolu, et mirent le régent
dans le plus grand embarras du monde.

Je vins après eux et lui demandai de leurs nouvelles. Je vis un homme
rouge bien plus qu'à son ordinaire, empêtré, et qui n'avait plus envie
de plaisanter. J'avais su du maréchal de Villeroy qu'il l'avait bourré
et imposé, des deux autres maréchaux qu'ils l'avaient extrêmement
embarrassé, et de tous les quatre que la déclaration de leur retraite
l'avait mis aux abois\,; qu'il avait tâché de leur persuader qu'ils
prenaient l'alarme mal à propos\,; leur avait fait tout plein de
caresses, assuré qu'il n'était point question de cela, mais sans jamais
leur dire que cela ne serait point. Chacun lui répéta sa protestation de
retraite si cela arrivait jamais, pour le lui mieux inculquer.

Le régent me dit que ces messieurs lui avaient parlé fort vivement\,;
puis me donna du même verbiage dont il les avait servis, sans me parler
de la retraite. Je lui répondis froidement qu'il devait savoir
maintenant dans quelle estime le grand prieur était dans le monde, quand
il l'aurait pu ignorer auparavant, depuis ce que ces messieurs lui en
avaient dit\,; qu'il me taisait le plus important de leur conversation,
quoiqu'il pût bien juger que je ne l'ignorais pas\,; que c'était
maintenant à lui à peser le mérite du grand prieur contre celui du
maréchal d'Harcourt si universellement reconnu, contre ses emplois et
ceux du maréchal de Villeroy pendant toute sa vie, contre ceux du
maréchal de Villars, tous trois si magnifiquement traités dans le
testament du feu roi, si grandement établis et si fort considérés dans
le monde\,; que je ne lui parlais plus de leur dignité à la façon dont
il s'en était joué, mais qui à force d'injures pouvaient s'en souvenir à
propos\,; que je me contentais du parallèle de ces trois hommes avec le
grand prieur, et de le supplier comme son serviteur, faisant abstraction
de tout autre intérêt que le sien, de réfléchir un peu sur l'effet que
ferait dans le monde le troc qu'il ferait au conseil de régence de ces
trois hommes-là pour y mettre un bandit, un homme de sac et de corde, à
qui, depuis tant d'années, il n'y avait pas un honnête homme qui voulût
lui parler.

Jamais je ne vis homme plus embarrassé que M. le duc d'Orléans le fut de
ce discours, que je lui fis lentement, tranquillement, posément, et
qu'il écouta sans m'interrompre. Il demeura court, et le silence dura un
peu. «\,Monsieur, lui dis-je\,; en le rompant le premier, nous savons
tous le respect que nous devons à un petit-fils de France et à un régent
du royaume\,; ainsi nos représentations seront toujours parfaitement
respectueuses. Nous sommes aussi parfaitement éloignés de nous écarter
assez de notre devoir pour oser vous faire une menace\,; mais rendre
compte à Votre Altesse Royale d'une résolution prise, et très fermement,
et des raisons qui nous engagent à la prendre, est un respect que nous
vous rendons pour que, le cas avenant, vous ne soyez pas surpris de
l'exécution. Ayez donc la bonté de ne vous pas méprendre en croyant
qu'on veut vous faire peur de vous remettre nos emplois à l'instant, et
que, le cas arrivant, nous nous en garderions bien\,; mais
persuadez-vous au contraire que nous le ferons, ainsi que ces messieurs
et moi avons eu l'honneur de vous le dire\,; que nous nous
déshonorerions autrement\,; que, de plus, nous nous en sommes donné
réciproquement parole positive, et que, quoi qu'il en pût arriver, nous
l'exécuterons, avec résolution de ne rien écouter, pas pour une minute,
et de rendre le public, même le pays étranger, juge de la préférence.\,»

Cette réplique, prononcée avec le même sang-froid, acheva d'accabler M.
le duc d'Orléans. Il demeura encore quelques moments en silence, puis me
dit que c'était bien du bruit pour une imagination. «\,Si cela est,
monsieur, repris-je, mettez-vous à votre aise et nous aussi\,: promettez
à chacun de ces messieurs et à moi, et donnez clairement et nettement
votre parole que jamais le grand prieur n'entrera dans le conseil de
régence, et trouvez bon en même temps que nous disions que vous nous
l'avez promis.\,» Il fit quelques pas, car nous étions debout, mais sans
marcher, puis revint à moi et me dit\,: «\,Mais volontiers, je vous la
donne, et vous le pouvez dire à ceux qui m'ont parlé\,: --- Non pas,
s'il vous plait, monsieur\,; mais, si vous le trouvez bon, je leur dirai
de votre part de la venir prendre de vous-même.\,»

Il rageait à part soi et ne le voulait pas montrer pour nous persuader
qu'il n'avait jamais songé à mettre le grand prieur dans le conseil,
mais à qui il l'avait promis et dont il ne savait comment se défaire. Il
voulut donc me faire entendre qu'il n'était pas besoin qu'il reparlât à
ces messieurs, qui ne pourraient, sans m'offenser, ne pas ajouter foi à
ce que je leur dirais de sa part. Je répondis qu'en telles matières je
ne m'offensais pas si aisément, mais qu'il me permettrait de lui dire
avec une respectueuse franchise qu'eux et moi désirions sûreté entière,
qui ne se pouvait trouver pour nous que dans ce que je lui proposais.
«\,Voilà un homme bien entêté et bien opiniâtre,\,» me dit-il\,; puis
tout de suite, avec un peu d'air de dépit\,: «\,Oh bien, ajouta-t-il, je
la leur donnerai s'ils veulent\,;» puis changea tout court de
conversation.

Après qu'elle eut un peu duré, et que je le vis remis avec moi à son
ordinaire, je pris congé et j'allai ce soir-là et le lendemain rendre
compte à d'Antin et aux trois maréchaux de ce que je venais d'emporter.
Tous me louèrent fort d'avoir insisté sur la parole à donner à chacun
d'eux, et sur la permission de n'en pas faire un mystère. Je m'en
applaudis plus qu'eux parce que j'évitai par là d'en être la dupe, de
voir entrer le grand prieur au conseil et M. le duc d'Orléans nier sa
parole. Ces quatre ducs ne tardèrent pas à aller recevoir la parole
positive de M. le duc d'Orléans, qui la leur donna très nette d'un air
aisé, et qui après leur voulut persuader qu'elle ne lui coûtait rien sur
une chose qu'il n'avait jamais pensé à faire. Ces messieurs prirent tout
pour bon, mais le supplièrent, en se retirant, de n'oublier pas qu'ils
avaient sa parole. On peut juger que nous n'en gardâmes pas longtemps le
secret avec la permission que j'en avais arrachée. Cela mit le grand
prieur aux champs, et M. le duc d'Orléans en proie à ses reproches, qui
en fut quitte pour un peu d'argent, avec quoi il fit taire le grand
prieur, lequel, se voyant la porte du conseil tout à fait fermée, fut
encore bien aise d'en tirer ce parti. Revenons maintenant où nous en
étions, après cet oubli réparé.

Le frère du roi de Portugal, lassé d'être depuis quelques mois à Paris
logé chez l'ambassadeur de cette couronne, sans distinction et sans
recevoir aucune honnêteté du roi, du régent, ni du monde à leur exemple,
songea à se raccommoder avec le roi son frère, qui lui envoya de
l'argent pour revenir à sa cour. Ce prince, toutefois, n'osa s'y fier et
s'en retourna à Vienne. Il avait fait deux campagnes en Hongrie avec
réputation.

Le duc de Saint-Aignan arriva d'Espagne et entra au premier conseil de
régence qui se tint après.

Saint-Germain-Beaupré, ennuyeux et plat important qui n'avait jamais été
de rien, mourut chez lui. Il avait cédé son petit gouvernement de la
marche à son fils, homme fort obscur, en le mariant à la fille de
Doublet de Persan, conseiller au parlement, qui trouva le moyen de
percer partout et d'être du plus grand monde.

Le prince d'Harcourt mourut aussi à Monjeu chez sa belle-fille, après
avoir mené une longue vie de bandit et presque toujours loin de la cour
et de Paris. Il en a été ici parlé ailleurs assez pour n'avoir rien à y
ajouter.

La marquise de Charlus, soeur de Mezières et mère du marquis de Lévi,
devenu depuis duc et pair, mourut riche et vieille. Elle était toujours
faite comme une crieuse de vieux chapeaux, ce qui lui fit essuyer
maintes avanies parce qu'on ne la connaissait pas, et qu'elle trouvait
fort mauvaises. Pour se délasser un moment du sérieux, je rapporterai
une aventure d'elle d'un autre genre.

Elle était très avare et grande joueuse. Elle y aurait passé les nuits
les pieds dans l'eau. On jouait à Paris les soirs gros jeu au lansquenet
chez M\textsuperscript{me} la princesse de Conti, fille de M. le Prince.
M\textsuperscript{me} de Charlus y soupait un vendredi, entre deux
reprises, avec assez de monde. Elle n'y était pas mieux mise
qu'ailleurs, et on portait en ce temps-là des coiffures qu'on appelait
des commodes, qui ne s'attachaient point et qui se mettaient et ôtaient
comme les hommes mettent et ôtent une perruque et un bonnet de nuit, et
la mode était que toutes les coiffures de femmes étaient fort hautes.
M\textsuperscript{me} de Charlus était auprès de l'archevêque de Reims,
Le Tellier. Elle prit un oeuf à la coque qu'elle ouvrit, et, en
s'avançant après pour prendre du sel, mit sa coiffure en feu, d'une
bougie voisine, sans s'en apercevoir. L'archevêque, qui la vit tout en
feu, se jeta à sa coiffure et la jeta par terre. M\textsuperscript{me}
de Charlus, dans la surprise et l'indignation de se voir ainsi décoiffée
sans savoir pourquoi, jeta son oeuf au visage de l'archevêque, qui lui
découla partout. Il ne fit qu'en rire, et toute la compagnie fut aux
éclats de la tête grise, sale et chenue de M\textsuperscript{me} de
Charlus et de l'omelette de l'archevêque, surtout de la furie et des
injures de M\textsuperscript{me} de Charlus qui croyait qu'il lui avait
fait un affront et qui fut du temps sans vouloir en entendre la cause,
et après de se trouver ainsi pelée devant tout le monde. La coiffure
était brûlée, M\textsuperscript{me} la princesse de Conti lui en fit
donner une, mais avant qu'elle l'eût sur la tête on eut tout le temps
d'en contempler les charmes et elle de rognonner toujours en furie. M.
de Charlus, son mari, la suivit trois mois après. M. de Lévi crut
trouver des trésors\,; il y en avait eu, mais ils se trouvèrent envolés.

Les jeux de hasard furent de nouveau sévèrement défendus\footnote{Le
  marquis d'Argenson, dans la partie de ses Mémoires qui est encore
  inédite, donne quelques détails sur la fureur du jeu pendant la
  régence\,: «\,J'ai vu, au commencement de la régence, s'introduire une
  irruption de jeux universelle\,; du moins ornait-elle Paris alors\,;
  car on voyait dans les cours et sur le devant des portes des pots à
  feux qui ornaient Paris. M. le duc d'Orléans fit cesser cela
  partout.\,»}.

M. le duc d'Orléans permit au président de Blamont de revenir du lieu de
son exil en une de ses terres\,; et il accorda au grand prévôt la
survivance de sa charge pour son fils, qui n'avait que six ans, et donna
quelques petites pensions. Il ordonna aussi une grande levée de milices
pour suppléer, mêlées avec quelques troupes, aux garnisons des places en
temps de guerre.

\hypertarget{chapitre-vii.}{%
\chapter{CHAPITRE VII.}\label{chapitre-vii.}}

1719

~

{\textsc{Quatre pièces, soi-disant venues d'Espagne, assez faiblement
condamnées par le parlement\,; discutées.}} {\textsc{- Prétendue lettre
circulaire du roi d'Espagne aux parlements.}} {\textsc{- Prétendu
manifeste du roi d'Espagne adressé aux trois états.}} {\textsc{-
Prétendue requête des états généraux de France au roi d'Espagne.}}
{\textsc{- Prétendue lettre du roi d'Espagne au roi.}} {\textsc{-
\emph{Philippiques}. La Peyronie premier chirurgien du roi.}} {\textsc{-
Belle entrée de Stairs, ambassadeur d'Angleterre.}} {\textsc{- Ses
vaines entreprises, et chez le roi et à l'égard des princes du sang.}}
{\textsc{- Mort de M\textsuperscript{me} de Seignelay.}} {\textsc{- La
bibliothèque de feu M. Colbert achetée par le roi.}} {\textsc{-
Archevêque de Malines\,; quel.}} {\textsc{- L'empereur lui impose
silence sur la constitution.}} {\textsc{- Sage et ferme conduite du roi
de Sardaigne sur la même matière.}} {\textsc{- Le P. Tellier exilé à la
Flèche, où il meurt au bout de six mois.}} {\textsc{- Ingratitude
domestique des jésuites.}} {\textsc{- Promotion d'officiers généraux.}}
{\textsc{- Duc de Mortemart vend au duc de Saint-Aignan le gouvernement
du Havre.}} {\textsc{- Dix mille livres de pension au vicomte de Beaune,
et vingt mille livres au duc de Tresmes, au lieu de son jeu, qui se
rétablit après, et la pension lui demeure.}} {\textsc{- L'abbaye de
Bourgueil à l'abbé Dubois.}} {\textsc{- Mariage de M. de Bournonville
avec M\textsuperscript{lle} de Guiche.}} {\textsc{- Profusion au grand
prieur.}} {\textsc{- Mariage du prince électoral de Saxe déclaré avec
une archiduchesse.}} {\textsc{- Le roi Jacques en Espagne.}} {\textsc{-
Retour de Turin et grâce faite à M. de Prie.}} {\textsc{- Rémond\,;
quel\,; son caractère.}} {\textsc{- Mimeur\,; quel\,; son caractère\,;
sa mort.}} {\textsc{- Mort et caractère de Térat.}} {\textsc{- La
Houssaye, conseiller d'État, lui succède.}} {\textsc{- Mort d'un fils de
l'électeur de Bavière, élu évêque de Munster.}} {\textsc{- Mort et
caractère de Puysieux.}} {\textsc{- Belle-Ile s'accommode lestement de
son gouvernement d'Huningue.}} {\textsc{- Cheverny a sa place de
conseiller d'État d'épée.}}

~

Le parlement rendit, le 4 février, un arrêt qui se contente de supprimer
quatre fort étranges pièces et qui défend de les imprimer, vendre ou
débiter, sous peine d'être poursuivis comme perturbateurs du repos
public et criminels de lèse-majesté. La première intitulée\,:
\emph{Copie d'une lettre du roi Catholique, écrite de sa main, que le
prince de Cellamare, ambassadeur, avait ordre de présenter au roi Très
Chrétien, du} 3 \emph{septembre} 1718. La seconde intitulée\,:
\emph{Copie d'une lettre circulaire du roi d'Espagne à tous les
parlements de France, datée du} 4 \emph{septembre} 1718. La troisième
intitulée\,: \emph{Manifeste du roi Catholique adressé aux trois états
de la France, du} 6 \emph{septembre} 1718. La quatrième intitulée\,:
\emph{Requête présentée au roi Catholique au nom des trois états de la
France}.

Il ne fallait pas être bien connaisseur pour s'apercevoir que pas une de
ces quatre pièces n'était venue d'Espagne. On ne pouvait les avoir
trouvées dans les valises de l'abbé Portocarrero ni de son compagnon, ni
dans les papiers de Cellamare qui avaient été pris les premiers à
Poitiers, les autres chez l'ambassadeur même, qui, dans la plus
tranquille confiance, ne se défiait de rien et se reposait pleinement
sur ses précautions, quand cet abbé et lui furent arrêtés et leurs
papiers pris, et qui, dans cette entière sécurité, ne les aurait confiés
à personne.

D'Espagne ils ne furent point avoués, quelque colère qui y fût allumée.
Outre que le style était peu digne d'un grand roi, on y était trop
instruit du gouvernement de France, de tous les siècles et de tous les
temps, pour confondre nos parlements d'aujourd'hui avec ce qui très
anciennement s'appelait le parlement de France, qui était l'assemblée
législative de la nation et à qui n'ont jamais ressemblé les états
généraux du royaume, qui ne sont connus que longtemps depuis, et qui
n'ont jamais eu que la voix de remontrance et quelquefois aussi
consultative, mais simplement et seulement quand il a plu aux rois de
les consulter, et limitée de plus à la chose qui faisait la matière de
la consultation et non davantage\,; on n'a pu encore moins confondre ces
anciennes et primordiales assemblées connues sous le nom de parlements
de France, avec les cours de justice si modernement et si fort par
degrés établies telles qu'elles sont aujourd'hui sous le nom de
parlement de Paris, parlement de Toulouse, etc., si modernement, dis-je,
en comparaison de ces anciens parlements de France.

On savait en Espagne, aussi bien qu'en France, que ces anciens
parlements ignoraient les légistes décorés à la fin du nom de
magistrats, qu'ils n'étaient composés que du roi et de ses grands et
immédiats vassaux\,; que là se décidaient en peu de jours les grandes
questions de fief, car la chicane était encore à naître, et cette
infinité de lois et de coutumes locales qui nourrissent et bouffissent
tant de rabats\,; que là se décidait la paix ou la guerre, et là les
moyens de celle-ci et les conclusions de celle-là\,; et que si on y
prenait la résolution de faire la guerre, c'était de l'assemblée même
que l'on partait pour attaquer l'ennemi ou pour défendre les
frontières\,; enfin là même que se proposaient les lois à faire et
qu'elles s'y faisaient quand il en était besoin.

On n'ignorait pas aussi en Espagne quelles sont nos cours judiciaires,
aujourd'hui connues sous le nom de parlements, et que ces cours, égales
entre elles, parfaitement indépendantes les unes des autres, sont
établies par les rois sur certains districts, plus ou moins étendus,
qu'on appelle ressorts, pour y connaître des affaires et des procès de
tous les sujets du roi du district qui leur a été affecté, et pour les
juger suivant les lois et ordonnances des rois et les coutumes des
lieux, au nom du roi, mais sans puissance législative, et seulement
coactive pour l'exécution de leurs arrêts, lesquels toutefois ne
laissent pas d'être cassés au conseil privé du roi, si la partie qui se
prétend mal jugée prouve que l'arrêt prononcé est en contradiction avec
une ou plusieurs des ordonnances des rois qui sont en vigueur\,: par où
il est évident que les parlements ont en ce conseil un supérieur, et
combien mal à propos ils avaient usurpé et s'étaient parés du nom de
cour souveraine, lorsque le feu roi le leur fit rayer avec d'autant plus
de justice\footnote{Voy. Mémoires de Louis XIV (t. Ier, p.~47 et suiv.
  des \emph{Oeuvres de Louis XIV}). Il dit à son fils\,: «\,Il fallait
  par mille raisons, même pour se préparer à la réformation de la
  justice qui en avait tant de besoin, diminuer l'autorité excessive des
  principales compagnies qui, sous prétexte que leurs jugements étaient
  sans appel, et, comme on parle, \emph{souverains et en dernier
  ressort}, ayant pris peu à peu le nom de \emph{cours souveraines}, se
  regardaient comme autant de souverainetés séparées et indépendantes.
  Je fis connaître que je ne souffrirais plus leurs entreprises. »}, que
ces cours ne tiennent leurs charges et leur autorité que du roi, seul
souverain dans son royaume, et ne peuvent prononcer d'arrêts qu'en son
nom. L'Espagne sait aussi bien que la France que ces tribunaux ne sont
compétents que des matières judiciaires, qu'ils ne le sont en aucune
sorte de celles d'État ni de celles du gouvernement, et que toutes les
fois qu'à la faveur des temps de besoins ou de troubles, ils ont essayé
de s'en arroger quelque connaissance, les rois les ont promptement et
souvent rudement repris et renfermés dans leurs bornes judiciaires.
L'Espagne, ainsi que la France, était parfaitement au fait de ce que
sont les enregistrements des édits, déclarations, ordonnances,
règlements que font les rois et des traités de paix.

On ne prend point en Espagne non plus qu'en France le change que ces
compagnies présentent si volontiers en jouant sur la chose et sur le
mot, comme elles ont tâché de faire sur celui de parlement commun à
l'ancien parlement de France, dont on vient de parler, et au parlement
d'Angleterre, qui est l'assemblée qui en représente toute la nation avec
un pouvoir législatif et de l'étendue que tout le monde sait. Les
enregistrements des parlements sont connus en Espagne comme en France
pour ce qu'ils valent intrinsèquement, c'est-à-dire comme n'ayant aucun
trait à ajouter rien à l'autorité du roi, devant laquelle toute autre
disparaît en France\,; mais simplement \emph{ut notum sit}, c'est-à-dire
pour rendre publique et solennellement publique la teneur de la pièce
qui s'enregistre, et pour faire une loi au parlement qui l'enregistre
d'y conformer ses jugements. Que si les rois ont permis les remontrances
aux parlements, chose dont l'usage ou l'exclusion dépend uniquement de
la volonté des rois, ce n'est que pour éviter les surprises et connaître
avec plus de justesse et de réflexion les conséquences du tout ou de
partie de la pièce envoyée pour enregistrer, qui se retire ou qui est
modifiée, si le roi est touché des raisons qui font la matière des
remontrances, ou s'il ne l'est pas, qui s'enregistre, nonobstant une ou
plusieurs remontrances.

À l'égard du rang que les parlements tiennent dans l'État, on le peut
voir plus haut, tome XI, pages 366 et suiv., et on y verra que ces
compagnies n'y en tiennent et n'y en ont jamais tenu, et qu'elles y sont
confondues dans le tiers état, sans jamais avoir fait corps à part. Que
si, dans des temps de troubles, comme dans ceux de la minorité de Louis
XIV et dans quelques autres, ceux qui voulaient troubler se soient
adressés au parlement de Paris, cela ne peut donner à cette compagnie un
droit de se mêler du gouvernement, qu'elle n'a pas\,; cela montre
seulement des gens qui vont à la seule assemblée toujours existante,
mais seulement pour juger des procès, qui la flattent dans sa chimère
d'être les tuteurs des rois, les protecteurs des peuples, le milieu
entre le roi et le peuple\,; des gens qui se veulent parer du nom et de
l'appui du parlement, et le parlement qui saisit les moments de figurer,
de se faire compter et d'essayer de se faire un titre d'autorité et de
puissance, qui s'évanouit avec les troubles dont la fin remet cette
compagnie en règle et dans son état naturel. Il en est en un autre sens
de même des trois dernières régences, les seules qui aient été déclarées
dans le parlement, comme on le pourra voir aux lieux ci-dessus où je
renvoie. Il est donc évident que rien n'était plus inutile au projet de
l'Espagne que d'écrire aux parlements, qui ne sont dans le royaume que
de simples juges supérieurs, dont tout le pouvoir et la fonction n'est
uniquement que de juger les procès, au nom et par l'autorité du roi, de
ceux de ses sujets qui sont dans leur ressort, à quoi ils sont tellement
bornés que c'est une autre cour {[}pour{]} les grands et immédiats
feudataires de la couronne, qui reçoit les hommages qu'ils doivent au
roi de leurs fiefs, ou le seul chancelier au choix des feudataires, mais
dont les hommages sont enregistrés dans cette autre cour, qui est la
chambre des comptes, laquelle aussi examine privativement au parlement
et à toutes autres cours les comptes des comptables du roi, les punit ou
les approuve. Mais M. du Maine et le premier président n'avaient garde
de manquer une si belle occasion de flatter le parlement, de tâcher de
l'engager avec eux, et d'éblouir le monde ignorant de ce vain nom en
telle matière\,; et Cellamare, qui regardait M. et M\textsuperscript{me}
du Maine comme les chefs et l'âme du parti qu'il voulait former, n'avait
garde aussi de s'éloigner en rien de ce qui leur convenait et de ce
qu'ils désiraient.

Le manifeste du roi d'Espagne adressé aux trois états de la France est
de même espèce que la lettre aux parlements. On vient de voir, et on a
vu plus haut, en plusieurs endroits, ce que c'est que les états
généraux, et qu'ils n'ont dans l'État ni puissance ni autorité
quelconque\,; qu'ils ne peuvent s'assembler que par la volonté et la
convocation du roi, ou, s'il est mineur, du régent, pour faire leurs
cahiers de plaintes et de représentations, et répondre uniquement aux
consultations, et non entamer rien au delà, quand il plaît au roi ou au
régent, le roi étant mineur, de leur en faire, et qui les sépare, quand
et comme il lui plaît. L'Espagne ne pouvait donc ignorer ces choses
fondamentales, ni se promettre plus qu'un vain bruit de l'adresse de ce
manifeste\,; mais que peut-on dire de l'adresse de ce manifeste aux
états généraux, qui n'étaient ni assemblés ni même convoqués, et qui,
par conséquent, n'étaient lors qu'un être de raison, puisque les états
généraux n'ont d'existence que lorsqu'ils sont convoqués, et
actuellement assemblés par et sous l'autorité du roi, ou, s'il est
mineur, du régent\,? C'était donc une adresse purement en l'air, qui ne
portait sur rien, et de laquelle il ne se pouvait rien attendre, par
conséquent ridicule, inepte, indigne de la majesté du roi d'Espagne.

Mais il en fut comme des lettres au parlement. Le duc du Maine, à faute
de mieux, voulait du bruit, éblouir, imposer par de grands noms aux
ignorants, qui font le très grand nombre. Cette méthode lui avait réussi
à museler et à se jouer de cette prétendue noblesse qu'il avait enivrée
des charmes de croire figurer et représenter le second ordre de l'État,
qu'il ravala ensuite avec la même facilité, jusqu'à présenter en son
prétendu corps une requête à \emph{nosseigneurs de parlement}, en faveur
de celui qui la mettait à tous usages, et qui enfin osa demander à
n'être jugé contre les princes du sang que par les états généraux qui
n'ont ni pouvoir ni autorité de juger rien. Le duc du Maine n'était pas
en mesure de parler des pairs\,; il y était trop avec le parlement pour
s'adresser ou faire adresser le roi d'Espagne à la noblesse seule ou au
clergé. Il fallut donc supposer des états généraux qui n'existaient
point, et qui, quand ils sont assemblés par et sous l'autorité royale,
comprennent l'un et l'autre avec le tiers état, mais duquel il eut le
soin de distinguer les parlements par cette lettre circulaire dont on
vient de parler.

La plus folle de ces quatre pièces est sans doute \emph{la requête au
roi d'Espagne des états généraux de la France}, qui n'étaient point, qui
n'existaient point, puisqu'ils n'étaient ni assemblés ni convoqués.
C'était donc un fantôme qui parlait en leur nom, et comme un de ces
rôles joués sur les théâtres, par ces héros morts depuis mille ans. La
simple inspection d'une puérilité qui en effet ne pouvait tromper que
des enfants ne permet pas d'imaginer que le cardinal Albéroni pût être
tombé dans des sottises si grossières. Mais tout était bon à M. du Maine
à qui l'aveuglement qu'il avait jeté sur cette prétendue noblesse avait
fait espérer qu'il aurait le même bonheur à infatuer tout le royaume.

À l'égard de la lettre \emph{du roi d'Espagne au roi}, que Cellamare
avait ordre de lui présenter en main propre, qui est une voie usitée
entre souverains de se parler et de se faire des représentations, elle
n'aurait rien contre la vraisemblance, si le style pouvait convenir
entre deux grands monarques. C'est donc la simple lecture de cette pièce
si étrange qui la rend indigne de passer pour venir du roi d'Espagne, et
très digne de l'esprit et de l'éloquence du cabinet de Sceaux. Ces
pièces firent du bruit, et tombèrent bientôt d'elles-mêmes. M. le duc
d'Orléans les méprisa, et n'en fut point affecté.

Il n'en fut pas de même d'une pièce de vers qui parut presque dans le
même temps sous le nom de \emph{Philippiques}, et qui fut distribuée
avec une promptitude et une abondance extraordinaire. La Grange, élevé
autrefois page de lime la princesse de Conti fille du roi, eu fut
l'auteur, et ne le désavouait pas. Tout ce que l'enfer peut vomir de
vrai et de faux y était exprimé dans les plus beaux vers, le style le
plus poétique, et tout l'art et l'esprit qu'on peut imaginer. M. le duc
d'Orléans le sut et voulut voir ce poème, car la pièce était longue, et
n'en put venir à bout, parce que personne n'osa la lui montrer.

Il m'en parla plusieurs fois, et à la fin il exigea si fort que je la
lui apporterais, qu'il n'y eut pas moyen de m'en défendre. Je la lui
apportai donc, mais de la lui lire, je lui déclarai que je ne le ferais
jamais. Il la prit donc, et la lut bas debout dans la fenêtre de son
petit cabinet d'hiver où nous étions. Il la trouva tout en la lisant
telle qu'elle était, car il s'arrêtait de fois à autre pour m'en parler
sans en paraître fort ému. Mais tout d'un coup, je le vis changer de
visage et se tourner à moi les larmes aux yeux, et près de se trouver
mal. «\,Ah\,! me dit-il, c'en est trop, cette horreur est plus forte que
moi.\,» C'est qu'il était à l'endroit où le scélérat montre M. le duc
d'Orléans dans le dessein d'empoisonner le roi, et tout prêt d'exécuter
son crime. C'est où l'auteur redouble d'énergie, de poésie,
d'invocations, de beautés effrayantes et terribles, d'invectives, de
peintures hideuses, de portraits touchants de la jeunesse, de
l'innocence du roi et des espérances qu'il donnait, d'adjurations à la
nation de sauver une si chère victime de la barbarie du meurtrier\,; en
un mot tout ce que l'art a de plus délicat, de plus tendre, de plus fort
et de plus noir, de plus pompeux et de plus remuant. Je voulus profiter
du morne silence où M. le duc {[}d'Orléans{]} tomba pour lui ôter cet
exécrable papier, mais je ne pus en venir à bout\,; il se répandit en
justes plaintes d'une si horrible noirceur, en tendresse sur le roi,
puis voulut achever sa lecture, qu'il interrompit encore plus d'une fois
pour m'en parler. Je n'ai point vu jamais homme si pénétré, si
intimement touché, si accablé d'une injustice si énorme et si suivie.
Moi-même, je m'en trouvai hors de moi. À le voir, les plus prévenus,
pourvu qu'ils ne le fussent que de bonne foi, se seraient rendus à
l'éclat de l'innocence et de l'horreur du crime dans laquelle il était
plongé. C'est tout dire que j'eus peine à me remettre, et que j'eus
toutes les peines du monde à le remettre un peu.

Ce La Grange, qui de sa personne ne valait rien en quelque genre que ce
fût, mais qui était bon poète, et n'était que cela, et n'avait jamais
été autre chose, s'était par là insinué à Sceaux, où il était devenu un
des grands favoris de M\textsuperscript{me} du Maine. Elle et son mari
en connurent la vie, la conduite, les moeurs et la mercenaire
scélératesse. Ils la surent bien employer. Il fut arrêté peu après et
envoyé aux îles de Sainte-Marguerite, d'où à la fin il obtint de sortir
avant la fin de la régence. Il eut l'audace de se montrer partout dans
Paris, et, tandis qu'il y paraissait aux spectacles et dans tous les
lieux publics, on eut l'impudence de répandre que M. le duc d'Orléans
l'avait fait tuer. Les ennemis de M. le duc d'Orléans et ce prince ont
été également infatigables\,; les premiers en toutes les plus noires
horreurs, lui à la plus infructueuse clémence, pour ne lui pas donner un
nom plus expressif.

Maréchal, premier chirurgien du roi, dont le fils avait la survivance,
mais si dégoûté du métier, qu'il ne voulait plus l'exercer, s'accommoda
de sa charge avec La Peyronie, fort grand chirurgien, qui parut depuis
grand et habile courtisan, et qui fit grand bruit à la cour et dans le
monde. Il avait beaucoup d'esprit et d'ambition.

Stairs fit une superbe entrée. Soit ignorance que les ambassadeurs
n'entrent à Paris dans la cour du roi qu'à deux chevaux, ou entreprise,
ses carrosses, attelés de huit chevaux, prétendirent entrer. La
contestation fut vive, mais enfin il fallut entrer à deux chevaux, et
dételer les six autres. Les jours suivants il alla voir les princes du
sang suivant l'usage. M. le prince de Conti lui rendit sa visite\,; mais
ne voyant pas Stairs au bas de son escalier, pour le recevoir, comme
c'est la règle, il attendit un peu dans son carrosse, puis le fit
tourner, et alla au Palais-Royal se plaindre de cette innovation. Stairs
avait déjà envoyé demander une audience à M\textsuperscript{me}s les
princesses de Conti, à qui M. le duc d'Orléans manda de ne le point
recevoir qu'il n'eût reçu les princes du sang comme il devait. M. le Duc
suspendit aussi la visite qu'il devait lui rendre. Stairs prétendit que
la réception au bas du degré n'était pas dans son protocole. Il s'en fit
approuver par les autres ambassadeurs, et blâmer par eux d'en avoir trop
fait pour M. le duc de Chartres, qui, quoique premier prince du sang, ne
devait pas être traité différemment des autres princes du sang. Enfin au
bout de deux mois de lutte et de négociations, M. le Duc et M. le prince
de Conti rendirent séparément leur visite à Stairs, qui les reçut au bas
de son degré. L'audace de cet ambassadeur d'Angleterre, qu'il portait
également peinte dans sa personne, dans ses discours et dans ses
actions, avait révolté toute la France. On a vu en son lieu que le
régent, d'abord par Canillac et par le duc de Noailles, puis par l'abbé
Dubois, dès qu'il fut à portée d'agir par lui-même, en fut subjugué, et
Stairs se crut assez le maître du terrain pour hasarder, seul de tous
les ambassadeurs des têtes couronnées, une entreprise sur les princes du
sang, dont la longue dispute fut honteuse à notre cour. Elle finit
pourtant sans innovation, mais uniquement par la persévérance des
princes du sang, et sans que Stairs en fût plus mal à Londres ni au
Palais-Royal.

M\textsuperscript{me} de Seignelay-Walsassine mourut en couche. Elle
avait épousé le dernier fils de Seignelay, ministre et secrétaire
d'État, qui avait quitté le petit collet\footnote{Le petit collet était
  une espèce de rabat, qui indiquait qu'on avait embrassé l'état
  ecclésiastique. Quitter le petit collet, c'était renoncer à cet état.},
et qui ne servit point. Il avait eu dans son partage l'admirable
bibliothèque de M. Colbert, son grand-père, qu'il vendit longtemps après
au roi.

L'archevêque de Malines, qui était Hennin-Liétard, des comtes de Bossut,
frère du prince de Chimay, était de ces ambitieux et ignorants dévots,
qui avait fait ses études à Rome. Il y avait jeté les fondements de la
fortune que dès lors il se proposait, en se dévouant aveuglément aux
jésuites et à toutes les chimères ultramontaines. Ses dévots manéges,
aidés de sa naissance, l'avaient mis à Malines, et obtenu de plus de
riches abbayes. La constitution lui parut une occasion de gagner la
pourpre, bien importante à ne pas manquer. Il s'y livra donc avec
fureur, et il trouva des travailleurs qui suppléèrent à son ignorance
par des écrits qui parurent sous son nom. L'empereur, moins dupe que
Louis XIV, et qui n'avait ni Maintenon ni Tellier, ne s'accommoda pas de
tout ce bruit, qu'il fit taire à l'instant par une lettre du prince
Eugène à ce prélat, qui lui manda que l'empereur lui défendait d'écrire
et parler sur la constitution.

Le roi de Sardaigne avait encore mieux fait chez lui dès les
commencements de cette affaire. Il sut qu'elle se glissait dans ses
États, et qu'elle commençait à y exciter des disputes. Il n'en fit pas à
deux fois. Il manda les supérieurs des jésuites de Turin et des maisons
les plus proches. Il leur dit ce qu'il apprenait, qu'il ne voulait point
se laisser mener comme la France, qu'il leur déclarait que, s'il
entendait parler davantage de cette affaire dans ses États, il en
chasserait tous les jésuites. Les bons pères lui protestèrent que ce
n'étaient point eux qui remuaient ces questions, et qu'ils seraient bien
malheureux d'être soupçonnés de ce qui se faisait sans eux et dont ils
ne se mêlaient point. Le roi de Sardaigne leur répondit qu'il ne
disputerait point avec eux\,; mais, encore une fois, qu'ils pouvaient
compter qu'au premier mot qu'il en entendrait parler, il les chasserait
tous de ses États et sans retour\,; et sans leur laisser l'instant
d'ouvrir la bouche, leur tourna le dos et s'en alla. Les révérends pères
le savaient homme de parole et de fermeté, et ne s'y jouèrent pas.
Oncques depuis il n'a été mention quelconque de la constitution dans
tous les États du roi de Sardaigne.

On a vu en son lieu le conseil que j'avais donné à M. le duc d'Orléans
sur le traitement à faire au P. Tellier, où je voulais accommoder la
reconnaissance des services qu'il en avait reçus avec la tranquillité
publique. Il l'approuva fort et en usa tout autrement. La pension fut
modérée, et la liberté ne la fut point. Il voulut aller chez l'évêque
d'Amiens, son intime confident, et l'obtint. Il en abusa en boute-feu
furieux et enragé de n'être plus le maître. Ses commerces en France, ses
intrigues aux Pays-Bas, ses cabales partout, ses machinations diverses
ne purent demeurer secrètes. Il se déroba, pour aller lui-même animer le
parti en Flandre, trop languissant pour son feu. Il en fit tant que
l'évêque d'Amiens fut fort réprimandé, et que le P. Tellier fut confiné
à la Flèche. Ce tyran de l'Église, indigné de ne pouvoir plus remuer, ce
qui était la seule consolation de la fin de son règne et de sa terrible
domination, se trouva dans une réduction à la Flèche également nouvelle
et insupportable.

Les jésuites, espions les uns des autres, et jaloux et envieux de ceux
qui ont le secret, l'autorité et la considération qu'elle leur donne
bien au-dessus des provinciaux et des autres supérieurs, sont encore
merveilleusement ingrats envers ceux mêmes qui, ayant été dans les
premières places ou qui ayant servi leur compagnie avec le plus grand
travail et le plus de succès, lui deviennent inutiles par leur âge ou
par leurs infirmités. Ils les regardent alors avec mépris et bien loin
des égards pour leur âge, leurs services et leur mérite, ils les
laissent dans la plus triste solitude et leur plaignent tout jusqu'à la
nourriture. J'en ai vu trois exemples de mes yeux dans trois jésuites,
gens d'honneur et de grande piété, qui avaient eu les emplois de talents
et de confiance, et à qui j'étais lié successivement d'une grande
amitié. Le premier avait été recteur de leur maison professe\footnote{La
  maison professe des jésuites était située rue Saint-Antoine\,; c'est
  aujourd'hui le lycée Charlemagne. On distinguait les maisons
  professes, des collèges et des noviciats. Les premières étaient
  habitées par les jésuites profès. Ces religieux faisaient, outre les
  trois voeux ordinaires de chasteté, de pauvreté, d'obéissance, un voeu
  particulier d'obéir au pape en tout ce qui regarde le bien des âmes et
  la propagation de la foi chrétienne.} à Paris, provincial de la même
province, distingué par d'excellents livres de piété, plusieurs années
assistant du général à Rome, à la mort duquel il revint à Paris, parce
que leur usage est que le nouveau général a aussi de nouveaux
assistants. De retour à la maison professe à Paris à quatre-vingts ans
et plus, ils le logèrent sous les tuiles au plus haut étage, dans la
solitude, le mépris et le manquement. La direction avait été la
principale occupation des deux autres, dont l'un fut même proposé pour
être confesseur de M\textsuperscript{me} la Dauphine, lui troisième, par
les jésuites, quand le P. le Comte fut renvoyé. Celui-là fut longtemps
malade, dont il mourut. Il n'était pas nourri, et je lui envoyai plus de
cinq mois, tous les jours, à dîner, parce que j'avais vu sa pitance, et
jusqu'à des remèdes, et qu'il ne put s'empêcher de m'avouer ce qu'il
souffrait du traitement qu'on lui faisait. Le dernier, fort vieux et
fort infirme, n'eut pas un meilleur sort. À la fin, n'y pouvant plus
résister, et me le laissant entendre, il me demanda retraite dans ma
maison de Versailles, sous prétexte chez eux d'aller prendre l'air. Il y
demeura plusieurs mois, et mourut au noviciat, à Paris, quinze jours
après qu'il y fut revenu. Tel est le sort de tous les jésuites sans
exception des plus fameux, si on en excepte quelques-uns qui, ayant
brillé à la cour et dans le monde par leurs sermons et leur mérite, et
s'y étant fait beaucoup d'amis, comme les PP. Bourdaloue, La Rue,
Gaillard, ont été garantis de la disgrâce générale, parce que, étant
visités souvent par des personnes principales de la cour et de la ville,
la politique ne permettait pas de les traiter à l'ordinaire, de peur de
faire crier tant de gens considérables qui s'en seraient bientôt
aperçus, et qui ne l'auraient pas souffert sans bruit et sans scandale.

C'est donc cet abandon, ce mépris et ce reproche tacite de tout
soulagement qu'éprouva le P. Tellier à la Flèche quoiqu'il eût quatre
mille livres de pension. Il avait maltraité jusqu'aux jésuites. Aucun
d'eux n'approchait de lui qu'en tremblant du temps qu'il était
confesseur\,; encore n'y avait-il que quelques gros bonnets et en très
petit nombre. Les premiers supérieurs, qu'il gouvernait à baguette,
éprouvaient ses duretés, et tous sa domination, sans la moindre
ouverture. Le général même fut réduit à ployer devant lui ce despotisme
absolu qu'il exerce sur toute la compagnie et sur tous les jésuites en
particulier. Tous, et ils me l'ont dit dans ces temps-là bien des fois,
désapprouvaient la violence de sa conduite et en étaient fort alarmés
pour la société\,; tous le haïssaient comme on déteste un maître
grossier, dur, inaccessible, plein de soi-même, qui se plaît à faire
sentir son pouvoir et son mépris. Son exil et la conduite qui le lui
attira leur fut un nouveau motif de dépit par le dévoilement des
intrigues secrètes où ils avaient grande part et qu'ils avaient grand
intérêt à cacher. Tout cela ensemble ne rendit pas au P. Tellier la
retraite forcée de la Flèche agréable. Il y trouva des supérieurs et des
confrères aigris qui, au lieu de la terreur générale qu'il avait imposée
aux jésuites mêmes, n'eurent plus que du mépris pour lui, et se plurent
à le lui faire sentir. Ce roi de l'Église et en partie de l'État, en
particulier de sa société, redevint un jésuite comme les autres, et sous
ses supérieurs on peut juger quel enfer ce fut à un homme aussi
impétueux et aussi accoutumé à une domination sans réplique et sans
bornes et à en abuser en toutes façons. Aussi ne la fit-il pas longue.
On n'entendit plus parler de lui depuis, et il mourut au bout de six
mois qu'il fut à la Flèche.

Il parut une promotion de six lieutenants généraux et d'un grand nombre
de maréchaux de camp et de brigadiers\,; ce qui fit aussi de nouveaux
colonels.

Le duc de Mortemart, piqué de ce que la lieutenance de roi vacante du
Havre de Grâce ne fût pas donnée à celui pour qui il la demandait,
vendit ce gouvernement au duc de Saint-Aignan. M. le duc d'Orléans donna
l'abbaye de Bourgueil à l'abbé Dubois\,; dix mille livres de pension, en
attendant un gouvernement, au vicomte de Beaune, à la sollicitation
pressante de M. le Duc et de M\textsuperscript{me} sa mère, et une de
vingt mille livres au duc de Tresmes. Comme gouverneur de Paris, il
avait un jeu public dans une maison qu'il louait pour cela, et dont il
tirait fort gros. Il l'avait prétendu comme un droit depuis qu'il en
avait vu s'établir d'autres par licence, et quelques-uns, depuis la
régence, par permission. Ces jeux étaient devenus des coupe-gorges qui
excitèrent tant de cris publics, qu'ils furent tous défendus, et celui
du duc de Tresmes comme les autres. Ce fut en dédommagement de ce jeu
que la pension lui fut donnée. Il ne laissa pas de s'en introduire de
temps en temps, mais plus modestement. Tout ayant changé de face sous le
gouvernement de M. le Duc, premier ministre, M\textsuperscript{me} de
Carignan, arrivée, ancrée, et point du tout oisive pour son intérêt,
obtint un jeu à l'hôtel de Soissons, qui lui valut extrêmement. Sur cet
exemple, le duc de Tresmes prétendit et obtint le rétablissement du
sien. Le rare fut qu'il ne laissa pas de conserver la pension de vingt
mille livres qu'il n'avait eue que pour le lui ôter.

Le jeune Bournonville, petit-fils, par sa mère, du duc de Luynes et
d'une soeur de M. de Soubise, et fils du cousin germain paternel de la
maréchale de Noailles, et frère de la duchesse de Duras, épousa la
seconde fille du duc de Guiche, mort maréchal duc de Grammont\,; c'est
celle qui épousa depuis mon fils aîné.

Le grand prieur attrapa de M. le duc d'Orléans un don sur les loteries
de Paris de plus de vingt-cinq mille écus de rente.

Le mariage du prince électoral de Saxe fut arrêté et déclaré avec une
des archiduchesses.

Le roi Jacques partit assez publiquement de Rome, s'embarqua à Nettuno,
8 février, et aborda en Espagne, d'où il se rendit à Madrid.

Prie revint avec sa femme de son ambassade de Turin. Je ne remarque ce
retour que par le bruit et le mal que fit cette femme, qui fut maîtresse
publique de M. le Duc, et de la cour, et de l'État, quand et tant qu'il
fut premier ministre. Prie eut douze mille livres de pension et
quatre-vingt-dix mille livres de gratification.

Rémond, dont il a été parlé ailleurs, fut introducteur des ambassadeurs.
Comme il devint une espèce de petit personnage, et, quoique subalterne,
fort dangereux, il est à propos de le faire encore mieux connaître. Il
était fils de Rémond, fermier général, connu sous le nom de Rémond le
Diable. Ce fils était un petit homme qui n'était pas achevé de faire, et
comme un biscuit manqué, avec un gros nez, de gros yeux ronds sortants,
de gros vilains traits, et une voix enrouée comme un homme réveillé en
pleine nuit en sursaut.

Il avait beaucoup d'esprit, il avait aussi de la lecture et des lettres,
et faisait des vers. Il avait encore plus d'effronterie, d'opinion de
soi et de mépris des autres. Il se piquait de tout savoir, prose,
poésie, philosophie, histoire, même galanterie\,; ce qui lui procura
force ridicules aventures et brocards. Ce qu'il sut le mieux, fut de
tacher de faire fortune, pour quoi tous moyens lui furent bons. Il fut
le savant des uns, le confident et le commode des autres, et de plus
d'une façon, et ne se cachait pas de la détestable\,; le rapporteur
quand on le voulut et que cela lui parut utile. Il s'attacha à Canillac,
à Nocé, aux ducs de Brancas, puis de Noailles, surtout à l'abbé Dubois,
dont il allait disant pis que pendre, pour faire parler les gens et le
lui aller redire\,; enfin à Stairs, dont il devint le panégyriste et
l'homme à tout faire. Sa souplesse, l'ornement de son esprit, son
aisance à parler et à frapper, sa facilité à adopter le goût de chacun,
une sorte d'agrément qu'on trouvait dans sa singularité, le mirent
quelque temps fort à la mode, dont il sut tirer un grand parti
pécuniaire. Il en avait espéré d'autres qui s'évanouirent avec son
cardinal Dubois. Tel qu'il était, il ne laissa pas de trouver et de
conserver des entrées et de la familiarité dans plusieurs maisons
distinguées. Il a fini par épouser une fille du joaillier Rondé, en quoi
il n'y a eu ni disparité ni mésalliance, et par donner souvent des
soupers à bonne et honorable compagnie. Il avait eu la charge de Magny.
Il ne la garda pas longtemps, voyant ses espérances trompées et qu'elle
ne le menait à rien.

Mimeur mourut officier général, dont je crois avoir parlé ailleurs. Il
était fils d'un président du parlement de Dijon. Je ne sais par quelle
protection il avait été attaché à Monseigneur dès sa jeunesse, chez qui
il avait les entrées\,; mais il n'alla jamais dans aucun lieu où on
mangeât avec lui. Son esprit souvent plaisant sans songer à l'être, et
l'ornement de son esprit joint `à beaucoup de modestie et de
savoir-vivre, l'avait mêlé avec le grand monde et fait désirer dans les
meilleures compagnies. Il était aimé et estimé sur un pied agréable, et
le méritait\,; il était honnête homme et fort brave, sans se piquer de
rien, et fort doux, aimable et sûr dans le commerce\,; il servit toute
sa vie, presque toujours dans la gendarmerie, avec réputation\,; il se
maria à la fin de sa vie et fut regretté de beaucoup d'amis.

Térat, chancelier et surintendant des affaires et finances de M. le duc
d'Orléans, mourut en même temps. Il avait un râpé de l'ordre. Il était
fort vieux et fort riche, fort homme d'honneur et fort désintéressé. Il
était chancelier de Monsieur quand, à la mort de Bechameil, qui était
surintendant, il eut sa charge, dont il refusa absolument les
appointements. Ce fut une perte pour M. le duc d'Orléans, dont il
gouvernait très bien les affaires. Il vivait fort honorablement et
n'était déplacé en rien\,; il était généralement aimé et estimé, et ne
laissa point d'enfants. Je n'ai point su qui il était\,; je crois que
c'était peu de chose\,; aussi était-il fort éloigné de s'en faire
accroire. Houssaye, conseiller d'État, eut les deux charges de Térat
chez M. le duc d'Orléans, qui le conduisirent à être enfin contrôleur
général des finances.

Un fils de l'électeur de Bavière fut élu évêque de Munster. Il était
allé se promener en Italie, et mourut à Rome sans avoir su son élection.

La mort de Puysieux, duquel on a déjà parlé lorsque son esprit et son
adresse le firent si singulièrement chevalier de l'ordre, devint le
commencement et la base de la prodigieuse fortune de Belle-Ile. Les
chartreux, qui sont accoutumés à donner quelquefois de grands repas, en
donnèrent un à beaucoup de gens distingués de la cour et des conseils.
J'en fus prié, et Puysieux, que tout le monde aimait, et qui était bon
et joyeux convive, en fut aussi. Le repas fut également grand et bon, et
la compagnie, quoique fort nombreuse, de très bonne humeur. Puysieux en
fit la joie\,; mais pour un homme fort près de quatre-vingts ans, gros
et court, il y mangea beaucoup, et tant que, la nuit même, il se sentit
d'une indigestion et de fièvre qui l'emporta en fort peu de jours. Ce
fut grand dommage pour sa probité, sa valeur, sa modestie, l'ornement de
son esprit, qui avait également l'agréable et le solide, et qui en
faisait tout à la fois un homme de guerre, un homme capable de bien
manier les affaires les plus délicates et un homme de la meilleure
compagnie, qui était estimé partout et recherché de ce qui était le plus
distingué. Son père s'était ruiné à ne rien faire\,; il était resté bien
peu de bien à Puysieux, et son frère, qui n'avait presque rien, avait
été trop heureux d'être écuyer de M. le prince de Conti, qui le traita
toujours avec distinction. Puysieux était conseiller d'État d'épée, dont
Cheverny eut la place\,; il avait aussi le gouvernement d'Huningue. Sa
famille le voyant moribond, et n'ayant que des filles, songea
promptement à profiter de la facilité du temps pour en faire une pièce
d'argent, et Belle-Ile, fort à l'affût de tout ce qui pouvait l'avancer,
conclut bientôt ce marché. Il était ami intime de Le Blanc, qui l'avait
mis dans quelque privance avec l'abbé Dubois et Law. Il ne faisait
qu'être maréchal de camp, par conséquent fort loin d'un gouvernement,
bien plus d'un de cette importance. Ces trois protecteurs, avec le
maréchal de Besons, frère de la mère de Le Blanc, qui entraîna d'Effiat,
joints avec la famille de Puysieux, emportèrent d'emblée l'agrément du
régent, et toute l'affaire fut menée si brusquement et si secrètement,
qu'on ne la sut que lorsqu'elle fut consommée, la veille de la mort de
Puysieux.

Une grâce si singulière excita les cris de tout ce qui se proposait de
demander cette récompense dès qu'elle serait vacante. L'adresse de
Belle-Ile excita ceux des moins à portée et le blâme des importants,
parmi lesquels les maréchaux de Villeroy, Villars, Huxelles, se
signalèrent autant que leur frayeur de toute la suite de l'affaire du
duc du Maine le leur permit, c'est-à-dire qu'ils ne se contraignirent
pas avec leurs familiers\,; qu'ils encouragèrent secrètement les
plaintes, et qu'ils se contentèrent d'ailleurs d'un silence de
désapprobation. Tant de bruit, et la réflexion tardive sur sa matière,
fit assez repentir le régent pour être tenté de révoquer la
permission\,; mais le marché était signé et l'argent compté\,; il ne se
trouvait d'autre moyen que l'autorité, par un changement subit de
volonté qui ne pouvait se couvrir de surprise. Ceux qui avaient obtenu
cette permission du régent lui firent honte de reculer, et Belle-Ile
demeura paisible gouverneur d'Huningue\,; mais il en resta une dent
contre lui à M. le duc d'Orléans, qu'il lui a toujours, mais assez
inutilement gardée.

\hypertarget{chapitre-viii.}{%
\chapter{CHAPITRE VIII.}\label{chapitre-viii.}}

1719

~

{\textsc{Inquiétude des maréchaux de Villeroy, Villars et
Huxelles-Villars, dans la frayeur, me prie de parler à M. le duc
d'Orléans.}} {\textsc{- Je le fais, et le veux rassurer.}} {\textsc{-
Manége et secret sur les prisonniers.}} {\textsc{- Politique de l'abbé
Dubois sur l'affaire du duc et de la duchesse du Maine et des leurs.}}
{\textsc{- La même politique fausse et très dangereuse pour M. le duc
d'Orléans.}} {\textsc{- Je le lui représente très fortement, ainsi que
l'énorme conduite à son égard du duc du Maine et de ses principaux
croupiers, et le danger d'une continuelle impunité.}} {\textsc{- Je ne
trouve que défaites et misères.}} {\textsc{- Trois crimes du duc du
Maine à punir à la fois\,: premièrement, attentat d'usurper l'habilité
de succéder à la couronne\,; secondement, les moyens pris pour soutenir
cette usurpation\,; troisièmement, sa conspiration avec l'Espagne.}}
{\textsc{- Conduite à tenir à l'égard du duc et de la duchesse du Maine,
de leurs principaux complices et des enfants du duc du Maine.}}
{\textsc{- Mollesse, faiblesse, ensorcellement du régent par Dubois.}}
{\textsc{- Je cesse de parler au régent du duc du Maine, qui peu à peu
est rétabli.}} {\textsc{- Adroit manége de Le Blanc et de Belle-Ile.}}
{\textsc{- Duc de Richelieu et Saillant à la Bastille.}} {\textsc{- Leur
folie.}} {\textsc{- Traité du premier.}} {\textsc{- Ils sont bientôt
élargis.}} {\textsc{- Singularité de la promotion de l'ordre, dont je
fus moins de dix ans après.}}

~

Ce qui tenait de si court les trois maréchaux dont on vient de parler,
était ce qu'ils sentaient en leur âme et conscience sur l'affaire du duc
du Maine. Orseau, des postes, avait été arrêté\,; Boisdavid en
Saintonge, et amené à la Bastille, où il arrivait journellement des gens
pris dans les provinces\,; même le duc de Richelieu fut mis à la
Bastille. La peur était grande que quelqu'un d'eux ne parlât, et qu'on
ne mît la main sur le collet à des gens de leur connaissance qui en
savaient encore plus, qui étaient encore libres, et tâchaient de faire
bonne contenance. Il courut même un bruit que le maréchal de Villars
allait être arrêté. Sa frayeur éclata sur son visage et dans sa
conduite. Il n'osait plus sortir de chez lui, et il s'informait de ce
qui se disait sur lui avec une inquiétude indécente.

Lui et sa femme m'avaient toujours extrêmement ménagé de tout temps. Ils
avaient fermé les yeux et les oreilles à mes façons et à mes propos sur
leur duché, et depuis encore sur leur pairie, et m'avaient sans cesse
également cultivé et M\textsuperscript{me} de Saint-Simon. Ils
m'envoyèrent prier d'aller chez eux, avec instance. J'y allai, et je
trouvai le maréchal dans des transes et dans un abattement incroyable.
Il me dit sans façon qu'il savait qu'il allait être arrêté, qu'il s'y
attendait à tous les instants, que ce n'était qu'avec la dernière
inquiétude qu'il sortait de chez lui pour le conseil de régence ou pour
aller au Palais-Royal le moins qu'il pouvait, même sans se croire en
sûreté chez lui. Que cela prenait fort sur sa santé, que les avis lui en
venaient de toutes parts, que le bruit en était public, qu'il n'y avait
pas moyen de vivre de la sorte\,; qu'il s'apercevait depuis du temps que
M. le duc d'Orléans ne le voyait plus de bon oeil, et qu'il était
embarrassé et froid avec lui, qu'il ne savait quel mauvais office on lui
avait rendu\,; s'étendit sur son attachement et sa fidélité, et me
conjura de parler à M. le duc d'Orléans, et de tâcher à le faire
expliquer sur son compte. Sa femme, beaucoup plus tranquille que lui, me
pria de la même chose. Je les assurai, comme il est vrai, que je n'avais
rien remarqué en M. le duc d'Orléans qui eût pu donner lieu aux bruits
qui couraient, et que je croyais qu'il se faisait tort à lui-même d'en
avoir de l'inquiétude.

Ce n'était pas que je fusse persuadé qu'il dût être dans la sécurité. On
a vu comme le hasard fit savoir si peu avant le lit de justice
l'assemblée mystérieuse du duc du Maine avec lui chez le maréchal de
Villeroy, et toutes ses liaisons y étaient conformes. Mais M. le duc
d'Orléans était si étouffé des deux tours de force qu'il n'avait pu
éviter de faire coup sur coup, si éloigné de ces coups d'éclat, si peu
capable encore de les soutenir, beaucoup moins de les oser pousser, que
j'ai toujours cru les gros complices en pleine sûreté, même les plus
médiocres. Je parlai donc à M. le duc d'Orléans qui n'était pas fâché de
la peur que le maréchal avait prise, mais qui me répondit ce qu'il
fallait pour le rassurer. Ils me remercièrent beaucoup tous deux, mais
le maréchal toujours fort dans l'inquiétude. Elle fit une telle
impression sur lui, qu'il en maigrit à vue d'oeil. Son sang se
corrompit, il lui vint un mal au cou qui menaça d'un cancer. Le remède
de Garrus l'en garantit, dont il prit souvent depuis, et en porta
toujours dans sa poche. Mais il languit toujours jusqu'à l'élargissement
du duc et de la duchesse du Maine, après quoi il reprit bientôt son
embonpoint et sa première santé, en sorte que la cause de son mal fut
manifestement visible.

Le Blanc allait souvent à la Bastille et à Vincennes, et sans que je le
lui eusse demandé ne manquait point de venir le même jour, le soir, chez
moi me rendre compte de ce qu'il avait appris des prisonniers, et de ce
qu'il s'était passé entre eux et lui, ainsi que de tout ce qui lui
revenait sur cette affaire\,; mais les prisonniers, à ce qu'il
m'assurait toujours, ne disaient rien ou que les riens qu'il me
rapportait. Belle-Ile, qui s'était fort initié chez moi par Charost et
par M\textsuperscript{me} de Lévi, qui n'était qu'un avec Le Blanc et
qui entrait dans tout ce qu'il pouvait, venait raisonner avec moi en
cadence des visites de Le Blanc. Je ne fus pas longtemps à démêler que
je n'en saurais jamais davantage, comme il arriva en effet, excepté ce
qu'il fallut tout à la fin en dire au conseil de régence pour excuser
les emprisonnements et les exécutions de Bretagne. M. le duc d'Orléans
n'en savait pas plus que moi, ou si on lui en disait quelque chose de
plus, ce fut sous un secret recommandé plus pour moi que pour personne.
L'abbé Dubois, maître absolu de M. le duc d'Orléans, faisait trembler,
excepté moi, tout ce qui approchait ce prince. L'abbé craignait le nerf
de mes conversations et de n'être pas le maître de son aiguière, s'il
venait jusqu'à moi des découvertes dont je pusse battre le régent, et
venir à bout de son incurie et de sa débonnaireté. On a vu, lors de
l'arrêt de l'abbé Portocarrero, l'adresse et la hardiesse dont Dubois se
saisit de tous les papiers. Il n'eut pas de soin de s'emparer de ceux de
Cellamare, que Le Blanc, qui l'y accompagnait, n'était pas pour lui
disputer. Il s'était donc ainsi rendu seul maître du secret et du fond
de l'affaire, et tellement que M. le duc d'Orléans ni personne n'en
pouvaient savoir que ce qu'il voulait bien leur dire. Le garde des
sceaux, qui allait rarement interroger les prisonniers, et Le Blanc, qui
les voyait bien plus souvent et à qui venaient tous les avis sur cette
affaire, étaient dans l'entière frayeur et la plus soumise dépendance de
l'abbé Dubois, avec lequel ils concertaient chaque jour ce qu'ils
devaient dire à M. le duc d'Orléans sur les avis et sur ce qu'ils
avaient tiré ou n'avaient pu tirer des prisonniers, et rendaient compte„
au sortir d'avec lui, au redoutable abbé de tout ce qui s'était passé
entre eux et le régent.

Dubois voulait faire la peur entière au duc et à la duchesse du Maine et
aux prisonniers pour tirer tout d'eux, et y mettre si bon ordre qu'il
n'y eût plus rien à craindre\,; il voulait aussi épouvanter les
maréchaux pour les humilier et les contenir. Mais il était bien éloigné
d'aller plus loin. Il voulait régner sans trouble et parvenir à la
pourpre et à la place et à toute l'autorité de premier ministre sans
embarras au dedans, pour n'avoir à vaincre que sur le chapeau, qui le
conduisait à l'autre, que les difficultés du dehors. Il voulait de plus
se préparer une domination absolue, sans contradiction. Il sentait quel
serait le cri public, le dépit et l'impétuosité de M. le Duc sur un
second maître et de son intimité\,; de combien de personnages il serait
escorté dans un mécontentement qui serait universel. Il y redoutait les
mouvements que le parlement y pourrait faire, à qui, dans un cas si
étrange, chacun se réunirait. Il se proposait donc de mettre entre ses
seules mains la vie et toute la fortune du duc du Maine et de ses
enfants et celle de ses complices, pour s'acquérir sur eux l'obligation
de leur avoir lui seul rendu le tout, et à ses plus importants
croupiers, pour s'en faire une protection sûre contre le cri public et
contre les princes du sang, et s'acquérir le parlement, au moins
l'arrêter et le rendre neutre et sans mouvement par le crédit du duc et
de la duchesse du Maine sur le premier président, qui s'y trouvait en
son particulier tout de son long, et sur les principaux moteurs de la
compagnie.

Je ne répondrais pas aussi que, sans s'être commis à confier le fond du
sac à M. le duc d'Orléans, il n'ait profité de son incroyable faiblesse,
de son insensibilité aux plus cruelles injures encore plus incroyable,
de son penchant à ne rien pousser et à des \emph{mezzo-termine}
déplorables, pour lui persuader cette politique à l'égard de tous ceux
qui avaient trempé dans le complot\,; et que, profitant des soeurs que
l'opiniâtre impétuosité de M. le Duc avait données au régent, lorsqu'il
lui força la main au dernier lit de justice sur la destitution du duc du
Maine, sur l'éducation du roi, sur un établissement pour M. le comte de
Charolais, sur une augmentation d'une pension de cent cinquante mille
livres pour soi-même, il n'ait fait comprendre au régent la nécessité
indispensable d'une barrière contre la hauteur et l'avidité des prince
du sang, et que cette barrière ne se pouvait trouver que dans la
conservation du duc du Maine, de ses rangs, de ses établissements, et de
ses complices les plus considérables. Je ne doute pas non plus qu'il
n'ait fait peur à son maître des maréchaux de Villeroy, dont Tallard
serait inséparable, Villars et Huxelles, du premier président et de
nombre d'autres qui venant à être publiquement convaincus, feraient avec
le duc du Maine un groupe formidable dont le régent serait d'autant plus
embarrassé par le nombre, les établissements, la parentelle et le poids
dans le mondé, que, criminels par les lois, il resterait vrai toutefois
qu'ils ne l'étaient directement que contre le régent, subsidiairement
contre l'État, mais pour le sauver du prétendu mauvais gouvernement,
point du tout contre la personne du roi, dont la conservation contre les
périls du poison deviendrait leur prétendue apologie, et produirait tôt
ou tard de funestes effets. Il n'en fallait pas tant pour étourdir un
prince au fond timide, ennemi des grands coups, parfaitement insensible
aux plus cruelles et aux plus dangereuses injures, bon et doux par
nature, choisissant toujours le plus aisé comme tel, par faiblesse, dans
les affaires grandes ou épineuses, et par incapacité de les suivre et
d'en soutenir le poids, enfin livré et abandonné à l'abbé Dubois, auquel
il ne pouvait plus résister sur quoi que ce fût.

Mais cette politique, si bonne et si fort dans le vrai pour la fortune
où tendait l'abbé Dubois, n'était ni bonne ni dans le vrai pour son
maître. Plus M. du Maine et ses plus considérables complices lui
auraient une obligation signalée de la vie, des honneurs, des
établissements, plus cette obligation à ne jamais l'oublier serait aux
dépens de M. le duc d'Orléans. Quelques marques de clémence et de
misère, quand elle est gratuitement poussée à l'extrême, que ce prince
eût données, jamais de grands coupables ne pardonnent à ceux contre qui
ils ont commis de grands crimes, et il était tout naturel qu'ils fussent
persuadés et que l'abbé Dubois leur fit délicatement entendre qu'il les
avait habilement arrachés des mains de son maître, sans quoi ils étaient
perdus. Le coup double et prodigieux que le régent venait si
nouvellement de frapper au dernier lit de justice sur le parlement et
sur le duc du Maine, n'avait causé ni trouble ni rumeur, mais une
frayeur extrême, un silence de tremblement, une soumission entière. Cet
exemple devait donc l'encourager, puisque c'était aux mêmes gens qu'il
avait affaire et prévenus de plus du crime d'État. C'est ce que je lui
avais représenté plus d'une fois, et que le pardon, ni le semblant de
manquer de preuves quand on en a, ne réconcilient jamais ceux qui ont
manqué un grand coup à celui contre qui il était préparé\,; que le péril
couru, plus il est grand, plus il irrite\,; qu'un tel bienfait reçu
redouble la haine et la rage de qui s'est vu dans la main et à la merci
de qui les pouvait exterminer, leur fait mépriser une générosité qu'ils
imputent à la faiblesse, qui les excite à prendre mieux leurs mesures,
ou s'ils ne le peuvent pendant le reste de la régence, à renverser le
régent auprès du roi majeur, avec d'autant plus de hardiesse qu'alors il
n'y a plus de crime\,; qu'il n'est point de régence dont le gouvernement
ne puisse être attaqué, ni de vie et de moeurs telles que celles de M.
le duc d'Orléans à couvert sous l'abri de son rang.

Je m'étendis un peu avec le régent sur les points de son gouvernement,
qu'on pourrait rendre très répréhensibles aux yeux d'un jeune roi
majeur, avec le secours d'une bonne et secrète cabale, en quoi le duc du
Maine était un grand et dangereux ouvrier, en quoi les maréchaux de
Villeroy, Villars, Huxelles, par leurs emplois dans la régence, comme
témoins de près, et d'autres joints à eux, aideraient le duc du Maine\,:
Law et sa banque\,; l'alliance d'Angleterre jusqu'à l'ensorcellement,
pour la fortune de l'abbé Dubois, conséquemment avec l'empereur, les
deux plus grands et plus naturels ennemis de la France\,; la rupture
pour eux seuls, et malgré la Hollande, entraînée de force contre
l'Espagne, après tant de sang et de trésors répandus pour la conserver,
et avec qui la plus étroite union était si naturelle et si utile\,; la
facilité de fasciner les yeux d'un jeune roi et de lui tourner toute
cette conduite à intérêt particulier contre celui de l'État, pour monter
sur le trône sans obstacle, s'il fût mésarrivé au roi ou s'il lui
mésarrivait encore sans enfant mâle, et de l à. revenir aux anciennes
horreurs pour lui faire craindre pour sa vie, tant que son précédent
régent ne serait pas mis en lieu de sûreté. Je ne trouvai que faiblesse
ou dissimulation.

Cela ne m'arrêta pas. Je lui demandai quel retour il trouvait dans le
maréchal de Villeroy pour l'avoir traité avec une distinction qui ne
différait pas du respect, sans jamais aucun refus ni aucun délai à
toutes ses demandes qui étaient continuelles pour faire montre de son
crédit et de sa protection, souvent en choses considérables\,; pour
avoir accru son autorité à Lyon fort au delà de raison et d'usage, au
point qu'il y était uniquement et absolument le maître de tout\,; enfin
pour l'avoir admis fort dangereusement au secret de la poste, et à la
lecture que Torcy lui venait faire des extraits, et encore en d'autres
confidences. Je lui demandai quel retour il trouvait dans le maréchal
d'Huxelles pour avoir comblé ses désirs en lui confiant le secret et
l'administration des affaires étrangères, et de son ami, le premier
président, en l'accablant d'argent et outre cela de pensions. Enfin je
vins au duc du Maine, et je lui demandai quel los\footnote{Vieux mot
  synonyme de \emph{louange}, et par suite de \emph{renom}.} il en avait
reçu, pour ne l'avoir pas destitué à la mort du roi, comme tout le
monde, tous les seigneurs, le parlement même s'y attendait et le
désirait alors avec un empressement qu'il ne pouvait ignorer\,: «\,Mais,
me répondit-il d'une voix basse, honteuse et faible, c'est mon
beau-frère. --- Comment votre beau-frère\,! repris-je avec feu\,: est-ce
donc un titre à lui pour vous étrangler comme il y a tâché et butté
toute sa vie\,? Avez-vous oublié la honte et le désespoir de Monsieur,
le vôtre alors à vous-même, la fureur et les larmes publiques de Madame
d'un mariage si étrangement disproportionné\,? Avez-vous oublié que
l'intérêt de ce beau-frère vous a éloigné du commandement des armées,
dont Monsieur mourut de colère et de dépit après la prise qu'il en avait
eue avec le roi le jour même\,? Avez-vous oublié jusqu'à quel point il
intéressa M\textsuperscript{me} de Maintenon à votre perte, lors de
votre affaire d'Espagne, malgré tous les efforts de
M\textsuperscript{me} la duchesse de Bourgogne auprès d'elle en votre
faveur et de combien près vous frisâtes les derniers malheurs\,?
Avez-vous oublié les horreurs dont ce cher beau-frère vous affubla à la
mort de Mgr le Dauphin et de M\textsuperscript{me} la Dauphine, du petit
prince leur fils, et de M. le duc de Berry ensuite\,; qu'il en persuada
le roi par M\textsuperscript{me} de Maintenon, et qu'ils l'ont toujours
été, la cour, Paris, les provinces, les pays étrangers\,; l'art et le
soin de répandre cette opinion jusqu'à en rendre le doute ridicule, et
le soin vigilant de la renouveler de temps en temps et de lui donner une
couleur nouvelle\,? Enfin avez-vous oublié le testament et le codicille
du roi, la dispute si forte de M. du Maine en plein parlement contre
vous, et si impudemment soutenue en faveur du codicille, et ce que vous
seriez devenu, si l'une de ces deux pièces que personne n'ignore que le
roi fit malgré lui, avait subsisté, bien pis si toutes deux avaient été
exécutées\,? Tous ces crimes à votre égard sont antérieurs à votre
régence, sans que vous ayez jamais donné le moindre ombrage à M. du
Maine, que celui qu'il a voulu prendre de votre naissance et de votre
droit. Vous avez cru par la conduite que vous avez si longtemps soutenue
et tant que vous l'avez pu à son égard, aux dépens des princes du sang
et de toute justice, regagner ce bâtard brûlant de la soif de régner. Il
vous en a payé dans le temps même qu'il jouissait de votre plus grand
déni de justice par la requête au parlement de cette prétendue noblesse,
et par son appel aux états généraux ou au roi majeur, avec la criminelle
audace de vous attaquer vous-même sur l'incompétence et le défaut de
pouvoir d'un régent. Enfin vous voyez ce qu'il vient de brasser, et par
tant d'expériences anciennes et nouvelles ce que vous devez attendre de
lui, si vous le laissez en état de continuer\footnote{Si l'on en croit
  \emph{les Mémoires du marquis d'Argenson} (éd. 1825, p.~178),
  Saint-Simon aurait pressé le duc d'Orléans de mettre en jugement le
  duc du Maine\,: «\,Que prétendait M. de Saint-Simon\,? Il voulait que
  l'on fît le procès à M. le duc du Maine\,; que l'on fit tomber sa tête
  et que l'on donnât à lui, Saint-Simon, la grande maîtrise de
  l'artillerie.\,»}.\,»

Ces propos, que je renouvelais de temps en temps, jetaient M. le duc
d'Orléans dans un trouble extrême. Il sentait tout le poids de mes
raisons\,; mais il était enchaîné par les prestiges de l'abbé Dubois.
Tantôt il s'excusait sur le défaut de preuves, et je lui remettais ce
qu'il en avait dit à M. le Duc et à moi, que M. et M\textsuperscript{me}
du Maine étaient des plus avant dans la conspiration, comme je l'ai
rapporté en son temps. Une autre fois, il alléguait le danger
d'entreprendre un homme si grandement établi, et je lui démontrais
qu'après le grand pas de l'avoir fait arrêter lui et
M\textsuperscript{me} du Maine, et confinés en deux prisons éloignées,
le danger du retour serait bien plus grand, mortellement offensés qu'ils
seraient, et que de plus ils se le devaient montrer comme innocents.
Enfin retranché sur l'embarras de leurs enfants, aussi grandement
établis que le père, dont ils avaient les survivances, et le
gouvernement de Guyenne de plus, qui sûrement ne trempaient point dans
le complot du père, et que par conséquent on ne pouvait dépouiller\,; je
lui demandai où il avait vu ou lu qu'on eût jamais laissé aux fils des
criminels d'État, convaincus et punis comme tels, des établissements
dont ils pussent abuser\,; qu'il prît garde qu'une telle condamnation
emportait confiscation des biens patrimoniaux, quoique les enfants ne
fussent pas coupables, à plus forte raison l'extinction des titres,
honneurs, etc., et la privation des gouvernements et des charges dans le
père, et des survivances dans ses fils, lesquels, bien que non
coupables, perdaient par la condamnation du père la succession entière
du patrimoine, qui, sans cela, leur était de tout droit acquis, à plus
forte raison des grâces dont le père était justement dépouillé\,; qu'il
était du plus évident danger de les leur laisser, et sur lesquelles ils
ne pouvaient avoir un droit en rien comparable au droit qu'ils avaient
aux biens de leur père, qui était leur patrimoine, duquel toutefois ils
ne laissaient pas d'être de tout droit totalement privés par la
confiscation inséparable de la condamnation\,; qu'à la vérité on n'y
touchait jamais au bien et aux reprises de la mère, qui demeuraient
après elle aux enfants\,; mais ici, la mère se trouvant aussi coupable
que le père, la condamnation emportait confiscation de tout le bien
maternel comme du bien paternel.

À cette réponse, M. le duc d'Orléans n'eut point de réplique, baissa la
tête et demeura quelque temps rêveur, puis me dit\,: «\,Mais
M\textsuperscript{me} du Maine, vous ne sauriez nier qu'elle ne soit
princesse du sang\,? --- Non, certes, lui répondis-je\,; mais vous ne me
prouverez pas aussi qu'elle la soit davantage que les deux ducs
d'Alençon, père et fils\footnote{Les deux ducs d'Alençon, dont il est
  ici question, sont Jean V et son fils René. Le premier fut arrêté en
  1456, et condamné à mort en 1458\,; la peine fut commuée en 1461 en
  une prison perpétuelle. Arrêté de nouveau en 1472, Jean V d'Alençon
  fut jeté par Louis XI dans un cachot, où il resta jusqu'à sa mort
  (1476). Son fils René fut arrêté en 1482 et condamné à demander pardon
  au roi et à recevoir garnison royale dans ses châteaux.}, que le
connétable de Bourbon, que M, le Prince, propre grand-père de
M\textsuperscript{me} du Maine, qui tous aussi étaient princes du sang,
bien reconnus pour tels, et néanmoins atteints, convaincus, et
solennellement jugés et condamnés comme criminels d'État. Vous savez
après combien de prison et à quelles conditions l'un de ces ducs
d'Alençon eut sa grâce\,; ce que devint lé connétable de Bourbon, et
que, quel désir qu'on eût d'une paix aussi avantageuse que fut alors
celle des Pyrénées, la passion extrême de la reine votre grand'mère du
mariage du roi avec l'infante sa nièce, quelque pressé qu'en fût le
cardinal Mazarin et la reine même, dans la frayeur qu'ils avaient eue
l'un et l'autre de ce qui avait pensé arriver de la nièce du
cardinal\footnote{Marie Mancini avait inspiré à Louis XIV une passion
  qui donna des inquiétudes sérieuses à Anne d'Autriche. Voy. les
  \emph{Nièces de Mazarin}, par M. Amédée Renée, et les \emph{Mémoires
  de M\textsuperscript{me} de Motteville}, à l'année 1659.}, qui épousa
depuis le connétable Colone, et de ce qui était toujours possible à
l'égard de quelque autre, tant que le roi ne serait pas marié\,; l'on
aima mieux hasarder la paix et le mariage, essuyer toutes les longueurs
à conclure, les persécutions et les propositions de toutes les sortes de
don Louis de Haro en faveur de M. le Prince, même aux dépens du roi
d'Espagne, que de souffrir qu'il tirât aucune sorte d'établissement des
Espagnols, ni qu'il rentrât dans son gouvernement, ni dans sa charge de
grand maître de France, qui à la fin, mais sans stipulation, furent
donnés à M. son fils, mais quelque temps après\,; grâce dont pour
conclure on n'était convenu que verbalement, secrètement et comme une
grâce et une galanterie personnelle au roi d'Espagne et à son ministre.
Aujourd'hui que vous commencez la guerre, vous ne traitez ni mariage
nécessaire et pressé, vous ne traitez point la paix, vous ne sauriez
craindre qu'on se persuade au dedans ni au dehors, après l'éclat fait
sur l'ambassadeur d'Espagne et ce que vous savez déjà sur M. et
M\textsuperscript{me} du Maine de leurs complots avec lui, qu'on leur
fasse accroire des crimes pour les perdre, et vous en saurez bien
davantage quand il plaira à l'abbé Dubois de vous instruire à fond par
les papiers dont vous convenez qu'il s'est saisi, qu'il a vus lui seul,
et qu'il ne vous a pas montrés. Grand Dieu\,! ajoutai-je avec dépit de
ne trouver que de la filasse pour ne pas dire du fumier, grand Dieu\,!
quel précieux présent avez-vous fait à ce prince de la plus difficile
vertu du christianisme, de cette vertu tellement surhumaine, si
contraire à la nature et à la plus droite raison quand elle n'est pas
miséricordieusement éclairée et entraînée par votre grâce
toute-puissante, cette vertu, l'écueil des plus grands hommes, le plus
dur et le plus continuel combat des plus grands saints, cette vertu
toutefois à qui vous prescrivez des bornes pour la conservation des
États et des hommes, enfin ce pardon des ennemis, sans lequel, ô mon
Dieu, nul ne vous verra\,; et vous l'accorderez à un prince qui vit
comme un homme, qui compte pour rien le bonheur éternel de vous voir. O
profondeur immense de vos jugements terribles qui, par l'usage et en
même temps par le mépris d'un présent si rare et si exquis, va faire
tout ce qui le peut conduire aux plus redoutables malheurs, et le va
faire non seulement sans éprouver en soi la plus légère violence
qu'éprouvent si fortement en ces occasions les personnes les plus à
Dieu, mais avec l'incurie, la facilité, l'insensibilité la plus
prodigieuse, la plus incroyable, la plus unique\,!»

Une si violente exclamation, précédée d'aussi fortes raisons, ébranla
assez M. le duc d'Orléans pour se mettre à raisonner sur le
dépouillement. Alors quoique sans espérance par sa mollesse, son peu de
tenue, l'intérêt et l'ensorcellement de l'abbé Dubois, mais pour n'avoir
rien à me reprocher à moi-même, je lui dis qu'il avait beau jeu à
réparer les fautes précédentes qui lui avaient fait tout pardonner au
plus cruel et au plus gratuit ennemi qui fut jamais, et au plus
continuellement acharné contre ses droits, son honneur et sa vie, ce que
lui-même ne se pouvait dissimuler\,; qu'au crime présent pour lequel le
duc du Maine se trouvait maintenant arrêté, il en pouvait rappeler deux
autres, et les faire d'autant mieux valoir, que le criminel avait
d'autant plus pernicieusement abusé du silence et de la patience à
l'égard de tous les deux\,: le premier, d'avoir attenté à se faire
prince du sang, puis à se faire déclarer capable de succéder à la
couronne, contre l'honneur de la loi de Dieu, contre la loi unanime de
la France et de tous les pays chrétiens, où le fils d'un double adultère
ne peut, en aucun cas, recueillir rien des biens de la famille dont il
est sorti, combien moins une couronne\,: contre le droit de la nation en
cas d'extinction de tous les mâles de la race régnante, contre le
respect et le droit des princes du sang, enfin contre la précieuse
vénération due à la loi salique qui distingue si grandement la couronne
de France de toutes les autres couronnes. Je le fis souvenir de ce que
je lui avais proposé à cet égard vers la fin de la vie du roi, pour
l'exécuter dès qu'il ne serait plus, et de la nécessité que je lui en
avais prouvée et de laquelle il n'était pas disconvenu de mettre un tel
frein à l'ambition de pouvoir être rendu capable de succéder à la
couronne, que la vue certaine de la profondeur du précipice retînt
bâtards, sujets trop puissants, premiers ministres, favoris démesurés,
princes étrangers trop établis et appuyés, d'attenter à ce crime qui en
prépare tant d'autres, et d'abuser ou de la folle tendresse, ou de la
faible complaisance, ou de l'âge, ou de l'imbécillité d'un roi, ou de
l'entêtement extravagant de sa toute-puissance même, pour renverser
l'État\,; que le silence sous lequel il l'avait laissé couler, avait
donné le temps au duc du Maine de commettre le second, de le tromper par
ce ramas de prétendue noblesse, dont plusieurs étaient, et de son aveu à
lui et des principaux de sa maison, en apparence, quoi qu'on eût pu lui
dire et follement, contre les ducs, en effet contre lui-même, comme il y
avait bientôt paru par leur belle requête au parlement, et de là par
l'appel des bâtards du régent, comme incompétent et impuissant, aux
états généraux ou au roi devenu majeur, autre crime d'État et toujours
connu et puni comme, tel de contester la puissance royale et d'en faire
aucune distinction du roi mineur ou majeur, et par là, M. du Maine
l'avait réduit en la presse où il s'était trouvé entre les princes du
sang et les bâtards, et après une longue et criante injustice, ou déni
de justice, en faveur des bâtards, forcé par leur audace à ventiler son
pouvoir de régent, de les déclarer déchus et non habiles à succéder à la
couronne, mais avec de tels ménagements de rangs et contre les termes
exprès de l'arrêt qu'il venait de rendre, que cette faiblesse avait
encouragé M. et M\textsuperscript{me} du Maine à entreprendre ce qui les
retenait maintenant en prison, dans la rage de n'avoir pas été maintenus
ou soufferts dans l'habilité de succéder à la couronne, et dans le
mépris de tout ce qui leur était conservé, compté par eux pour rien,
sinon pour une faiblesse sur laquelle ils pouvaient compter, quelque
chose qu'ils osassent entreprendre.

Après ce tableau ramassé et raccourci, je représentai à M. le duc
d'Orléans qu'au moins pouvait-il maintenant mettre deux aussi lourdes
fautes à profit et les faire bien payer à ces deux premiers crimes à
l'appui du troisième qui en était la suite et le fruit\,: reprendre le
premier, en montrer l'énormité, le danger extrême de l'exemple dans un
royaume très chrétien et l'unique qui suive la loi salique comme loi
fondamentale pour la succession à la couronne depuis tant de siècles,
l'exposer au sort de la Russie, à l'ambition de quiconque aurait la
force des établissements en main et qui posséderait un roi\,; faire
sentir que de se faire prince du sang et habile à succéder à la
couronne, après tous les princes du sang, comme fils de roi, de le
transmettre à sa postérité, à se faire préférer aux princes du sang,
comme bien plus proches qu'eux, par la qualité de fils du roi, il n'y
avait guère de distance, avec la force en main, et à quiconque obtient
ce droit, une violente tentation de se faire place nette et s'abréger le
chemin du trône\,; dire que le respect pour la mémoire du roi et la
considération d'une alliance, quoiqu'elle n'eût jamais dû être, l'estime
de la probité du comte de Toulouse, qui n'avait eu ni voulu avoir aucune
part aux démarches de son frère pour s'élever aussi monstrueusement,
avait arrêté Son Altesse Royale sur la justice qu'il devait aux princes
du sang, à la nation entière, à soi-même, d'une entreprise si
criminelle, qui ri allait à rien moins qu'à déshonorer la mémoire du feu
roi, quoiqu'on sût bien qu'il avait eu là-dessus la main forcée comme
sur les dispositions de son testament et de son codicille en faveur du
duc du Maine\,; que, le cas avenant, cette prétention à la couronne
pouvait renverser l'État par le choc des forces de l'intrus et de celles
de la nation qui ne se laisserait pas priver d'un si beau droit, qui lui
était si certainement et si constamment acquis, et dont les étrangers
sauraient profiter pour s'agrandir des provinces à leur bienséance\,; et
de là s'étendre sur la nécessité d'un châtiment tel qu'il ôtât pour
toujours un pareil dessein de la tête des plus ambitieux et des plus
puissants, et de celle des rois par orgueil ou par faiblesse, auxquels
le royaume n'appartient point comme une terre à un particulier, mais
comme un fidéicommis qui est perpétuellement affecté à l'aîné de
génération en génération, à moins qu'une couronne présente, une vaste
monarchie, un trône étranger vacant où un prince français est appelé,
par le testament du dernier roi mort sans postérité de lui ni de ses
prédécesseurs rois de sa maison, testament appuyé de l'exprès
consentement et des voeux de toute cette nation, ne fasse préférer une
couronne présente aux futurs les plus contingents, et que toute
l'Europe, avec la monarchie vacante, ne stipule la renonciation à la
possible succession, avec le gré et le consentement du roi de France et
les solennités célébrées pour cette renonciation\,; qu'un roi de France
n'a pas le pouvoir de disposer de sa couronne, laquelle suit de droit et
par elle-même cette aînesse de génération en génération\,; et si la race
masculine vient à manquer, le droit commun acquiert alors tout son
droit, qui donne à la nation celui de se choisir un roi et sa postérité
légitime masculine pour lui succéder tant qu'elle durera de génération
en génération par aînesse\,; appuyer sur l'attentat de troubler cet
ordre, et sur tous les points qui viennent d'être mis sous les yeux.

Passer de là au second crime\,: ameutement de gens à qui on fait usurper
le nom de la noblesse, sans convocation du roi, ou du régent en son nom,
s'il est mineur, à qui seul elle appartient, par conséquent sans
légitimes assemblées des bailliages pour le choix des députés, par
conséquent sans mission, sans pouvoir de personne, des gens ramassés de
toutes parts pour faire nombre, et dont plusieurs se trouveraient bien
empêchés de prouver leur noblesse\,; éblouir des gens distingués par la
leur à fraterniser en égaux avec ce vil mélange\,; abuser des fantaisies
qu'on leur a inspirées de loin pour les ramasser et les animer, se les
dévouer après à soi pour tout faire, jusqu'à avilir le nom du second,
mais du plus illustre des trois états, que ce ramas se prétend être, par
une requête au parlement, plus basse et plus humble que celle du moindre
particulier\,; de traiter le parlement de nosseigneurs, en nom collectif
de la noblesse, et avoir recours à sa justice, à son autorité, à sa
protection, au nom de la noblesse, et en chose où ces mêmes suppliants
prétendent le droit de juger. Se peut-il rien de plus contradictoire en
soi, de plus injurieux au second corps de l'État, en tous les points et
en tous les genres, de plus insultant au pouvoir du régent et à la
majesté royale, de plus visiblement et prochainement tendant à révolte
et à félonie, et sous un roi mineur, à nier toute autorité, pour n'en
reconnaître qu'autant qu'on le veut bien, et qu'elle peut et veut bien
servir aux vues qu'on s'est formées\,? Montrer enfin l'énormité de cet
attentat, le crime et le danger de ses diverses branches, qui ne
viennent d'être touchées qu'en deux mots.

Joindre à ces deux crimes le troisième qui a fait arrêter le duc et la
duchesse du Maine. Les preuves des deux premiers sont claires. De ce
dernier, qui est le fruit des deux premiers, les preuves seront
évidentes quand il plaira à l'abbé Dubois de montrer les papiers de
Cellamare et ceux de l'abbé Portocarrero, qui n'ont été vus que de lui
seul, et qui ne sont pas sortis de sous sa clef, et quand il plaira à
son maître de se faire l'effort de le lui commander de façon à se faire
obéir\footnote{Les représentations de Saint-Simon au régent n'étaient
  pas ignorées. Madame, mère du régent, écrivait le 7 juillet 1719\,:
  «\,Le duc de Saint-Simon s'impatienta une fois de la bonté de mon fils
  et lui dit en colère\,: \emph{Ah\,! vous voilà bien débonnaire\,;
  depuis Louis le Débonnaire on n'a rien vu d'aussi débonnaire que
  vous}. Mon fils faillit se rendre malade à force de rire.\,»}.

C'était bombarder rudement la faiblesse du régent, et tâcher à l'exciter
à force de boulets rouges. Je lui laissai prendre haleine et voulus voir
quel effet la batterie aurait produit. Il m'avait laissé tout dire sans
aucune interruption, et je lui voyais l'âme fort en peine. Nous fûmes
quelques moments en silence. Il le rompit le premier pour me répondre
que ce que je lui avais représenté était bel et bon sur M. et
M\textsuperscript{me} du Maine, mais que je ne prenais pas garde à ce
qui était avec eux de personnages engagés peut-être dans la même affaire
et sous les mêmes preuves, et à faire un si grand coup de filet, que le
filet en pourrait rompre.

Ma réplique fut prompte. Je l'assurai qu'il ne devait pas avoir assez
mauvaise opinion de mon jugement de n'avoir pas pensé à une partie si
principale de cette affaire, dont j'avais bien compté de l'entretenir,
après avoir achevé sur M. et M\textsuperscript{me} du Maine\,; que pour
venir à cette autre partie, je le suppliais de se représenter toutes les
conspirations qu'il avait lues, dont il n'y avait aucune qui n'eût son
chef, et des complices principaux et distingués par la force qu'ils y
pouvaient ajouter, outre le nombre des autres, dont les personnes
étaient de peu ou rien\,; qu'en cela on dépendait des preuves\,; qu'il
n'était pas permis de retrancher ni de grossir\,; que plus le nombre des
complices considérables serait grand, plus le crime du chef le serait,
et le danger de l'État aussi, plus la punition très sévère deviendrait
indispensable\,; plus la clémence et la justice devraient marcher de
front\,; plus le crime des personnages que le chef de la conspiration
aurait débauchés de leur devoir devait à plomb retomber sur sa tête\,;
plus la bonté du régent aurait de quoi se satisfaire, en montrant ne
chercher que la sûreté présente et future du royaume, et de la
succession à la couronne, par la punition du chef et du criminel de
trois grands crimes, comme du plus grand coupable, du plus dangereux ou
du seul dangereux, de celui qui ferait exemple à la postérité, et en
pardonnant généreusement aux personnages qu'il aurait entraînés, qui,
ensemble et par eux-mêmes, n'étaient point à craindre, et par la
timidité qu'il en avait éprouvée, et par les qualités de leur esprit, et
par l'impuissance de leurs établissements qui ne sont plus que des noms,
sans force et sans autorité dangereuse\,; qu'il prît bien garde que
passer les yeux clos à côté d'un tel complot, précédé de tant d'autres
par le même, était la plus insigne preuve de crainte et de faiblesse, et
le plus puissant convi à recommencer avec plus de succès\,; que voir le
crime d'une façon publique, telle que de mettre en prison le duc et la
duchesse du Maine, et leur pardonner après sans plus d'examen, revient
au premier\,; mais qu'articuler les preuves juridiquement, ne punir que
le chef et pardonner aux autres, si ce n'est à quelques gens obscurs
trop signalés, c'est courage, c'est justice, c'est exemple, c'est
sûreté, c'est générosité, c est clémence, c'est rendre à jamais les
personnages pardonnés hors de mesure d'oser remuer, et quelque
malveillants qu'ils puissent être, hors d'état de toute sorte
d'opposition, et par crainte et par honneur, en un mot c'est savoir
discerner, laisser les boucheries aux Christiern\footnote{Allusion aux
  cruautés de Christiern II ou Christian II, roi de Suède et de Danemark
  de 1520 à 1523. Ses actes de cruauté en Suède et en Danemark
  provoquèrent un soulèvement contre lui et le firent déposer en 1523.}
et aux Cromwell, ne vouloir que l'indispensable à l'exemple et à la
sûreté, n'être sévère que par la nécessité, et clément et généreux par
grandeur et par nature. Mais pour arriver à ce point il faut un jugement
juridique, où tous les pairs soient juridiquement convoqués et sans
excuses admises, parce qu'en cas de pairie et de crime, nulle sorte de
cause de récusation ne peut en exclure aucun\,; et appeler avec eux les
officiers de la couronne. J'ajoutai que le comte de Toulouse, n'ayant
trempé dans aucun des trois crimes de son frère, sa considération ne
devait ni ne pouvait retenir, puisqu'il était en pleine innocence, et
qu'à l'égard même de M\textsuperscript{me} du Maine, sa condamnation se
pouvait commuer à passer le reste de sa vie bien et sûrement enfermée,
sans communication avec personne, en faveur de sa qualité de princesse
du sang.

Le régent écouta tout, puis me dit\,: «\,Mais les enfants, qui sont
innocents, qu'en ferez-vous\,? --- Les enfants, repris-je, il est vrai
qu'ils sont innocents\,; mais il les faut empêcher de devenir coupables,
et leur ôter les ongles pour qu'ils ne pussent venger leurs malheurs
domestiques, ne leur laisser ni charge, ni gouvernement, ni le comté
d'Eu, petite province trop sur le bord de la mer et d'un petit port, et
trop voisine de l'Angleterre\,; ni Dombes, trop près de Savoie, qui ne
fut jamais qu'un franc alleu\footnote{C'est-à-dire une terre non soumise
  aux droits seigneuriaux.}, encore tout au plus, que les ducs de
Montpensier ont par degrés fait souveraineté, Mademoiselle encore plus,
à quoi M. du Mairie a fait mettre la dernière main, depuis le don que
Mademoiselle fut forcée de lui en faire, avec Eu et d'autres encore,
pour tirer M. de Lauzun de Pignerol. Il restera encore le duché d'Aumale
et de grands biens aux enfants de M. du Maine, dont vous leur ferez
prisent sur la confiscation, sans compter l'immensité de meubles, les
maisons et les pierreries, dont vous savez que lime du Maine en cacha et
en emporta pour un million, que La Billarderie découvrit et qu'il
rapporta, ce qui, pour le dire en passant, vous montre bien que
M\textsuperscript{me} du Maine n'avait perdu ni jugement ni desseins,
pour être arrêtée, et que ce million de pierreries n'était pas destiné à
la parer dans sa prison. J'appelle cela, ajoutai-je, faire un bon et
grand parti aux enfants qui sont innocents, et les mettre seulement hors
d'état de devenir criminels.\,»

M. le duc d'Orléans fut un peu ébranlé de ce plan et des raisons qui le
soutenaient. Il raisonna assez dessus avec moi. Mais je n'en conçus pas
une meilleure espérance. Ce plan, tout juste, tout sage, tout nécessaire
qu'il me paraissait, se trouvait en contradiction avec le naturel du
maître et, qui bien pis, avec les vues et l'intérêt de l'abbé Dubois, et
ce valet avait ensorcelé M. le duc d'Orléans. Je ne me trompai pas. Je
retrouvai ce prince s'affaiblissant tous les jours sur cette affaire, de
sorte que, content d'avoir fait ce que je croyais de mon devoir à tous
égards, je ne lui en parlai plus, et le mis ainsi fort à son aise sur
les divers et prompts adoucissements qu'il donna par reprises au duc et
à la duchesse du Maine jusqu'à leur liberté, et depuis. Je l'avais
pourtant fort flatté sur la distribution de leurs charges et
gouvernements, et je lui avais bien déclaré que je ne voulais d'aucun de
ces grands morceaux, ni même de leurs cascades, parce que je lui parlais
là-dessus sans aucun intérêt.

Je ne songeai donc plus à percer les mystères du complot et des
complices que l'abbé Dubois se réservait à lui seul, ni les dispositions
des prisonniers, dont Le Blanc ne me disait que des riens souvent
absurdes, parce qu'il ne lui était pas permis de me dire mieux\,; mais,
après le retour du duc et de la duchesse du Maine en leur précédent
état, je n'eus pas de peine à m'apercevoir, par l'amitié qu'ils ont
toujours depuis témoignée à Belle-Ile et à Le Blanc, qu'ils les avaient
bien et efficacement servis, même auprès de l'abbé Dubois, dont ils
avaient très bien suivi l'esprit et imité la politique. Elle réussit si
bien que bientôt, c'est-à-dire au commencement d'avril, Mine la
Princesse obtint que M\textsuperscript{me} du Maine, qui faisait la
malade, fût conduite de Dijon à Châlon-sur-Saône, avec la permission de
l'y aller voir.

On sut néanmoins en ce même temps par M. le duc d'Orléans, qui le rendit
public, qu'il avait quatre lettres au cardinal Albéroni du duc de
Richelieu\footnote{Le marquis d'Argenson écrit à la date de mars 1719,
  dans ses Mémoires manuscrits\,: «\,Le duc de Richelieu était
  véritablement coupable, quand on le fit mettre à la Bastille, environ
  ce temps-ci. Mon père (le garde des sceaux) fut cause de son arrêt\,;
  il s'en prit à lui et nous en voulait bien du mal. Cependant il est
  certain que ce duc avait des liaisons avec l'Espagne.\,» Le marquis
  d'Argenson raconte ensuite une anecdote qui se trouve dans les
  Mémoires imprimés, p.~192. (\emph{Mém. du marquis d'Argenson}, I
  vol.~in-8, 1825.)}, dont trois étaient signées de lui, qu'il
s'engageait à livrer Bayonne, où son régiment et celui de Saillant
étaient en garnison, pour quoi Saillant, qui était du complot, avait été
mis à la Bastille, et que le marché du duc de Richelieu était d'avoir le
régiment des gardes. Le rare est que, quatre jours après ce récit public
de M. le duc d'Orléans, auquel il ajouta que, si M. de Richelieu avait
quatre têtes, il avait dans sa poche de quoi les faire couper toutes
quatre, on donna à M. de Richelieu un de ses valets de chambre, des
livres, un trictrac et une basse de viole, qu'il demanda. On se moqua
dans le monde avec raison de la belle idée de deux jeunes colonels qui
se crurent assez maîtres de leurs régiments, et leurs régiments assez
maîtres de Bayonne, pour se figurer de pouvoir livrer cette
place\footnote{L'abbé Dubois écrivait au maréchal de Berwick, le 1er
  avril 1719\,: «\,Vous aurez été surpris sans doute d'apprendre, par le
  courrier que M. Le Blanc a dût vous dépêcher hier, que M. le duc de
  Richelieu devait livrer Bayonne aux Espagnols, et qu'il a été mis à la
  Bastille, où il n'est pas disconvenu de son intelligence avec
  Albéroni.\,» Le maréchal lui répondit le 17 avril\,: «\,Je n'ai point
  été surpris de l'aventure de M. de Richelieu, dont la conduite,
  jusqu'à présent, n'a pas été d'un homme sensé.\,»}. Qui m'aurait dit
que, moins de dix ans après, je serais chevalier de l'ordre, en même
promotion de huit que les deux fils du duc du Maine en princes du sang,
M. de Richelieu, Cellamare et d'Alègre, m'aurait bien étonné\footnote{La
  promotion de chevaliers de l'ordre, à laquelle Saint-Simon fait
  allusion, eut lieu le 1er janvier 1728\,; elle comprit les huit
  personnages suivants Le prince de Dombes, le comte d'Eu, les ducs de
  Richelieu, de Saint-Simon, de Giovenazzo (Cellamare), grand écuyer de
  la reine d'Espagne, les maréchaux de Roquelaure et d'Aligre, le comte
  de Grammont.}.

\hypertarget{chapitre-ix.}{%
\chapter{CHAPITRE IX.}\label{chapitre-ix.}}

1719

~

{\textsc{Conduite étrange de M\textsuperscript{me} la duchesse de Berry,
de Rion et de la Mouchy.}} {\textsc{- Conduite de M\textsuperscript{me}
de Saint-Simon.}} {\textsc{- Scandaleuse maladie de
M\textsuperscript{me} la duchesse de Berry, à {[}au{]} Luxembourg.}}
{\textsc{- Rion, conduit par le duc de Lauzun, son grand-oncle, épouse
secrètement M\textsuperscript{me} la duchesse de Berry.}} {\textsc{-
M\textsuperscript{me} la duchesse de Berry rouvre le jardin de
Luxembourg\,; se voue au blanc pour six mois\,; change de capitaine des
gardes.}} {\textsc{- Canillac et le marquis de Brancas entrent au
conseil des parties.}} {\textsc{- Prince Clément de Bavière est
{[}élu{]} évêque de Munster et de Paderborn.}} {\textsc{- Le cardinal
Albano est fait camerlingue.}} {\textsc{- Le duc d'Albret épouse de
nouveau la fille de feu Barbezieux.}} {\textsc{- Mort de
M\textsuperscript{me} de Maintenon.}} {\textsc{- Sa vie et sa conduite à
Saint-Cyr.}} {\textsc{- Mort d'Aubigny, archevêque de Rouen.}}
{\textsc{- Besons, archevêque de Bordeaux, lui succède\,; et le frère du
garde des sceaux, à Besons.}} {\textsc{- Érection de grands officiers de
l'ordre de Saint-Louis à l'instar de ceux de l'ordre du Saint-Esprit.}}
{\textsc{- Nouveaux règlements sur l'ordre de Saint-Louis, et leurs
inconvénients.}} {\textsc{- Extraction, caractère, fortune de Monti.}}
{\textsc{- Laval, dit la Mentonnière, mis, à la Bastille.}} {\textsc{-
Cellamare, duc de Giovenazzo, arrive en Espagne\,; est aussitôt fait
vice-roi de Navarre.}} {\textsc{- Rare baptême de Marton.}} {\textsc{-
L'abbesse de Chelles, soeur du maréchal de Villars, se démet et se
retire dans un couvent à Paris avec une pension de douze mille livres du
roi.}} {\textsc{- M\textsuperscript{me} d'Orléans lui succède, se démet,
se retire à la Madeleine.}} {\textsc{- Leur caractère.}} {\textsc{-
Diminution d'espèces.}} {\textsc{- Élargissement du quai du Louvre.}}
{\textsc{- Guichet, place et fontaine du Palais-Royal.}} {\textsc{-
Efforts peu heureux sur l'Écosse.}} {\textsc{- Tyrannie maritime des
Anglais.}} {\textsc{- Cilly prend le port du Passage et y brûle toute la
marine renaissante de l'Espagne.}} {\textsc{- Les plus confidents du duc
et de la duchesse du Maine sortent de la Bastille et sont mis en pleine
liberté.}} {\textsc{- Merveilles du Mississipi.}} {\textsc{- Law et le
régent me pressent d'en recevoir.}} {\textsc{- Je le refuse, mais je
reçois le payement d'anciens billets de l'épargne.}} {\textsc{- Blamont,
rappelé à sa charge, devient l'espion du régent, et le mépris et
l'horreur du parlement.}} {\textsc{- Mort de Pécoil père, digne d'un
avare, mais affreuse.}} {\textsc{- Digne refus, belle et sainte
retraite, curieuse, mais inintelligible déclaration, de l'abbé
Vittement, sur le règne sans bornes et sans épines du cardinal Fleury.}}
{\textsc{- Douze mille livres d'augmentation d'appointements et de
gouvernement à Castries.}}

~

M\textsuperscript{me} la duchesse de Berry vivait à son ordinaire dans
le mélange de la plus altière grandeur, et de la bassesse et de la
servitude la plus honteuse\,; des retraites les plus austères,
fréquentes, mais courtes aux Carmélites du faubourg Saint-Germain, et
des soupers les plus profanés par la vile compagnie, et la saleté et
l'impiété des propos\,; de la débauche la plus effrontée, et de la plus
horrible frayeur du diable et de la mort, lorsqu'elle tomba malade à
Luxembourg. Il faut tout dire, puisque cela sert à l'histoire, d'autant
plus qu'on ne trouvera dans ces Mémoires aucunes autres galanteries
répandues, que celles qui tiennent nécessairement à l'intelligence
nécessaire de ce qu'il s'est passé d'important ou d'intéressant dans le
cours des années qu'ils renferment. M\textsuperscript{me} la duchesse de
Berry ne voulait se contraindre sur rien\,; elle était indignée que le
monde osât parler de ce qu'elle-même ne prenait pas la peine de lui
cacher, et toutefois elle était désolée de ce que sa conduite était
connue. Elle était grosse de Rion, elle s'en cachait tant qu'elle
pouvait. M\textsuperscript{me} de Mouchy était leur commode, quoique les
choses à cet égard se passassent tambour battant. Rion et la Mouchy
étaient amoureux l'un de l'autre, et vivaient avec toute sorte de
privances et de facilité pour les avoir. Ils se moquaient ensemble de la
princesse qui était leur dupe, et de qui ils tiraient de concert tout ce
qu'ils pouvaient. En un mot, ils étaient les maîtres d'elle et de sa
maison, et l'étaient avec insolence, jusque-là que M. {[}le duc{]} et
M\textsuperscript{me} la duchesse d'Orléans qui les connaissaient et les
haïssaient, les craignaient et les ménageaient. M\textsuperscript{me} de
Saint-Simon, fort à l'abri de tout cela, extrêmement aimée et respectée
de foute la maison, et respectée même de ce couple qui se faisait tant
redouter et compter, ne voyait M\textsuperscript{me} la duchesse de
Berry que pour les moments de représentation qu'elle arrivait à
Luxembourg, dont elle revenait dès qu'elle était finie, et ignorait
parfaitement tout ce qu'il s'y passait, quoiqu'elle en fût parfaitement
instruite.

La grossesse vint à terme, et ce terme mal préparé par les soupers
continuels fort arrosés de vins et de liqueurs les plus fortes devint
orageux et promptement dangereux. M\textsuperscript{me} de Saint-Simon
ne put éviter de s'y rendre assidue dès que le péril parut, mais jamais
elle ne céda aux instances de M. {[}le duc{]} et de
M\textsuperscript{me} la duchesse d'Orléans et de toute la maison, ni
pour y coucher dans l'appartement qu'on lui avait toujours réservé, et
où elle ne mit jamais le pied, ni même pour y passer les journées, sous
prétexte de venir se reposer chez elle. Elle trouva
M\textsuperscript{me} la duchesse de Berry retranchée dans une petite
chambre de son appartement, qui avait des dégagements commodes et hors
de portée, et qui que ce fût dans cette chambre que la Mouchy et Rion et
une femme ou deux de garde-robe affidées. Le nécessaire au secours avait
les dégagements libres. M. {[}le duc{]} et M\textsuperscript{me} la
duchesse d'Orléans, Madame même n'entraient pas quand ils voulaient, à
plus forte raison la dame d'honneur ni les autres dames, la première
femme de chambre ni les médecins. Tout cela entrait de fois à autre,
mais des instants. Un grand mal de tête ou le besoin de sommeil les
faisait souvent prier de vouloir bien ne point entrer, et quand ils
entraient de s'en aller après quelques instants. Eux-mêmes, qui ne
voyaient que trop de quoi il s'agissait, ne se présentaient pas le plus
souvent pour entrer, se contentaient de savoir des nouvelles par
M\textsuperscript{me} de Mouchy qui entre-bâillait à peine la porte, et
ce manége ridicule qui se passait devant la foule du Luxembourg, du
Palais-Royal, et de beaucoup d'autres gens qui, par bienséance ou par
curiosité venaient savoir des nouvelles, devint la conversation de tout
le monde.

Le danger redoublant, Languet, célèbre curé de Saint-Sulpice, qui déjà
s'était rendu assidu, parla des sacrements à M. le duc d'Orléans. La
difficulté fut qu'il pût entrer pour les proposer à
M\textsuperscript{me} la duchesse de Berry. Mais il s'en trouva bientôt
une plus grande. C'est que le curé, en homme instruit de ses devoirs,
déclara qu'il ne les administrerait point, ni ne souffrirait qu'ils lui
fussent administrés, tant que Rion et M\textsuperscript{me} de Mouchy
seraient non seulement dans sa chambre, mais dans le Luxembourg. Il le
fit tout haut, et devant tout le monde, exprès à M. le duc d'Orléans qui
en fut moins choqué qu'embarrassé. Il prit le curé à part, et le tint
longtemps à tâcher de lui faire goûter quelques tempéraments. Le voyant
inflexible, il lui proposa à la fin de s'en rapporter au cardinal de
Noailles. Le curé l'accepta sur-le-champ, et promit de déférer à ses
ordres comme étant son évêque, pourvu qu'il eût la liberté de lui
expliquer ses raisons. L'affaire pressait, et M\textsuperscript{me} la
duchesse de Berry se confessait pendant cette dispute à un cordelier son
confesseur. M. le duc d'Orléans se flatta sans doute de trouver le
diocésain plus flexible que le curé avec lequel il était très opposé de
sentiment sur la constitution, et qui pour la même affaire était si fort
entre les mains du régent\,; s'il l'espéra, il se trompa.

Le cardinal de Noailles arriva\,; M. le duc d'Orléans le prit à l'écart
avec le curé, et la conversation dura plus d'une demi-heure. Comme la
déclaration du curé avait été publique, le cardinal-archevêque de Paris
jugea à propos que la sienne la fût aussi. En se rapprochant tous les
trois du monde et de la porte de la chambre, le cardinal de Noailles dit
tout haut au curé qu'il avait fait très dignement son devoir, qu'il n'en
attendait pas moins d'un homme de bien, éclairé comme il l'était, et de
son expérience\,; qu'il le louait de ce qu'il exigeait, avant
d'administrer ou de laisser administrer les sacrements à
M\textsuperscript{me} la duchesse de Berry\,; qu'il l'exhortait à ne
s'en pas départir et à ne se laisser pas tromper sur une chose aussi
importante\,; que, s'il avait besoin de quelque chose de plus pour être
autorisé, il lui défendait, comme son évêque diocésain et son supérieur,
de laisser administrer ou d'administrer lui-même les sacrements à
M\textsuperscript{me} la duchesse de Berry, tant que M. de Rion et
M\textsuperscript{me} de Mouchy seraient dans la chambre, même dans le
Luxembourg, et n'en seraient pas congédiés. On peut juger de l'éclat
d'un si indispensable scandale, de l'effet qu'il fit dans cette pièce si
remplie, de l'embarras de M. le duc d'Orléans, du bruit que cela fit
incontinent partout. Qui que ce soit, pas même les chefs de la
constitution, les plus violents ennemis du cardinal de Noailles, les
évêques du plus bel air, les femmes du plus grand monde, les libertins
même, pas un seul ne blâma ni le curé ni son archevêque, les uns par
savoir les règles ou par n'oser les impugner, le gros et le plus
nombreux par l'horreur de la conduite de M\textsuperscript{me} la
duchesse de Berry, et par la haine que son orgueil lui attirait.

Question après entre le régent, le cardinal et le curé, tous trois dans
le coin de la porte, qui d'eux porterait cette résolution à
M\textsuperscript{me} la duchesse de Berry, qui ne s'attendait à rien
moins, et qui toute confessée, comptait à tous moments de voir entrer le
saint sacrement et le recevoir. Après un court colloque, que l'état de
la malade pressa, le cardinal et le curé s'éloignèrent un peu tandis que
M. le duc d'Orléans se fit entr'ouvrir la porte et appeler
M\textsuperscript{me} de Mouchy. Là, toujours la porte entr'ouverte,
elle dedans, lui dehors, il lui déclara de quoi il était question. La
Mouchy, bien étonnée, encore plus indignée, le prit sur le haut ton, dit
ce qu'il lui plut sur son mérite et sur l'affront que des cagots
entreprenaient de lui faire et à M\textsuperscript{me} la duchesse de
Berry, qui ne le souffrirait et n'y consentirait jamais, et qui la
ferait mourir dans l'état où elle était, si on avait l'imprudence et la
cruauté de le lui dire. La conclusion pourtant fut que la Mouchy se
chargea d'aller dire à M\textsuperscript{me} la duchesse de Berry ce qui
était résolu sur les sacrements\,; on peut juger ce qu'elle y sut
ajouter du sien. La réponse négative ne tarda pas à être rendue par la
même à M. le duc d'Orléans, en entre-bâillant la porte. Avec une telle
commissionnaire, il devait bien s'attendre à la réponse qu'il en reçut.
Aussitôt après, il fut la rendre au cardinal et au curé\,; le curé ayant
là son archevêque, et de même avis que lui, se contenta de hausser les
épaules. Mais le cardinal dit à M. le duc d'Orléans que
M\textsuperscript{me} de Mouchy, l'une des deux personnes indispensables
à renvoyer et sans retour, n'était guère propre à faire entendre règle
et raison à M\textsuperscript{me} la duchesse de Berry\,; que c'était à
lui, son père, à lui porter cette parole et à la porter à faire le
devoir d'une chrétienne, si près de paraître devant Dieu, et le pressa
d'aller lui parler. On n'aura pas peine à croire que son éloquence n'y
gagna rien. Ce prince craignait trop sa fille et aurait été un faible
apôtre avec elle.

Le refus réitéré fit prendre sur-le-champ au cardinal le parti de parler
lui-même à M\textsuperscript{me} la duchesse de Berry, accompagné du
curé\,; et comme il voulait s'y acheminer tout de suite, M. le duc
d'Orléans, qui n'osa l'en empêcher, mais qui eut peur de quelque
révolution subite et dangereuse dans M\textsuperscript{me} sa fille, à
l'aspect et au discours des deux pasteurs, le conjura d'attendre qu'on
l'eût disposée à les voir. Il alla donc faire un autre colloque dans
cette porte qu'il se fit entre-bâiller, dont le succès fut pareil au
précédent. M\textsuperscript{me} la duchesse de Berry se mit en furie,
répondit des emportements contre ces cafards qui abusaient de son état
et de leur caractère pour la déshonorer par un éclat inouï, et n'épargna
pas M. son père de sa sottise et de sa faiblesse de le souffrir. Qui
l'aurait crue, on aurait fait sauter les degrés au cardinal et au curé.
M. le duc d'Orléans revint à eux fort petit et fort en peine, et qui ne
savait que faire entre sa fille et eux. Il leur dit qu'elle était si
faible et si souffrante qu'il fallait qu'ils différassent, et les
entretint comme il put. L'attention et la curiosité de tout ce grand
monde qui remplissait cette pièce était extrême, qui sut enfin ce détail
par-ci par-là, et tout de suite après dans la journée.
M\textsuperscript{me} de Saint-Simon, avec quelques dames de
M\textsuperscript{me} la duchesse de Berry, et quelques autres qui
étaient venues savoir des nouvelles, était assise dans une embrasure de
fenêtre, un peu au loin, qui voyait tout ce manége, et qui de temps en
temps était instruite de ce qui se passait.

Le cardinal de Noailles demeura plus de deux heures avec M. le duc
d'Orléans, desquels à la fin le monde principal se rapprocha. Le
cardinal voyant enfin qu'il ne pouvait entrer dans la chambre, sans une
sorte de violence et fort contraire à la persuasion, trouva indécent
d'attendre inutilement davantage. En s'en allant il réitéra ses ordres
au curé, et lui recommanda de veiller à n'être point trompé sur les
sacrements qu'on tenterait peut-être d'administrer clandestinement. Il
s'approcha ensuite de M\textsuperscript{me} de Saint-Simon, la prit en
particulier, lui conta ce qui s'était passé, s'en affligea avec elle et
de tout l'éclat qu'il n'avait pu éviter. M. le duc d'Orléans se hâta
d'annoncer à M\textsuperscript{me} sa fille le départ du cardinal, dont
lui-même se trouva fort soulagé. Mais en sortant de la chambre, il fut
étonné de trouver le curé collé tout près de la porte, et encore plus de
la déclaration qu'il lui fit que c'était là le poste qu'il avait pris et
dont rien ne le ferait sortir, parce qu'il ne voulait pas être trompé
sur les sacrements. En effet, il y demeura ferme quatre jours, et les
nuits de même, excepté de courts intervalles pour la nourriture et
quelque repos qu'il allait prendre chez lui, fort près de Luxembourg, et
laissait en son poste deux prêtres jusqu'à son retour\,; enfin, le
danger passé, il leva le siège.

M\textsuperscript{me} la duchesse de Berry, bien accouchée d'une fille,
n'eut plus qu'à se rétablir, mais dans un emportement égal contre le
curé et contre le cardinal de Noailles auxquels elle ne l'a jamais
pardonné, et fut de plus en plus ensorcelée des deux amants qui se
moquaient d'elle, et qui ne lui étaient attachés que pour leur fortune
et leur intérêt, qui restèrent encore du temps enfermés avec elle sans
voir M. {[}le duc{]} et M\textsuperscript{me} la duchesse d'Orléans qu'à
peine et des moments, Madame de même, mais qui, excepté les premiers
jours, n'y allait presque point\footnote{Voy. la \emph{Correspondance de
  M\textsuperscript{me} la duchesse d'Orléans}, lettres du 23 mai 1719,
  18 juin, 17, 18, 19, et 22 juillet de la même année.}.

M\textsuperscript{me} la duchesse de Berry ne se voulait pas montrer à
qui que ce fût en couche, ni se contraindre là-dessus pour personne.
Personne aussi, à commencer par M\textsuperscript{me} de Saint-Simon,
n'eut d'empressement à la voir, parce que personne n'ignorait ce qui
tenait la porte close. M\textsuperscript{me} de Saint-Simon la vit
pourtant des instants, mais c'était toujours M\textsuperscript{me} la
duchesse de Berry qui lui mandait d'entrer, sans que
M\textsuperscript{me} de Saint-Simon lui en eût fait rien dire, ni
qu'elle s'y fût présentée\,; elle y demeurait des moments, prenait pour
bon ce que M\textsuperscript{me} la duchesse de Berry lui disait de sa
santé, et se retirait au plus vite.

Rion, comme on l'a dit, cadet de Gascogne qui n'avait rien, quoique de
bonne maison, était petit-fils d'une soeur du duc de Lauzun, dont les
aventures avec Mademoiselle, qui voulut l'épouser, ne sont ignorées de
personne. Cette parité de son neveu et de lui leur mit en tête le même
mariage. Cette pensée délectait l'oncle qui se croyait revivre en la
personne de son neveu, et qui le conduisait dans cette trame. L'empire
absolu qu'il avait usurpé sur cette impérieuse princesse, à qui, de
propos délibéré, il faisait chaque jour essuyer des caprices qui lui
ôtaient jusqu'à la moindre liberté, et des humeurs brutales qui la
faisaient pleurer tous les jours et plus d'une fois, le danger qu'elle
avait couru dans sa couche, l'horreur de l'éclat où elle s'était vue
entre les derniers sacrements, et la rupture entière avec ce dont elle
était affolée, la peur du diable qui la mettait hors d'elle-même au
moindre coup de tonnerre, qu'elle n'avait jamais craint jusqu'alors,
enhardirent l'oncle et le neveu. C'était l'oncle qui avait conseillé à
son neveu de traiter sa princesse comme il avait lui-même traité
Mademoiselle. Sa maxime était que les Bourbons voulaient être rudoyés et
menés le bâton haut, sans quoi on ne pouvait se conserver sur eux aucun
empire. Rion, maître du coeur de la Mouchy, qui l'était de l'esprit de
leur princesse, lui fut d'un merveilleux usage à son dessein. Tous deux
y trouvaient leur compte. Ils avaient tremblé de l'éclat qui venait
d'arriver sur eux, dont l'occasion pouvait revenir encore et les perdre.
La peur du diable et des réflexions pouvaient à la fin produire le même
effet, au lieu que Rion n'avait plus rien à craindre et n'avait
{[}qu'à{]} jouir de la plus incompréhensible fortune en réussissant à
épouser, et la Mouchy à se tout promettre d'une union où elle aurait
tant de part et tous deux sûrs de se posséder l'un l'autre, sans
appréhender rien pour leurs secrets plaisirs. Je m'en tiens ici à cette
préparation de scène, qui commença au plus tard à l'époque de cette
maladie et de l'éclat dont on vient de parler. Il n'est pas temps encore
d'en dire davantage.

M\textsuperscript{me} la duchesse de Berry, infiniment peinée de la
façon dont tout le monde, jusqu'au peuple, avait pris sa maladie et ce
qu'il s'y était passé, crut regagner quelque chose en faisant rouvrir au
public les portes du jardin de Luxembourg, qu'elle avait fait fermer il
y avait longtemps. On en fut bien aise\,: on en profita\,; mais ce fut
tout. Elle se voua au blanc pour six mois. Ce voeu fit un peu rire le
monde. Il survint quelques piques avec le marquis de La Rochefoucauld,
qui remit sa place de capitaine des gardes, que M\textsuperscript{me} la
duchesse de Berry donna au comte d'Uzès, car, pourvu qu'elle eût des
noms, elle n'en cherchait pas davantage.

Canillac et le marquis de Brancas, qui avaient des expectatives de
conseiller d'État, obtinrent, en attendant les places, d'en faire les
fonctions avec les appointements.

Le prince Clément fut élu évêque de Munster, au lieu de son frère, mort
à Rome, et aussitôt après, de Paderborn. Le pape donna au cardinal
Albano, son neveu, la charge de camerlingue\footnote{Le camerlingue
  était autrefois le président de la chambre apostolique, et en cette
  qualité il était à proprement parler le représentant de la puissance
  temporelle de l'Église. Les clercs de la chambre apostolique qui
  formaient son conseil se partageaient les attributions réparties
  aujourd'hui entre les différents ministères. Si le camerlingue a perdu
  de son pouvoir en temps ordinaire, il a conservé le privilège
  d'exercer l'autorité temporelle pendant les premiers jours qui suivent
  la mort du pape. Durant le conclave, il ne fait plus que partager le
  pouvoir avec ce que l'on nomme les chefs d'ordre, c'est-à-dire un
  cardinal-évêque, un cardinal-prêtre et un cardinal-diacre, délégués
  comme représentants du sacré collège.}, {[}vacante{]} par la mort du
cardinal Spinola.

Le duc d'Albret, qui avait épousé une fille de feu M. et
M\textsuperscript{me} de Barbezieux, malgré toute la famille, et plaidé
fortement là-dessus au parlement, puis au conseil de régence, refit son
mariage, suivant l'arrêt de ce conseil. Il épousa donc une seconde fois
sa femme chez Caumartin, conseiller d'État, dont le frère, évêque de
Vannes, leur donna à minuit la bénédiction nuptiale dans la chapelle de
la maison. Si on savait et si on se souciait en l'autre monde de ce qui
se passe en celui-ci, je pense que M. de Turenne et M. de Louvois
seraient tous deux bien étonnés.

Le samedi au soir 15 avril, veille de la Quasimodo, mourut à Saint-Cyr
la célèbre et fatale M\textsuperscript{me} de Maintenon. Quel bruit cet
événement en Europe, s'il fût arrivé quelques années plus tôt\,! On
l'ignora peut-être à Versailles, qui en est si proche\,; à peine en
parla-t-on à Paris. On s'est tant étendu sur cette femme trop et si
malheureusement fameuse, à l'occasion de la mort du roi, qu'il ne reste
rien à en dire que depuis cette époque. Elle a tant, si puissamment et
si funestement figuré pendant trente-cinq années, sans la moindre
lacune, que tout, jusqu'à ses dernières années de retraite, en est
curieux.

Elle se retira à Saint-Cyr au moment même de la mort du roi, et eut le
bon sens de s'y réputer morte au monde, et de n'avoir jamais mis le pied
hors de la clôture de cette maison. Elle ne voulut y voir personne du
dehors sans exception, que du très petit nombre dont on va parler, rien
demander, ni recommander à personne, ni se mêler de rien où son nom pût
être mêlé. M\textsuperscript{me} de Caylus, M\textsuperscript{me} de
Dangeau, M\textsuperscript{me} de Lévi étaient admises, mais peu
souvent, les deux dernières encore plus rarement, à dîner. Le cardinal
de Rohan la voyait toutes les semaines, le duc du Maine aussi, et
passait trois et quatre heures avec elle tête à tête. Tout lui riait
quand on le lui annonçait. Elle embrassait son mignon avec la dernière
tendresse, quoiqu'il puât bien fort, car elle l'appelait toujours ainsi.
Assez souvent le duc de Noailles, dont elle paraissait se soucier
médiocrement, de sa femme encore moins, quoique sa propre nièce, qui y
allait fort rarement et d'un air contraint, et mal volontiers\,; aussi
la réception était pareille\,; le maréchal de Villeroy, tant qu'il en
pouvait prendre le temps et toujours avec grand accueil\,; presque point
le cardinal de Bissy\,; quelques évêques obscurs et fanatiques
quelquefois\,; assez souvent l'archevêque de Rouen, Aubigny\,; Bloin de
temps en temps\,; et l'évêque de Chartres, Mérinville, diocésain et
supérieur de la maison.

Une fois la semaine, quand la reine d'Angleterre était à Saint-Germain,
{[}elle{]} allait dîner avec elle, mais de Chaillot, où elle passait des
temps considérables, elle n'y allait pas. Elles avaient chacune leur
fauteuil égal, vis-à-vis l'une de l'autre. À l'heure du dîner, on
mettait une table entre elles deux, leur couvert, les premiers plats et
une cloche. C'était les jeunes demoiselles de la chambre qui faisaient
tout ce ménage, et qui leur servaient à boire, des assiettes et un
nouveau service quand la cloche les appelait\,; la reine leur témoignait
toujours quelques bontés. Le repas fini, elles desservaient et ôtaient
tout de la chambre, puis apportaient et rapportaient le café. La reine y
passait deux ou trois heures tête à tête, puis elles s'embrassaient\,;
M\textsuperscript{me} de Maintenon faisait trois ou quatre pas en la
recevant et en la conduisant\,; les demoiselles, qui étaient dans
l'antichambre, l'accompagnaient à son carrosse, et l'aimaient fort,
parce qu'elle leur était fort gracieuse.

Elles étaient charmées surtout du cardinal de Rohan, qui ne venait
jamais les mains vides, et qui leur apportait des pâtisseries et des
bonbons de quoi les régaler plusieurs jours. Ces bagatelles faisaient
plaisir à M\textsuperscript{me} de Maintenon. Il est pourtant vrai
qu'avec ce peu de visites, qui ne se hasardaient point qu'elle n'en
marquât le jour et l'heure, qu'on envoyait lui demander, excepté son
mignon, toujours reçu à bras ouverts, il arrivait rarement des journées
où elle n'eût personne. Ces temps-là et les vides des matinées étaient
remplis par beaucoup de lettres qu'elle recevait et de réponses qu'elle
faisait, presque toutes à des supérieurs de communautés de prêtres ou de
séminaires, à des abbesses, même à de simples religieuses\,; car le goût
de direction surnagea toujours à tout, et comme elle écrivait
singulièrement bien et facilement, elle se plaisait à dicter ses
lettres. Tous ces détails, je les ai sus de M\textsuperscript{me} de
Tibouville, qui était Rochechouart, sans aucun bien, et mise enfant à
Saint-Cyr.

M\textsuperscript{me} de Maintenon, outre ses femmes de chambre, car nul
homme de ses gens n'entrait dans la clôture, avait deux, quelquefois
trois anciennes demoiselles et six jeunes pour être de sa chambre, dont,
vieilles et jeunes, elle changeait quelquefois. M\textsuperscript{lle}
de Rochechouart fut une des jeunes\,; elle la prit en amitié, et autant
en une sorte de petite confiance que son âge le pouvait permettre\,; et
comme elle lui trouvait de l'esprit et la main bonne, c'était à elle
qu'elle dictait toujours. Elle n'est sortie de Saint-Cyr qu'après la
mort de M\textsuperscript{me} de Maintenon, qu'elle a toujours fort
regrettée, quoiqu'elle ne lui ait rien donné. Le mariage que son total
manquement de bien fit faire pour elle à d'Antin, qui l'eut toujours
chez lui depuis sa sortie de Saint-Cyr, ne fut pas heureux. Tibouville
mangea son bien à ne rien faire, quoique très considérable, vendit son
régiment dès que la guerre pointa, et se conduisit de façon que sa femme
n'eut de ressource qu'à se retirer chez l'évêque d'Évreux, son frère. La
maison de campagne de l'évêché d'Évreux n'est qu'à cinq petites lieues
de la Ferté\,; nous voisinions continuellement, et ils passaient souvent
des mois entiers à la Ferté. Ce détail est peu intéressant\,; mais ce
que je n'ai pas vu ou manié moi-même, je veux citer comment je le sais,
et d'où je l'ai pris.

M\textsuperscript{me} de Maintenon, comme à la cour, se levait matin et
se couchait de bonne heure. Ses prières duraient longtemps\,; elle
lisait aussi elle-même des livres de piété, quelquefois elle se faisait
lire quelque peu d'histoire par ses jeunes filles, et se plaisait à les
faire raisonner dessus et à les instruire. Elle entendait la messe d'une
tribune tout contre sa chambre, souvent quelques offices, très rarement
dans le choeur. Elle communiait, non comme le dit Dangeau dans ses
Mémoires, ni tous les deux jours, ni à minuit, mais deux fois la
semaine, ordinairement entre sept et huit heures du matin, puis revenait
dans sa tribune, où ces jours-là elle demeurait longtemps.

Son dîner était simple, mais délicat et recherché dans sa simplicité, et
très abondant en tout. Le duc de Noailles, après Mornay et Bloin, ne la
laissaient pas manquer de gibier de Saint-Germain et de Versailles, ni
les bâtiments de fruits. Quand elle n'avait point de dames de dehors,
elle mangeait seule, servie par ces demoiselles de sa chambre, dont elle
faisait mettre quelques-unes à table trois ou quatre fois l'an tout au
plus. M\textsuperscript{lle} d'Aumale, qui était vieille, et qu'elle
avait eue longtemps à la cour, n'était pas de ce côté la plus
distinguée. Il y avait un souper neuf pour cette M\textsuperscript{lle}
d'Aumale et pour les demoiselles de la chambre, dont elle était comme la
gouvernante. M\textsuperscript{me} de Maintenon ne prenait rien le
soir\,; quelquefois, dans les fort beaux jours sans vent, elle se
promenait un peu dans le jardin.

Elle nommait toutes les supérieures, première et subalternes, et toutes
les officières. On lui rendait un compte succinct du courant\,; mais, de
tout ce qui était au delà, la première supérieure prenait ses ordres.
Elle était Madame tout court dans la maison, où tout était en sa main\,;
et, quoiqu'elle eût des manières honnêtes et douces avec les dames de
Saint-Cyr, et de bonté avec les demoiselles, toutes tremblaient devant
elle. Il était infiniment rare qu'elle en vît d'autres que les
supérieures et les officières, encore n'était-ce que lorsqu'elle en
envoyait chercher, ou, encore plus rarement, quand quelqu'une se
hasardait de lui faire demander une audience, qu'elle ne refusait pas.
La première supérieure venait chez elle quand elle voulait, mais sans en
abuser\,; elle lui rendait compte de tout et recevait ses ordres sur
tout. M\textsuperscript{me} de Maintenon ne voyait guère qu'elle. Jamais
abbesse, fille de France, comme il y en a eu autrefois, n'a été si
absolue, si ponctuellement obéie, si crainte, si respectée, et, avec
cela, elle était aimée de presque tout ce qui était enfermé dans
Saint-Cyr. Les prêtres du dehors étaient dans la même soumission et dans
la même dépendance. Jamais, devant ses demoiselles, elle ne parlait de
rien qui pût approcher du gouvernement ni de la cour, assez souvent du
feu roi avec éloge, mais sans enfoncer rien, et ne parlant jamais des
intrigues, des cabales, ni des affaires.

On a vu que lorsque, après la déclaration de la régence, M. le duc
d'Orléans alla voir M\textsuperscript{me} de Maintenon à Saint-Cyr, elle
ne lui demanda quoi que ce soit, que sa protection pour cette maison. Il
l'assura, elle, M\textsuperscript{me} de Maintenon, que les quatre mille
livres que le roi lui donnait tous les mois lui seraient payées de même
avec exactitude chaque premier jour des mois, et cela fut toujours très
ponctuellement exécuté. Ainsi, elle avait du roi quarante-huit mille
livres de pension. Je ne sais même si elle n'avait pas conservé celle de
gouvernante des enfants du roi et de M\textsuperscript{me} de Montespan,
quelques autres qu'elle avait dans ce temps-là, et les appointements de
seconde dame d'atours de M\textsuperscript{me} la dauphine-Bavière,
comme la maréchale de Rochefort, première dame d'atours de la même,
conservait encore les siens, et comme la duchesse d'Arpajon, dame
d'honneur, avait touché les siens tant qu'elle avait vécu, depuis la
mort de M\textsuperscript{me} la dauphine-Bavière. Outre cela,
M\textsuperscript{me} de Maintenon jouissait de la terre de Maintenon et
de quelques autres biens. Saint-Cyr, par sa fondation, était chargé, en
cas qu'elle s'y retirât, de la loger, elle et tous ses domestiques et
équipages, et de les nourrir, gens et chevaux, tant qu'elle en voudrait
avoir, pour rien, aux dépens de la maison, ce qui fut fidèlement exécuté
jusqu'aux bois, charbon, bougie, chandelle, en un mot, sans que, pour
elle, ni pour pas un de ses gens ni chevaux, il lui en coûtât un sou, en
aucune sorte que ce puisse être, que pour l'habillement de sa personne
et de sa livrée. Elle avait au dehors un maître d'hôtel, un valet de
chambre, des gens pour l'office et la cuisine, un carrosse, un attelage
de sept ou huit chevaux, et un ou deux de selle, et, au dedans,
M\textsuperscript{lle} d'Aumale et ses femmes de chambre, et les
demoiselles dont on a parlé, mais qui étaient de Saint-Cyr\,: toute sa
dépense n'était donc qu'en bonnes oeuvres et en gages de ses
domestiques.

J'ai souvent admiré que les maréchaux, d'Harcourt si intrinsèquement lié
avec elle, Tallard, Villars qui lui devait tant, M\textsuperscript{me}
du Maine et ses enfants pour qui elle avait fait fouler aux pieds toutes
les lois divines et humaines, le prince de Rohan et tant d'autres ne
l'aient jamais vue.

La chute du duc du Maine au lit de justice des Tuileries lui donna le
premier coup de mort. Ce n'est pas trop présumer que de se persuader
qu'elle était bien instruite des mesures et des desseins de ce mignon,
et que cette espérance l'ait soutenue, mais quand elle le vit arrêté,
elle succomba\,; la fièvre continue la prit, et elle mourut à
quatre-vingt-trois ans, avec toute sa tête et tout son esprit.

Les regrets de sa perte, qui ne furent pas universels dans Saint-Cyr,
n'en passèrent guère les murailles. Je n'ai su qu'Aubigny, archevêque de
Rouen, son prétendu cousin, qui fut assez sot pour en mourir. Il fut
tellement saisi de cette perte qu'il en tomba malade et la suivit
bientôt. Besons, archevêque de Bordeaux, passa à Rouen, et Argenson,
archevêque d'Embrun, frère du garde des sceaux, passa à l'archevêché de
Bordeaux.

M. le duc d'Orléans fit ériger des officiers de l'ordre de Saint-Louis
presqu'à l'instar de celui du Saint-Esprit, avec des appointements et
des marques, moyennant finance à proportion. Le garde des sceaux fut
chancelier et garde des sceaux de cet ordre\,; Le Blanc, prévôt et
maître des cérémonies\,; Armenonville, en râpé\,; et Morville, son fils,
en titre de greffier. Bientôt après, le garde des sceaux, conservant les
marques, fit passer sa charge à son second fils, dont l'aîné eut le
râpé. Tous ceux-là portèrent le grand cordon rouge et la croix brodée
d'or, cousue sur leurs habits. Trois gros trésoriers de la marine et de
l'extraordinaire des guerres\footnote{On appelait \emph{extraordinaire
  des guerres} un fonds spécial destiné à payer les dépenses
  extraordinaires de la guerre.} furent trésoriers de l'ordre et
portèrent le grand cordon rouge comme les commandeurs, mais non la croix
brodée sur leurs habits, comme les grand'croix et comme les trois
principales charges, ci-devant dites. D'autres gens moindres, la plupart
des bureaux, eurent les autres petites charges avec la croix à la
boutonnière, comme les simples chevaliers. Bientôt après il fut réglé,
au conseil de régence, que les rachats qui revenaient au roi seraient
affectés par un édit enregistré à l'ordre de Saint-Louis, et que les
grand'croix commandeurs et même les chevaliers de Saint-Louis qui
avaient des pensions sur cet ordre les perdraient s'ils devenaient
chevaliers du Saint-Esprit.

Ces deux règlements passèrent\,: le premier en forme, l'autre par
l'usage, malgré leurs inconvénients. Celui du premier regardait
essentiellement tout le monde, parce qu'il ôtait au roi la liberté de
remettre les rachats qui lui étaient dus, et à ses sujets de toute
qualité une gratification qui s'accordait aisément pour peu que les
débiteurs de ces rachats fussent graciables par leurs services ou par
leur considération\,; le second, parce que le cordon bleu ne valant que
mille écus\,; et les grandes croix, les unes six mille livres, les
autres huit mille livres\,; les commanderies, les unes quatre mille
livres, les autres six mille livres\,; et les pensions des chevaliers,
plusieurs de mille livres, de quinze cents livres et de deux milles
livres, il se pouvait trouver parmi tous ceux-là des maréchaux de France
et d'autres à être chevaliers du Saint-Esprit, mais pauvres, qui
perdraient, à devenir chevaliers du Saint-Esprit, un revenu qui faisait
toute leur aisance, comme il arriva en effet. Il fut réglé aussi qu'ils
demeureraient par simple honneur ce qu'ils étaient dans l'ordre de
Saint-Louis, et que leurs pensions seraient distribuées en détail dans
le même ordre. Au moins eût-il mieux valu rendre vacant ce qu'ils y
étaient, pour faire en leur place d'autres grand'croix et d'autres
commandeurs, puisque, recevant l'ordre du Saint-Esprit, ils quittaient
la croix d'or brodée sur leurs habits pour y porter celle d'argent du
Saint-Esprit, et tous le grand cordon rouge, et ne gardaient que le
petit ruban rouge et la petite croix de Saint-Louis attachés au bas du
cordon bleu. On fut encore choqué de voir des hommes de robe et des gens
de plume et de finances porter, pour de l'argent, des marques
précisément militaires et des croix sur eux et à leurs armes (car qui
n'a pas des armes aujourd'hui\,?) sur lesquelles on voyait écrites ces
paroles en lettres d'or\,: \emph{Praemium bellicae virtutis}.

Monti, dont il a souvent été parlé ici dans ce qui y a été copié de M.
de Torcy sur les affaires étrangères, eut ordre, par une lettre de
cachet, de sortir incessamment du royaume, et défense en même temps
d'aller en Espagne. Il était colonel réformé, et comme il avait de
l'esprit et du sens, il était bien reçu dans les meilleures compagnies,
et avec cela fort honnête homme quoique ami intime d'Albéroni. Il était
pauvre et de Bologne, où il avait plusieurs frères et un à Rome, fort
distingué dans la prélature, qui à la fin est devenu cardinal. Il y a
deux familles Monti, qui ne sont point parentes\,: l'une ancienne et
fort noble, l'autre qui n'est ni l'un ni l'autre, dont était celui dont
il s'agit ici. Son mérite, et des hasards qui dépassent de beaucoup le
temps de ces Mémoires, lui procurèrent des emplois fort importants au
dehors et un très principal lors de la seconde catastrophe du roi
Stanislas en Pologne, dont il s'acquitta très judicieusement\footnote{On
  trouvera dans la note sur Charles XII, publiée à la fin du t. XIV,
  quelques détails sur ce Monti, qui avait été employé dans les
  négociations d'Albéroni. La date de sa mort est aussi indiquée dans
  cet extrait des Mémoires inédits du marquis d'Argenson.}. Il y avait
la disposition de grandes sommes fournies par la France, dont il
rapporta plus d'un million, qu'il pouvait très aisément s'approprier
sans qu'on en pût avoir nulle connaissance. Le ministère même fut très
agréablement surpris de revoir ce million, auquel il était bien loin de
s'attendre. Monti, qui avait déjà le régiment Royal-Italien, fut fait
chevalier de l'ordre, mais ce fut tout. On le laissa mourir de faim, et
il en mourut en effet peu après, quoique en grande considération et en
grande estime. Le ministère lui parlait même quelquefois des affaires.
Il était encore dans la force de l'âge quand il mourut de déplaisir de
sa misère, et n'avait point été marié. Il fut fort regretté et mérita de
l'être.

M. de Laval, dit \emph{la Mentonnière}, d'une blessure qu'il avait reçue
au menton, qui lui en faisait porter une par besoin ou pour se faire
remarquer, fut mis à la Bastille. Cette détention renouvela très
vivement et d'une façon marquée les alarmes de ceux qui ne sentaient pas
nets de l'affaire de Cellamare et du duc du Maine. Il venait d'attraper
une pension, et il se trouva à la fin qu'il était une clef de meute et
le plus coupable de tous, sans qu'il lui en soit rien arrivé qu'une
courte prison. C'est le même Laval dont il a été parlé à propos de la
prétendue noblesse et de l'effronterie de ses mensonges en confondant
hardiment les Laval-Montfort avec les Laval-Montmorency dont il était,
et neveu paternel de la duchesse de Roquelaure.

Peu de temps après le prince de Cellamare, conduit par du Libois,
gentilhomme ordinaire du roi, qui ne l'avait point quitté depuis le jour
qu'il fut arrêté à Paris, arriva à la frontière et passa en Espagne. Il
fut aussitôt déclaré vice-roi de Navarre, et comme son père était mort
il prit tout à fait le nom de duc de Giovenazzo, auquel on n'avait pu
s'accoutumer en France par l'usage de l'y avoir toujours appelé prince
de Cellamare.

Je ne puis passer sous silence une bagatelle de soi très peu
intéressante, mais parfaitement ridicule, pour ne rien dire de pis. On
obtint mille écus de pension pour Marton, fils de Blansac, et colonel du
régiment de Conti. Il avait vingt-quatre ou vingt-cinq ans. Quand il
fallut lui expédier sa pension, point de nom de baptême. On chercha, il
se trouva qu'il avait été ondoyé tout au plus. On suppléa donc les
cérémonies pour lui donner un nom. On le dispensa de l'habit blanc\,; il
fut tenu par M. le prince de Conti et M\textsuperscript{me} la duchesse
de Sully.

M\textsuperscript{me} d'Orléans, religieuse professe à Chelles par
fantaisie, humeur et enfance, ne put durer qu'en régnant où elle était
venue pour obéir. L'abbesse, fille de beaucoup de mérite, soeur du
maréchal de Villars, se lassa bientôt d'une lutte où Dieu et les hommes
étaient pour elle, mais qui lui était devenue insupportable, et qui
troublait toute la paix et la régularité de sa maison. Elle ne songea
donc qu'à céder et à avoir de quoi vivre ailleurs. Elle obtint douze
mille livres de pension du roi, vint à Paris loger chez son frère en
attendant un appartement dans un couvent. Elle le trouva chez les
Bénédictines du Cherche-Midi, près la Croix-Rouge\,; elle s'y retira,
elle y vécut plusieurs années faisant l'exemple et les délices de la
maison, et y est enfin morte fort regrettée. Pour achever de suite une
matière qui ne vaut pas la peine d'être reprise, et dont la fin passe
les bornes du temps de ces Mémoires, la princesse qui lui succéda se
lassa bientôt de sa place. Tantôt austère à l'excès, tantôt n'ayant de
religieuse que l'habit, musicienne, chirurgienne, théologienne,
directrice, et tout cela par sauts et par bonds, mais avec beaucoup
d'esprit, toujours fatiguée et dégoûtée de ses diverses situations,
incapable de persévérer en aucune, aspirant à d'autres règles et plus
encore à la liberté, mais sans vouloir quitter son état de religieuse,
se procura enfin la permission de se démettre et de faire nommer à sa
place une de ses meilleures amies de la maison, dans laquelle néanmoins
elle ne put durer longtemps. Elle vint donc s'établir pour toujours dans
un bel appartement du couvent des Bénédictines de la Madeleine de
Tresnel, auprès duquel M\textsuperscript{me} la duchesse d'Orléans, qui
avait quitté Montmartre, s'était fait un établissement magnifique et
délicieux, avec une entrée dans la maison, où elle allait passer les
bonnes fêtes et quelquefois se promener. M\textsuperscript{me} de
Chelles peu à peu reprit la dévotion et la régularité, et, quoique en
princesse, mena une vie qui édifia toujours de plus en plus jusqu'à sa
mort, qui n'arriva que plusieurs années après dans la même maison sans
en être sortie.

On diminua les espèces par un arrêt du conseil. On commença aussi le
très nécessaire élargissement du quai le long du vieux Louvre, et
d'accommoder la place du Palais-Royal en symétrie d'architecture en
face, avec une fontaine et un grand réservoir. Je fis tout ce que je pus
auprès de M. le duc d'Orléans pour faire changer le guichet du Louvre,
le mettre vis-à-vis la rue Saint-Nicaise, et le faire de la largeur de
cette rue, sans avoir pu, en faveur d'une telle commodité pour un
passage qui fait la communication d'une partie de Paris, surmonter la
rare considération du régent pour Launay, fameux et très riche orfèvre
du roi, qui était logé dans l'emplacement de ce guichet, et qu'il aurait
fallu déranger et Loger ailleurs.

Le chevalier de Saint-Georges avait été très bien reçu en Espagne.
Albéroni, enragé contre l'Angleterre, et qui n'avait de ressource qu'à y
jeter des troubles, fit équiper une flotte, mais, à peine fut-elle en
mer qu'une tempête la dispersa et la maltraita fort. Cependant les lords
Maréchal, Tullybaldine et Seaford, partis du port du Passage sur des
frégates avec beaucoup d'armes, étaient heureusement arrivés en Écosse.

Ce port du Passage qu'Albéroni avait entrepris de fortifier et où il
avait le dépôt principal de construction pour l'Océan, était le point
secret de la jalousie de l'Angleterre depuis que ce cardinal s'était
sérieusement appliqué à rétablir la marine d'Espagne. Les Anglais ne
voulaient souffrir de marine à aucune puissance de l'Europe. Elle était
venue à bout par l'intérêt de l'abbé Dubois à obtenir formellement qu'il
ne s'en format point en France, et qu'on y laissât tomber le peu qui en
restait. La ruine de la flotte d'Espagne par une anglaise très
supérieure avait été l'objet du secours de Naples et de Sicile pour le
moins autant que l'attachement aux intérêts de l'empereur\,; et la
guerre déclarée à l'Espagne en conséquence de la quadruple alliance
avait en point de vue principal la destruction de la marine d'Espagne
renaissante au Passage. L'union de l'Angleterre avec la Hollande
n'empêchait pas cette couronne d'abuser de sa supériorité sur la
république, et de lui donner souvent des occasions de plaintes sur le
trouble de ses navigations et de son commerce, et les plus clairvoyants
de ces pays de liberté sentaient le poids de cette alliance léonine, et
que, si l'Angleterre avait jamais autant de moyens que de volonté, elle
ne traiterait pas mieux leur marine, pour en avoir seule en Europe, et
c'est ce qui avait rendu les Hollandais si rétifs à la quadruple
alliance dans laquelle ils n'étaient enfin entrés qu'après coup, malgré
eux et faiblement, parce qu'ils étaient fâchés de la destruction de la
marine renaissante de l'Espagne, à quoi ils voyaient que tout tendait
principalement. En effet, dès que Cilly se fut emparé de quelques petits
forts sur la Bidassoa, il marcha secrètement et brusquement au port du
Passage, le prit et les forts commencés pour le défendre, brûla six
vaisseaux qui étaient sur les chantiers, un amas immense d'autres bois
et de toutes les choses nécessaires aux constructions et n'y laissa
chose quelconque dont on pût faire le moindre usage. Ce coup fit exulter
l'Angleterre, et fixa la certitude du chapeau sur la tête de Dubois. Il
montra une joie odieuse de cette funeste expédition, et toute la France
une douleur dont personne ne se contraignit, et qui embarrassa le régent
pendant quelques jours. Le grand but se trouvant rempli, on se soucia
médiocrement depuis des expéditions militaires sur la frontière
d'Espagne. Dans cette satisfaction anglaise et si peu française de
l'abbé Dubois et de son maître, M\textsuperscript{lle} de Montauban fort
attachée à M\textsuperscript{me} du Maine, le fils de Malézieu, Davisart
et l'avocat Bargetton, qui étaient à la Bastille, furent mis en pleine
liberté, quoique Saillant, en sortant de cette prison, eût été exilé
chez son père en Auvergne.

Law faisait toujours merveilles avec son Mississipi. On avait fait comme
une langue pour entendre ce manège et pour savoir s'y conduire, que je
n'entreprendrai pas d'expliquer, non plus que les autres opérations de
finances. C'était à qui aurait du Mississipi. Il s'y faisait presque
tout à coup des fortunes immenses. Law, assiégé chez lui de suppliants
et de soupirants, voyait forcer sa porte, entrer du jardin par ses
fenêtres, tomber dans son cabinet par sa cheminée. On ne parlait que par
millions. Law, qui, comme je l'ai dit, venait chez moi tous les mardis
entre onze heures et midi, m'avait souvent pressé d'en recevoir sans
qu'il m'en coûtât rien, et de le gouverner sans que je m'en mélasse pour
me valoir plusieurs millions. Tant de gens de toute espèce y en avaient
gagné plusieurs par leur seule industrie, qu'il n'était pas douteux que
Law ne m'en fit gagner encore plus et plus rapidement\,; mais je ne
voulus jamais m'y prêter. Law s'adressa à M\textsuperscript{me} de
Saint-Simon, qu'il trouva aussi inflexible. Enrichir pour enrichir, il
eût bien mieux aimé m'enrichir que tant d'autres, et m'attacher
nécessairement à lui par cet intérêt dans la situation où il me voyait
auprès du régent. Il lui en parla donc pour essayer de me vaincre par
cette autorité. Le régent m'en parla plus d'une fois\,: j'éludai
toujours.

Enfin, un jour qu'il m'avait donné rendez-vous à Saint-Cloud, où il
était allé travailler pour s'y promener après, étant tous deux assis sur
la balustrade de l'orangerie qui couvre la descente dans le bois des
Goulottes, il me parla encore du Mississipi, et me pressa infiniment
d'en recevoir de Law\,; plus je résistai, plus il me pressa, plus il
s'étendit en raisonnements\,; à la fin il se fâcha, et me dit que
c'était être trop glorieux aussi, parmi tant de gens de ma qualité et de
ma dignité qui couraient après, de refuser obstinément ce que le roi me
voulait donner, au nom duquel tout se faisait. Je lui répondis que cette
conduite serait d'un sot et d'un impertinent encore plus que d'un
glorieux\,; que ce n'était pas aussi la mienne\,; que, puisqu'il me
pressait tant, je lui dirais donc mes raisons\,; qu'elles étaient que,
depuis la fable du roi Midas, je n'avais lu nulle part, et encore moins
vu, que personne eût la faculté de convertir en or tout ce qu'il
touchait\,; que je ne croyais pas aussi que cette vertu fût donnée à
Law, mais que je pensais que tout son savoir était un savant jeu, un
habile et nouveau tour de passe-passe, qui mettait le bien de Pierre
dans la poche de Jean, et qui n'enrichissait les uns que des dépouilles
des autres\,; que tôt ou tard cela tarirait, le jeu se verrait à
découvert, qu'une infinité de gens demeureraient ruinés, que je sentais
toute la difficulté, souvent l'impossibilité des restitutions, et de
plus à qui restituer cette sorte de gain\,; que j'abhorrais le bien
d'autrui, et que pour rien je ne m'en voulais charger, même d'équivoque.

M. le duc d'Orléans ne sut trop que me répondre, mais néanmoins,
parlant, rebattant et mécontent, revenant toujours à son idée de refuser
les bienfaits du roi. L'impatience heureusement me prit\,: je lui dis
que j'étais si éloigné de cette folie que je lui ferais une proposition
dont je ne lui aurais jamais parlé sans tout ce qu'il me disait, et dont
non seulement je ne m'étais pas avisé, mais, comme il était vrai, qui me
tombait en ce moment dans l'esprit pour la première fois. Je lui
expliquai ce qu'autrefois je lui avais conté, dans nos conversations
inutiles, des dépenses qui avaient ruiné mon père à la défense de Blaye
contre le parti de M. le Prince, à y être bloqué dix-huit mois, à avoir
payé la garnison, fourni des vivres, fait fondre du canon, muni la
place, entretenu dedans cinq cents gentilshommes qu'il y avait ramassés,
et fait plusieurs dépenses pour la conserver au roi sans rien prendre
sur le pays, et n'ayant tiré que du sien\,; qu'après les troubles on lui
avait expédié pour cinq cent mille livres d'ordonnances dont il n'avait
jamais eu un sou, et dont M. Fouquet allait entrer en payement lorsqu'il
fut arrêté. Je dis après à M. le duc d'Orléans que, s'il voulait entrer
dans la perte de cette somme et dans celle d'un si long temps sans en
rien toucher, tandis que mon père et moi portions, pour ce service
essentiel rendu au roi, bien plus que la somme, et de plus les intérêts
tous les ans depuis, ce serait une justice que je tiendrais à grande
grâce, et que je recevrais avec beaucoup de reconnaissance, en lui
rapportant mes ordonnances à mesure des payements pour être brûlées
devant lui. M. le duc d'Orléans le voulut bien il en parla dès le
lendemain à Law\,; mes billets et ordonnances furent peu à peu brûlés
dans le cabinet de M. le duc d'Orléans, et c'est ce qui a payé ce que
j'ai fait à la Ferté.

Le président Blamont eut permission de revenir à Paris et d'y faire sa
charge aux enquêtes\,; il avait fait son marché avec le régent qui,
moyennant quelque gratification secrète, fit de ce beau magistrat, si
ferme et si zélé pour sa compagnie, un très bon espion qui lui rendit
compte depuis avec exactitude de tout ce qui se passait de plus
intérieur dans le parlement. Il en fut reçu comme le défenseur et le
martyr, et jouit quelque temps des applaudissements républicains\,; mais
à la fin il fut découvert et parfaitement haï, méprisé et déshonoré dans
sa compagnie et dans le monde.

Pécoil mourut en ce temps-ci. C'était un vieux et plat maître des
requêtes, qui n'avait jamais su rapporter un procès ni aller en
intendance, fort obscur et riche à millions, ne laissant qu'une fille.
Cet article ne semble pas fait pour tenir place ici, mais l'étrange
singularité au rapport de laquelle il donne lieu m'a engagé à ne pas
l'omettre. Ce Pécoil était petit-fils d'un regrattier de Lyon, dont le
fils, père du maître des requêtes, travailla si bien et fut si
prodigieusement avare qu'il gagna des millions, mourant de faim et de
froid auprès, n'habillant presque pas ni soi ni sa famille\,; et le
magot croissant toujours. Il avait fait chez lui à Lyon une cave pour y
déposer son argent avec toutes les précautions possibles, avec plusieurs
portes dont lui seul gardait les clefs. La dernière était de fer et
avait un secret à la serrure qui ri était connu que de lui et de celui
qui l'avait fait, qui était difficile et sans lequel cette porte ne
pouvait s'ouvrir. De temps en temps il y allait visiter son argent et y
en porter de nouveau, tellement qu'on ne laissa pas de s'apercevoir chez
lui qu'il allait quelquefois dans cette cave, qu'on soupçonna exister
par ces voyages à la dérobée.

Un jour qu'il y était allé, il ne reparut plus. Sa femme, son fils, un
ou deux valets qu'ils avaient, le cherchèrent partout, et ne le trouvant
ni chez lui ni dans le peu d'endroits où quelquefois il allait, se
doutèrent qu'il était allé dans cette cave. Ils ne la connaissaient que
par sa première porte qu'ils avaient découverte dans un recoin de la
cave ordinaire. Ils l'enfoncèrent avec grand'peine, puis une autre et
parvinrent à la porte de fer\,; ils y frappèrent, prièrent, appelèrent
ne sachant comment l'ouvrir ou la rompre. N'entendant rien, la crainte
redoubla\,; ils se mirent à tâcher d'enfoncer la porte\,; mais elle
était trop épaisse et trop bien prise dans la muraille pour en venir à
bout\,; il fallut du secours. Avec {[}celui{]} de leurs voisins et un
pénible travail ils se firent un passage\,; mais que trouvèrent-ils\,?
des coffres forts de fer bien armés de grosses barres et le misérable
vieillard le long de ces coffres, les bras un peu mangés, le désespoir
peint encore sur ce visage livide, une lanterne près de lui dont la
chandelle était usée, et la clef dans la porte qu'il n'avait pu ouvrir
cette fois après l'avoir ouverte tant d'autres. Telle fut l'horrible fin
de cet avare. L'horreur et l'effroi les firent bientôt remonter\,; mais
les voisins qui avaient aidé au travail et les mesures qu'il fallut
prendre quoique avec le moindre bruit qu'il fût possible, empêchèrent
que l'affaire fût assez étouffée. Elle est si épouvantable et le
châtiment y est si terriblement marqué que j'ai cru qu'elle ne devait
pas être oubliée\footnote{Tout ce passage, depuis \emph{Pécoil mourut},
  a été supprimé dans les éditions précédentes. On trouvera plus loin la
  même anecdote, mais avec des variantes considérables.}.

La fille unique de Pécoil et d'une fille de Le Gendre, riche, honnête et
fameux marchand de Rouen, épousa depuis le duc de Brissac, car, excepté
ma soeur et la Gondi, sa belle-mère, il est vrai que MM. de Brissac
n'ont pas été heureux ni délicats en alliances.

On a parlé ailleurs de l'abbé Vittement, que son seul mérite fit
sous-précepteur du roi, chose bien rare à la cour, et sans qu'il y
pensât ni personne pour lui. Il y vécut en solitaire, mais sans être
farouche ni singulier et s'y fit généralement aimer et fort estimer. Il
vaqua en ce temps-ci une abbaye de douze mille livres de rente. M. le
duc d'Orléans proposa au roi de la lui donner et de le lui apprendre
lui-même. Le roi en fut ravi, l'envoya chercher sur-le-champ et le lui
dit. Vittement lui témoigna toute sa reconnaissance, et le supplia avec
modestie de le dispenser de l'accepter. Il fut pressé par le roi, par le
régent, par le maréchal de Villeroy qui était présent. Il répondit qu'il
avait suffisamment de quoi vivre. Le maréchal insista, et lui dit qu'il
en ferait des aumônes. Vittement répondit humblement que ce n'était pas
la peine de recevoir la charité pour la faire, tint bon et se retira.

Cette action, qui a si peu d'exemples et faite avec tant de simplicité,
fit grand bruit et augmenta l'estime et le respect même, que sa vertu
lui avait acquis. Mais elle incommoda M. de Fréjus, qui voyait croître
l'affection du roi pour Vittement. Dès que celui-ci s'en aperçut, il
compta sa vocation finie, d'autant plus que, s'il avait su se faire
aimer et goûter, il n'en espérait rien pour le but qu'il avait
uniquement en vue. Bientôt après, M. de Fréjus, qui s'inquiétait de lui,
lui conseilla doucement la retraite. Il la fit sur-le-champ avec joie à
la Doctrine chrétienne, d'où il ne sortit plus, et où il ne voulut
presque recevoir personne.

On a de lui une prophétie aussi célèbre que surprenante, dont on a
vainement cherché la clef, et que Bidault m'a contée. Bidault était un
des valets de chambre que le duc de Beauvilliers avait choisis pour
mettre auprès de Mgr le duc de Bourgogne. Il avait de l'esprit, des
lettres, du sens, encore plus de vraie et solide piété. Son mérite,
joint à une grande et respectueuse modestie, l'avait distingué dans son
état. M. de Beauvilliers l'aimait, et Mgr le duc de Bourgogne avait
beaucoup de bonté pour lui. Il avait le soin de ses livres\,; cela me
l'avait fait connaître et encore plus familièrement depuis le soin dont
il voulut bien se charger des affaires que la Trappe pouvait avoir à
Paris. On le mit auprès du roi dès son enfance, et quand il commença à
avoir quelques livres il en fut chargé. Cela lui donna du rapport avec
Vittement et les lia bientôt d'amitié et de confiance. Bidault venait
chez moi quelquefois et voyait Vittement dans sa retraite. Effrayé des
premiers rayons de la toute-puissance de Fréjus, devenu tout
nouvellement cardinal, il en parla à Vittement qui, sans surprise
aucune, le laissa dire. Bidault, étonné du froid tranquille et
silencieux dont il était écouté, pressa Vittement de lui en dire la
cause. «\,Sa toute puissance, répondit-il tranquillement, durera autant
que sa vie, et son règne sera sans mesure et sans trouble. Il a su lier
le roi par des liens si forts, que le roi ne les peut jamais rompre. Ce
que je vous dis là, c'est que je le sais bien. Je ne puis vous en dire
davantage\,; mais si le cardinal meurt avant moi, je vous expliquerai ce
que je ne puis faire pendant sa vie.\,» Bidault me le conta quelques
jours après, et j'ai su depuis que Vittement avait parlé en mêmes termes
à d'autres. Malheureusement il est mort avant le cardinal et a emporté
ce curieux secret avec lui. La suite n'a que trop montré combien
Vittement avait dit vrai\footnote{Le marquis d'Argenson rapporte le même
  fait dans ses Mémoires manuscrits\,: «\,J'oubliais de dire que l'abbé
  Vittement disait à ses amis, à qui il confiait ce secret, que, s'il
  survivait au cardinal, il disait quel était ce lien indissoluble entre
  le roi et le cardinal.\,»}.

Jamais, depuis sa retraite, il n'a songé à voir le roi ni à visiter
personne. Il a vécu dans la Doctrine chrétienne, dans la pénitence et
dans la médiocrité la plus frugale, dans une séparation entière, dans
une préparation continuelle à une meilleure vie, et il y est saintement
mort au bout de quelques années. Le maréchal de Villeroy l'allait voir
quelquefois malgré lui, et en revenait toujours charmé, quoiqu'il y
trouvât souvent des morales courtes mais bien placées, que peut-être il
n'y cherchait pas.

Castries, gouverneur de Montpellier et chevalier d'honneur de
M\textsuperscript{me} la duchesse d'Orléans, et dont il a été parlé
quelquefois ici, obtint que le port de Cette fût mis en gouvernement
pour lui, uni à celui de Montpellier, avec des appointements
particuliers de douze mille livres payés par la province.

\hypertarget{chapitre-x.}{%
\chapter{CHAPITRE X.}\label{chapitre-x.}}

1719

~

{\textsc{M\textsuperscript{me} la duchesse de Berry va demeurer à
Meudon, où sa maladie empire, et sa volonté de déclarer son mariage
augmente.}} {\textsc{- M. le duc d'Orléans me le confie et fait
subitement partir Rion pour l'armée du maréchal de Berwick.}} {\textsc{-
M\textsuperscript{me} la duchesse de Berry, déjà considérablement mal,
se fait transporter à la Muette.}} {\textsc{- Mort d'Effiat.}}
{\textsc{- Singularité étrange de sa dernière maladie.}} {\textsc{-
Biron premier écuyer de M. le duc d'Orléans.}} {\textsc{- Mort de La
Vieuville et de M\textsuperscript{me} de Leuville\,; quelle elle
était.}} {\textsc{- Pensions données à Coettenfao, à Fourille, à Ruffey,
à Savine, à Béthune, à La Billarderie.}} {\textsc{- La duchesse du Maine
à Châlon-sur-Saône, presque en pleine liberté.}} {\textsc{- L'épouse du
roi Jacques se sauve d'Inspruck, est reçue à Rome en reine.}} {\textsc{-
Le roi en pompe à Notre-Dame.}} {\textsc{- Étrange arrangement de son
carrosse.}} {\textsc{- Siège de Fontarabie.}} {\textsc{- Folle lettre
anonyme à M. le prince de Conti.}} {\textsc{- Mort du fils de
d'Estaing.}} {\textsc{- Prise de Fontarabie, puis de Saint-Sébastien.}}
{\textsc{- On brûle à Santona trois vaisseaux espagnols prêts à être
lancés à la mer.}} {\textsc{- Mort, fortune et caractère de La Berchère,
archevêque de Narbonne.}} {\textsc{- Beauvau, archevêque de Toulouse,
lui succède.}} {\textsc{- Mort, caractère et infortune de Dupin.}}
{\textsc{- Misère de notre conduite à l'égard de Rome.}} {\textsc{-
Impudence des \emph{Te Deum}. Mort, fortune et caractère de Nyert. Le
roi à l'hôtel de ville, voit le feu de la Saint-Jean. Fatuités du
maréchal de Villeroy. Mort et caractère de Chamlay. La cour des monnaies
obtient la noblesse. Le chevalier de Bouillon obtient trente mille
livres de gratification. Sainte-Menehould brûlée. Autre incendie à
Francfort-sur-le-Mein. Mort et caractère de Nancré. Mort de la duchesse
d'Albret (Le Tellier). Clermont-Chattes\,; quel\,; est capitaine des
Suisses de M. le duc d'Orléans. Le garde des sceaux marie son second
fils\,; perd sa femme\,; pousse ses deux fils. Mort de Chauvelin,
conseiller d'État. Mort, extraction, fortune du duc de Schomberg. Mort,
fortune et caractère de Bonrepos.}}

~

La maladie de M\textsuperscript{me} la duchesse de Berry, dont on a
parlé, la prit le 26 mars, et le jour de Pâques se trouva le 9 avril.
Elle était tout à fait bien, mais sans vouloir voir personne. La semaine
de Pâques après la semaine sainte était fâcheuse à Paris, après le
scandale qu'on a raconté. D'ailleurs les visites de M. le duc d'Orléans
devenaient rares et pesantes. Le mariage de Rion causait de violentes
querelles et force pleurs. Pour s'en délivrer et sortir en même temps de
l'embarras des Pâques, elle résolut de s'aller établir à Meudon le lundi
de Pâques. On eut beau lui représenter le danger de l'air, du mouvement
du carrosse et du changement de lieu au bout de quinze jours, et de
beaucoup moins depuis le grand danger où elle s'était vue, rien ne put
lui faire supporter Paris plus longtemps. Elle partit donc, suivie de
Rion et de la plupart de ses dames et de sa maison.

M. le duc d'Orléans m'apprit alors le dessein arrêté de
M\textsuperscript{me} la duchesse de Berry de déclarer le mariage secret
qu'elle avait fait avec Rion. M\textsuperscript{me} la duchesse
d'Orléans était à Montmartre pour quelques jours, et nous nous
promenions dans le petit jardin de son appartement. Le mariage ne me
surprit que médiocrement par cet assemblage de passion et de peur du
diable, et par le scandale qui venait d'arriver. Mais je fus étonné au
dernier point de cette fureur de le déclarer dans une personne si
superbement glorieuse. M. le duc d'Orléans s'étendit avec moi sur mon
embarras, sa colère, celle de Madame, qui se voulait porter aux
dernières extrémités, le dépit extrême de M\textsuperscript{me} la
duchesse d'Orléans. Heureusement le gros des officiers destinés à servir
sur les frontières d'Espagne partaient tous les jours, et Rion n'était
resté qu'à cause de la maladie de M\textsuperscript{me} la duchesse de
Berry. M. le duc d'Orléans trouva plus court de se donner une espérance
de délai en faisant partir Rion, se flattant que cette déclaration se
différerait plus aisément en absence qu'en présence. J'approuvai fort
cette pensée, et dès le lendemain Rion reçut à Meudon un ordre sec et
positif de partir sur-le-champ pour joindre son régiment dans l'armée du
duc de Berwick. M\textsuperscript{me} la duchesse de Berry en fut
d'autant plus outrée qu'elle en sentit la raison et par conséquent son
impuissance de retarder le départ, à quoi Rion, de son côté, n'osa se
commettre. Il obéit donc\,; et M. le duc d'Orléans, qui n'avait pas
encore été à Meudon, fut plusieurs jours sans y aller.

Ils se craignaient l'un l'autre, et ce départ n'avait pas mis d'onction
entre eux. Elle lui avait dit et répété qu'elle était veuve, riche,
maîtresse de ses actions, indépendante de lui, répétait ce qu'elle avait
ouï dire des propos de Mademoiselle quand elle voulut épouser M. de
Lauzun, grand-oncle de Rion\,; y ajoutait les biens, les honneurs, les
grandeurs qu'elle prétendait pour Rion dès que leur mariage serait
déclaré, et se mettait en furie jusqu'à maltraiter fortement de paroles
M. le duc d'Orléans, dont elle ne pouvait supporter les raisons ni les
oppositions. Il avait essuyé de ces scènes à Luxembourg dès qu'elle fut
mieux, et il n'en essuya pas de moins fortes à Meudon dans le peu de
visites qu'il lui fit. Elle y voulait déclarer son mariage, et tout
l'esprit, l'art, la douceur, la colère, les menaces, les prières et les
instances les plus vives de M. le duc d'Orléans ne purent qu'à
grand'peine pousser en délais le temps avec l'épaule. Si on en avait cru
Madame, l'affaire aurait été finie avant le voyage de Meudon, car M. le
duc d'Orléans aurait fait jeter Rion par les fenêtres de Luxembourg.

Le voyage si prématuré de Meudon et des scènes si vives n'étaient pas
pour rétablir une santé si nouvellement revenue des portes de la mort.
Le désir extrême qu'elle eut de cacher son état au public et de
soustraire à sa connaissance la situation où elle se trouvait avec M.
son père, dont on remarquait la rareté des visites qu'il lui faisait,
l'engagèrent à lui donner un souper sur la terrasse de Meudon, sur les
sept heures du soir. En vain on lui représenta le danger du serein et du
frais du soir sitôt après l'état où elle avait été et dans l'état
chancelant où sa santé se trouvait encore. Ce fut pour cela même qu'elle
s'y opiniâtra dans la pensée qu'un souper sur la terrasse, sitôt après
l'extrémité où elle avait été, ôterait à tout le monde la persuasion de
sa couche et ferait croire qu'elle était toujours avec M. le duc
d'Orléans comme elle y avait été, nonobstant la rareté inusitée de ses
visites, qui avait été remarquée. Ce souper en plein air ne lui réussit
pas. Dès la nuit même elle se trouva mal. Elle fut attaquée d'accidents
causés par l'état où elle était encore et par une fièvre irrégulière,
que la contradiction qu'elle trouvait à la déclaration de son mariage ne
contribuait pas à diminuer. Elle se dégoûta de Meudon comme les malades
de corps et d'esprit, qui, dans leur chagrin, se prennent à l'air et aux
lieux.

Elle était embarrassée de ce que les visites de M. le duc d'Orléans ne
se rapprochaient point, et de ce que Madame et M\textsuperscript{me} la
duchesse d'Orléans n'allaient presque point la voir, quoique
considérablement malade. Son orgueil en souffrait plus que sa tendresse,
qui était nulle pour ces princesses, et qui commençait à se tourner en
haine par leur résistance à ses plus ardents désirs. La même raison
commençait à lui faire prendre les mêmes sentiments pour M. son père\,;
mais elle espérait le ramener à ses volontés par l'empire qu'elle avait
sur lui, et elle était de plus peinée que le monde s'aperçût de la
rareté de ses visites et ne diminuât la considération qu'elle tirait du
pouvoir si connu qu'elle avait sur lui, quand il paraîtrait qu'il
n'était plus le même. Quelque contraire que lui fût l'air, le mouvement,
le changement de lieu dans l'état où elle se trouvait, rien ne put
l'empêcher de se faire transporter de Meudon à la Muette, couchée entre
deux draps, dans un grand carrosse, le dimanche 14 mai, où elle espéra
que la proximité de Paris engagerait M. le duc d'Orléans à la venir voir
plus souvent, et M\textsuperscript{me} la duchesse d'Orléans aussi, au
moins par bienséance. Ce voyage fut pénible par les douleurs qui
s'étaient jointes aux autres accidents que ce trajet augmenta et que le
séjour de la Muette ni les divers remèdes ne purent apaiser que par de
courts intervalles, et qui devinrent très violentes.

Le marquis d'Effiat, dont on a parlé ici en plusieurs endroits et
suffisamment pour le faire connaître, se trouva fort mal à
quatre-vingt-un ans dans sa belle maison de Chilly, près Paris, où il
était allé prendre du lait. Il fut ramené à Paris le 23 mai, mais si mal
qu'on n'en espérait plus. Le maréchal de Villeroy, son bon ami et sa
dupe en bien des choses, courut chez lui, et pour se donner le vernis de
sa conversion, si convenable à sa place de gouverneur du roi, vint à
bout de lui faire recevoir ses sacrements sur-le-champ. Sa maladie
diminua et traîna. C'était, comme on l'a vu ici, un homme dont le fond
de la vie était obscur par goût, par habitude et par la plus sordide
avarice. Il avait toujours quelques femmes de rien et de mauvaise vie
qui l'amusaient, qui en espéraient et qui lui coûtaient peu. Il avait la
meute de Monsieur, que M. le duc d'Orléans lui avait conservée. Il était
maître de leur écurie comme leur premier écuyer. Ainsi c'était à leurs
dépens qu'il courait le cerf, tous les étés, chez lui à Montrichard, ou
dans les forêts voisines de Montargis dont il était capitaine. Il y
voyait peu de noblesse du pays, à qui il faisait très courte chère.

La chasse et les filles l'avaient peu à peu apprivoisé avec du Palais,
qui chassait les étés avec lui et le voyait les hivers. Il n'en voyait
guère d'autres avec familiarité, et malgré cette liaison, du Palais, qui
avait de l'esprit et du monde, était honnête homme, connu pour tel, et
voyait bonne compagnie à Paris, et avait très bien servi. Il eut grand
soin d'Effiat pendant sa maladie, qui ne voulut voir que lui. Tous les
jours sur les sept heures du soir, Effiat le renvoyait et, comme par
politesse et amitié, il le forçait de s'en aller. Du Palais, au bout de
quelques jours, s'aperçut de la régularité de l'heure et de l'inquiétude
d'Effiat à se défaire de lui. Comme de longue main il était familier
dans la maison, il en parla aux valets de chambre. Ils se regardèrent et
lui dirent ensuite qu'ils étaient dans le même cas et dans la même
curiosité\,; qu'eux-mêmes étaient chassés de la chambre à cette même
heure, avec des défenses si expresses d'y rentrer et d'y laisser
personne sans exception quelconque, et par quelque raison que ce pût
être, jusqu'à ce qu'il sonnât, qu'ils ne savaient ce que ce pouvait
être. Mais ce qu'ils ajoutèrent est bien plus étrange. Ils dirent à du
Palais qu'ils s'étaient mis à écouter à la porte\,; que tantôt plus tôt,
tantôt plus tard, ils y entendaient parler leur maître et une autre voix
avec lui, étant très sûrs qu'il n'y avait et ne pouvait y avoir que le
malade dans la chambre\,; qu'ils ne pouvaient distinguer que rarement
quelques mots qui leur avaient paru indifférents, que ce colloque durait
souvent une heure et plus, très rarement court\,; que rentrant dans la
chambre au bruit de la sonnette, ils n'y remarquaient aucun changement
en rien, mais leur maître fort concentré en lui-même, et d'ailleurs
comme ils l'avaient laissé. Ce récit augmenta tellement la curiosité de
du Palais, qu'il accepta la proposition que lui firent les valets de
chambre d'éprouver lui-même ce qu'ils lui racontaient. Du Palais sortant
de chez d'Effiat qui à l'ordinaire l'avait congédié, demeura avec eux,
écouta, et entendit comme eux parler d'Effiat et l'autre voix, et
quelquefois l'élever l'un et l'autre, mais sans en entendre que quelques
mots rares, indifférents et seuls. Du Palais voulut se donner encore le
même passe-temps, et se le donna deux ou trois fois encore. Il raisonna
avec les valets de chambre, et ne purent deviner ce que ce pouvait être,
d'autant que du Palais, qui connaissait cet appartement comme le sien,
savait comme eux que, depuis sa sortie de la chambre d'Effiat, il était
impossible que par aucune voie il s'y fût glissé personne.

Il fut tenté de tourner d'Effiat là-dessus\,; mais n'osant trop, il se
contenta de lui montrer sa surprise de l'heure fixe de son renvoi.
Effiat fit la sourde oreille, puis battit la campagne sur l'heure de la
société, et qu'il ne voulait pas abuser de son amitié et de son
assiduité\,; puis l'heure venue, le renvoya comme de coutume. Du Palais
fit semblant de sortir, et demeura près de la porte\,; un peu après, du
Palais ne sait s'il lui échappa quelque mouvement\,; mais d'Effiat
s'aperçut qu'il était là, se mit en colère, lui dit que, quand il le
priait de s'en aller, il voulait qu'il s'en allât\,; qu'il ne savait par
quel esprit il se cachait dans sa chambre\,; que c'était l'offenser
cruellement\,; qu'en un mot, s'il voulait continuer à le voir, et qu'il
demeurât son ami, il le priait de sortir sur-le-champ, et de ne lui
faire pareil tour de sa vie. Du Palais répondit d'où il était ce qu'il
put, l'autre à répéter avec empressement\,: «\,Sortez donc\,; mais
sortez.\,» Il sortit en effet, et se tint en dehors de la porte. Le
colloque, à ce qu'il entendit, ne tarda pas à commencer. Ni lui ni les
valets de chambre n'en ont jamais pu découvrir davantage.

Sur les neuf heures, quelque femme de l'espèce dont j'ai parlé, et
quelque complaisant, venaient l'amuser. Quelquefois du Palais y
revenait. Effiat ne sortait point de son lit, et eut sa tête libre et
entière jusqu'à sa mort qui arriva le 3 juin. Il laissa un prodigieux
argent comptant, de grands biens et de belles terres, fit des legs
considérables, et des fondations fort utiles pour l'éducation de pauvres
gentilshommes. Il donna Chilly à M. le duc d'Orléans, qui ne le voulut
pas accepter, et le rendit à la famille. Le duc Mazarin, fils de sa
soeur, en hérita, et de la plupart de ses biens. Il fit du Palais
exécuteur de son testament, et lui donna un diamant de mille pistoles.
Il avait beaucoup de pierreries. C'est le premier particulier à qui
j'aie vu une croix du Saint-Esprit de diamants fort belle sur son habit,
au lieu de la croix d'argent brodée, et tout l'habit garni de boutons et
de boutonnières de diamants. À la considération que M. le duc d'Orléans
lui avait toujours témoignée, on fut surpris et lui mortifié de ce qu'il
ne l'alla point voir, et il parut si peu touché de sa maladie et de sa
mort, que les maréchaux de Villeroy, Villars, Tessé, Huxelles et autres
en prirent une nouvelle inquiétude. L'écurie et les équipages de M. le
duc d'Orléans qu'Effiat entretenait moyennant une somme, se trouvèrent
dans un grand délabrement. Biron fut deux jours après choisi par M. le
duc d'Orléans pour remplir cette charge lucrative.

Il faut dire maintenant où j'ai pris ce récit curieux\,; car j'étais
fort éloigné d'avoir jamais eu aucun commerce avec d'Effiat. Du Palais
avait épousé la mère de Lanmary, et vivait avec lui dans la plus étroite
amitié, contre l'ordinaire de telles parentelles\,; il conta tout ce que
je viens d'écrire à Lanmary qui était fort de mes amis et en est encore,
qui me le rendit incontinent après.

La Vieuville mourut à Paris\,; il était veuf de la dame d'atours de
M\textsuperscript{me} la duchesse de Berry, et avait été chevalier
d'honneur de la reine, mais le plus pauvre et le plus obscur homme du
monde.

M\textsuperscript{me} de Leuville mourut aussi à soixante-sept ans. Son
mari, mort très jeune, était frère de la femme d'Effiat, duquel on vient
de parler, morte jeune aussi et tous deux sans enfants. Le chancelier
Olivier était leur trisaïeul paternel, mort en 1560, dont le père fut
premier président du parlement de Paris, après avoir été avocat du roi,
comme on parlait alors, c'est-à-dire avocat général, et président à
mortier. Ce fut lui qui commença la race, car son père, qui était de
Bourgneuf, près de la Rochelle, ne fut jamais que procureur au
parlement. M\textsuperscript{me} de Leuville dont on parle ici était
nièce de Laigues, un des importants de la Fronde, qu'on prétendit que la
fameuse M\textsuperscript{me} de Chevreuse avait, à la fin, épousé
secrètement. Sa nièce tâcha aussi d'être importante. Elle avait beaucoup
d'esprit, de domination, d'intrigue et d'amis qui se rassemblaient chez
elle et qui lui donnaient de la considération. C'était une femme qui,
sans tenir à rien, eut l'art de se faire compter\,: elle était riche et
médiocrement bonne.

Je fis rendre à Coettenfao une ancienne pension qu'il avait eue du feu
roi de six mille livres, et donner parole de l'ordre, par M. le duc
d'Orléans, pour la première promotion qui se ferait. Fourille, aveugle,
et ancien capitaine aux gardes, fort pauvre, eut quatre mille livres de
pension, et Ruffey, sous-gouverneur du roi, une de six mille. Savine
obtint six mille livres d'augmentation d'appointements à son
gouvernement d'Embrun. Béthune, distingué dans la marine, eut une
pension de trois mille livres, et La Billarderie, conducteur de
M\textsuperscript{me} du Maine à Dijon, en eut une de six mille livres.
Trois semaines après, il y fut chercher la même avec un chirurgien et
deux femmes de chambre, et la mena à Châlon-sur-Saône presque en pleine
liberté\,; elle y arriva le 24 mai.

La fille aînée du prince Jacques Sobieski, arrêtée avec sa mère à
Inspruck par ordre de l'empereur, depuis quelques mois, allant à Rome
épouser le roi Jacques, trouva moyen de se sauver la nuit en chaise de
poste escortée par quatre hommes à cheval. On trouva sur sa table un
écrit par lequel elle marquait que c'était par ordre de sa famille. Elle
arriva le 2 mai à Bologne\,; elle y fut épousée le 7 par lord Murray,
chargé de la procuration du roi Jacques, en partit le 9 pour Rome où
elle fut reçue et traitée en reine.

Quelle que fût la persécution sans bornes et sans mesure et ouverte
depuis si longtemps et avec une si scandaleuse animosité contre le
cardinal de Noailles, elle ne put empêcher que le roi fît une démarche
publique qui ne sentait ni le prélat réprouvé ni son Église hérétique.
Il fut, l'après-dînée du jour de la Pentecôte, après avoir entendu le
sermon aux Tuileries, à Notre-Dame en pompe. Il fut reçu à la porte par
le cardinal de Noailles pontificalement revêtu, à la tête de son
chapitre, avec les cérémonies accoutumées, et par lui conduit au choeur
où ce prélat entonna le \emph{Te Deum}, qui fut continué par la musique
et terminé par la bénédiction que le cardinal donna. Le choeur était
nouvellement achevé et la chapelle de la Vierge aussi, qui fut trouvée
très magnifique, laquelle fut toute aux dépens du cardinal, ainsi que
l'admirable vitrage sur la porte collatérale, que le cardinal avait tout
refait, quoiqu'il ne fût obligé à aucune de ces deux grandes dépenses.
Après la bénédiction, il conduisit le roi autour du choeur et à cette
chapelle, et de là à son carrosse. Le roi y était avec peu de dignité et
comme si on eût voulu le mettre incognito, malgré la pompe de sa suite.
Il y fut entre M. le duc d'Orléans et M. le comte de Clermont sur le
derrière\,; le prince Charles, grand écuyer, sur le devant, entre M. le
duc de Chartres et M. le Duc\,; le maréchal de Villeroy, gouverneur, et
le duc de Charost, capitaine des gardes en quartier aux portières. On
fut très étonné de cet arrangement\,; le roi en cérémonie, comme il
était là, devait être seul sur le derrière. M. le duc d'Orléans, régent,
et M. le Duc, surintendant de l'éducation, seuls sur le devant, les
portières comme elles étaient. M. le duc de Chartres et M. le comte de
Clermont n'y avaient que faire pour offusquer le roi, et faire de son
carrosse un coche, le prince Charles encore moins. Bien est vrai que le
grand écuyer entre les grands officiers y a la première place, mais il
n'en est pas moins vrai que le grand chambellan, le premier gentilhomme
de la chambre, et le même premier écuyer y entrent de préférence à
lui\,; c'est ce qui a été expliqué ailleurs ici assez clairement pour
n'avoir pas besoin d'être répété. On trouva aussi fort singulier que M.
le duc de Chartres fût sur le devant, tandis que M. le comte de Clermont
était sur le derrière. Il avait neuf ans et M. de Chartres quinze, qui,
de la taille dont il était, n'aurait pas plus pressé le roi que M. le
comte de Clermont.

Le maréchal de Berwick fit ouvrir la tranchée le 27 mai devant
Fontarabie. Pendant ce siège, où était M. le prince de Conti, il reçut
une lettre anonyme par laquelle on lui promettait de le faire roi de
Sicile, s'il voulait passer en Espagne. Il s'en moqua avec raison, et
l'envoya à M. le duc d'Orléans. La proposition ne pouvait venir
d'Espagne. M. le prince de Conti n'avait ni place, ni suite, ni parti,
ni réputation\,; son acquisition n'eût pas valu que l'Espagne se
dépouillât de la Sicile pour l'avoir, et il n'y aurait été que fort à
charge. La proposition de plus était ridicule\,; quinze mille Impériaux
venaient d'y passer de Naples, et avaient déjà obligé le marquis de Lede
de leur abandonner son camp de Melazzo, avec ses malades, ses blessés et
toutes les provisions de vivres et de fourrage qu'il y avait amassées.
Il y recommanda ceux qu'il y laissait au général Zumzungen, qui,
aussitôt après, laissa le commandement de l'armée impériale à Mercy, et
la Sicile ne fut pas longtemps à changer de maître. Mais la conjuration
du duc et de la duchesse du Maine enhardie après les frayeurs des
emprisonnements, par leur courte durée, et par la conduite du régent et
de l'abbé Dubois à cet égard, faisait bois de toute flèche et ne
désespérait pas encore de réussir.

Le fils unique d'Estaing, aide de camp de Joffreville, fut tué devant
Fontarabie, sans enfants de la fille unique de M\textsuperscript{me} de
Fontaine-Martel. L'armée d'Espagne était vers Tafalla à trois lieues de
Fontarabie. Coigny, par ordre du duc de Berwick, visitait cependant,
avec un léger détachement, les gorges et les passages de toute la chaîne
des Pyrénées pour les bien reconnaître. Fontarabie capitula le 16 juin.
Tresnel, gendre de Le Blanc, en apporta la nouvelle. Le duc de Berwick
fit aussitôt après le siège de Saint-Sébastien. Il y eut quelque
désertion dans ses troupes, mais pas d'aucun officier. L'armée d'Espagne
n'était pas en état de se commettre avec celle du maréchal de Berwick.
Saint-Sébastien capitula le 1er août. Bulkley, frère de la maréchale de
Berwick, en apporta la nouvelle. Quinze jours après, M. de Soubise
apporta celle du château, et qu'on avait brûlé, dans, un petit port près
de Bilbao, nommé Santona, trois gros vaisseaux espagnols, qui étaient
sur le chantier prêts à être lancés à la mer.

L'archevêque de Narbonne mourut dans son diocèse. Il s'appelait Le
Goust\,: il était frère de La Berchère qui avait passé sa vie maître des
requêtes, dont le fils, guère plus esprité mais fort riche, était devenu
conseiller d'État et chancelier de M. le duc de Berry, parce qu'il avait
épousé une fille du chancelier Voysin. Le prélat avait été évêque de
Lavaur, puis archevêque d'Aix, après de Toulouse, enfin de Narbonne.
C'était un grand vilain homme, sec et noir avec des yeux
bigles\footnote{Louches.}, qui avait été ami intime du P. de La Chaise.
L'âme en était aussi belle que le corps en était désagréable\,; très bon
évêque et pieux, sans fantaisie et sans faire peine à personne, adoré
partout où il avait été, beaucoup d'esprit et facile, et l'esprit
d'affaires et sage, possédant au dernier point toutes celles du clergé,
et venant à bout des plus difficiles sans faire peine à personne, allant
au bien, parlant franchement aux ministres et en étant cru et considéré.
Ce fut une perte qui ne fut pas réparée par M. de Beauvau qui lui
succéda, après avoir été évêque de Bayonne, ensuite de Tournay, puis
archevêque de Toulouse.

Dupin, célèbre docteur de Sorbonne par sa vaste et profonde érudition,
et par le grand nombre et la qualité de ses ouvrages, mourut en même
temps. Il fut un étrange exemple de la conduite, si funestement répétée
en France par la suggestion des jésuites et de leurs adhérents. Dans les
temps de brouillerie avec Rome, sur les propositions de l'assemblée du
clergé de 1682, etc., la cour se servit très avantageusement de sa
plume, et, pour plaire à Rome depuis, le laissa manger aux poux. Il fut
réduit à imprimer pour vivre\,: c'est ce qui a rendu ses ouvrages si
précipités, peu corrects, et ce qui enfin le blasa de travail et
d'eau-de-vie qu'il prenait en écrivant pour se ranimer, et pour épargner
d'autant sa nourriture, bel et bon esprit, juste, judicieux quand il
avait le temps de l'être, et un puits de science et de doctrine, avec de
la droiture, de la vérité et des moeurs.

M\textsuperscript{me} la Duchesse, qui avait été longtemps fort mal, fut
si considérablement mieux qu'on la crut guérie. Il y eut pour cela un
\emph{Te Deum} aux Cordeliers, que l'hôtel de Condé fit chanter plus que
très mal à propos. Le \emph{Te Deum} est une action publique jusqu'alors
réservée au public et aux rois pour remercier Dieu solennellement, au
nom du public, des grâces qui intéressent l'un ou l'autre, ou plutôt
inséparablement tous les deux. Celui-ci ne porta pas bonheur à
M\textsuperscript{me} la Duchesse. C'était la jeune, soeur de M. le
prince de Conti\,; des princes du sang on les vit tôt après tomber aux
moindres particuliers.

Nyert, premier valet de chambre, mourut en ce même temps\,: c'était un
des plus méchants singes, auquel il ressemblait fort, et des plus
gratuitement dangereux qu'il y eût parmi ce qu'on pouvait appeler les
affranchis du feu roi, qui, par leurs entrées à toute heure et leur
familiarité avec lui, étaient des personnages fort comptés et
redoutables aux ministres mêmes. Celui-ci l'amusait aux dépens de tout
le monde avec le jugement d'un valet d'esprit et d'expérience. Aussi
l'avarice, l'envie et la haine étaient peintes sur son visage décharné.

Il était fils d'un excellent musicien dont la voix et le luth étaient
admirables\,; il était au marquis de Mortemart, premier gentilhomme de
la chambre de Louis XIII, du temps que mon père l'était aussi, père de
la trop fameuse M\textsuperscript{me} de Montespan, et duc et pair des
quatorze de 1663. Louis XIII, s'opiniâtrant dans les Alpes en 1629, à
forcer le célèbre pas de Suze malgré la nature, et ce qui était
peut-être plus, malgré le cardinal de Richelieu, et malgré tous ses
généraux qui jugeaient l'entreprise impraticable, s'ennuyait fort les
soirs au retour de ses recherches assidues des passages, parce que le
cardinal lui écartait le monde à dessein, dans l'espérance de l'abandon
plus prompt d'un projet que tous jugeaient impossible. Mon père, alors
en grandes charges et en grande faveur, cherchait à amuser le roi qui
aimait fort la musique, et lui proposa, dans cette solitude des soirs,
d'entendre Nyert. Le roi le goûta fort, tellement qu'au retour de ce
triomphant voyage où le roi s'était couvert de lauriers si purs et si
uniquement dus à lui seul, mon père trouva jour à lui donner Nyert\,; il
en parla à M. de Mortemart avant de rien entreprendre, qui fut ravi de
faire cette fortune, et qui même pria mon père d'en parler au roi. Le
héros le prit, et mon père, dans la suite, le fit premier valet de
chambre. Son fils, dont on parle ici, ne lui ressembla en rien, et le
fils que celui-ci laissa ressembla encore moins au père. Il fut modeste,
très honnête homme, et un saint\,; il dura peu, il laissa deux fils de
même caractère que lui, qui ne durèrent pas non plus. Le singe qui a
donné lieu à cet article avait attrapé le petit gouvernement de Limoges
et celui des Tuileries, lequel passa à son fils avec sa charge de
premier valet de chambre.

On donna le plaisir au roi d'aller voir le feu de la Saint-Jean à
l'hôtel de ville, qui fut, à cause de lui, beaucoup plus beau qu'à
l'ordinaire. Quantité de dames de la cour et de seigneurs y furent
conviés par le duc de Tresmes\,; on ne doutait point que le roi ayant
huit ans, la galanterie dont le maréchal de Villeroy s'était piqué toute
sa vie et se piquait encore, ne fit manger les dames avec lui. La
pédanterie de gouverneur l'emporta. Il fit souper le roi seul dans une
chambre particulière, et à son heure accoutumée\,: le premier maître
d'hôtel, soutenu de M. le Duc comme grand maître, prétendit le servir,
parce que le souper du roi fut fait par la bouche. Le prévôt des
marchands revendiqua son droit\,; un \emph{mezzo-termine}, si chéri du
régent, finit la dispute. Il fit signer un billet au prévôt des
marchands, par lequel il reconnut que ce serait sans conséquence à
l'égard du premier maître d'hôtel qu'il servirait le roi, et en effet il
le servit. Après ce solitaire souper, la fatuité du maréchal de Villeroy
se déploya tout entière. Il fit faire au roi la prière comme s'il allait
se coucher, et se fit moquer par tout le monde. Après, le roi vit le
feu. Le roi parti, il y eut plusieurs tables magnifiquement servies pour
tout ce qui avait été convié, et un bal à l'hôtel de ville termina la
fête.

On a tant parlé de Chamlay dans ces Mémoires, qu'on n'a rien à y
ajouter. Il était extrêmement gros\,; sa grande sobriété et un exercice
à pied journalier et prodigieux ne purent le garantir de l'apoplexie. Il
en eut plusieurs attaques qui lui avaient fort abattu le corps et
l'esprit. Il en mourut à Bourbon. C'était un homme d'un mérite très
rare, qui, en quelque état qu'il fût, fut fort regretté. Il était
grand'croix de Saint-Louis, dès la fondation de l'ordre, et maréchal
général des logis des armées du roi, qu'il avait exercé avec la plus
grande capacité et distinction, et la confiance de M. de Turenne et des
meilleurs généraux des armées. On a vu ailleurs combien il eut toujours
la confiance du roi, et la probité, la modestie, et le désintéressement
avec lequel il en usa.

M. le duc d'Orléans, à qui tout coulait d'entre les doigts, accorda la
noblesse aux officiers de la cour des monnaies, et dix mille écus au
chevalier de Bouillon. Il y eut un grand incendie à
Francfort-sur-le-Mein, et en Champagne toute la ville de
Sainte-Menehould fut brûlée.

On a souvent parlé de Nancré, assez nouvellement revenu d'Espagne,
charmé d'Albéroni avec qui il était aussi assez homogène, lorsqu'il vint
mourir ici en vingt-quatre heures. C'était un des hommes du monde le
plus raffiné et dont le coeur et l'âme étaient le plus parfaitement
corrompus, avec beaucoup d'esprit, des connaissances et beaucoup de
souplesse et de liant. Il avait servi, puis fait le philosophe\,; après,
s'était accroché au Palais-Royal par Canillac et par les maîtresses, de
là à M. de Torcy, et le plus sourdement qu'il avait pu à tout ce qui
approchait du feu roi\,; il ne tint pas à lui d'en devenir l'espion,
puis l'organe. On a vu ici qu'il le fut bien étrangement lors des
renonciations. Valet de Nocé, enfin âme damnée de l'abbé Dubois qui le
porta aux négociations étrangères, et à d'autres plus intérieures. Nocé
comptait voler haut, lorsque tout à coup il lui fallut quitter ce monde.

Ce n'était pas la peine de tant de bruit de part et d'autre,
d'importuner les tribunaux, le régent et le conseil de régence sur le
mariage du duc d'Albret avec une fille de Barbezieux. Elle mourut
presque incontinent après en couche d'un fils qui mourut dix ou douze
ans après.

M. le duc d'Orléans remplit dignement la place de Nancré, capitaine de
ses Suisses, de vingt mille livres de rente par les profits. Nancré
n'était point marié, était sans suite, et n'avait point de brevet de
retenue. Le régent la donna à Clermont-Chattes, frère de Roussillon et
de l'évêque-duc de Laon, qui n'avait rien vaillant, et qui, des plus
riantes espérances, était tombé dans la plus cruelle disgrâce, à
laquelle la mort de Monseigneur avait mis le dernier sceau, et qui a été
racontée ici sous l'an\footnote{Tome Ier, p.~208 et 209.} {[}1694{]}
avec l'aventure célèbre de M\textsuperscript{lle} Choin et de
M\textsuperscript{me} la princesse de Conti. Clermont, en naissance, en
honneur, en probité, était le parfait contraste de Nancré. Ce choix fut
fort applaudi.

Le garde des sceaux maria son second fils à la fille, fort riche, du
président Larcher. Ce mariage ne fut pas heureux, mais le jeune époux
fit dans la suite la plus brillante fortune de son état. Le mariage de
son père avec une soeur de Caumartin, intendant des finances, fort
accrédité et conseiller d'État, n'avait pas été, non plus, fort
heureux\,; il perdit sa femme de la petite vérole quelques mois après le
mariage de son fils. Il en avait deux fils\,: celui-ci plein d'esprit et
d'ambition, et fort galant de plus, et un aîné qui était et fut toujours
un balourd\footnote{Les deux fils du garde des sceaux d'Argenson et de
  Marguerite Le Fèvre de Caumartin, furent René-Louis Le Voyer de
  Paulmy, marquis d'Argenson, et Marc-Pierre Le Voyer de Paulmy, comte
  d'Argenson. L'aîné, que Saint-Simon traite sévèrement, a laissé des
  Mémoires, dont on n'a publié que des fragments (voy. \emph{Mémoires du
  marquis d'Argenson}, édit. 1825). Nous avons publié quelques passages
  des Mémoires inédits. Du reste, Saint-Simon n'a fait que reproduire
  l'opinion de ses contemporains, qu'exprime en ces termes un des
  biographes du marquis d'Argenson\,: «\,Plus froid, plus mesuré {[}que
  son frère{]}, ne se livrant qu'à des amis intimes\,; raisonnant juste,
  mais sans la même grâce dans la façon de s'exprimer, les habitants de
  Versailles, à une époque où il était d'usage de donner à tout le monde
  des sobriquets ridicules, le désignèrent sous celui d'\emph{Argenson
  la Bête}.\,» On prépare en ce moment même une édition plus complète
  des Mémoires du marquis d'Argenson.}. Le père ne fut pas longtemps à
les mettre dans les emplois de leur état, et, malgré leur jeunesse, à
les faire conseillers d'État, tous deux à peu de distance l'un de
l'autre.

Chauvelin, conseiller d'État, mourut aussi. Il avait été intendant de
Picardie, avec peu de lumières, mais beaucoup de probité. Il était père
de l'avocat général, dont il a été parlé ici, et de Chauvelin, dont la
prodigieuse élévation et la lourde chute\footnote{Voy., sur ce
  Chauvelin, t. XII, p. 474.} ont fait depuis tant de bruit.

Le duc de Schomberg mourut subitement en une de ses maisons, près de
Londres, à soixante-dix-neuf ans. Il était fils du dernier maréchal de
Schomberg, qui avait commandé les armées de Portugal, et depuis celles
de France avec réputation. Il était Allemand et gentilhomme, mais point
du tout parent des deux précédents maréchaux de Schomberg, père et fils,
lequel fut duc et pair d'Halluyn, en épousant l'héritière, par de
nouvelles lettres.

Ce dernier maréchal de Schomberg dont on parle ici était huguenot, et se
retira en Allemagne avec sa famille, à la révocation de l'édit de
Nantes. L'électeur de Brandebourg le mit à la tête de son conseil et de
ses troupes, et le donna après au prince d'Orange comme un homme utile
dans les affaires et dans les armées. Lorsqu'il fut question de la
révolution d'Angleterre, le maréchal en eut le secret tout d'abord et en
dirigea la mécanique avec le prince d'Orange. Il passa avec lui en
Angleterre, puis avec lui en Irlande, où il commanda son armée sous lui,
et fut tué à la bataille de La Boyne, que le prince d'Orange gagna
contre le roi d'Angleterre, laquelle fut le dernier coup de son
accablement.

Le fils du maréchal de Schomberg fut fait duc par le roi Guillaume, et
commanda les troupes anglaises en chef en divers pays et diverses
armées, et se retira à la fin mécontent. Il avait épousé une soeur
bâtarde de Madame, que l'électeur palatin avait eue d'une demoiselle de
Degenfeldt, et qu'il fit faire comtesse par l'empereur.

Bonrepos mourut subitement dans sa maison à Paris, dans une heureuse
vieillesse, sain de corps et d'esprit, sans avoir été marié. Il avait
été longtemps dans les bureaux de la marine, du temps de M. Colbert,
ensuite un des premiers commis de Seignelay, dont il eut la confiance. À
sa mort il se retira des bureaux, qui lui avaient servi à se faire à la
cour des amis et à être depuis bien reçu dans toute la bonne compagnie.
Il alla en Angleterre faire un traité de commerce, puis aux villes
hanséatiques, enfin ambassadeur en Danemark, puis en Hollande, où il
réussit fort bien. Le roi le traitait avec bonté, M\textsuperscript{me}
de Maintenon aussi\,; il était estimé, et sur un pied de considération
dans le monde, avec de l'esprit, de l'honneur, de la capacité et des
talents. Bonac, fils de son frère aîné, hérita de lui. Il était gendre
de Biron, qui lors n'avait rien à donner à ses filles, et à
Constantinople où il était ambassadeur. Bonrepos avait près de trente
mille livres du roi.

\hypertarget{chapitre-xi.}{%
\chapter{CHAPITRE XI.}\label{chapitre-xi.}}

1719

~

{\textsc{M\textsuperscript{me} la duchesse de Berry se fait transporter
de Meudon à la Muette.}} {\textsc{- Conduite de M\textsuperscript{me} de
Saint-Simon à l'égard de M\textsuperscript{me} la duchesse de Berry.}}
{\textsc{- Raccourci de M\textsuperscript{me} la duchesse de Berry.}}
{\textsc{- M\textsuperscript{me} la duchesse de Berry reçoit superbement
ses sacrements, fait après à M\textsuperscript{me} de Mouchy présent
d'un baguier de deux cent mille écus.}} {\textsc{- M. le duc d'Orléans
le prend, et elle demeure perdue.}} {\textsc{- M\textsuperscript{me} la
duchesse de Berry reçoit une seconde fois ses sacrements, et
pieusement.}} {\textsc{- Scélératesse insigne de Chirac, impunie.}}
{\textsc{- Ma conduite à l'égard de M\textsuperscript{me} la duchesse de
Berry en sa dernière extrémité.}} {\textsc{- Je vais à la Muette auprès
de M. le duc d'Orléans.}} {\textsc{- Il me charge de ses ordres sur tout
ce qui devait suivre la mort.}} {\textsc{- J'empêche toute cérémonie et
l'oraison funèbre.}} {\textsc{- Mort de M\textsuperscript{me} la
duchesse de Berry regrettée, sans exception, de personne que de M. le
duc d'Orléans, et encore peu de jours.}} {\textsc{- Scellés mis par La
Vrillière, secrétaire d'État.}} {\textsc{- Convois du coeur et du
corps.}} {\textsc{- Ni manteaux ni mantes au Palais-Royal.}} {\textsc{-
Les appointements et logements continués à toutes les dames de
M\textsuperscript{me} la duchesse de Berry.}} {\textsc{- Mouchy et sa
femme chassés.}} {\textsc{- Gouvernement de Meudon rendu à du Mont.}}
{\textsc{- Désespoir de Rion, qui à la fin se console.}} {\textsc{-
Maladie de M\textsuperscript{me} de Saint-Simon à Passy.}} {\textsc{- Le
régent nous prête le château neuf de Meudon.}} {\textsc{- Deuil de la
cour prolongé six semaines au delà de celui du roi.}} {\textsc{- Il
visite Madame, M. {[}le duc{]} et M\textsuperscript{me} la duchesse
d'Orléans.}} {\textsc{- Le roi au Louvre, en visite toutes les académies
pendant qu'on nettoie les Tuileries.}} {\textsc{- M. et
M\textsuperscript{me} du Maine fort relâchés.}} {\textsc{- Aveux de la
duchesse du Maine.}} {\textsc{- Misérable comédie entre elle et son
mari.}} {\textsc{- Le secrétaire du prince de Cellamare mis au château
de Saumur.}} {\textsc{- MM. d'Allemans, Renaud et le P. Malebranche\,;
quels.}} {\textsc{- Mémoires d'Allemans sur la manière de lever la
taille.}} {\textsc{- La Muette donnée au roi, et le gouvernement à
Pezé.}} {\textsc{- Vingt mille livres de pension à M\textsuperscript{me}
la princesse de Conti la mère.}} {\textsc{- Cent cinquante mille livres
de brevet de retenue à Lautrec sur la lieutenance générale de Guyenne.}}
{\textsc{- Toutes pensions se payent.}} {\textsc{- Forte augmentation de
troupes.}} {\textsc{- M. le duc d'Orléans achète pour M. le duc de
Chartres le gouvernement de Dauphiné, de La Feuillade, qu'il accable
d'argent.}} {\textsc{- La Vrillière présente au roi les députés des
états de Languedoc, de préférence à Maillebois, lieutenant général de la
province.}} {\textsc{- Extraction de Maillebois.}} {\textsc{- Belle
action des moines d'Orcamp.}} {\textsc{- M\textsuperscript{me} la
duchesse d'Orléans refuse audience à tous députés d'états, depuis la
prison du duc du Maine.}} {\textsc{- Le duc de Richelieu peu à peu en
liberté.}}

~

M\textsuperscript{me} la duchesse de Berry était à Meudon du lendemain
de Pâques, 10 avril, d'où elle s'était fait transporter à la Muette le
14 mai, couchée dans un carrosse entre deux draps. Elle ne s'y trouva
point soulagée. Le mal eut son cours, les accidents et les douleurs
augmentèrent avec des intervalles courts et légers, et la fièvre le plus
ordinairement marquée et souvent forte. Des irrégularités de crainte et
d'espérance se soutinrent jusqu'au commencement de juillet. Cet état, où
les temps de soulagement passaient si promptement et où la souffrance
était si durable, donna des trêves à l'ardeur {[}de{]} déclarer le
mariage de Rion, et engagea, outre la proximité de lieu, M. le duc
d'Orléans à rapprocher ses visites, et même M\textsuperscript{me} la
duchesse d'Orléans et Madame aussi, laquelle passait l'été à
Saint-Cloud. Le mois de juillet devint plus menaçant par la suite
continuelle des accidents et des douleurs et par beaucoup de fièvre. Ces
maux augmentèrent tellement le 14 juillet, qu'on commença tout de bon à
tout craindre.

La nuit fut si orageuse qu'on envoya éveiller M. le duc d'Orléans au
Palais-Royal. En même temps, M\textsuperscript{me} de Pons écrivit à
M\textsuperscript{me} de Saint-Simon, et la pressa d'aller s'établir à
la Muette. On a vu qu'elle ne voyait M\textsuperscript{me} la duchesse
de Berry que pour des cérémonies, et les soirs pour l'heure de sa cour,
où elle ne soupait presque jamais, et retenait seulement les dames qui
étaient choisies pour y souper, entre celles qui s'y trouvaient ou au
jeu ou à voir jouer, ce qui était le temps de sa cour publique. Elle ne
la suivait guère que chez le roi, ce qui était rare\,; et quoiqu'elle
eût un logement à la Muette, elle n'y allait comme point\,; c'était
excès de complaisance si elle y couchait une nuit, quoique la princesse
et sa maison n'y fussent occupées que d'elle, et que ce fût une fête et
toutes sortes de soins quand elle faisait tant que d'y aller une fois,
et rarement deux pendant tout le séjour qu'on y faisait. Elle se rendit
à l'avis de M\textsuperscript{me} de Pons, et s'y en alla sur-le-champ
pour y demeurer.

Elle trouva le danger grand. Il y eut une saignée faite au bras, puis au
pied ce même jour 15 juillet, et on envoya chercher un cordelier son
confesseur. J'interromps ici la suite de cette maladie, qui dura encore
sept jours, et qui finit le 21 juillet, parce que ce qui reste à en
rapporter s'entendra mieux après avoir vu d'un même coup d'oeil cette
princesse tout entière, au hasard peut-être de quelques légères redites
de ce qui se trouve d'elle ici en différents endroits.

M\textsuperscript{me} la duchesse de Berry a fait tant de bruit dans
l'espace d'une très courte vie que, encore que la matière en soit
triste, elle est curieuse et mérite qu'on s'y arrête un peu. Née avec un
esprit supérieur, et, quand elle le voulait, également agréable et
aimable, et une figure qui imposait et qui arrêtait les yeux avec
plaisir, mais que sur la fin le trop d'embonpoint gâta un peu, elle
parlait avec une grâce singulière, une éloquence naturelle qui lui était
particulière, et qui coulait avec aisance et de source, enfin avec une
justesse d'expressions qui surprenait et charmait. Que n'eût-elle point
fait de ces talents avec le roi et M\textsuperscript{me} de Maintenon,
qui ne voulaient que l'aimer, avec M\textsuperscript{me} la duchesse de
Bourgogne, qui l'avait mariée, et qui en faisait sa propre chose, et
depuis avec un père régent du royaume, qui n'eut des yeux que pour elle,
si les vices du coeur, de l'esprit et de l'âme, et le plus violent
tempérament n'avaient tourné tant de belles choses en poison le plus
dangereux. L'orgueil le plus démesuré et la fausseté la plus
continuelle, elle les prit pour des vertus, dont elle se piqua toujours,
et l'irréligion, dont elle croyait parer son esprit, mit le comble à
tout le reste.

On a vu en plus d'un endroit ici son étrange conduite avec M. le duc de
Berry, son horreur pour une mère bâtarde\,; ses mépris pour un père
qu'elle avait dompté\,; ses extravagantes idées à l'égard de
Monseigneur\,; son désespoir de rang et d'ingratitude pour M. {[}le
duc{]} et M\textsuperscript{me} la duchesse de Bourgogne, à qui elle
devait tout\,; son peu d'égards pour le roi et pour
M\textsuperscript{me} de Maintenon\,; sa haine déclarée pour tous ceux
qui avaient contribué à son mariage, parce que, disait-elle, il lui
était insupportable d'avoir obligation à quelqu'un\,; ses grossières
tromperies et ses hauteurs\,; l'inégalité d'une conduite si peu d'accord
avec elle-même\,; enfin jusqu'à la honte de l'ivrognerie complète et de
tout ce qui accompagne la plus basse crapule en convives, en ordures et
en impiétés. On a vu que, dès les premiers jours du mariage, la force du
tempérament ne tarda pas à se déclarer, les indécences journalières en
public, ses courses après plusieurs jeunes gens avec peu ou point de
mesure, et jusqu'à quelles folies fut porté son abandon à La Haye,
ensuite à Rion, enfin ses projets d'avoir de grands noms et des braves
dans sa maison pour se faire compter entre l'Espagne et son père, se
tourner du côté qui lui semblerait le plus avantageux des deux, se
figurer que cela lui serait possible, usurper aussi le rang de reine en
plusieurs occasions, et une fois de plus que reine, avec les
ambassadeurs.

Ce qui parut de plus extraordinaire fut l'étonnant contraste d'un
orgueil qui la portait sur les nues, et de la débauche qui la faisait
manger non seulement avec quelques gens de qualité, elle dont le rang ne
souffrait point d'autres hommes à sa table que des princes du sang, même
en particulier uniquement et à des parties de campagne, mais d'y
admettre le P. Riglet, jésuite, qui en savait dire des meilleures, et
d'autres espèces de canailles, qui n'auraient été admis dans aucune
honnête maison, et souper souvent avec les roués de M. le duc d'Orléans,
avec lui et sans lui, et se plaire à exciter leurs gueulées et leurs
impiétés. Ce court crayon rappelle en peu de mots ce qu'on a vu épars
ici plus au long à mesure que les occasions s'en sont présentées,
quoique écrit le plus succinctement qu'il a été possible, qui a montré
jusqu'à quel point elle manquait de tout jugement et de tout honnête,
même naturel sentiment.

Parmi une dépravation si universelle et si publique, elle était indignée
qu'on osât en parler. Elle débitait hardiment qu'il n'était jamais
permis de parler des personnes de son rang, non pas même de blâmer ce
qui pouvait le mériter dans leurs actions les plus publiques, et qu'on
aurait vues soi-même, combien moins de ce qui ne se passait qu'en
particulier. C'est ce qui l'irritait contre tout le monde, comme d'un
droit sacré violé en sa personne, le plus criminel manquement de
respect, le plus indigne de pardon. Sa mort aussi fut un étrange
spectacle. C'est maintenant à quoi il faut revenir.

Les longues douleurs dont elle fut accablée ne purent la persuader de
penser à cette vie par un régime nécessaire à son état, ni à celle qui
la devait bientôt suivre, jusqu'à ce qu'enfin parents et médecins se
crurent obligés de lui parler un langage qu'on ne tient aux princes de
ce rang qu'à grand'peine dans la plus urgente extrémité, mais que
l'impiété de Chirac déconcerta. Néanmoins, comme il fut seul de son
avis, et que tous les autres, qui avaient parlé, continuèrent à le
faire, elle se soumit aux remèdes pour ce monde et pour l'autre. Elle
reçut ses sacrements à portes ouvertes, et parla aux assistants sur sa
vie et sur son état, mais en reine de l'une et de l'autre. Après que ce
spectacle fut fini, et qu'elle se fut renfermée avec ses familiers, elle
s'applaudit avec eux de la fermeté qu'elle avait montrée, et leur
demanda si elle n'avait pas bien parlé, et si ce n'était pas mourir avec
grandeur et avec courage.

Un peu après, elle ne retint que M\textsuperscript{me} de Mouchy, lui
indiqua clef et cassette, et lui dit de lui apporter son baguier\,; il
fut apporté, et ouvert. M\textsuperscript{me} la duchesse de Berry lui
en fit un présent après quantité d'autres\,; car, outre ce qu'elle avait
eu souvent, il n'y avait guère de jours, depuis qu'elle était malade,
qu'elle n'en tirât tout ce qu'elle pouvait, souvent de l'argent et des
pierreries\,: le moins était des bijoux. Ce baguier valait seul plus de
deux cent mille écus. La Mouchy, tout avide qu'elle était, ne laissa pas
d'en être étourdie. Elle sortit et le montra à son mari. C'était le
soir. M. {[}le duc{]} et M\textsuperscript{me} la duchesse d'Orléans
étaient partis. Le mari et la femme eurent peur d'être accusés de vol,
tant leur réputation était bonne. Ils crurent donc en devoir dire
quelque chose à ce qui leur était le moins opposé dans la maison, où ils
étaient généralement haïs et méprisés.

De l'un à l'autre la chose fut bientôt sue, et vint à
M\textsuperscript{me} de Saint-Simon. Elle connaissait ce baguier et en
fut si étonnée, qu'elle crut en devoir informer M. le duc d'Orléans, à
qui elle le manda sur-le-champ. L'état où était M\textsuperscript{me} la
duchesse de Berry faisait qu'on ne se couchait guère à la Muette, où on
se tenait dans un salon. M\textsuperscript{me} de Mouchy, voyant que
l'affaire du baguier devenait publique et réussissait mal, s'approcha
fort embarrassée de M\textsuperscript{me} de Saint-Simon, lui conta
comment cela s'était passé, tira le baguier de sa poche, et le lui
montra. M\textsuperscript{me} de Saint-Simon appela les dames les plus
proches d'où elle était pour le voir aussi, et devant elles (car elle ne
les avait appelées que dans ce dessein), elle dit à
M\textsuperscript{me} de Mouchy que c'était là un beau présent, mais
qu'il était si beau qu'elle lui conseillait d'en aller rendre compte au
plus tôt à M. le duc d'Orléans, et {[}de{]} le lui porter. Ce conseil,
et donné en présence de témoins, embarrassa étrangement
M\textsuperscript{me} de Mouchy. Elle répondit néanmoins qu'elle le
ferait, et alla retrouver son mari, avec qui elle monta dans sa chambre.

Le lendemain matin ils furent ensemble au Palais-Royal, et demandèrent à
parler à M. le duc d'Orléans, qui, averti par M\textsuperscript{me} de
Saint-Simon, les fit aussitôt entrer, et sortir le peu qui était dans
son cabinet\,; car il était fort matin. M\textsuperscript{me} de Mouchy,
son mari présent, fit son compliment comme elle put. M. le duc
d'Orléans, pour toute réponse, lui demanda où était le baguier. Elle le
tira de sa poche et le lui présenta. M. le duc d'Orléans le prit,
l'ouvrit, considéra bien si rien n'y manquait, car il le connaissait
parfaitement, le referma, tira une clef de sa poche, l'enferma dans un
tiroir de son bureau, puis les congédia par un signe de tête, sans dire
un mot, ni eux non plus. Ils firent la révérence, et se retirèrent
également outrés et confus. Oncques depuis ils ne reparurent à la
Muette. Bientôt après M. le duc d'Orléans y arriva, qui, dès qu'il eut
vu un moment M\textsuperscript{me} sa fille, prit M\textsuperscript{me}
de Saint-Simon en particulier, la remercia beaucoup de ce qu'elle lui
avait mandé et fait, lui conta ce qu'il venait de faire, et que le
baguier ne sortirait plus de ses mains. Il était si en colère de cette
effronterie, qu'il ne put se tenir d'en parler dans le salon en termes
fort désavantageux pour M. et M\textsuperscript{me} de Mouchy, au grand
applaudissement de toute la compagnie, même jusque des valets.

Je ne sais si l'absence de la Mouchy fit quelque impression heureuse sur
M\textsuperscript{me} la duchesse de Berry\,; mais elle n'en parla
jamais, et peu après elle parut fort rentrée en elle-même, et souhaita
de recevoir encore une fois Notre-Seigneur. Elle le reçut, à ce qu'il
parut, avec beaucoup de piété, et tout différemment de la première fois.
Ce fut l'abbé de Castries, son premier aumônier, nommé à l'archevêché de
Tours, qui le fut après d'Albi, et enfin commandeur de l'ordre, qui le
lui administra et qui le fut chercher à la paroisse de Passy, et l'y
reporta, suivi de M. le duc d'Orléans et de M. le duc de Chartres. Cet
abbé fit une exhortation courte, belle, touchante et tellement
convenable, qu'elle fut admirée de tout ce qui l'entendit.

Dans cette extrémité où les médecins ne savent plus que faire et où on a
recours à tout, on parla de l'élixir d'un nommé Garus, qui faisait alors
beaucoup de bruit, et dont le roi a depuis acheté le secret. Garus fut
donc mandé et arriva bientôt après. Il trouva M\textsuperscript{me} la
duchesse de Berry si mal qu'il ne voulut répondre de rien. Le remède fut
donné et réussit au delà de toute espérance. Il ne s'agissait plus que
de continuer. Sur toutes choses, Garus avait demandé que rien sans
exception ne fût donné à M\textsuperscript{me} la duchesse de Berry que
par lui, et cela même avait été très expressément commandé par M. {[}le
duc{]} et par M\textsuperscript{me} la duchesse d'Orléans.
M\textsuperscript{me} la duchesse de Berry continua d'être de plus en
plus soulagée, et si revenue à elle-même que Chirac craignit d'en avoir
l'affront. Il prit son temps que Garus dormait sur un sofa, et avec son
impétuosité présenta un purgatif à M\textsuperscript{me} la duchesse de
Berry, qu'il lui fit avaler sans en dire mot à personne et sans que deux
garde-malades, qu'on avait prises pour la servir, et qui seules étaient
présentes, osassent branler devant lui. L'audace fut aussi complète que
la scélératesse, car M. {[}le duc{]} et M\textsuperscript{me} la
duchesse d'Orléans étaient dans le salon de la Muette. De ce moment à
celui de retomber pis que l'état d'où l'élixir l'avait tirée, il n'y eut
presque pas d'intervalle. Garus fut réveillé et appelé. Voyant ce
désordre, il s'écria qu'on avait donné un purgatif qui, quel qu'il fût,
était un poison dans l'état de la princesse. Il voulut s'en aller, on le
retint, on le mena à M. {[}le duc{]} et à M\textsuperscript{me} la
duchesse d'Orléans. Grand vacarme devant eux, cris de Garus, impudence
de Chirac et hardiesse sans égale à soutenir ce qu'il avait fait. Il ne
pouvait le nier, parce que les deux gardes avaient été interrogées et
l'avaient dit. M\textsuperscript{me} la duchesse de Berry, pendant ce
débat, tendait à sa fin sans que Chirac ni Garus eussent de ressource.
Elle dura cependant le reste de la journée et ne mourut que sur le
minuit. Chirac, voyant avancer l'agonie, traversa la chambre, et faisant
une révérence d'insulte au pied du lit, qui était ouvert, lui souhaita
un bon voyage en termes équivalents, et de ce pas s'en alla à Paris. La
merveille est qu'il n'en fut autre chose, et qu'il demeura auprès de M.
le duc d'Orléans comme auparavant.

Depuis la légèreté, pour ne pas employer un autre nom, que M. le duc
d'Orléans avait eue de parler à M\textsuperscript{me} la duchesse de
Berry d'un avis que je lui avais donné, si important à l'un et à
l'autre, au lieu d'en profiter, et de la haine qu'elle en conçut, ce qui
arriva dès les premiers mois de son mariage, je ne la vis plus qu'aux
occasions indispensables, qui n'arrivaient presque jamais, et d'ailleurs
quand il n'en arrivait point, une fois ou deux l'an tout au plus, à une
heure publique, et un instant à chaque fois. M\textsuperscript{me} de
Saint-Simon, voyant que la fin s'approchait, et qu'il n'y avait personne
à la Muette avec qui M. le duc d'Orléans fût bien libre, me manda
qu'elle me conseillait d'y venir pour être auprès de lui dans ces
tristes moments. Il me parut en effet que mon arrivée lui fit plaisir,
et que je ne lui fus pas inutile au soulagement de s'épancher en liberté
avec moi. Le reste du jour se passa ainsi et à entrer des moments dans
la chambre. Le soir je fus presque toujours seul auprès de lui.

Il voulut que je me chargeasse de tout ce qui devait se faire après que
M\textsuperscript{me} la duchesse de Berry {[}serait morte{]}, sur
l'ouverture de son corps, et le secret en cas qu'elle se trouvât grosse,
sur tous les détails qui demandaient ses ordres et sa décision, pour
n'être point importuné de ces choses touchantes, et de tout ce qui
regardait les funérailles et les ordres qu'il y avait à y donner. Il me
parla avec toute sorte d'amitié et de confiance, ne voulut point
qu'ensuite je lui demandasse ses ordres sur rien, et dit en passant à
toute la maison de la princesse, qui se trouvait là toute rassemblée,
qu'il m'avait donné ses ordres, et que c'était à moi, qu'il en avait
chargé, à les donner sur tout ce qui pourrait demander les siens. Il me
dit, de plus, qu'il ne comptait plus M\textsuperscript{me} de Mouchy
pour être de la maison, avec sa chimère de charge de seconde dame
d'atours\,; qu'elle avait perdu sa fille, qu'elle l'avait pillée,
n'oublia pas le baguier qu'il lui avait ôté, et me chargea,
conjointement avec M\textsuperscript{me} de Saint-Simon, d'empêcher
qu'elle demeurât à la Muette si elle s'y présentait, encore plus de lui
laisser faire aucune fonction, ni d'entrer dans les carrosses pour
accompagner le corps à Saint-Denis, ou le coeur au Val-de-Grâce.

Je proposai à M. le duc d'Orléans qu'il n'y eût ni garde du corps, ni
eau bénite, ni aucune cérémonie\,; que le convoi fût décent, mais au
plus simple, et les suites de même, surtout qu'au service de
Saint-Denis, où on ne pouvait éviter le cérémonial ordinaire, il n'y eût
point d'oraison funèbre\,: je lui en touchai légèrement les raisons,
qu'il sentit très bien, me remercia, et convint avec moi que les choses
se passeraient ainsi, et que de sa part je les ordonnasse de la sorte.
Je fus le plus court que je pus avec lui sur ces funèbres matières, et
je le promenais tant que je pouvais de temps en temps dans les pièces de
suite de la maison et dans l'entrée du jardin, et le détournais de la
chambre de la mourante autant qu'il me fut possible.

Le soir bien avancé, et M\textsuperscript{me} la duchesse de Berry de
plus en plus mal et sans connaissance depuis que Chirac l'avait
empoisonnée, comme on a vu en son lieu que les médecins de la cour en
firent autant au maréchal de Boufflers, en pareil cas, à Fontainebleau,
et avec même succès, M. le duc d'Orléans rentra dans la chambre et
approcha du chevet du lit, dont tous les rideaux étaient ouverts\,; je
ne l'y laissai que quelques moments et le poussai dans le cabinet, où il
n'y avait personne. Les fenêtres y étaient ouvertes, il s'y mit appuyé
sur le balustre de fer, et ses pleurs y redoublèrent au point que j'eus
peur qu'il ne suffoquât. Quand ce grand accès se fut un peu passé, il se
mit à me parler des malheurs de ce monde et du peu de durée de ce qui
est de plus agréable. J'en pris occasion de lui dire ce que Dieu me
donna, avec toute la douceur, l'onction et la tendresse qu'il me fut
possible. Non seulement il reçut bien ce que je lui disais, mais il y
répondit et en prolongea la conversation.

Après avoir été là plus d'une heure, M\textsuperscript{me} de
Saint-Simon me fit avertir doucement qu'il était temps que je tâchasse
d'emmener M. le duc d'Orléans, d'autant plus qu'on ne pouvait sortir de
ce cabinet que par la chambre. Son carrosse était prêt, que
M\textsuperscript{me} de Saint-Simon avait eu soin de faire venir. Ce ne
fut pas sans peine que je pus venir doucement à bout d'arracher de là M.
le duc d'Orléans plongé dans la plus amère douleur. Je lui fis traverser
la chambre tout de suite, et le suppliai de s'en retourner à Paris. Ce
fut une autre peine à l'y résoudre. À la fin il se rendit. Il voulut que
je demeurasse pour tous les ordres. Il pria M\textsuperscript{me} de
Saint-Simon avec beaucoup de politesse d'être présente à tous les
scellés, après quoi je le mis dans son carrosse, et il s'en alla. Je
rendis ensuite à M\textsuperscript{me} de Saint-Simon les ordres qu'il
m'avait donnés sur l'ouverture du corps, pour qu'elle les fît exécuter,
et sur tout le reste, et je l'empêchai de demeurer dans le spectacle de
cette chambre où il n'y avait plus que de l'horreur.

Enfin sur le minuit du 21 juillet, M\textsuperscript{me} la duchesse de
Berry mourut, deux jours après le forfait de Chirac. M. le duc d'Orléans
fut le seul touché. Quelques perdants s'affligèrent\,; mais qui d'entre
eux eut de quoi subsister ne parut pas même regretter sa perte.
M\textsuperscript{me} la duchesse d'Orléans sentit sa délivrance, mais
avec toutes les mesures de la bienséance. Madame ne s'en contraignit que
médiocrement. Quelque affligé que fût M. le duc d'Orléans, la
consolation ne tarda guère. Le joug auquel il s'était livré et qu'il
trouvait souvent pesant, était rompu. Surtout il se trouvait affranchi
des affres de la déclaration du mariage de Rion et de ses suites,
embarras d'autant plus grand, qu'à l'ouverture du corps, la pauvre
princesse fut trouvée grosse\,; on trouva aussi un dérangement dans son
cerveau. Cela ne promettait que de grandes peines et fut soigneusement
étouffé pour le temps.

Sur les cinq heures du matin, c'est-à-dire cinq heures après cette mort,
La Vrillière arriva à la Muette, où il mit le scellé en présence de
M\textsuperscript{me} de Saint-Simon. Dès que cela fut fait, elle monta
dans son carrosse avec lui, que les gens nécessaires au scellé suivirent
dans le carrosse de La Vrillière, et s'en allèrent en faire autant à
Meudon, puis au Luxembourg, de là au Palais-Royal en rendre compte à M.
le duc d'Orléans, après quoi M\textsuperscript{me} de Saint-Simon revint
à la Muette, où une plus cruelle nuit l'attendait par l'horreur de ses
fonctions à l'ouverture du corps, de laquelle j'allai rendre compte à M.
le duc d'Orléans, et de l'exécution de ses ordres. Le corps fut déposé
ensuite dans la chapelle de la Muette sans être gardé, où les messes
basses furent continuelles tous les matins.

Je m'établis à Passy chez M. et M\textsuperscript{me} de Lauzun pour
être plus près de la Muette, sans y être toujours, d'où j'allais presque
tous les jours voir M. le duc d'Orléans, outre les jours de conseil de
régence. Comme il n'y eut point de cérémonie, tout le monde fut dispensé
des manteaux et des mantes au Palais-Royal, où on se présenta en deuil,
mais en habits ordinaires. Il ne se trouva point de testament, et
M\textsuperscript{me} la duchesse de Berry ne donna rien à personne, que
ce que M\textsuperscript{me} de Mouchy s'était fait donner. Elle
jouissait de sept cent mille livres de rente, sans ce que depuis la
régence elle tirait de M. le duc d'Orléans.

Le soir du samedi 22, l'abbé de Castries, nommé à l'archevêché de Tours
et son premier aumônier, porta le coeur au Val-de-Grâce, ayant à sa
gauche M\textsuperscript{lle} de La Roche-sur-Yon, M\textsuperscript{me}
de Saint-Simon au-devant et la duchesse de Louvigny nommée par le roi.
M\textsuperscript{me} de Brassac, dame de M\textsuperscript{me} la
duchesse de Berry, à une portière, et ce qui fut fort étrange, la dame
d'honneur de M\textsuperscript{me} la princesse de Conti, mère de
M\textsuperscript{lle} de La Roche-sur-Yon, à l'autre. Le deuil du roi
fut de six semaines, celui du Palais-Royal de trois mois par respect du
rang, et M\textsuperscript{me} de Saint-Simon drapa pour six mois, parce
qu'elle avait, comme on l'a vu en son lieu, drapé par excès de
complaisance à d'autres deuils où M. le duc de Berry drapait sans que le
roi drapât.

Le dimanche 23 juillet, sur les dix heures du soir, le corps de
M\textsuperscript{me} la duchesse de Berry fut mis dans un carrosse dont
les huit chevaux étaient caparaçonnés. Il n'y eut aucune tenture à la
Muette. L'abbé de Castries et les prêtres suivaient dans un autre
carrosse, et les dames de M\textsuperscript{me} la duchesse de Berry
dans un autre. Il n'y eut qu'une quarantaine de flambeaux portés par ses
pages et ses gardes. Le convoi passa par le bois de Boulogne et la
plaine de Saint-Denis, avec beaucoup de simplicité, et fut reçu de même
dans l'église de l'abbaye.

La veille du convoi, M. le duc d'Orléans, sans que je lui en parlasse,
me dit que le roi conservait à M\textsuperscript{me} de Saint-Simon ses
appointements en entier qui étaient de vingt et un mille livres. Je l'en
remerciai, et en même temps je lui dis que ce serait faire à
M\textsuperscript{me} de Saint-Simon et à moi la grâce entière, de
conserver aux dames de M\textsuperscript{me} la duchesse de Berry leurs
appointements\,; il me les accorda sur-le-champ\,; ensuite je lui
demandai la même grâce pour la première femme de chambre qui était une
fille d'un singulier mérite, je l'obtins aussi. Au sortir du
Palais-Royal, j'allai à la Muette, où je dis à M\textsuperscript{me} de
Saint-Simon ce que je venais de faire\,; elle envoya prier toutes les
dames de venir dans sa chambre, et leur manda que j'y étais et que
j'avais à leur parler. J'eus la malice de ne leur rien dire jusqu'à ce
que toutes fussent arrivées\,; alors je leur appris les grâces du régent
qui leur conserva aussi en même temps leurs logements au Luxembourg. La
joie fut grande et sans contrainte, et je fus bien embarrassé\,; je leur
conseillai d'aller toutes ensemble le lendemain remercier M. le duc
d'Orléans\,; elles le firent et furent reçues de très bonne grâce. En
même temps, M\textsuperscript{me} de Saint-Simon lui remit l'appartement
qu'elle avait au Luxembourg, et lui demanda de le rendre à Mille de
Langeais et à ses frères qui l'avaient auparavant, et elle l'obtint. On
a vu ailleurs que M\textsuperscript{me} de Saint-Simon ne s'en était
jamais servie, mais on n'avait pas voulu le reprendre, et qu'il parût
qu'elle n'avait point d'appartement au Luxembourg.

M\textsuperscript{me} de Mouchy fit demander une audience à M. le duc
d'Orléans qui ne voulut pas la voir, et lui fit dire d'aller parler à La
Vrillière. Elle y fut donc avec son mari. Elle y reçut l'ordre de sortir
tous deux en vingt-quatre heures de Paris et de n'y pas revenir.
Longtemps après ils y revinrent, mais aucun des événements arrivés dans
la suite n'a pu les rétablir dans le monde, ni les tirer d'obscurité, de
mépris et d'oubli.

Les spectacles furent interrompus huit jours à Paris.

M. le duc d'Orléans, dès les premiers jours, envoya chercher du Mont,
lui rendit le gouvernement de Meudon, et lui ordonna d'y faire revenir
tous les gens qui y étaient lorsque M\textsuperscript{me} la duchesse de
Berry eut Meudon, et que leurs emplois leur seraient rendus. On peut
juger en quel état tomba Rion en apprenant à l'armée une aussi terrible
nouvelle pour lui\,; quel affreux dénouement d'une aventure plus que
romanesque, au point qu'il touchait à tout ce que l'ambition peut
procurer même de plus imaginaire\,; aussi fut-il plus d'une fois sur le
point de se tuer, et longtemps gardé à vue par des amis que la pitié lui
fit. Il vendit bientôt après la fin de la campagne son régiment et son
gouvernement. Comme il avait été doux et poli avec ses amis, il en
conserva, et fit bonne chère avec eux pour se consoler. Mais au fond, il
demeura obscur, et cette obscurité l'absorba.

Le service de M\textsuperscript{me} la duchesse de Berry se fit à
Saint-Denis avec les cérémonies accoutumées, mais sans oraison funèbre,
les premiers jours de septembre.

M\textsuperscript{me} de Saint-Simon, qui, comme on l'a vu en son lieu,
avait été forcée, et moi aussi, à consentir qu'elle fût dame d'honneur
de M\textsuperscript{me} la duchesse de Berry, n'avait pu, en aucun
temps, trouver le moindre jour à quitter cette triste place. On avait
pour elle toute sorte de considération, et on lui laissait toute sorte
de liberté\,; mais tout cela ne la consolait point de cette place, de
sorte qu'elle sentit tout le plaisir, pour ne pas dire toute la
satisfaction, d'une délivrance qu'elle n'attendait pas d'une princesse
de vingt-quatre ans. Mais l'extrême fatigue des derniers jours de la
maladie, et de ceux qui suivirent la mort, lui causèrent une fièvre
maligne dont elle fut six semaines à l'extrémité dans une maison que
Fontanieu lui avait prêtée à Passy pour prendre l'air et des eaux de
Forges, et s'y reposer\,; elle fut deux mois à s'en remettre. Cet
accident, qui me pensa tourner la tête, me séquestra de tout pendant
deux mois sans sortir de cette maison et presque de sa chambre, sans
ouïr parler de rien, et sans voir que le peu de proches ou d'amis
indispensables. Lorsqu'elle commença à se rétablir, je demandai à M. le
duc d'Orléans quelques logements au château neuf de Meudon. Il me le
prêta tout entier et tout meublé. Nous y passâmes le reste de l'été et
plusieurs autres depuis. C'est un lieu charmant pour toute espèce de
promenades. Nous comptions de n'y voir que nos amis, mais la proximité
nous accabla de monde, en sorte que tout le château neuf fut souvent
tout rempli, sans les gens de simple passage.

Pour ne plus revenir à la même matière, le deuil de
M\textsuperscript{me} la duchesse de Berry eut une chose jusqu'alors
sans exemple, et qui n'en a pas eu depuis\,: c'est que le roi, ne le
portant que six semaines, la cour ne comptait pas le porter davantage,
parce que les deuils de cour ne se portent que par respect pour le roi,
et se prennent et se quittent en même temps que lui. Cependant il y eut
ordre de le continuer au delà du roi et de le porter trois mois,
c'est-à-dire autant que M. le duc d'Orléans le porta.

Les logements au Luxembourg furent conservés aux deux premiers
officiers, et au premier maître d'hôtel\,; et le chevalier d'Hautefort,
premier écuyer, obtint de conserver les livrées et un carrosse aux armes
de M\textsuperscript{me} la duchesse de Berry sur le dernier exemple de
Sainte-Maure, premier écuyer de feu M. le duc de Berry.

Le roi alla voir sur cette mort Madame, M. {[}le duc{]} et
M\textsuperscript{me} la duchesse d'Orléans.

Le roi, qui était depuis trois semaines dans l'appartement de la reine
mère au Louvre pour laisser nettoyer les Tuileries, alla, pendant ce
séjour, voir toutes les académies et le balancier. Le maréchal de
Villeroy voulut parler aux Académies française, des sciences et des
belles-lettres\,; on ne comprit ni pourquoi ni trop ce qu'il y dit\,;
les directeurs de ces académies firent chacun une harangue au roi, qui
retourna après aux Tuileries.

M\textsuperscript{me} du Maine obtint d'aller demeurer dans un château
voisin de Châlon-sur-Saône où La Billarderie la fut conduire, et le duc
du Maine, celle de chasser autour de Dourlens, mais sans en découcher.
En même temps le secrétaire du prince de Cellamare, qui avait eu enfin
permission de retourner en Espagne, fut arrêté en chemin à Orléans, et
mené dans le château de Saumur. C'est que la duchesse du Maine avait
enfin commencé à parler, à avouer beaucoup de choses, peut-être à en
cacher davantage\,; car, comme je l'ai dit au commencement de cette
affaire, et pourquoi, je n'y ai jamais vu bien clair, et je suis très
persuadé que M. le duc d'Orléans, qui sûrement en a su davantage, en a
ignoré plus qu'il n'en a su, et que l'abbé Dubois s'est bien gardé de ne
retenir pas pour soi tout seul le fond et le très fond de l'affaire,
n'en a dit à son maître que ce qu'il n'a pu lui cacher, et lui a
soigneusement tu tout ce qui ne le conduisait pas aux vues que j'ai
expliquées.

M\textsuperscript{me} du Maine avoua donc enfin, par une espèce de
mémoire qu'elle envoya, signé d'elle, à M. le duc d'Orléans, que le
projet d'Espagne était véritable, nomma comme complices ceux dont j'ai
parlé, mais fort diversement. Elle y traita Pompadour avec un grand
mépris, et les gens de peu qui étaient arrêtés, confirma la chimère du
duc de Richelieu sur Bayonne pour avoir le régiment des gardes, et de
Saillant qui y avait aussi son régiment, et qui s'était laissé
entraîner. Boisdavid y était fort chargé, et Laval plus qu'aucun autre,
comme la clef de meute, l'homme de confiance et d'expédients, qui
conduisait Cellamare en beaucoup de choses, le seul qui allât
directement de lui à elle et d'elle à lui, qui avait la créance de la
noblesse qui leur était attachée, et qu'il savait conduire où il
convenait sans leur rien dire qu'avec grande mesure pour les temps et
pour le choix des personnes\,; enfin qu'ils avaient compté de faire une
révolte à Paris et dans les provinces contre le gouvernement, de le
changer, d'y faire déclarer le roi d'Espagne régent, de mettre à la tête
de toutes les affaires et de toutes les troupes celui que le roi
d'Espagne nommerait pour exercer la régence en son nom et en sa place,
de faire enregistrer ces changements dans tous les parlements, et que
pour opérer ces choses, ils avaient formé un grand parti en Bretagne
avec promesse réciproque que le roi d'Espagne leur rendrait tous leurs
privilèges, tels qu'ils en jouissaient du temps d'Anne de Bretagne et
des deux rois successivement ses époux, Charles VIII et Louis XII, et
que la Bretagne recevrait toutes les troupes que l'Espagne voudrait
envoyer en France, et lui livrerait le Port-Louis pour en être le seul
maître absolu. Plusieurs Bretons furent nommés\,; je n'ai point su
qu'aucun membre des parlements de Paris et de Rennes l'aient été,
peut-être bien M. le duc d'Orléans l'a-t-il ignoré lui-même. Si elle a
chargé des seigneurs de la cour qui ont montré avoir grand'peur, mais
qui ne furent pas arrêtés, c'est encore ce qui n'est pas venu jusqu'à
moi.

Laval, interrogé à la Bastille sur ces aveux, entra en furie contre la
duchesse du Maine, jusqu'à lui donner toutes sortes de noms, s'écria que
c'était bien la dernière personne dont il aurait soupçonné la faiblesse
et l'infamie de révéler et de perdre ses amis, qu'il y avait plus de dix
ou douze ans qu'il la voyait peu en public, très fréquemment en
secret\,; que c'était elle qui l'avait embarqué dans toute cette
affaire, dont la colère lui fit dire plusieurs détails, sans que ces
détails soient revenus à moi ni à personne qu'à M. le duc d'Orléans,
qui, à ce que je crus voir, n'en fut même que légèrement instruit, et ne
les approfondit pas.

Un seul fut su\,: c'est qu'une nuit, qu'après avoir été souper à
l'Arsenal, M\textsuperscript{me} du Maine allait en bonne fortune voir
Cellamare sans valets, n'ayant que quelques gens affidés dedans et
derrière son carrosse, et Laval le menant au lieu de cocher et sans
flambeaux, elle fut accrochée par un autre carrosse, dont ils eurent
toutes les peines du monde à se débarrasser, et la plus grande frayeur
d'en être reconnus.

Ce furent ces aveux qui valurent plus de liberté à M. et à
M\textsuperscript{me} du Maine, et qui firent mettre à Saumur le
secrétaire de Cellamare. Ce fut aussi où commença cette comédie entre
eux deux, dont qui que ce soit ne put être la dupe. Ces aveux furent
accompagnés de toutes sortes d'assurances et de protestations que le duc
du Maine n'avait jamais su un mot de toute cette affaire\,; qu'ils
n'avaient garde d'en rien laisser apercevoir à sa timidité naturelle,
car, pour le sauver, elle ne le ménageait pas\,; qu'ils se seraient
exposés à voir rompre leur projet à l'instant, et très possiblement
encore à la révélation qu'il en aurait faite dans la peur où il en
aurait été\,; que leur plus épineux embarras avait été de se cacher de
lui, ce qui avait souvent retardé et quelquefois déconcerté toutes leurs
mesures par les contre-temps des rendez-vous et la fréquente nécessité
de les abréger. Ce fut à cette momerie que tout l'esprit de la duchesse
du Maine s'aiguisa, comme celui du duc du Maine, quand il apprit ces
aveux, à jurer de son ignorance, de son aveuglement, de son imbécillité
à ne s'être ni aperçu ni même douté de rien, à détester le projet et
ceux qui y avaient embarqué sa femme, et à se déchaîner contre elle avec
peu de ménagement.

M. le duc d'Orléans me conta toutes ces choses en attendant qu'il en
parlât au conseil de régence. Il eut l'air avec moi de mépriser la
conspiration, et de rire de la comédie entre le mari et la femme, de la
male-peur du duc du Maine et de l'usage que M\textsuperscript{me} du
Maine ne doutait pas de faire de son esprit à cet égard, et de son sexe
et de sa naissance pour elle-même, et du plein succès qu'elle s'en
promettait sûrement. Je me contentai de sourire et de lui répondre un
peu dédaigneusement que je serais bien de moitié avec elle, parce qu'il
n'est rien de si certain que de persuader qui veut absolument être
persuadé, et aussitôt je changeai de discours. Il y avait longtemps que
nous ne nous étions parlé de cette affaire. Il sentait bien que j'avais
raison\,; mais il sentait encore plus le poids du joug de l'abbé Dubois,
et j'avais bien reconnu, comme je l'ai dit plus haut, à quoi aboutirait
tout ce vacarme, et l'indignation m'avait fermé la bouche là-dessus. On
verra bientôt les suites de ces aveux sur la Bretagne, et à quel point
la comédie fut poussée entre M. et M\textsuperscript{me} du Maine.

Quoique je fasse profession dans ces Mémoires de ne les charger pas de
deux matières, dont l'une a produit une infinité de volumes, qui sont
entre les mains de tout le monde, et dont l'autre n'en fournirait guère
moins par son étendue et l'excès de ses révolutions, je veux dire la
constitution \emph{Unigenitus} et la finance, il se trouve néanmoins en
mon chemin des choses là-dessus que je me crois quelquefois obligé de
raconter.

La taille et la manière de la lever plus à charge que la taille même
avaient été un objet sur lequel on avait sans cesse médité depuis la
régence\footnote{Voy. les notes à la fin du volume.}. Les inconvénients
en étaient extrêmement moindres en Languedoc et en Bretagne\,; mais
c'étaient les seuls pays d'états\,; car le peu d'autres pays d'états
sont si petits, et objets si peu considérables, que ce n'étaient pas des
objets. M. d'Allemans, qui était un homme fort distingué parmi la
noblesse du Périgord par la sienne et par son mérite, et qui, depuis
qu'il s'y était retiré, y était considéré par tout ce qui y vivait,
comme un arbitre général, à qui chacun avait recours pour sa probité, sa
capacité et la douceur de ses manières, et comme un coq de province, où
il vivait très honorablement, était venu faire un tour à Paris, revoir
ses anciens amis, et il en avait beaucoup, et quelques-uns fort
considérables\,; car il avait longtemps vécu à la cour et à Paris, où il
s'était fait généralement estimer. Il était des miens dès ma jeunesse,
et son fils aussi, qui est devenu lieutenant-colonel du régiment du roi
infanterie, brigadier et commandeur de Saint-Louis, et qui n'a quitté
que par une grande blessure à la bataille de Parme, avec des pensions,
parce qu'elle l'avait mis hors d'état de servir. Le père et le fils
avaient beaucoup d'esprit, de savoir et de monde. Je les avais connus
chez le célèbre P. Malebranche, de l'Oratoire, dont la science et les
ouvrages ont fait tant de bruit, et la modestie, la rare simplicité, la
piété solide ont tant édifié, et dont la mort dans un âge avancé a été
si sainte, la même année de la mort du roi. D'autres circonstances
l'avaient fait connaître à mon père et à ma mère. Il avait bien voulu
quelquefois se mêler de mes études\,; enfin il m'avait pris en amitié,
et moi lui, qui a duré autant que sa vie. Le goût des mêmes sciences
l'avait fait ami intime de MM. d'Allemans père et fils, et c'était chez
lui que j'étais devenu le leur. Cette préface semble bien étrangère à ce
qui est annoncé. Elle y va pourtant paraître nécessaire, parce qu'elle y
montre là raison qui m'a fait mêler d'un projet de finance, moi dont le
goût et l'aptitude en sont si éloignés.

M. d'Allemans, excellent citoyen, qui était depuis longtemps témoin
oculaire des malheurs de la campagne, chercha des remèdes à ces maux. Il
crut en avoir trouvé un dans une manière de taille proportionnelle. Il
travailla son projet, et il en apporta des mémoires à Paris. Il me vint
voir et il m'en parla. Je lui dis que le petit Renaud avait eu une idée
pareille, et que M. le duc d'Orléans aussi l'avait envoyé en quelques
provinces faire quelques essais sur des paroisses en petit nombre, et
Silly d'un autre côté, qui s'y était présenté, qui est le même Silly
dont j'ai ailleurs raconté par avance la fortune et la catastrophe. Je
crois avoir aussi fait connaître ailleurs ce petit Renaud, que tout le
monde, et le meilleur, avec qui son mérite l'avait mêlé, appelait ainsi
de sa très petite taille. Il était très savant, très homme d'honneur,
modeste, désintéressé, zélé citoyen, avec de l'esprit et du monde, des
distractions plaisantes de géomètre, consommé dans toutes les parties de
la marine, fort brave, lieutenant général des armées navales,
grand'croix de Saint-Louis, qui avait fait en chef diverses expéditions,
fort estimé du feu roi dont il avait des pensions, et de ses ministres,
et de tout temps aimé de M. le duc d'Orléans. Il était ami intime de
Louville. Il était des miens, et, comme il était grand disciple du P.
Malebranche, il avait connu aussi M. d'Allemans. Ce dernier me lut un
mémoire tiré de ses observations. Louville, qui le connaissait, et qui
avait dîné avec lui chez moi, demeura présent à cette lecture.

Le mémoire était beau et solide et nous parut mériter d'aller plus
loin\,; mais avant d'en parler à M. le duc d'Orléans, nous jugeâmes
qu'il fallait éviter d'être croisés, et qu'il était à propos de
rassembler les lumières. Renaud était venu faire un tour à Paris\,; nous
en voulûmes profiter. Louville aboucha d'Allemans avec lui\,; ils eurent
plusieurs conférences chez Louville et une dernière chez moi.
Réciproquement ils approuvèrent leurs vues et leurs moyens de les
remplir. Réciproquement aussi ils trouvèrent des embarras et des
obstacles. Deux hommes d'honneur et d'esprit qui sincèrement ne
cherchent que le bien et ne se proposent aucun but particulier
conviennent aisément, même sur ce qui reste en dispute entre eux\,;
ainsi, tout bien examiné, ils jugèrent tous deux que ce plan devait être
proposé et lu en leur présence, pour qu'il jugeât lui-même des points
qui demeuraient indécis entre eux. Louville n'avait pas laissé de
travailler aussi à la refonte des points convenus, sur plusieurs
desquels Renaud et d'Allemans s'étaient conciliés\,; il entendait bien
la matière, et nous crûmes qu'il ne serait pas inutile.

Je parlai donc à M. le duc d'Orléans de ce mémoire et je lui proposai
d'en entendre la lecture en présence de ces trois hommes pour en
raisonner en même temps avec eux. Il me parut que la proposition lui
plut, il l'accepta avec plaisir, il voulut aussi que j'y assistasse, et
me donna jour au 2 août, trois ou quatre jours après\,; nous allâmes
donc ce jour-là de bonne heure l'après-dînée chez lui. Lecture ou
conférence durèrent quatre bonnes heures sans dispute et chacun ne
cherchant que les meilleurs moyens à lever les embarras et les
difficultés. La conclusion fut louanges et remercîments du régent et
approbation du mémoire\,; mais il fut convenu de voir pendant un an les
difficultés et les succès de Renaud dans la généralité de la Rochelle,
et de Silly dans une des élections\footnote{Les élections étaient des
  circonscriptions territoriales de l'ancienne monarchie, soumises, pour
  la juridiction financière, au tribunal des magistrats appelés
  \emph{élus}. Ceux-ci connaissaient en première instance de l'assiette
  des tailles et des aides, ou impôts prélevés sur les personnes, les
  propriétés et les denrées.} de Normandie, où ils travaillaient à
établir la taille proportionnelle, pour ensuite revoir avec eux ce même
mémoire, et sur l'expérience de leur travail et les lumières que donnait
le mémoire, se déterminer, se fixer et travailler en conséquence dans
tout le royaume sur la manière de lever la taille.

Ce projet, qui fut de l'avis de tous, et qui était sage, n'eut pas le
temps d'être exécuté. Renaud, malade de fatigue et du chagrin que lui
causaient les obstacles qu'il rencontrait dans la généralité de la
Rochelle, et de la haine que, sans savoir pourquoi, la nouveauté qu'il
voulait introduire avait excitée contre lui, malgré la netteté de ses
mains très reconnue, parce que toute nouveauté est suspecte en matière
d'impôts et de levée, Renaud, dis-je, voulut se presser de retourner à
son travail. Il voulut prendre des eaux de Pougues\,; il en prit par
excès, car par principe, comme le père Malebranche, il était grand
buveur d'eau, et mourut à Pougues les derniers jours de septembre. M.
d'Allemans, retourné chez lui, ne le survécut que de peu de mois\,;
ainsi tout ce projet s'en alla en fumée.

M. le duc d'Orléans fit au roi une galanterie très convenable à son âge,
ce fut de lui proposer de prendre la maison de la Muette pour s'en
amuser, et y aller faire des collations. Le roi en fut ravi. Il crut
avoir quelque chose personnellement à lui, et se fit un plaisir d'y
aller, d'en avoir du pain, du lait, des fruits, des légumes, et de s'y
amuser de ce qui divertit à cet âge. Ce lieu changeant de maître changea
aussi de gouverneur. Le duc d'Humières me parla pour Pezé\,; je le lui
fis donner, et il en sut tirer parti pour se rendre de plus en plus
agréable au roi. Il eut aussi la capitainerie du bois de Boulogne, comme
Rion avait l'un et l'autre.

M. le Duc, qui avait un procès fort aigre avec M\textsuperscript{me} la
princesse de Conti sa tante, l'accommoda\,; mais ce fut aux dépens du
roi à qui il en coûta une pension de vingt mille livres à
M\textsuperscript{me} la princesse de Conti, outre celles qu'elle avait
déjà. M. le duc d'Orléans accorda aussi à Lautrec cent cinquante mille
livres de brevet de retenue sur sa lieutenance générale de Guyenne. Il
profita aussi du bon état de la banque de Law pour faire payer toutes
les pensions, vieux et courant. Il fit aussi une grande augmentation de
troupes pour environ sept à huit millions.

Peu de jours après, il fit un marché qui scandalisa étrangement, après
tout ce qui s'était passé à Turin de La Feuillade à lui, et les
exécrables propos que ce dernier s'était piqué de tenir à tous venants
sur la mort de M. le Dauphin et de M\textsuperscript{me} la Dauphine.
Ils furent tels et si publics et si connus, que j'eus toutes les peines
du monde à empêcher M. le duc d'Orléans de lui faire donner des coups de
bâton, lui, si insensible à tout ce qui s'est fait et dit contre lui,
comme on le voit en tant d'endroits de ces Mémoires. Mais Canillac, ami
intime de La Feuillade de tout temps, voulut faire éclater son crédit et
la puissance de sa protection aux dépens de M. le duc d'Orléans même,
raccommoder avec lui un homme si gratuitement et si démesurément
coupable envers lui, et lui ouvrir un large robinet d'argent. Il
persuada donc à M. le duc d'Orléans, qui ne songeait à rien moins,
d'acheter de La Feuillade, pour M. le duc de Chartres, le gouvernement
de Dauphiné cinq cent cinquante mille livres comptant, trois cent mille
livres en outre pour le brevet de retenue que La Feuillade avait, et de
plus les appointements d'ambassadeur à Rome depuis le jour que le même
Canillac l'avait fait nommer, en obtenant son pardon jusqu'à son départ.
Ce fut donc près d'un million pour un gouvernement de soixante mille
livres de rente, et dix ans d'appointements d'ambassadeur à Rome où il
n'alla jamais. On verra, dans la suite, la rare reconnaissance de ce
galant homme, le plus corrompu et le plus méprisable que j'aie jamais
connu. Clermont qui, comme on l'a dit, avait les Suisses de M. le duc
d'Orléans, fut aussi capitaine des gardes de M. le duc de Chartres,
comme gouverneur de Dauphiné\,: il n'avait rien et grand besoin de
subsistance.

L'audience ordinaire du roi à la députation des états de Languedoc donna
lieu à une étrange dispute à qui les présenterait, par l'absence du duc
du Maine et du prince de Dombes, gouverneurs de cette province, entre
Maillebois qui en était un des lieutenants généraux, et La Vrillière,
secrétaire d'État, qui avait le Languedoc dans son département, qui,
plus étrangement encore, l'emporta. Voilà ce que perdent les charges à
tomber à des gens infimes. On n'a jamais contesté au lieutenant général
d'une province d'y faire les fonctions de gouverneur en son absence,
quand le lieutenant général y est de l'agrément du roi. Or, c'en est une
constante de présenter au roi les députés des états en l'absence du
gouverneur, et qui n'a pas besoin de l'agrément du roi, parce que cette
fonction est très passagère, et n'emporte ni détail ni commandement.
Toutefois La Vrillière osa la prétendre, et l'emporta parce qu'il n'eut
affaire qu'à Maillebois, et de là en avant, voilà cette fonction ôtée
aux lieutenants généraux par les secrétaires d'État, dans un pays où
rien de suivi par règle, par principes, par maximes, tout par exemple et
par considération.

À ce propos, puisque dans la suite ce Maillebois a voulu faire du
seigneur, si faut-il que je dise au vrai d'où il vient. Desmarets était
laboureur de l'abbaye d'Orcamp, comme l'avait été son père. Peu à peu il
en prit des ferres et s'y enrichit. M. Colbert, fort petit compagnon
alors, mais déjà dans les bureaux, n'avait pas encore oublié Reims, sa
patrie ni ses environs. Il sut que ces Desmarets, père et fils, étaient
devenus de gros marchands de blés, et qu'ils y avaient fait fortune. Il
trouva le nid bon pour sa soeur, et la leur fit proposer pour le fils.
Les Desmarets ne se firent pas prier pour s'allier à un homme qui
travaillait dans les bureaux du premier ministre, et le mariage se fit.
Colbert, de degré en degré, parvenu à la place d'intendant des affaires
du cardinal Mazarin et d'intendant des finances, voulut recrépir son
beau-frère. Il lui fit acheter une charge de trésorier de
France\footnote{Les trésoriers de France étaient des officiers de
  finance chargés principalement de l'administration des domaines
  royaux. Ils formaient des bureaux de finance qui siégeaient à Alençon,
  Amiens, Bordeaux, Bourges, Grenoble, la Rochelle, Limoges, Lyon,
  Montauban, Moulins, Orléans, Paris, Poitiers, Reims, Rouen, Soissons
  et Tours.} à Soissons, où il alla s'établir, sans avoir jamais monté
plus haut, et ne laissa pas tout doucement de continuer son commerce et
d'accumuler. Il eut trois fils de la soeur de Colbert, dont l'aîné fut
Desmarets dont il a été suffisamment parlé en plusieurs endroits ici
pour n'avoir rien de plus à en dire, et qui, à la mort du roi, était
ministre d'État et contrôleur général des finances, lequel, d'une fille
de Bechameil, surintendant de Monsieur, a eu Maillebois, qui a donné
lieu à ce récit.

Le même, mot pour mot, m'a été fait dans l'abbaye d'Orcamp par le prieur
et par ses principaux religieux, et m'a été confirmé unanimement par
tout le pays. Ce qu'ils ne m'ont pas dit, et ce que j'ai appris de tout
leur voisinage, mérite de n'être pas oublié, pour la beauté et encore
plus pour l'extrême rareté de l'action. Il y avait trente ans, lorsque
je l'appris, que le prieur et les principaux religieux de l'abbaye
d'Orcamp surent que deux enfants gentilshommes, dont les ascendants
paternels avaient fait de grands biens à leur abbaye et l'avaient
presque fondée, étaient tombés dans la nécessité. Ils les prirent chez
eux, les élevèrent, et leur firent apprendre tout ce qui convenait à
leur état\,; ensuite ils trouvèrent moyen de les faire officiers, leur
achetèrent après des compagnies, et tous les hivers défrayaient leurs
équipages chez eux\,; enfin au printemps leur faisaient une bourse pour
leur campagne, et ont toujours continué tant que ces gentilshommes ont
eu besoin et ont bien voulu recevoir ce secours. Aussi ces moines, tout
riches qu'ils sont, en ont recueilli la vénération de tout leur pays\,:
ils la méritent sans doute et d'être proposés en exemple. J'ai regret
d'avoir oublié le nom de ces gentilshommes, qui doivent être d'ancienne
race. Orcamp est si près de Paris que ce nom est aisé à retrouver.

Avant de quitter Maillebois et la députation des états de Languedoc, il
ne faut pas oublier cette singularité. Cette députation, après avoir
fait sa harangue au roi, allait toujours en faire une à Madame, et à M.
{[}le duc{]} et M\textsuperscript{me} la duchesse d'Orléans, ainsi que
les députés des états de Bretagne. Cela se pratiquait de même sous le
feu roi. M\textsuperscript{me} la duchesse d'Orléans ne voulut point la
recevoir cette année, pour marquer le deuil qu'elle demenait\footnote{Le
  mot \emph{demenait} est pris ici dans le sens de \emph{affectait de
  mener}.} de la situation du duc du Maine, quoique si étrangement
adoucie, d'une manière plus solennelle et plus publique.

Peu de jours après, le duc de Richelieu sortit de la Bastille et alla
coucher à Conflans chez le cardinal de Noailles. Il était veuf sans
enfants de sa nièce, mais, par son traité avec l'Espagne, il avait voulu
dépouiller le duc de Guiche, autre neveu du cardinal de Noailles, du
régiment des gardes, et l'avoir. Il devait s'en aller à Richelieu\,; il
obtint d'aller faire une pause à Saint-Germain, où il avait une maison,
puis d'y demeurer, après d'être à Paris sans voir le roi ni le
régent\,;au bout de trois mois il eut permission de les saluer, et tout
fut bientôt oublié.

\hypertarget{chapitre-xii.}{%
\chapter{CHAPITRE XII.}\label{chapitre-xii.}}

1719

~

{\textsc{Paix de la Suède avec l'Angleterre.}} {\textsc{- Le duc de
Lorraine échoue pour l'érection de Nancy en évêché.}} {\textsc{-
Vaudemont en tombe fort malade à Paris.}} {\textsc{- Maximes absurdes,
mais suivies toujours et inhérentes, du parlement sur son autorité.}}
{\textsc{- J'empêche le régent d'en rembourser toutes les charges avec
le papier de Law.}} {\textsc{- Raisons secrètes contre le remboursement
des charges du parlement.}} {\textsc{- Seconde tentative du projet du
remboursement des charges du parlement finalement avortée.}} {\textsc{-
Le parlement informé du risque qu'il a couru, qui le lui a paré, et qui
y a poussé.}} {\textsc{- Duchesse du Maine à Chamlay, où
M\textsuperscript{me} la Princesse la visite.}} {\textsc{- Officiers du
sang, et leur date.}} {\textsc{- Usurpations et richesses.}} {\textsc{-
Le chevalier de Vendôme vend au bâtard reconnu de M. le duc d'Orléans le
grand prieuré de France, et veut inutilement se marier.}} {\textsc{-
Retour de Plénoeuf en France.}} {\textsc{- Raisons d'en parler.}}
{\textsc{- Plénoeuf, sa femme et sa fille\,; quels.}} {\textsc{- Courte
reprise de sa négociation de Turin avortée par l'intérêt personnel et la
ruse singulière de l'abbé Dubois.}} {\textsc{- Étrange trait de
franchise de Madame, qui rompt tout court la négociation de Turin.}}
{\textsc{- Digression sur les maisons d'Este et Farnèse.}} {\textsc{-
Maison d'Este.}} {\textsc{- Bâtards d'Este, ducs de Modène et de Reggio
jusqu'à aujourd'hui.}} {\textsc{- Maison Farnèse.}} {\textsc{- Farnèse
bâtards, duc de Parme et de Plaisance.}}

~

Enfin l'alliance du nord se démancha. Le roi de Suède n'était plus, et
la faiblesse où son règne avait réduit ce royaume contribua beaucoup à
la paix qu'il conclut enfin avec le roi d'Angleterre. Le czar, déjà
adouci par la même raison, même du temps dernier de Charles XII, était
plus occupé du dedans que du dehors\,; le roi de Danemark demeura seul,
faisant la guerre en Norvège. C'est grand dommage que les Mémoires de M.
de Torcy ne soient pas venus jusqu'à ce temps-ci, et que le joug de
l'abbé Dubois n'ait pas laissé la liberté à M. le duc d'Orléans de me
parler aussi librement, qu'il avait accoutumé de l'intérieur, des
affaires étrangères\,: c'est ce qui m'y rendra sec désormais, parce que
je ne veux dire que ce que je sais par moi-même ou par des gens assez
instruits pour que je puisse m'y fier, et les citer pour garants.

Le roi d'Espagne, qui s'était approché de son armée, et qui même l'était
venu voir, s'en retourna à Madrid. Le prince Pio, qui la commandait, ne
se trouva pas en état de s'opposer à rien. Il se contenta de bien faire
rompre autour de l'abbaye de Roncevaux les chemins qu'on y avait faits à
grand'peine pour le canon et les autres voitures, dans un temps où on
n'imaginait pas qu'il pût jamais arriver de rupture avec Philippe V.

On vit au conseil de régence tous les ressorts que le duc de Lorraine
remuait pour obtenir l'érection d'un évêché à Nancy. Cet objet avait été
celui de ses pères et le sien pour se tirer du spirituel de l'évêché de
Toul, à quoi, par la raison contraire, la France s'était toujours
opposée. Il était temps d'arrêter les menées là-dessus. Le pape, qui
tremblait toujours devant l'empereur, le lui avait comme accordé. Il
espérait brusquer l'affaire avant que la France intervînt. Je ne sais si
M. le duc d'Orléans, abandonné ou plutôt entraîné comme il l'était à
tout ce qui convenait au duc de Lorraine par Madame, par
M\textsuperscript{me} la duchesse de Lorraine et par d'autres gens, en
aurait été bien fâché. J'ai soupçonné que l'affaire n'avait pu être
conduite si près du but sans qu'il en eût su quelque chose, et qu'il
l'avait voulu ignorer ou négliger. Mais enfin l'abbé Dubois, qui n'avait
rien personnellement à y gagner, ne crut pas devoir salir son ministère
d'une tolérance si préjudiciable et qui ferait crier contre lui, de
sorte qu'il y fit former à Rome une opposition solennelle et parler si
ferme au pape et au duc de Lorraine qu'il abandonna ses poursuites.
Ainsi le voyage précipité de Commercy ici, où M. de Vaudemont venait
d'arriver, fut inutile\,; deux jours après il tomba malade à
l'extrémité. Le dépit du peu de succès de sa conversation avec le régent
le piqua. Il n'avait pas l'habitude d'être contredit. Il n'avait pas
compté avoir grand'peine à tirer le consentement, au moins tacite, à une
chose si avancée et que le duc de Lorraine désirait si ardemment. Il y
fut trompé et ne fut plaint que de ses chères nièces, aussi dépitées que
lui, et de ses complaisants, dont quelques-uns encore étaient ou se
réputaient du plus haut parage.

Le parlement, comme on l'a déjà dit, plus irrité du lit de justice des
Tuileries, qu'abattu, était revenu du premier étourdissement. Après
quelque temps d'inaction et de crainte il ne trouva dans la conduite du
régent à l'égard du duc du Maine, que de quoi se rassurer. Il ne
s'appliqua donc plus qu'à éluder tout ce qui le regardait dans les
enregistrements que le roi avait fait faire en sa présence. Cette
compagnie est très conséquente pour ses intérêts\,: elle se prétend,
quoique très absurdement, la modératrice de l'autorité des rois mineurs,
même majeurs. Quoique si souvent battue sur ce grand point, elle n'a
garde de l'abandonner. De cette maxime factice, elle en tire une autre
sur les enregistrements\,; elle ne les prend point comme une publication
qui oblige parce qu'elle ne peut être ignorée\,; elle n'en regarde point
la nécessité comme étant celle de la notoriété, de laquelle résulte
l'obéissance à des lois qu'on ne peut plus ignorer\,; mais elle prétend
que l'enregistrement est en genre de lois, d'ordonnances, de levées,
etc., l'ajoutement d'une autorité nécessaire et supérieure à l'autorité
qui peut faire les lois, les ordonnances, etc., mais qui, en les
faisant, ne peut les faire valoir ni les faire exécuter sans le concours
de la première autorité, qui est celle que le parlement ajoute par son
enregistrement à l'autorité du roi, laquelle par son concours rend
celle-ci exécutrice, sans laquelle l'autorité du roi ne la serait pas.
De cette dernière maxime suit, dans les mêmes principes, que tout effet
d'autorité nécessaire, mais forcée, est nul de droit\,; par conséquent
que tout ce que le roi porte au parlement et y fait enregistrer par
crainte et par force, est vainement enregistré, est nul de soi et sans
force\,: enfin qu'il n'y a d'enregistrement valable et donnant aux
édits, déclarations, règlements, lois, levées, etc., l'ajoutement
nécessaire à l'autorité du roi qui les a faits, l'autorité qui les passe
en loi et qui les rende exécutoires, que l'enregistrement libre, et
qu'il n'est libre qu'autant que ce qui se porte au parlement pour y être
enregistré y soit communiqué, examiné et approuvé\,; ou que, porté
directement par le roi au lit de justice, y est, non pas approuvé du
bonnet, parce que nul n'ose parler, mais discuté en pleine liberté pour
être admis ou rejeté.

Dans cet esprit, il était très naturel et parfaitement conséquent que
non seulement le parlement ne se crût pas tenu d'observer rien de tout
ce qui avait été enregistré au lit de justice des Tuileries malgré lui
et contre ses prétentions, mais encore qu'il se crût en droit d'agir
d'une manière tout opposée à la teneur de ce qui y avait été ainsi
enregistré. C'est aussi ce que le parlement fit pas à pas, avec toute la
suite et la fermeté possible, et toute la circonspection aussi qui pût
assurer l'effet de son intention, en s'opposant à tous les
enregistrements nécessaires aux diverses opérations de Law, et vainement
tentées sous toutes les formes.

M. le duc d'Orléans était exactement informé et très peiné de cette
conduite, et Law infiniment embarrassé\,; il avait bien des manèges et
des opérations à faire qui demandaient un parlement soumis, et il avait
affaire à un régent qui n'aimait pas les tours de force, et qui semblait
épuisé sur ce point par ceux où il avait été contraint d'avoir recours.
Dans cette perplexité Law imagina de trancher ce noeud gordien. Il se
trouvait au plus haut point de son papier\,: le feu du François y
était\,; il n'y avait que peu de gens, en comparaison du grand nombre,
qui préférassent l'argent à ce papier. Il proposa donc à M. le duc
d'Orléans de rembourser avec ce papier toutes les charges du parlement
de gré ou de force, de se parer à l'égard du public d'ôter la vénalité
des charges qui a tant fait crier autrefois, et qui nécessairement
entraîne de si grands abus\,; de les remettre toutes en la main du roi
pour n'en plus disposer que gratuitement, comme avant que les charges
fussent vénales, et le rendre ainsi maître du parlement, par de simples
commissions qu'il donnerait, pour le tenir d'une vacance à l'autre, et
qui seraient ou continuées ou changées à chaque tenue du parlement, en
faveur dés mêmes, ou d'autres sujets, selon son bon plaisir.

Un spécieux si avantageux, et sans bourse délier, éblouit le régent. Le
duc de La Force appuya cette idée de concert avec l'abbé Dubois qui n'y
voulait pas trop paraître, mais qui faisait agir, et qui, dans la
crainte des revers et dans la connaissance qu'il avait et du parlement
et de son maître, se tenait derrière la tapisserie d'où il dirigeait ses
émissaires. Lui-même trouvait son compte à ce remboursement, dans ses
vues de se rendre maître absolu du gouvernement sous le nom du régent,
et tout de suite après sous le nom du roi majeur\,; mais il sentait tous
les hasards de la transition, et ne voulait pas se commettre.

Law, qui, comme je l'ai déjà dit, venait chez moi tous les mardis matin,
ne m'avait pas ouvert la bouche de rien qui pût me faire sentir ce
projet\,; j'ai lieu de croire, sans pourtant rien d'évident, qu'ils
n'osèrent se hasarder à un examen de ma part, et qu'ils voulurent
surprendre ce qu'ils imaginaient de mon goût, de ma haine, de mon
intérêt par la proposition que m'en ferait M. le duc d'Orléans, et
m'engager ainsi à l'improviste à une approbation qui se tournerait
incontinent en impulsion. C'est ce qui m'a toujours fait pencher à
croire que ce fut de cet artifice que vint à M. le duc d'Orléans la
volonté de me consulter là-dessus. Ils me connaissaient tous pour être
un des hommes du monde qui portait le plus impatiemment les prétentions
et les entreprises sur l'autorité royale, et qui, par attachement à ma
dignité, demeurait le plus ouvertement et le plus publiquement ulcéré de
toutes les usurpations que cette compagnie lui avait faites, et de tout
ce qui s'était passé en dernier lieu sur le bonnet dans les fins du feu
roi et depuis sa mort. C'était aussi par là que M. le duc d'Orléans,
dont les soupçons n'épargnaient pas les plus honnêtes gens ni ses plus
éprouvés serviteurs, avait regardé de cet oeil tout ce que je lui avais
dit dans les commencements des entreprises du parlement sur son
autorité, et pourquoi j'étais demeuré depuis à cet égard dans un silence
entier et opiniâtre avec lui, et qui n'avait été que forcément rompu de
ma part, quand il me parla du lit de justice peu de jours avant qu'il
fût tenu aux Tuileries, comme il a été rapporté en son lieu. Les mêmes
raisons, les mêmes soupçons, le même naturel de M. le duc d'Orléans le
devaient éloigner de me parler du remboursement du parlement, s'il n'y
avait été poussé d'ailleurs. Mais si j'étais celui contre lequel, à son
sens, il devait être le plus en garde là-dessus, c'était, à ce qu'il
pouvait sembler aux intéressés, un coup de partie d'engager M. le duc
d'Orléans à consulter un homme qu'ils comptaient être si fait exprès
pour seconder leurs désirs, et qui rassemblait en soi tout ce qu'il
fallait pour les faire réussir pleinement et avec promptitude.

Quoi qu'il en fût, une après-dînée que je travaillais à mon ordinaire
tête à tête avec M. le duc d'Orléans, il se mit avec moi sur le
parlement sans que rien n'y eût donné lieu, et à me conter et à
m'expliquer les entraves que cette compagnie lui donnait sans cesse, le
peu de compte qu'elle faisait publiquement du lit de justice des
Tuileries, le peu de fruit qu'il en tirait, puis tout de suite me
proposa l'expédient qu'on lui avait trouvé, et en même temps tira de sa
poche un mémoire bien raisonné du projet, dont jusqu'à ce moment il ne
m'était pas revenu la moindre chose. J'entrai fort dans ses plaintes de
la conduite du parlement, et dans les raisons de le ranger à son devoir
à l'égard de l'autorité royale. Je n'oubliai pas d'alléguer les causes
personnelles de mon désir de le voir mortifier et remis dans les bornes
où il devait être, et les avantages que ma dignité ne pouvait manquer de
trouver dans l'exécution de ce projet\,; mais j'ajoutai tout de suite
que de première vue il me paraissait d'un côté bien injuste, et de
l'autre bien hardi, et que ce n'était pas là matière à prendre une
résolution sans beaucoup de mûre délibération, et sans en avoir bien
reconnu et pesé toutes les grandes suites et l'importance très étendue.
Il ne m'en laissa pas dire davantage, et voulut lire le mémoire d'abord
de suite et sans interruption, malgré sa mauvaise vue, puis une seconde
fois en s'arrêtant et raisonnant dessus.

Cette lecture première me confirma dans l'éloignement que j'avais conçu
du projet dès sa première proposition, et que je n'avais pu tout à fait
cacher. Quand ce fut à la seconde lecture je raisonnai, et mes
raisonnements allaient toujours à la réfutation. M. le duc d'Orléans,
surpris au dernier point de m'y trouver contraire, mais déjà entraîné et
enchanté du projet, ne fut pas content de ma résistance. Il me témoigna
l'un et l'autre\,; il n'oublia rien pour me piquer, et me ramener par
l'intérêt de ma dignité, me dit qu'il fallait donc laisser le parlement
le maître, ou en venir à bout par l'unique moyen qu'on en avait, puis se
répandit sur l'odieux et les inconvénients infinis de la vénalité dès
charges\footnote{Voy. notes à la fin du volume.}, sur le bonheur public
que ce changement apporterait, et sur les acclamations qu'on en devait
attendre.

Le voyant si prévenu, et reployer le mémoire pour le remettre dans sa
poche, je sentis tout le danger où on l'allait embarquer. Je lui dis
donc qu'encore qu'il y eût déjà fort longtemps que nous en étions
là-dessus, cette matière était pour ou contre trop importante pour
n'être pas examinée plus mûrement\,; que j'avais dit ce qui s'était
présenté d'abord à mon esprit\,; qu'en y pensant davantage, et faisant
tout seul plus de réflexion sur ce mémoire, et avec plus de loisir,
peut-être que je changerais d'avis\,; que je le souhaitais passionnément
pour lui complaire, pour l'intérêt de ma dignité, pour l'extrême plaisir
de ma vengeance personnelle, mais qu'il ne devait pas avoir oublié aussi
ce que je lui avais protesté en plus d'une occasion, et qu'il m'avait vu
pratiquer si fermement et si opiniâtrement, quoique presque si
inutilement sur celle du changement de main de l'éducation du roi, et
sur la réduction des bâtards au rang et ancienneté de leurs pairies\,;
que je le lui répétais en celle-ci, que j'aimais incomparablement mieux
ma dignité que ma fortune, mais que l'une et l'autre ne me seraient
jamais rien en comparaison de l'État. Je le priai ensuite que je pusse
emporter le mémoire pour le mieux considérer tout à mon aise. Il y
consentit à condition qu'il ne serait vu que de moi seul. Il me le
donna, mais avec promesse de le lui rapporter le surlendemain, sans
m'avoir jamais voulu accorder un plus long terme.

Je tins parole et plus, car je fis de ma main une réponse si péremptoire
que je lus à M. le duc d'Orléans, qu'il demeura convaincu que le projet
était la chimère du monde la plus dangereuse. Cette réponse, je l'ai
encore\,; elle se trouvera parmi les Pièces. En effet, il ne fut plus
parlé du projet. Ceux qui l'avaient fait et conseillé trouvèrent M. le
duc d'Orléans si armé contre leurs raisons, qu'ils n'y trouvèrent point
de réplique, et qu'ils se continrent dans le silence\,; mais ce ne fut
pas pour toujours.

Outre les raisons contre ce remboursement, expliquées dans le mémoire
qui persuada alors M. le duc d'Orléans, trop long pour être inséré ici,
mais qu'il faut voir dans les Pièces, j'en eus deux autres non moins
puissantes, non moins inhérentes à l'intérêt de l'État, mais qui
n'étaient pas de nature à mettre dans mon mémoire\,: la première est
que, quelque fausses et absurdes que soient les maximes du parlement qui
viennent d'être expliquées, et quelque abus énorme et séditieux qu'il en
ait fait trop souvent, surtout dans la minorité du feu roi, il ne
fallait pas oublier le service si essentiel qu'il rendit dans le temps
de la Ligue, ni se priver d'un pareil secours dans les temps qui
pouvaient revenir, puisqu'on les avait déjà éprouvés, en même temps ne
pas ôter toute entrave aux excès de la puissance royale tyranniquement
exercée quelquefois sous des rois faibles, par des ministres, des
favoris, des maîtresses, des valets même, pour leurs intérêts
particuliers contre celui de l'État, de tous les particuliers, de ceux
d'un roi même qui les autoriserait à tout faire et à employer son nom
sacré et son autorité entière à la ruine de son État, de ses sujets et
de sa réputation. Mon autre raison fut l'importance d'opposer l'unique
barrière que l'État pût avoir contre les entreprises de Rome, du clergé
de France, d'un régulier\footnote{C'est-à-dire ecclésiastique soumis à
  une règle monastique.} impétueux qui gouvernerait la conscience d'un
roi ignorant, faible, timide, ou qui n'étant d'ailleurs ni timide ni
faible, le serait par la grossièreté d'une conscience délicate et
ténébreuse sur toutes les matières ecclésiastiques, ou qu'on lui
donnerait pour l'être. Il n'y a qu'à ouvrir les histoires de tous les
pays et du nôtre en particulier, pour voir la solidité de ces raisons.
Celles de mon mémoire ne me parurent ni moins fortes ni moins solides,
mais celles-ci qui ne s'y pouvaient mettre, me semblèrent encore plus
importantes.

Tandis que je suis sur cette matière, je suis d'avis de l'achever pour
n'avoir pas à y revenir sur l'année prochaine, où il n'y aurait qu'un
mot à en dire. Ce projet était trop cher à Law et à l'abbé Dubois pour
l'abandonner\,: à Dubois pour s'ôter toutes sortes d'obstacles présents
et à venir pour l'établissement et la conservation de sa
toute-puissance\,; à Law pour son propre soutien par ce prodigieux
débouchement de papier dont il sentait de loin tout le poids en quelque
vogue qu'il fût alors. On verra sur l'année prochaine, qu'elle se passa
en luttes entre le gouvernement et le parlement. Ces luttes donnèrent
lieu aux promoteurs du projet abandonné de tâcher de le ressusciter,
sans qu'en aucun temps ni l'un ni l'autre m'en ait parlé, sinon une fois
ou deux quelques regrets échappés courtement à Law d'un si bon coup
manqué.

J'étais allé, dans l'été, passer quelques jours à la Ferté, dans un
intervalle d'affaires et du conseil de régence. Peut-être que mon
absence leur fit naître l'espérance de le brusquer. Le lendemain de mon
arrivée, j'allai faire ma cour à M. le duc d'Orléans, comme je faisais à
tous mes retours. Je le trouvai avec assez de monde. Après quelques
moments de conversation générale, M. le duc d'Orléans me tira à part
dans un coin\,; il me dit qu'il avait bien à m'entretenir de choses
instantes et pressées, et que ce serait pour le lendemain. Je le pressai
de m'en dire la matière\,; il eut quelque peine à s'expliquer, puis me
dit qu'il était excédé du parlement, qu'il fallait reprendre le projet
du remboursement et voir enfin aux moyens de l'exécuter. Je lui
témoignai toute ma surprise de le voir revenir encore une fois à un
expédient si ruineux, et de l'abandon duquel il était demeuré si
pleinement convaincu. Le régent insista, mais coupa court, et me donna
son heure pour le lendemain\,; je lui dis que j'étais tout prêt, mais
que je n'avais rien de nouveau à lui exposer sur cette matière, et que
je serais surpris si on lui en proposait quelque solution praticable. La
nuit suivante, la fièvre me prit assez forte\,; je m'envoyai donc
excuser d'aller au Palais-Royal. Le jour d'après, M. le duc d'Orléans
envoya savoir de mes nouvelles, et quand je pourrais le voir. Ce fut une
fièvre double-tierce, qui impatienta d'autant plus les promoteurs du
projet qu'apparemment ils trouvèrent le régent arrêté à n'y avancer pas
sans moi, car deux jours après, le duc de La Force vint forcer ma porte
de la part de M. le duc d'Orléans. Il me trouva au lit, dans l'accès, et
hors d'état de raisonner sur la mission qui l'amenait, et qu'il me dit
être le projet du remboursement du parlement. Il me demanda avec
empressement quand il en pourrait conférer avec moi, parce que l'affaire
pressait. Je sus après que c'était la première fois que M. le duc
d'Orléans lui en avait parlé. Je répondis au duc de La Force que je ne
prévoyais pas être sitôt en état de raisonner, ni d'aller au
Palais-Royal, mais que si l'affaire pressait tant, que j'avais tellement
dit à M. le duc d'Orléans, il y avait plus d'un an, tout ce que je
pouvais lui en dire, que je n'avais plus rien à y ajouter\,; que tout ce
que je pouvais faire, c'était de lui prêter à lire un mémoire que
j'avais fait là-dessus et que par hasard j'avais gardé. En effet, je le
lui envoyai l'après-dînée du même jour. Apparemment qu'ils le trouvèrent
péremptoire, car le duc de La Force me le rapporta quelques jours après.
Je n'étais pas lors encore trop en état de parler d'affaires, et moins
en volonté d'entrer sur celle-là en matière avec lui, aussi n'y
insista-t-il pas, et se contenta d'avouer en général que le mémoire
était bon. Ils n'y purent apparemment rien répondre, parce que la
première fois ensuite que je vis M. le duc d'Orléans, il me dit d'abord
qu'il n'y avait pas moyen de songer davantage à ce projet, et en effet
il n'en fut plus du tout parlé depuis.

Ce qui ne peut se comprendre, et qui pourtant est arrivé quelquefois
dans la régence, c'est que tout cela fut su en ce même détail par le
premier président avec qui j'étais demeuré en rupture plus qu'ouverte,
sans le saluer, et quelquefois pis encore, depuis l'affaire du bonnet,
dès avant la mort du roi. Peu après ceci, le parlement, comme on le
verra en son lieu, fut envoyé à Pontoise. Le premier président, en y
allant avec sa famille, dit en carrosse à M\textsuperscript{me} de
Fontenelle, sa sueur, le risque que le parlement avait couru, et lui
donna à deviner qui l'avait sauvé, dont il ne sortait pas de surprise,
et me nomma. Sa soeur n'en fut pas moins étonnée\,; elle-même me l'a
raconté après que nous fûmes raccommodés. Ils surent aussi la part
contradictoire que le duc de La Force y avait eue, et surent après s'en
venger cruellement. Pour moi, qui n'avais pas prétendu à leur
reconnaissance, je demeurai avec eux tel que j'étais auparavant, et eux
avec moi.

M\textsuperscript{me} la Princesse fut refusée du séjour d'Anet pour la
duchesse du Maine, où elle aurait voulu la faire venir et y passer
quelque temps avec elle. Mais peu après elle obtint le séjour du château
de Chamlay, près de Joigny, qui était à vendre depuis la mort de
Chamlay\,; et comme cette mort était récente, le lieu qu'il avait fort
accommodé était encore entretenu et meublé. M\textsuperscript{me} la
Princesse eut permission d'y aller voir M\textsuperscript{me} sa fille.

À propos de princes du sang, il faut réparer ici, bien ou mal à propos,
l'oubli d'une remarque qui aurait dû être placée lors de l'achat du
gouvernement du Dauphiné, et que Clermont-Chattes, capitaine des Suisses
de M. le duc d'Orléans, fut aussi capitaine des gardes de M. le duc de
Chartres, comme gouverneur du Dauphiné. Les princes du sang, comme tels,
n'ont ni gardes ni capitaines des gardes, mais quand ils sont
gouverneurs de province, ils ont en cette qualité des gardes, mais dans
leur province, et un capitaine des gardes comme en ont tous les autres
gouverneurs de province. Le seul premier prince du sang a un gentilhomme
de la chambre. Ils l'appellent maintenant premier gentilhomme de la
chambre et en ont tous un. La date de cette nouveauté, peu après
imperceptiblement introduite, est depuis la mort du roi, et n'a paru que
longtemps après. Qui voudrait expliquer leurs diverses usurpations en
tous genres depuis la mort du roi, et les millions qu'ils ont eus, et
les augmentations immenses en sus de pensions, ferait un volume.

Le chevalier de Vendôme, grand prieur de France, dont on a assez parlé
ailleurs pour le faire connaître, avait passé sa vie à se ruiner et à
manger tout ce qu'il avait pu d'ailleurs. Les biens du grand prieuré
étaient tombés dans le dernier désordre, et l'ordre de Malte avait à cet
égard une action toujours prête contre lui. Il avait tiré infiniment de
Law, et n'était pas d'avis d'en réparer ses bénéfices. Les
accroissements prodigieux et parfaitement inattendus qu'il avait vu
arriver à son rang par le feu roi, à cause de ses bâtards, et que son
impudence avait augmentés depuis par les tentatives hardies, que la
faiblesse, ou peut-être la prétendue politique de M. le duc d'Orléans,
avait souffertes, lui avaient tellement tourné la tête, que la chute de
ce rang arrivée au dernier lit de justice des Tuileries n'avait pu le
rappeler à la première moitié de sa vie, ni le détacher de la folle
espérance de revenir au rang de prince du sang. Il la combla par vouloir
avoir postérité, et ne put comprendre que cette postérité même serait un
obstacle de plus à ses désirs. Il s'abandonna donc à sa chimère, et Law,
son ami et son confident, en profita pour faire sa cour au régent, et
procurer au bâtard qu'il avait reconnu de M\textsuperscript{me}
d'Argenton le grand prieuré de France. Le marché en fut bientôt fait et
payé gros. Pas un de ceux qui y entrèrent de part et d'autre n'étaient
pas pour en avoir plus de scrupule que du marché d'une terre ou d'une
charge, et l'ordre de Malte, ni le grand maître, pour oser refuser un
régent de France. L'affaire se fit donc avec si peu de difficulté qu'on
la sut consommée avant d'en avoir eu la moindre idée. Il s'en trouva
davantage pour la dispense des voeux du chevalier de Vendôme, et pour
celle de se pouvoir marier\,; mais il l'obtint enfin par la protection
de M. le duc d'Orléans, et au moyen des sûretés qu'il donna à la maison
de Condé de ne répéter rien de la succession du feu duc de Vendôme, son
frère, qui par la donation entre vifs de son contrat de mariage avec la
dernière fille de feu M. le Prince, fondée sur la profession de cet
unique frère, était passée tout entière aux héritiers de la feue
duchesse de Vendôme, excepté ce qui se trouva réversible à la couronne.
Cela fait, il chercha partout à se marier, et partout personne ne voulut
d'un vieux ivrogne de soixante-quatre ou soixante-cinq ans, pourri de
vérole, vivant de rapines, sans autre fonds de bien que le portefeuille
qu'il s'était fait et dont tout le mérite ne consistait que dans son
extrême impudence\,; lui, au contraire se persuadait qu'il n'y avait
rien de trop bon pour lui. Il chercha donc en vain et si longtemps qu'il
se lassa enfin d'une recherche vaine et ridicule. 11 continua sa vie
accoutumée qu'il était incapable de quitter, qui l'obscurcit de plus en
plus, et qui ne dura que peu d'années depuis cette dernière scène de sa
vie.

Ce fut en ce temps-ci que Plénoeuf revint en France en pleine liberté,
après s'être accommodé avec ses créanciers à peu près comme il voulut.
Je ne barbouillerais pas ces Mémoires du nom et du retour de ce bas
financier sans les raisons curieuses qui s'en présenteront d'elles-mêmes
en cet article, et qui m'engageront même à une courte, mais nécessaire
répétition. Il était de la famille des Berthelot, tous gens d'affaires,
et frère de la femme du maréchal de Matignon. Il entra dans plusieurs
affaires, enfin dans les vivres et les hôpitaux des armées, où tant de
soldats périrent par son pillage, et où il amassa tant de trésors.
Embarrassé de tant de proie, il se mit à l'abri en se faisant connaître
à Voysin comme un homme consommé dans la science des vivres et des
fourrages, qui le fit un de ses premiers commis. Il ne s'oublia pas dans
cet emploi, et en profita dans le peu qu'il dura pour cacher si bien
tout ce qu'il avait amassé que lorsqu'il se vit recherché par la chambre
de justice, après la mort du roi, il fit une banqueroute frauduleuse et
prodigieuse, se sauva hors du royaume, et ne craignit point qu'on
trouvât ce qu'il avait caché. Ce fut d'au delà des Alpes qu'il plaida en
sûreté et mains garnies, et qu'il se servit sans qu'il lui en coûtât
rien, de ce qui corrompt tant de gens, de l'argent et de la beauté.

Sa femme en avait, des agréments encore plus, tout l'esprit, et la sorte
d'esprit de suite, d'insinuation et d'intrigue, qui est la plus propre
au grand monde, et à y régner autant que le pouvait une bourgeoise que
sa figure, son esprit, ses manières, ses richesses y avaient mêlée d'une
façon fort au-dessus de son état, et avec un empire qu'elle ne déployait
qu'avec discrétion, mais qu'elle eut toujours l'art de faire aimer à
ceux qu'elle avait entrepris d'y soumettre. Elle était mère de la trop
fameuse M\textsuperscript{me} de Prie, qui avait autant d'esprit et
d'ambition qu'elle, et plus de beauté. Elle enchaîna M. le Duc, le
gouverna entièrement, et pendant qu'il fut premier ministre fit des maux
infinis à la cour et à l'État, dont il se peut dire que les trésors
immenses qu'elle ramassa de toutes parts fut le moindre mal qu'elle fit,
si on excepte la pension d'Angleterre, pareille à celle qu'avait eue
l'abbé Dubois, et qui ne coûta guère moins cher au royaume. La rivalité
de beauté brouilla la mère et la fille, les rendit ennemies implacables,
et {[}elles{]} y entraînèrent leurs adorateurs. C'est ce qui mit Le
Blanc et Belle-Ile à une ligne de leur perte après une longue et dure
prison. On se contente d'en faire ici la remarque\,; le règne funeste et
cruel de M\textsuperscript{me} de Prie dépasse le temps de ces Mémoires,
qui ne doivent pas aller plus loin que la vie de M. le duc d'Orléans.

Plénoeuf, d'extérieur grossier, lourd, stupide, était le plus délié
matois, qui allait le mieux et le plus à ses fins, qui n'était retenu
par aucun scrupule et dont l'esprit financier était propre aussi aux
affaires et à l'intrigue. Ce dernier talent l'initia dans la cour de
Turin, et le mit en situation de mettre sur le tapis le mariage de
M\textsuperscript{lle} de Valois avec le prince de Piémont, sans en
avoir nulle charge. On a vu ailleurs ce qui se passa là-dessus, comme je
fus chargé malgré moi de la correspondance sur cette affaire avec
Plénœuf, comme sa femme s'insinua chez M\textsuperscript{me} la duchesse
d'Orléans et chez moi, sous prétexte de rendre elle-même les lettres de
son mari, et comme, l'affaire avortée, elle sut se maintenir toujours
auprès de M\textsuperscript{me} la duchesse d'Orléans et m'a toujours
cultivé depuis. On a vu aussi qu'alors l'abbé Dubois était auprès du roi
d'Angleterre, et que, dès qu'il fut arrivé, las de la correspondance
avec un homme tel que Plénœuf, et connaissant la jalousie de l'abbé
Dubois et la faiblesse de M. le duc d'Orléans pour lui, enfin qu'il
goûtait très médiocrement ce mariage, quoique très mal à propos, je lui
proposai de ne pas faire un pot à part de cette seule affaire étrangère,
et de trouver bon que je la remisse à l'abbé Dubois, pour ne m'en plus
mêler, ce que je fis en même temps, au grand regret de
M\textsuperscript{me} la duchesse d'Orléans, et dont
M\textsuperscript{me} de Plénoeuf fut aussi bien fâchée, mais à ma
grande satisfaction. Celle-ci bâtissait déjà beaucoup en espérance, si
son mari concluait ce mariage. M\textsuperscript{me} la duchesse
d'Orléans le désirait passionnément\,; elle était informée de tout par
moi, ce qu'elle n'espérait pas de l'abbé Dubois, et craignait tout de
lui, avec raison, pour le faire manquer. M\textsuperscript{me} de
Plénœuf, le voyant en de telles mains, le comptait déjà rompu et ses
espérances perdues.

En effet ce mariage n'était pas le compte personnel de l'abbé Dubois. Sa
boussole était sa fortune particulière, comme on l'a remarqué ici bien
des fois, et ses vues étaient trop avancées pour leur tourner le dos par
quelque considération que ce pût être. Il avait sacrifié l'Espagne, sa
marine et la nôtre à l'Angleterre\,; il ne restait plus qu'à sacrifier
la même Espagne et le roi de Sicile à l'empereur. Le sacrifice déjà fait
aux dépens de l'État et à ceux de son maître lui avait assuré les
offices de l'Angleterre les plus efficaces auprès de l'empereur, qui en
profitait, et qui alors était très intimement avec le roi Georges. Le
sacrifice qui restait à faire étant directement à l'empereur, le rendait
son obligé et le disposait personnellement à ce que le roi Georges lui
demandait, qui ne lui coûtait rien que de faire dire au pape, qui
tremblait devant lui et qui ne cherchait qu'à prévenir ses désirs, qu'il
voulait, et promptement, un chapeau pour l'abbé Dubois. Dans cette
position, l'abbé Dubois n'avait dans la tête que la quadruple alliance,
dont la Sicile devait être le premier fruit pour l'empereur, aux dépens
du roi de Sicile à qui était destiné, aux dépens encore de l'Espagne, le
triste dédommagement de la Sardaigne, pour lui conserver le titre et le
rang de roi. Dubois n'avait donc garde de vouloir le mariage à la veille
de le dépouiller. Il fit donc languir la négociation pour se préparer à
la rompre, la laissa transpirer exprès et revenir à Madame, sans y
paraître, parce qu'il en était méprisé et haï, mais dans l'espérance de
quelque trait de férocité allemande. Il la connaissait et il devina.

Madame était la droiture, la vérité, la franchise même, avec de grands
défauts, dont l'un était de pousser à l'extrême les vertus dont on vient
de parler. Aussi, dans cette occasion, n'en fit-elle pas à deux fois.
Elle aimait tellement à écrire à ses parents et à ses amis, comme on l'a
pu voir ici, par ce qui lui en arriva à la mort de Monsieur, qu'elle y
passait sa vie\footnote{Des extraits des lettres de la duchesse
  d'Orléans ont été publiés plusieurs fois. La dernière édition a été
  donnée par M. Brunet. (Paris, Charpentier, 2 vol. in-12, 1855.)}. La
reine de Sicile et elle s'écrivaient toutes les semaines. Madame lui
manda sans détour qu'elle apprenait qu'il était sérieusement question du
mariage du prince de Piémont avec M\textsuperscript{lle} de Valois\,;
qu'elle l'aimait trop pour lui vouloir un si mauvais présent et pour la
tromper\,; qu'elle l'avertissait donc\footnote{Madame fait mention de
  M\textsuperscript{lle} de Valois dans plusieurs lettres de 1719 et
  notamment dans les lettres du 13 mai, du 8 juin, 9 novembre, 30
  novembre, 3 décembre, 17 décembre. Dans les dernières lettres, Madame
  parle du mariage prochain de M\textsuperscript{lle} de Valois avec le
  fils aîné du duc de Modène\,; dans celle du 13 mai, il est question de
  ses intrigues avec le duc de Richelieu.}, etc.\,; et lui raconta tout
de suite tout ce qu'elle en savait, ou ce qu'elle en croyait savoir\,;
puis, la lettre partie et hors de portée de pouvoir être arrêtée et
prise, elle dit tout ce qu'elle contenait à M. {[}le duc{]} et à
M\textsuperscript{me} la duchesse d'Orléans, qui en fut outrée. M. le
duc d'Orléans, qui n'avait jamais été de bon pied en cette affaire, et
beaucoup moins depuis qu'elle avait été remise à l'abbé Dubois, ne lit
qu'en rire, et Dubois rit encore de bien meilleur coeur de ce rare et
subit effet de son artifice. Ce mariage tomba donc de la sorte.

Plénœuf en fut éconduit avec assez peu de ménagement\,; ses affaires en
France s'étaient accommodées\,; il se hâta de quitter Turin et revint
avec l'air de l'importance, le fruit et la sécurité de sa banqueroute.
Il n'en jouit pas longtemps et ne vécut pas longues années.

Six semaines après cette aventure, M. le duc d'Orléans, qui avait ses
raisons de se soucier peu de M\textsuperscript{lle} de Valois, et
beaucoup de s'en défaire, conclut et déclara son mariage avec le fils
aîné du duc de Modène. Personne malheureusement n'ignorait pourquoi le
régent se hâtait tant de se défaire de cette princesse et avec si peu de
choix. Je ne pus m'empêcher pourtant de le lui reprocher. «\,Pourquoi ne
mérite-t-elle pas mieux\,? me répondit-il\,: tout m'est bon, pourvu que
je m'en défasse.\,» Il n'y eut rien qui n'y parût\,: on lui donnait un
des plus petits princes d'Italie quant à la puissance et aux richesses,
qui avait à attendre longtemps à être souverain, et dont le père était
connu pour être d'un caractère et d'une humeur fort difficile, comme il
le leur montra bien tant qu'il vécut. Il est vrai que la reine d'Espagne
ri était pas de meilleure maison, et que Philippe V était fort au-dessus
de M\textsuperscript{lle} de Valois en bien des manières. Aussi on a vu
ici en son lieu de quelle façon ce mariage se fit, et que le feu roi ne
le pardonna pas à M\textsuperscript{me} des Ursins. Il n'est peut-être
pas inutile d'expliquer ici en peu de mots ce que sont les d'Este
d'aujourd'hui, et ce que sont aussi les Farnèse.

Je ne me donne pas pour être généalogiste, mais je suivrai Imhoff qui
passe pour exact et savant sur les maisons allemandes, espagnoles et
italiennes\footnote{Il a déjà été question, t. III, p.~249, des
  \emph{Recherches historiques et généalogiques des grands d'Espagne},
  par Imhoff. On a encore de lui une histoire généalogique de la maison
  royale de Portugal, sous le titre de \emph{Stemma regium Lusitanicum}
  (Amsterdam, 1708, in-fol.). Saint-Simon fait allusion, à la fin de sa
  phrase, à l'ouvrage du même auteur, intitulé \emph{Excellentium
  familiarum in Gallia genealogiae} (Nuremberg, 1687, in-fol.)}, et fort
peu l'un et l'autre sur les françaises. Peut-être que si nous
connaissions autant ces maisons étrangères que nous faisons celles de
notre pays, cet auteur n'aurait pas pris tant de réputation\,; mais ce
qui regarde l'origine des Farnèse et l'étrange déchet des Este
d'aujourd'hui est si moderne et si connu qu'il n'y a pas de méprise à
craindre.

Imhoff donne pour tige, dont la maison d'Este est sortie, Azon, seigneur
d'Este, marchis en Lombardie, c'est-à-dire général et gardien des
marches ou des frontières de ces pays, qui épousa en premières noces
Cunégonde, qui était Allemande et héritière de sa maison, (héritage
difficile à entendre dans une fille en Germanie à la fin du Xe siècle où
cela se passait)\,; et en secondes noces Ermengarde, fille du comte du
Maine en France. Du premier lit il eut Guelfe, héritier des biens de sa
mère. Il fut créé duc de Bavière en 1071, répudia sa première femme,
fille d'Otton le Saxon, duc de Bavière, épousa ensuite Judith, fille de
Baudouin le Pieux, comte de Flandre, mourut en 1101 dans l'île de
Chypre, laissa deux fils\,: Guelfe l'aîné, duc de Bavière, mort sans
postérité en 1119\,; et Henri, dit le Noir, duc de Bavière après son
frère. Il épousa Walflide, fille de Magnus, duc de Saxe, mourut 1123, et
laissa un fils nommé Henri comme lui, qui fut duc de Bavière et de Saxe.
Celui-ci épousa Gertrude, fille de l'empereur Lothaire II, et de ce
mariage est sortie la maison de Brunswick et Lunebourg, à ce qu'on
prétend.

Hugues, second fils d'Azon tige de cette maison, et fils de son second
lit, hérita des biens de sa mère, fut comte du Maine en France, et vécut
peu\,; il ne lui paraît point de postérité, et le comté du Maine
disparaît avec lui.

Son frère Foulques fut seigneur d'Este et marchis. Obizzo son fils eut
les mêmes titres, y ajouta en 1177 celui de podestat\footnote{Les
  podestats étaient des gouverneurs établis dans certaines villes
  d'Italie avec droit de haute justice.} de Pavie, et de Ferrare l'année
suivante. Il mourut en 1196. Son fils Azon II devint en 1196 marquis
d'Este et de Ferrare, en 1199 podestat de Padoue, en 1207 podestat de
Vérone, en 1208 marquis d'Ancône\,; il mourut en 1212. Son fils Obizzo
III devint premier marquis d'Este et de Ferrare, fut aussi seigneur de
Modène et de Parme. Il épousa Élisabeth, fille d'Albert duc de Saxe,
électeur. Nicolas, fils de son fils, ajouta à ces titres ceux de
seigneur de Reggio, Forli et Romandiole. Borsus son fils fut créé duc de
Modène et de Reggio par l'empereur Frédéric III, 18 mai 1452, et duc de
Ferrare par le pape Paul III (Farnèse), 14 avril 1470. Borsus ne se
maria point, et mourut en 1471. Hercule son frère lui succéda\,; il fut
gendre de Ferdinand d'Aragon, roi de Naples, et mourut en 1505.

Son fils Alphonsele lui succéda. Il épousa en premières noces Anne
Sforce, fille de Galéas Marie duc de Milan\,; en secondes noces Lucrèce
Borgia, fille du pape Alexandre VI. Il faut ici expliquer sa famille
avant d'aller plus loin. De trois frères qu'il eut, deux ne se marièrent
point, tous deux moururent longtemps avant lui, dont un des deux en
prison. L'autre frère fut évêque de Ferrare, archevêque de Strigonie, de
Milan, de Capoue, de Narbonne, fut cardinal en 1493, mourut en 1520. Cet
Alphonse Ier, frère aîné de ce cardinal, eut un fils de Laure Eustochie
degli Dianti, dont le père était un artisan de Ferrare. Il avait perdu
ses deux femmes longtemps avant sa mort. On a prétendu qu'il épousa
enfin cette maîtresse\,; mais il n'est pas contesté que le fils qu'il en
eut, et qui s'appela aussi Alphonse, ne soit né avant ce dernier
mariage, si tant est qu'il ait été fait. Le duc Alphonse Ier mourut en
1534 et laissa\,: Hercule II qui lui succéda\,; Hippolyte, élevé en
France, évêque de Ferrare, de Tréguier, d'Autun, de Saint-Jean de
Maurienne, archevêque de Strigonie, de Milan, de Capoue, de Narbonne,
d'Arles, de Lyon, cardinal en 1538, mort en décembre 1572, à
soixante-trois ans\,; un fils qui n'eut que deux filles\,; le bâtard
Alphonse susdit\,; un fils mort dès 1545 sans alliance\,; et une fille
religieuse.

Hercule II, fils aîné susdit d'Alphonse Ier, fut son successeur, duc de
Ferrare, de Modène et de Reggio. Il épousa, en 1527, Renée de France,
fille du roi Louis XII, et ce mariage fut peu concordant. Il mourut en
octobre 1558 à cinquante ans. Renée se retira en France, où elle mourut
en juin 1571 avec un grand apanage et une grande considération. Elle fut
la protectrice des savants\,; et quoique belle-mère du duc de Guise,
elle protégea aussi les huguenots. De ce mariage, deux fils et quatre
filles\,: Alphonse II, successeur de son père, Louis, évêque de Ferrare,
archevêque d'Auch, cardinal, 1561, mort à Rome 3 décembre 1586, chargé
des affaires de France, après son oncle Hippolyte, et toujours très
français et très opposé à la Ligue et aux Guise ses cousins germains.
Les filles, leurs soeurs, furent la trop célèbre Anne d'Este, duchesse
de Guise, née en 1531, mariée décembre 1549, veuve par l'assassinat de
Poltrot, février 1563\,; remariée, 1566, à Jacques de Savoie, duc de
Nemours, mère des duc et cardinal de Guise, tués, décembre 1588, aux
derniers états de Blois, du duc de Mayenne, de la duchesse de
Montpensier, etc., et du duc de Nemours, et du marquis de Saint-Sorlin,
duc de Nemours après son frère\,; elle mourut mai 1607, à
soixante-dix-sept ans\,; Lucrèce, épouse de François-Marie della Rovere,
duc d'Urbin, en 1570, morte en 1598\,; Marfise et Bradamante, mariées
aux marquis de Carrare-Cibo et comte Bevilaqua.

Alphonse II, duc de Ferrare, de Modène et de Reggio, fils aîné et
successeur d'Hercule II, épousa, en février 1560, Lucrèce, fille de
Cosme de Médicis, grand-duc de Toscane\,; en février 1565, Barbe
d'Autriche, fille de l'empereur Ferdinand Ier\,; enfin, Marguerite,
fille de Guillaume Gonzague, marquis de Mantoue. Il mourut sans enfants,
27 octobre 1597, à soixante-quatre ans, le dernier de la véritable et
illustre maison d'Este.

Ici commence la maison bâtarde d'Este, présentement régnante.

Alphonse, fils du duc Alphonse Ier et de la fille de cet artisan de
Ferrare, était frère bâtard du duc Hercule, gendre du roi Louis XII et
oncle de son fils Alphonse II, mort sans enfants, en 1597. Ce bâtard
avait pourtant épousé, en 1549, Julie, fille de François-Marie della
Rovere, duc d'Urbin. Elle mourut en 1563 et lui en 1582, quinze ans
avant le dernier duc de Ferrare, de Modène et de Reggio, de la véritable
maison d'Este. Ce bâtard Alphonse laissa César, son aîné, et Alexandre,
évêque de Reggio, cardinal, 1598, mort 1624, et deux filles mariées,
l'une à Charles Gesualdo, prince de Venose au royaume de Naples, l'autre
à Frédéric Pic, prince de la Mirandole.

César, fils aîné du bâtard, se trouva le seul à prétendre à la
succession de son cousin germain le duc Alphonse II, mort sans enfants
en 1597 et le dernier de l'ancienne et véritable maison d'Este. Il fut
protégé par l'empereur, et, sans difficulté, duc de Modène et de Reggio.
Clément VIII ne fut pas si facile pour Ferrare qui ne relevait pas de
l'Empire comme Modène et Reggio, mais du saint-siège, et qu'il prétendit
lui être dévolu faute d'hoirs légitimes. Il ne voulut pas voir l'envoyé
de César, lequel prit les armes pour soutenir sa prétention et se
maintenir dans Ferrare. Le pape s'arma de son côté, et n'oublia pas en
même temps de se servir des foudres de l'Église. Henri IV, qui avait
grand intérêt de se montrer ami du pape, lui offrit le secours de ses
armes. Cette démonstration finit tout. César, hors d'état de résister,
ne pensa plus qu'à tirer de sa soumission le meilleur parti qu'il pût.
Il conclut donc un traité avec le pape à la fin de 1597, par lequel il
céda au pape la ville et le duché de Ferrare avec la Romandiole. Le pape
lui céda quelques terres dans le Bolonais, lui laissa ses biens
allodiaux \footnote{L'alleu, ou domaine allodial, était un bien possédé
  en toute propriété à la différence du fief qui relevait d'un seigneur
  dominant ou suzerain.}, lui garantit ses biens mouvants de l'Empire,
lui accorda le rang à Rome que les ducs ses prédécesseurs y avaient eu,
enfin donna à son frère Alexandre, évêque de Reggio, le chapeau de
cardinal, en mars 1598, lequel mourut en mai 1624. Après ce traité,
Clément VIII alla lui-même à Ferrare prendre possession de la ville et
du duché qui fait encore aujourd'hui une des plus belles possessions de
l'État ecclésiastique. César, seulement duc de Parme et de Reggio,
épousa, en 1586, Virginie, fille de Cosme de Médicis, grand-duc de
Toscane, qui mourut en 1615, et César en 1628 à soixante-six ans.

Alphonse, son fils, épousa en 1608 Isabelle, fille de Charles-Emmanuel
duc de Savoie, et la perdit en 1626. Il se dégoûta en moins d'un an de
la souveraineté à laquelle il avait succédé à son père, et s'alla faire
capucin à Munich en Bavière en 1629, et mourut dans cet ordre en 1644, à
cinquante-trois ans, ayant porté cet habit quinze ans. Il laissa entre
autres enfants François, son aîné, qui lui succéda\,; Renaud, évêque de
Reggio, cardinal 1641, mort 1672, qui fut attaché à la France, chargé de
ses affaires à Rome, et qui l'était lors de l'insulte que les Corses de
la garde du pape firent au duc de Créquy, ambassadeur de France en
{[}1662{]}, et qui sut en tirer un si bon parti pour sa maison par
l'accommodement de cette affaire\,; et une fille mariée à ce fameux muet
prince de Carignan.

François duc de Modène et de Reggio, par la retraite d'Alphonse, son
père, épousa les deux filles de Ranuce Farnèse duc de Parme, l'une après
l'autre, en 1630 et 1648, et en troisièmes noces Lucrèce fille de Tadée
Barberin prince de Palestrina en 1654. Il mourut en 1658 à quarante-huit
ans, et sa dernière femme en 1699. Entre autres enfants il laissa
Alphonse II, son fils aîné et son successeur\,; François, cardinal, puis
duc de Parme à son tour, et deux filles qui, l'une après l'autre, furent
la seconde et la troisième femme de Ranuce Farnèse duc de Parme.

Alphonse II, fils et successeur de François, duc de Modène et de Reggio.
Il épousa en 1655 Laure, fille de Jérôme Martinozzi et de Marguerite
soeur du cardinal Mazarin. Il mourut en juillet 1662, et son épouse qui
était soeur de M\textsuperscript{me} la princesse de Conti, mourut à
Rome, 19 juillet 1687. De ce mariage il n'y eut qu'un fils et une fille
à remarquer\,: François II, successeur\,; et Marie Béatrix qui épousa en
1673 le duc d'York, depuis roi d'Angleterre, Jacques II, et détrôné par
le prince d'Orange, réfugié en France, mort à Saint-Germain {[}16
septembre 1701{]}, et elle morte aussi à Saint-Germain {[}7 mai 1718{]},
mère de Jacques III, réfugié et traité en roi à Rome.

François II fils et successeur d'Alphonse II, duc de Modène et de
Reggio, gendre de Ranuce II Farnèse duc de Parme, mort sans enfants
1694, à trente-quatre ans.

Renaud, frère d'Alphonse II, oncle paternel de François II, cardinal en
1686 à trente un ans, n'entra point dans les ordres sacrés. Il succéda
en 1694 à François II, duc de Parme et de Reggio, son neveu, remit son
chapeau au pape, épousa en février 1696 Charlotte-Félicité, soeur de
l'impératrice Amélie, femme de l'empereur Joseph, qui ne l'épousa que
depuis\,; filles de Joseph Frédéric, duc de Brunswick-Lunebourg, et de
la soeur de la princesse de Salm, dont le mari avait été gouverneur et
grand maître de l'archiduc, puis empereur Joseph, et de
M\textsuperscript{me} la princesse de Condé, femme du dernier M. le
Prince.

François Marie, fils et depuis successeur de Renaud duc de Parme et de
Reggio, né en 1698, qui a épousé M\textsuperscript{lle} de Valois, fille
de M. le duc d'Orléans, lors régent de France.

Ainsi la bâtardise de ces derniers Este ne peut être plus clairement ni
plus évidemment prouvée. Passons maintenant à la maison Farnèse.

Elle est d'Orvieto et a pris le nom de son fief de Farnèse en Toscane.
On prétend qu'ils ont paru dès l'an 1000 entre les principaux citadins
d'Orvieto. Ce qui est certain, c'est qu'ils en ont été, plusieurs de
suite, consuls, et vers 1226 podestats. De là ils ont commandé les
troupes de Bologne, puis celles de Florence. On en connaît en tout cinq
générations avant le pape qui a fait les ducs de Parme, et six
générations légitimes sorties du père ou de l'oncle paternel de ce pape,
et qui ont duré jusque vers 1700 qu'elles se sont éteintes, la plupart
connues par des emplois militaires distingués, par des fiefs qui
l'étaient aussi, par des alliances bonnes, et plusieurs grandes, comme
des maisons Olonne, Ursins, Savelli, Conti, Acquaviva, Piccolomini,
Sforce, etc. On parle ici des Farnèse légitimes\,; venons maintenant aux
bâtards qui seuls des Farnèse ont été ducs de Parme et de Plaisance, de
Castro et de Camerino aux dépens de l'Église.

Alexandre, second fils de Louis Farnèse, seigneur de Montalte et de
Jeanne Cajetan, fille de Jacques seigneur de Sermoneta, né dernier
février 1468, cardinal 1493, évêque de Parme, puis d'Ostie, et doyen du
sacré collège, pape 1534, sous le nom de Paul III, mort 2 novembre 1549
à quatre-vingt-un ans\,; il eut un frère aîné, Barthélemy Farnèse qui,
de Violente Monaldeschi de Corvara, laissa une postérité légitime qui a
été illustre, et qui, avec celle de ses autres frères et cousins, n'a
fini qu'un peu avant 1700, et avec elle toute la maison Farnèse
légitime. Ce pape eut aussi deux soeurs dont l'aînée épousa Jules des
Ursins de Bracciano, et l'autre un Pucci de Florence, puis Gilles comte
de l'Anguilliara.

Farnèse bâtards\,: Alexandre Farnèse, depuis pape Paul III, avait
commencé par être évêque de Montefiascone et de Corneto. Étant cardinal
et évêque sacré, il eut deux bâtards Pierre-Louis et Ranuce, et une
bâtarde, Constance, qu'il maria depuis qu'il fut pape à Étienne Colone,
prince de Palestrine.

Ce pape acheta de Lucrèce della Rovere, veuve de Marc-Antoine Colone, la
terre de Frescati qu'elle avait eue en dot du pape son oncle, puis il
échangea avec l'Église Frescati pour les terres de Castro et de
Ronciglione qu'il donna à son bâtard Pierre-Louis. Ensuite il acheta
chèrement Camerino de ceux qui y avaient droit, se fondant sur ce que ce
fief était dévolu à l'Église par la mort de Jean-Marie Varani sans
enfants mâles, et qu'il avait droit de l'ôter aux héritiers de
Guidobaldo della Rovere, son gendre, qui était mort. Il maria son bâtard
Pierre-Louis à une fille de Louis des Ursins comte de Petigliano, et
Ranuce, son autre bâtard, à Virginie Gambara. Il fut général des
Vénitiens en 1526, du pape son père en 1527, du roi de France 1529\,; il
mourut sans postérité.

Il maria Octave, fils de Pierre-Louis, qu'il fit duc de Camerino, à
Marguerite, bâtarde de l'empereur Charles-Quint, veuve d'Alexandre de
Médicis, et ne se flatta pas de moins que d'obtenir le duché de Milan en
dot de ce mariage. Cette espérance fut le grand motif de la conférence
de Nice entre ce pape et Charles-Quint. Il y fut trompé\,: il se
réduisit donc à l'échange de Camerino avec Parme et Plaisance que Léon X
avait réclamés et acquis à l'Église comme ayant fait partie de
l'exarchat de Ravenne\,; son prétexte fut la proximité de Camerino qui
par là convenait mieux à l'Église que Parme et Plaisance qui étaient
éloignés et qui ne pouvaient s'entretenir et se conserver qu'avec
beaucoup de dépense. La plupart des cardinaux s'y opposèrent, mais le
pape passa outre, fit remettre à l'Église Camerino par Octave, fils de
Pierre-Louis, et le retira aussitôt après et le redonna au même Octave,
avec la qualité de duc et de duché, en le soumettant envers l'Église au
tribut annuel de dix mille écus d'or.

Ainsi ce bon pape fit ses deux bâtards l'un duc de Parme et de
Plaisance, l'autre duc de Castro, et le fils de son bâtard aîné duc de
Camerino, en attendant qu'il eût la succession de son père.

Pierre-Louis, bâtard aîné de Paul III, ne fut pas deux ans duc de Parme
et de Plaisance. C'était un homme perdu de toutes sortes de débauches et
de crimes, et qui s'était enrichi au pillage de Rome, par l'armée du
connétable de Bourbon, quoiqu'il ne fût point dans les troupes. Un
dernier crime énorme et de la nature de ceux qu'on ne peut nommer, mit
le comble à l'exécration publique. Il se fit une conjuration dont le
pape son père l'avertit\,; l'un et l'autre étaient fort enclins à la
magie. On prétend que Pierre sut par cette voie qu'il trouverait le nom
des conspirateurs écrits sur sa monnaie. Elle portait cette inscription
P. Aloïs. Farn. Parm. et Place. Dux. Il eut beau l'examiner, il n'en fut
pas plus savant. Il se trouva pourtant que les quatre premières lettres,
P. Aloïs, les désignaient. Les comtes Camille Palavicin, Jean
Anguisciola, Auguste Landi et Jean Louis gonfalonier, surprirent la
forteresse de Plaisance, tuèrent les gardes, et Anguisciola le tua dans
sa chambre. Aussitôt après cette exécution qui se fit le 10 septembre
1547, les Impériaux envoyés au voisinage par Gonzague, qui était du
complot, se saisirent de Plaisance pour l'empereur. Octave, fils de
l'assassiné, se retira auprès du pape son grand-père, qui pourvut à la
conservation de Parme, par les troupes qu'il y envoya sous Camille des
Ursins. Quelque temps après Octave, à l'insu du pape, tenta d'être reçu
dans la citadelle de Parme, comme dans son héritage, et en fut refusé
par Camille des Ursins, qui la gardait pour le pape. Octave menaça le
pape de s'accommoder avec Ferdinand Gonzague et de se rendre maître de
Parme par son secours, si le pape refusait de lui faire remettre la
place. Le pape entra sur cette menace dans une si étrange colère, qu'il
en mourut le 2 novembre 1549, s'écriant et répétant ce verset du psaume
18\,: «\,Si mei non fuissent dominati tunc immaculatus essem et
emundatus a delicto maximo.\,» Louis XIV, qui se trouvait dans le même
cas, y mit le comble en mourant, bien loin du repentir de ce pape, entre
les bras de ses bâtards déifiés, de la Maintenon leur gouvernante, du
jésuite Tellier, des cardinaux de Rohan et de Bissy, et de Voysin, leur
fidèle ministre, et leur immola de plus son royaume, autant qu'il fut en
lui, et l'éducation du roi son successeur et son arrière-petit-fils, en
plein.

Les enfants de Pierre-Louis furent\,: Octave, qui lui succéda\,;
Alexandre et Ranuce à dix ans l'un l'autre, que le pape leur grand-père
fit cardinaux, chacun à quinze ans, et leur donna force grands évêchés
et archevêchés, et les premières charges de la cour de Rome, dont ils
furent l'un et l'autre l'ornement à tous égards\,: Alexandre mourut en
1589, à soixante-neuf ans, doyen du sacré collège, et Ranuce en 1565, à
quarante-cinq ans\,; Horace duc de Castro, tué à la guerre en 1554, un
an après avoir épousé Diane, bâtarde d'Henri II, et de Diane de
Poitiers, laquelle fut remariée au duc de Montmorency maréchal de
France, fils et frère des deux derniers connétables de Montmorency\,;
elle n'eut point d'enfants de ses deux maris\,: enfin une fille Victoire
mariée à Guidobaldo della Rovere duc d'Urbin.

Octave avait épousé en 1535, comme on l'a déjà dit, Marguerite, bâtarde
de l'empereur Charles-Quint, qui ne fut pas heureuse avec lui. Brouillé
avec Charles-Quint, lors de la mort du pape son grand-père, il se jeta
dans le service de France jusqu'à ce qu'il se fut raccommodé avec lui en
1556. Il joignit alors le duché de Plaisance à celui de Parme\,; mais il
ne put jamais avoir la citadelle de Plaisance. Il servit toute sa vie la
maison d'Autriche dans toutes ses guerres, et vint mourir à Parme, en
octobre 1586, à soixante-deux ans. Marguerite, son épouse, fut la
célèbre gouvernante des Pays-Bas pendant huit ans, à qui succéda le duc
d'Albe\,; elle vint se retirer à Ortone, dans le royaume de Naples,
qu'elle avait eu en dot, et y mourut dans la plus haute réputation en
tout genre, en janvier 1586. Ils laissèrent Alexandre, leur fils unique,
qui fut duc de Parme et de Plaisance, et quatre filles. L'aînée épousa
Jules Cesarini, puis Marc Pio, marquis de Sassolo\,; les trois autres,
Alexandre marquis Palavicini, Renaud comte Borromée, Alexandre Sforce
comte de Borgonovo.

Alexandre, duc de Parme et de Plaisance, fut un des plus grands
capitaines de son siècle, si connu par la guerre qu'il fit dans les
Pays-Bas pour l'Espagne, et en France pour la Ligue. Il épousa, en 1566,
Marie, fille d'Édouard, prince de Portugal, qui mourut en 1577, et lui
en Artois, 11 décembre 1592, à quarante-sept ans. Ils laissèrent deux
fils et une fille Ranuce qui succéda à son père\,; Odoard, cardinal
1591, mort 1626, à soixante-deux ans\,; et Marguerite, mariée à Vincent
Gonzague, duc de Mantoue\,; elle en fut séparée pour cause de parenté et
se fit religieuse à Plaisance.

Ranuce, duc de Parme et de Plaisance, après le fameux Alexandre son
père, épousa Marguerite Aldobrandin, fille du frère de Clément VIII. Il
fut gonfalonier de l'Église, et mourut plus craint qu'aimé, en 1622, à
cinquante-deux ans, et sa femme en 1646. Ils laissèrent deux fils et
deux filles Odoard qui succéda\,; François-Marie, cardinal, 1645, mort,
1647, à trente ans\,; Marie, femme de François d'Este, duc de Modène, et
Victoire, seconde femme du même. Ranuce laissa encore une bâtarde, qu'il
maria à Jules-César Colone, prince de Palestrine.

Odoard, duc de Parme et de Plaisance, après Ranuce son père, épousa, en
1628, Marguerite de Médicis, fille de Cosme II, grand duc de Toscane. Il
se brouilla avec les Espagnols, qui lui firent une cruelle guère\,; il
en essuya une autre des Barberins, non moins fâcheuse, du temps d'Urbin
VIII\,; il mena une vie fort agitée, et la finit, en 1646, à
trente-quatre ans. Sa femme mourut en 1679. Leurs enfants furent, Ranuce
II, qui succéda\,; Alexandre, qui fut vice-roi de Navarre, puis
gouverneur des Pays-Bas en 1680, et qui mourut sans alliance, en 1689, à
cinquante-quatre ans\,; et Horace, général des Vénitiens, mort sans
alliance, en 1656, à vingt ans.

Ranuce II, duc de Parme et de Plaisance, épousa, en 1660, Marguerite,
fille de Victor-Amédée, duc de Savoie, et la perdit en 1663\,; en
secondes noces, en 1664, Isabelle d'Este, fille de François duc de
Modène, qu'il perdit en 1666\,; en troisièmes noces, en 1668, Marie
d'Este, soeur de la dernière\,: elle mourut en 1684. Ranuce ne fut pas
moins embarrassé de la guerre de Castro que son père l'avait été, et des
crimes d'un favori de néant. Il fut malheureux et battu, et réduit à
souffrir l'\emph{incamération}\footnote{Le mot \emph{incamération}
  signifie que le duché de Castro fut réuni aux domaines de la Chambre
  Apostolique. C'est un terme de la chancellerie romaine.} de Castro. Sa
vie ne fut pas moins agitée, mais plus triste encore que celle de son
père\,; il mourut, en 1694, à soixante-deux ans. Il eut une fille,
mariée, en 1692, à François d'Este, duc de Modène, et deux fils qui lui
succédèrent l'un après l'autre.

Odoard II, qui épousa, en 1690, Dorothée-Sophie, fille de
Philippe-Guillaume, électeur palatin, duc de Neubourg\,: de ce mariage
une fille unique, seconde femme de Philippe V, roi d'Espagne. Odoard
mourut en 1693, à trente-trois ans. Son frère François lui succéda. Il
épousa sa veuve, dont il n'eut point d'enfants. Il mourut en {[}1727{]}
et en lui finirent les ducs de Parme et de Plaisance bâtards de la
maison Farnèse.

On voit ainsi qu'Élisabeth Farnèse, fille unique d'Odoard II, duc de
Parme et de Plaisance, est la seule héritière de ses États et de ceux de
Toscane par la grand'mère de son père.

\hypertarget{chapitre-xiii.}{%
\chapter{CHAPITRE XIII.}\label{chapitre-xiii.}}

1719

~

{\textsc{Le roi Jacques repasse en Italie.}} {\textsc{- Le prince
électoral de Saxe épouse une archiduchesse, Joséphine.}} {\textsc{-
Bénédiction de M\textsuperscript{me} de Chelles.}} {\textsc{- Mort de
Marillac, doyen du conseil\,; de M\textsuperscript{me} de Croissy\,; son
caractère.}} {\textsc{- Mort de Courcillon\,; de Louvois, capitaine des
Cent-Suisses.}} {\textsc{- Sa charge donnée à son fils à la mamelle.}}
{\textsc{- Mort du comte de Reckem, du duc de Bisaccia\,; sa famille.}}
{\textsc{- Mort du marquis de Crussol\,; de l'évêque d'Avranches,
Coettenfao\,; d'Orry\,; de M\textsuperscript{me} de Bellegarde, puis de
son mari\,; du duc de La Trémoille.}} {\textsc{- Mort de
M\textsuperscript{me} de Coigny\,; extraction de son mari.}} {\textsc{-
Mort de l'abbé de Montmorel.}} {\textsc{- Mort du président
Tambonneau.}} {\textsc{- M. le comte de Charolais comblé d'argent du
roi, fait gouverneur de Touraine.}} {\textsc{- Comte d'Évreux achète le
gouvernement de l'Ile-de-France et la capitainerie de Monceaux, où il
désole le cardinal de Bissy.}} {\textsc{- Le nonce Bentivoglio, près
d'être cardinal, prend congé et part.}} {\textsc{- Ses horreurs.}}
{\textsc{- L'abbé de Lorraine et l'abbé de Castries obtiennent enfin
leurs bulles de Bayeux et de Tours, et sont sacrés par le cardinal de
Noailles.}} {\textsc{- Commission de juges du conseil envoyée à
Nantes.}} {\textsc{- Bretons arrêtés\,; d'autres en fuite.}} {\textsc{-
Berwick en Roussillon, prend la Ceu-Urgel\,; y finit la campagne.}}
{\textsc{- Le Guerchois gouverneur d'Urgel.}} {\textsc{- M. le duc
d'Orléans se fait appeler mon oncle.}} {\textsc{- Le feu roi
n'apparentait que lui, Monsieur et la vieille Mademoiselle.}} {\textsc{-
Conseil de régence entièrement tombé.}} {\textsc{- Besons, archevêque de
Rouen, puis l'abbé Dubois, y entrent.}} {\textsc{- Je propose à M. le
duc d'Orléans un conseil étroit, en laissant subsister celui de
régence\,; {[}chose{]} que l'abbé Dubois empêcha.}} {\textsc{- Davisard
mis en liberté.}} {\textsc{- La Chapelle\,; quel\,; exilé, aussitôt
rappelé, mort peu après.}} {\textsc{- Quatre millions payés en
Bavière\,; trois en Suède.}} {\textsc{- Quatre-vingt mille livres
données à Meuse, et huit cent mille francs à M\textsuperscript{me} de
Châteauthiers, dame d'atours de Madame.}} {\textsc{- Abbé Alary\,;
quel\,; obtient deux mille livres de pension.}} {\textsc{- Le marquis de
Brancas obtient quatre mille livres de pension pour son jeune frère, et
la survivance de sa lieutenance générale de Provence à son fils, à neuf
ans.}} {\textsc{- Maréchal de Matignon obtient six mille livres
d'augmentation d'appointements de son gouvernement.}} {\textsc{- Fureur
du Mississipi et de la rue Quincampoix.}} {\textsc{- Diminution
d'espèces\,; refonte.}} {\textsc{- Prince de Conti retire Mercoeur à
Lassai.}} {\textsc{- Largesses aux officiers employés contre
l'Espagne.}} {\textsc{- Affaires de cour à Vienne.}} {\textsc{- Prince
d'Elboeuf\,: quel\,; obtient son abolition et revient en France.}}
{\textsc{- Nominations d'évêchés où l'abbé d'Auvergne et le jésuite
Lafitau sont compris.}} {\textsc{- Conduite de ce dernier.}}

~

Le roi Jacques, qui avait été bien reçu en Espagne, et qui avait tenté
avec son secours de passer en Écosse, essuya une tempête qui endommagea
et sépara toute la flotte d'Espagne. La mort du roi de Suède et les
affaires domestiques de Russie avaient fort déconcerté ses projets\,:
ainsi il repassa en Italie, et s'en retourna à Rome achever son mariage,
où la fille du prince Sobieski, qu'il avait épousée par procureur,
l'attendait. C'était la crainte de cette tentative et de son succès qui
avait si fort pressé l'abbé Dubois de la déclaration de la guerre à
l'Espagne.

Le prince électoral de Saxe épousa à Vienne l'archiduchesse, fille aînée
du feu empereur Joseph avec les plus fortes renonciations en faveur de
la maison d'Autriche, contenues dans le contrat de mariage, et
solennellement ratifiées devant et après la célébration.

M\textsuperscript{me} de Chelles fut enfin bénite à Chelles par le
cardinal de Noailles au milieu de trente abbesses. Il y eut des tables
pour six cents personnes. Elle en tint une de cinquante couverts. M. le
duc d'Orléans mangea en particulier avec quelques dames qu'il avait
menées. Madame n'y alla point, et M\textsuperscript{me} la duchesse
d'Orléans passa toute cette journée dans sa nouvelle maison de Bagnolet.

Il mourut en ce temps-ci un grand nombre de personnes distinguées ou
connues\,: Marillac, doyen du conseil, en la place duquel Pelletier de
Sousi monta. On a vu ailleurs que la conversion forcée des huguenots fit
Marillac conseiller d'État, qui était intendant à Poitiers, et Vérac,
chevalier de l'ordre, qui était lieutenant général de Poitou. Marillac
fut le dernier de cette famille assez récemment sortie d'un avocat, que
l'élévation et les malheurs du garde des sceaux et du maréchal de
Marillac, frères, avaient fort décorée.

M\textsuperscript{me} de Croissy, mère de Torcy, qui était fort vieille,
mais tout entière de corps et d'esprit, dont elle avait beaucoup. Elle
était fille unique de Braud, qui de médecin s'était fait grand
audiencier\footnote{Officier de la grande chancellerie chargé de
  présenter au sceau les lettres de grâce, de noblesse, etc. Voy., t. X,
  p.~451, une note sur la manière dont le chancelier et le garde des
  sceaux tenaient le sceau dans l'ancienne monarchie.}, après être
devenu fort riche. Les ambassades de son mari l'avaient fort accoutumée
au grand monde, et la cour ensuite lorsqu'il fut devenu secrétaire
d'État\,; elle y était fort propre. Son goût était d'accord avec son
génie pour la grande représentation, la magnificence et le jeu, qui
l'avaient suivie à Paris dans son veuvage. Elle y tint toujours une
grande et florissante maison où la cour, ce qu'il y avait de meilleur
dans la ville, et tous les étrangers de distinction, étaient toujours.
Elle excellait à la tenir et en bien faire les honneurs, avec une
politesse et un discernement particulier\,; hors de chez elle impérieuse
et insupportable. Son démêlé sur un rien, car il ne s'agissait ni de
cérémonial ni encore moins d'affaires, avec la femme du comte
Olivencrantz, premier ambassadeur de Suède, et dont une dispute au jeu
fut le plus essentiel, se poussa si loin, que les maris prirent parti,
dont les suites ne furent pas heureuses pour la France par la haine que
cet ambassadeur remporta chez lui, et qu'il inspira au conseil de son
maître.

Courcillon mourut de la petite vérole. On a eu lieu de parler de lui ici
assez pour n'avoir rien à y ajouter. C'était un homme très singulier,
qu'une cuisse de moins n'avait pu attrister\,; qui, par faveur de sa
mère et la sienne personnelle auprès de M\textsuperscript{me} de
Maintenon, et son état mutilé, s'était mis sur le pied de tout dire et
de tout faire, et qui en faisait d'inouïes avec beaucoup d'esprit et une
inépuisable plaisanterie et facétie. Il avait aussi beaucoup de lecture,
de valeur et de courage d'esprit, mais au fond ne valait rien, et de la
plus étrange débauche et la plus outrée. Sa femme, fille unique de
Pompadour, belle comme le jour, eut de quoi être toute consolée. Dangeau
et sa femme, qui n'avaient point d'autres enfants, en furent très
affligés. Courcillon ne laissa qu'une fille unique.

Louvois mourut aussi de la petite vérole à Rambouillet, chez le comte de
Toulouse. Il était fils de Courtenvaux, fils aîné du trop célèbre
Louvois, et d'une fille et soeur des deux derniers maréchaux d'Estrées,
et capitaine des Cent-Suisses de la garde du roi, que son père lui avait
cédés. Il avait épousé une fille de la maréchale de Noailles, dont il
laissa un fils qui n'avait que seize mois. Le lendemain de sa mort le
maréchal de Villeroy, le duc de Noailles et le maréchal d'Estrées
n'eurent pas honte de demander la charge pour un enfant à la mamelle, ni
M. le duc d'Orléans de la leur accorder. Ajoutez à cela la naissance,
les services, le mérite de Courtenvaux et de son fils, et on trouvera
cette grâce encore mieux placée.

Le comte de Reckem, chanoine de Strasbourg, avec deux belles abbayes. Il
avait servi assez longtemps à la tête d'un des régiments du cardinal de
Fürstemberg quoique dans les ordres. Dès que le roi le sut il le lui fit
quitter.

Le duc de Bisaccia (Pignatelli). Il avait été pris à Gaëte avec le
marquis de Villena, vice-roi de Naples, par les Impériaux, conduit avec
lui à Pizzighitone, et chargé comme lui de chaînes, en haine de la belle
défense qu'ils avaient faite et avaient été pris combattant. Après une
longue prison, il était venu à Paris. C'était un très galant homme. Sa
mère était del Giudice, et sa femme la dernière de cette grande et
illustre maison d'Egmont. Elle était morte, et en avait laissé le nom,
les armes, la grandesse et les biens à son fils, que le père avait
marié, comme on l'a vu, à la seconde fille du feu duc de Duras. Il avait
aussi marié sa fille au duc d'Aremberg-Ligne, un des plus grands
seigneurs de Flandre.

La petite vérole emporta encore le comte de Crussol, à Villacerf, chez
son beau-père. Il était jeune et avait un régiment. Il était fils de
Florensac, qui était menin de Monseigneur et frère du duc d'Uzès, gendre
du duc de Montausier. Le comte de Crussol laissa des enfants.

Coettenfao, dont il a été parlé ici plusieurs fois, et fort de mes amis,
perdit son frère, évêque d'Avranches, très bon et digne prélat.

Orry mourut enfin dans son lit, après avoir frisé de si près, et par
deux fois, la corde qu'il méritait à tant de titres. Il avait été
fermier de Villequier, puis solliciteur de procès, après homme d'affaire
de la duchesse de Portsmouth, qui le chassa pour ses friponneries. Il a
depuis été par deux fois maître de l'Espagne sous la princesse des
Ursins. Il y a eu lieu ici d'en parler assez pour n'avoir rien à y
ajouter.

M\textsuperscript{me} de Bellegarde, femme du second fils d'Antin,
depuis assez peu, fille unique et héritière de Vertamont, premier
président du grand conseil, mourut de la petite vérole également riche
et laide, mais bonne créature. Elle n'eut point d'enfants. Son mari, qui
avait la survivance des bâtiments, fut fort sensible à cette perte, et
mourut quatre ou cinq mois après.

Le duc de La Trémoille mourut de la petite vérole, laissant un seul
fils, enfant, survivancier de sa charge de premier gentilhomme de la
chambre.

M\textsuperscript{me} de Coigny mourut aussi fort vieille\,: elle était
soeur du comte de Matignon, chevalier de l'ordre, et du maréchal de
Matignon. On l'avait mariée à grand regret, mais pour rien à Coigny qui
était fort riche. Le fâcheux était qu'il les avaisinait, et que ce qu'il
était ne pouvait être ignoré dans la Normandie. Son nom est Guillot, et
lors du mariage, tout était plein de gens dans le pays qui avaient vu
ses pères avocats et procureurs du roi, des petites juridictions
royales, puis présidents de ces juridictions subalternes. Ils
s'enrichirent et parvinrent à cette alliance des Matignon. Coigny se
trouva un honnête homme, bon homme de guerre, qui ne se méconnut point,
et qui mérita l'amitié de ses beaux-frères\,; c'est lui qu'on a vu en
son lieu refuser le bâton de maréchal de France, sans le savoir, en
refusant de passer en Bavière, dont il mourut peu après de douleur.
Marsin en avait profité. Coigny s'arrondit plus que n'avaient fait ses
pères. Il acheta tout près de son bien la terre de Franquetot de gens de
condition en Normandie. Il vit cette maison s'éteindre. Alors il obtint
des lettres patentes pour changer son nom de Guillot en celui de
Franquetot, et les fit enregistrer au parlement, etc., de Normandie, par
quoi son ancien nom, conséquemment son ancien état, est pour toujours
solennellement constaté. Que dirait cette dame de Coigny si elle
revendit au monde\,? Pourrait-elle croire la fortune de son fils et la
voir sans en pâmer d'effroi et sans en mourir aussitôt de joie\,?

L'abbé de Montmorel, qui avait été aumônier de la dernière Dauphine et
proposé pour être confesseur du roi. Son rare mérite l'avait fort
distingué, duquel il s'était toujours contenté avec grande modestie. On
a de lui plusieurs ouvrages de piété pleins d'érudition et d'onction,
deux choses qu'on allie rarement.

Tambonneau, qui avait été président à la chambre des comptes et
longtemps ambassadeur en Suisse où il avait bien fait. Il était fils de
la vieille Tombonneau, soeur de la mère du feu maréchal et du cardinal
de Noailles, qui avait eu l'art de se faire un tribunal dans Paris, où
abondait chez elle, jusqu'à sa mort, la fleur de la cour et de la ville.
On en a parlé ici en son temps. Son fils, dont elle ne fit jamais aucun
cas, se fourra tant qu'il put dans le monde, et sa femme aurait bien
voulu imiter sa belle-mère, mais les phénomènes ne se redoublent pas.
Tambonneau était bon homme et honnête homme.

Dangeau n'ayant plus d'enfants, M. le Duc obtint de M. le duc d'Orléans
que le roi payât comptant quatre cent mille livres à Dangeau pour le
gouvernement de Touraine qu'il avait acheté autrefois peu de chose, je
ne me souviens plus de qui, et qui avait toujours été sur le pied des
petits gouvernements de province, d'environ vingt mille livres au plus
d'appointements, et de le donner à M. le comte de Charolais sur le pied
des grands, c'est-à-dire de soixante mille livres d'appointements au
moins\,; ce n'était pas que M. de Charolais n'eût de grosses pensions du
roi et pour immensément d'actions en pur présent, à faire valoir sur le
roi au centuple.

Le comte d'Évreux acheta du duc d'Estrées le gouvernement de
l'Ile-de-France, et du duc de Tresmes la capitainerie de Monceaux, avec
laquelle il désola le cardinal de Bissy sur la chasse, par cent procès
et procédés, pour sa maison de campagne de son évêché de Meaux.

Le nonce Bentivoglio, près enfin d'être cardinal et sûr de trouver sa
calotte en entrant en Italie, prit congé du roi et du régent, après
avoir fait, ou voulu et travaillé à faire tous les maux dont les chiens
et les loups enragés peuvent être capables. Il emporta le mépris et la
malédiction publique, même de ceux de son parti. Il ne fut regretté que
d'une fille de l'Opéra qu'il entretenait chèrement, et dont il eut une
fille, qui à son tour monta sur le théâtre de l'Opéra, où elle a été
fort connue et toujours sous le nom de \emph{la Constitution}, en
mémoire de son éminentissime père, qui en tout était un fou et un
scélérat qui aurait mis le feu aux quatre coins de l'Europe, s'il avait
cru et pu en hâter sa promotion d'un jour. Il avait si bien noirci à
Rome l'abbé de Lorraine, nommé à Bayeux, et l'abbé de Castries, nommé à
Tours, que le pape leur refusa leurs bulles. D'autres, nommés par
compagnie, essuyèrent la même vexation. Je m'étais employé pour l'abbé
de Castries, conjointement avec M\textsuperscript{me} la duchesse
d'Orléans qui m'en avait prié avant que nous fussions brouillés, et
l'amitié pour cet abbé et pour son frère m'y aurait bien porté seul. On
voit par cette date combien ces bulles se différèrent. Enfin, on fit
parler si haut à Rome, qu'à la fin les bulles arrivèrent\,; le grand
crime de ces deux nommés était leur liaison d'amitié avec le cardinal de
Noailles. Tous deux s'en moquèrent devant et après\,; tous deux se
firent sacrer par le cardinal de Noailles, l'abbé de Castries, à
l'ordinaire, dans la chapelle de l'archevêché\,; l'abbé de Lorraine,
quelque peu après, dans le choeur de Notre-Dame à la prière du chapitre,
ce qui, depuis l'épiscopat du cardinal de Noailles, ne s'était fait que
pour son frère, qui lui succéda à l'évêché de Châlon.

Les déclarations de la duchesse du Maine qu'on a vues ici en son lieu
donnèrent lieu à des découvertes importantes en Bretagne, et enfin à une
commission de douze maîtres des requêtes, à la tête desquels
Châteauneuf, conseiller d'État, de retour de ses ambassades, fut mis.
Vattan, maître des requêtes, en fut le procureur général, et deux
conseillers du Châtelet pour substituts. Plusieurs gentilshommes furent
arrêtés en Bretagne, d'autres en fuite, entre ces derniers Pontcallet,
Bonamour, du Poulduc\footnote{On écrit ordinairement Polduc.} de la
maison de Rohan. La commission se rendit à Nantes\,; on avait eu soin
auparavant de prendre des prétextes pour la faire soutenir par des
troupes, et pour que l'arrivée de ces troupes n'effarouchât personne.

Le maréchal de Berwick, n'ayant plus rien à exécuter du côté de la
Navarre, était passé en Roussillon, où il prit la Ceu-d'Urgel et nettoya
divers postes en présence du prince Pio, qui l'avait suivi à la tête de
l'armée d'Espagne par le dedans du pays, et ce fut là que finit la
campagne. Le Guerchois, lieutenant général, en eut le gouvernement avec
douze mille livres d'appointements.

Sur la fin d'octobre, M. le duc d'Orléans, je n'ai point su à
l'instigation de qui, car il n'était guère capable d'y penser lui-même,
désira que le roi, parlant à lui, l'appelât mon oncle, au lieu de lui
dire Monsieur, et cela fut ainsi désormais. Le feu roi n'apparentait
personne sans exception que Monsieur et M. le duc d'Orléans. Il les
appelait mon frère et mon neveu, parlant à eux et parlant d'eux. Il
appelait aussi ma cousine et disait ma cousine en parlant de
Mademoiselle, fille de Gaston, morte en 1693\,; jamais ses petits-fils
ni Monseigneur. Il était très rare qu'il lui dît quelquefois mon fils ou
en parlant de lui\,; jamais Madame ni pas un prince ni princesse du
sang.

Besons, archevêque de Rouen, entra en ce même temps au conseil de
régence, où il se disait et ne se faisait presque plus rien d'important.
L'abbé Dubois, qui n'y entrait que pour les affaires étrangères depuis
qu'il en était secrétaire d'État, y entra bientôt après tout à fait. Le
ridicule où ce conseil commençait à tomber, et que je prévis devoir
s'augmenter par la facilité de M. le duc d'Orléans à y admettre, parce
qu'on n'y faisait rien, et qu'il s'en moquait tout bas le premier, me
fit sentir de plus en plus le danger de son cabinet, où tout se réglait,
et celui du crédit de l'abbé Dubois qui y était le maître, et qui n'y
laissait rien communiquer à personne qu'à ceux-là seulement, dont il ne
pouvait {[}se{]} passer pour l'exécution, et encore pour le moment du
besoin\,; rarement, M. le duc d'Orléans prenait la liberté d'étendre
cette confiance. Je lui parlai de l'indécence du conseil de régence, du
dégoût de ceux qui le composaient principalement, des inconvénients de
son cabinet, où tout passait et se réglait, et qui donnait aux
mécontents une toute autre prise que si les affaires se portaient dans
un conseil de régence sérieux et peu nombreux, à l'exception des choses
rares qui avaient besoin d'un entier secret, comme cela était dans les
deux premières années. Je lui représentai que la confiance ne pouvait
plus être la même\,; qu'il donnait lieu par là à tous les soupçons qu'on
voudrait prendre et qu'on prenait en effet, et beau jeu dans la suite à
prévenir le roi contre lui, et peut-être à lui demander des comptes et à
lui imputer bien des choses, dont il se trouverait embarrassé.

C'était l'homme du monde qui convenait le plus aisément de ce qu'on lui
disait de vrai, mais qui en convenait le plus inutilement. Il m'avoua
que je pouvais avoir raison, et ajouta qu'à tout ce qui était dans le
conseil de régence, il n'y avait plus moyen d'y rien porter que des
choses de forme. Alors je souris et lui demandai à qui en était la
faute, ainsi que de la confusion des autres conseils qui les avait fait
supprimer\,: «\,Cela est encore vrai, me dit-il en riant, mais cela est
rait, et quel remède\,? --- Quel remède\,? repris-je, il est bien
nécessaire, et en même temps bien aisé\,; mais il faut le vouloir, et ne
s'arrêter pas à des considérations personnelles de gens qui, s'ils
pouvaient vous tenir, n'en auraient aucune pour vous, comme vous-même
n'en sauriez douter\,; et la fermeté après de ne pas retomber dans
l'inconvénient où peu à peu votre facilité a mis le conseil de
régence\,: c'est le laissant tel qu'il est, mais n'y ajoutant plus
personne et continuant à y porter les choses de forme, vous faire un
conseil de quatre personnes, et vous en cinquième, les bien choisir à
vous, mais tels aussi que le monde en puisse approuver le choix, et y
prendre confiance\,; que ce soit tous gens de tel état qu'il vous
plaira, mais qui n'aient aucun département, et ne soient point entraînés
par cet intérêt d'un côté plus que d'un autre\,; que tout sans exception
passe par ce conseil, et que vous vous gardiez surtout de lui rien
cacher, et de ces petits pots à part de travail avec un homme et avec un
autre, surtout avec aucun qui ait un département, et qui ne manqueront
pas de prétexte. À cela, vous avez beau jeu. Il n'est personne, à
commencer par ceux du conseil de régence, qui ne sente qu'à son nombre
et à sa composition, il n'est plus possible d'y traiter rien de sérieux,
et qui n'aime mieux vous voir avec un conseil particulier qu'entre les
seules mains de l'abbé Dubois, et par-ci par-là, du premier venu pour
d'autres affaires. Vous n'êtes point gêné en ce choix, comme vous l'avez
été pour le conseil de régence, d'y mettre des gens de contrebande, même
en le formant, et de l'un à l'autre depuis, d'autres parfaitement
inutiles ou même embarrassants. Vous avez eu depuis la mort du roi sans
parler des temps qui l'ont précédée, vous avez eu, dis-je, le temps et
les occasions de connaître le fort et le faible, la conduite et les
inclinations de tout ce qui peut être choisi. Choisissez donc bien et
avec mûre réflexion, mais sans lenteur, parce que vous avez toutes les
connaissances, et qu'il ne s'agit que de repasser les différentes
personnes dans votre esprit, et ce que vous connaissez de chacune
d'elles\,; d'en faire le triage, et de vous déterminer. Vous n'avez
point à craindre là-dessus ce qui a passé au parlement sur votre
régence. Vous avez supprimé les conseils particuliers sans lui, quoique
établis avec lui, et le parlement n'en a pas soufflé\,; en laissant donc
le conseil de régence comme il est, et y portant les choses seulement de
forme, comme aujourd'hui il ne s'y en porte guère d'autres, le parlement
n'a rien à dire. Vous travaillez chez vous avec qui il vous plaît\,; que
ce soit toujours avec les mêmes gens ou avec un seul, ou quelquefois
avec différentes personnes, le parlement n'a que voir à cela. Il n'a
rien dit là-dessus jusqu'à cette heure. À l'humeur qu'il vous a montrée,
il aurait bien dit là-dessus, s'il avait cru pouvoir l'entreprendre\,;
il ne s'agit donc que de votre volonté et d'aucune autre difficulté. Je
trouve la chose si nécessaire que, pour vous en persuader mieux, je vous
déclare de très bonne foi, et vous ne sauriez me nier que je vous aie
parlé toute ma vie de même, je vous déclare, dis-je, que je ne veux
point être de ce conseil, par conséquent qu'aucune autre vue ne me meut
à vous le proposer, que le bien de l'État et que le vôtre.\,»

M. le duc d'Orléans se promena trois ou quatre tours dans sa petite
galerie, devant son cabinet d'hiver, et moi avec lui sans dire un mot et
la tête basse, comme il avait accoutumé quand il était embarrassé, puis
il se tourna à moi qui ne disais mot, et me dit que cela avait du bon,
et qu'il y fallait penser. «\,Penser, soit, lui répondis-je, pourvu que
cela ait son terme court, car les raisons en sautent aux yeux et je n'en
vois pas une contre\,; il ne s'agit que de prendre une résolution, vous
déterminer sur le choix, et exécuter.\,»

Je laissai le régent pensif et mal à son aise\,; il sentait combien ce
que je proposais blesserait l'abbé Dubois, et l'abbé Dubois était son
maître. Il ne se pouvait défendre aussi de sentir le ridicule du conseil
de régence, et le murmure général que tout passât par l'abbé Dubois et
rien que par lui\,; et pour le danger, s'il le sentait, le Rubicon en
était passé par les chaînes anglaises dont il s'était laissé entraver et
de concomitance par les impériales, et cette folle et funeste guerre
contre l'Espagne, qui en était la suite nécessaire, et qui, formant et
laissant une haine personnelle contre le régent et l'Espagne, l'en
séparait pour toujours, et nécessairement par cela même le livrait pour
les suites de plus en plus à l'Angleterre, et par l'Angleterre à
l'empereur, qui était le but où l'abbé Dubois avait toujours tendu pour
son chapeau, et de là pour être premier ministre. C'est ce que le
conseil que je proposais aurait utilement empêché, s'il avait été établi
à temps, mais dont l'établissement alors aurait du moins prévenu les
funestes suites et celles du chapeau et de la toute-puissance\,; par
conséquent, ce conseil était ce qui pouvait être proposé de plus
contradictoire et de plus odieux à l'abbé Dubois, à l'opposition duquel
et de toutes ses forces il fallait s'attendre. Aussi en regardai-je
l'établissement comme une chimère, mais chimère toutefois que le devoir
ne me permettait pas de ne pas proposer, et de ne pas poursuivre auprès
d'un prince, duquel l'expérience montrait qu'il ne fallait ou plutôt
qu'on pouvait n'espérer et ne désespérer de rien.

Il permit à Davisard, cette plume si hardie du duc et de la duchesse du
Maine, malade ou qui le faisait, de sortir de la Bastille, c'est-à-dire
qu'il fut mis en liberté. En même temps il exila à Bourges La Chapelle,
secrétaire de M. le prince de Conti, qui cria tant qu'il le fit revenir
au bout d'un mois. Je n'ai point su quelle sottise ce compagnon avait
faite. C'était un très hardi et très dangereux fripon, recrépi de bel
esprit, et de l'Académie française. Il ne vécut pas longtemps depuis son
retour.

L'argent était en telle abondance, c'est-à-dire les billets de la banque
de Law qu'on préférait alors à l'argent, qu'on paya quatre millions à
l'électeur de Bavière et trois millions à la Suède, la plupart
d'anciennes dettes. Peu après M. le duc d'Orléans fit donner
quatre-vingt mille francs à Meuse, et huit cent mille livres à
M\textsuperscript{me} de Châteauthiers, dame d'atours de Madame, qui
l'aimait fort depuis bien des années. L'abbé Alary obtint deux mille
livres de pension. Il était fils d'un apothicaire de Paris, et une
dangereuse espèce, avec de l'esprit et de l'érudition, du monde et de la
politesse\footnote{Voy., sur l'abbé Alary, les détails donnés par le
  marquis d'Argenson. \emph{Mémoires} (édit. de 1825, p.~229, 247, 272).}.
Il trouva depuis le moyen de se faire des amis, de se fourrer à la cour,
d'avoir des bénéfices. Il intrigua tant qu'après quelques années il se
fit chasser.

Le marquis de Brancas, mon ami depuis longtemps, avait eu, comme on l'a
vu en son temps, la lieutenance générale unique de Provence, à la mort
de Simiane, gendre du vieux comte de Grignan. Brancas en voulait avoir
la survivance pour son fils qui n'avait que neuf ans, et il venait
d'obtenir une pension de quatre mille livres pour son jeune frère, le
comte de Cereste\,; je ne sais pourquoi il me pria d'en parler à M. le
duc d'Orléans, duquel il était très à portée de l'obtenir directement\,;
je le fis et cela ne fut pas difficile\,; M. le duc d'Orléans la lui
donna.

Le maréchal de Matignon, on ne sait pas pourquoi, eut une augmentation
d'appointements de six mille livres sur son gouvernement du pays
d'Aunis.

Le commerce des actions de la Compagnie des Indes, appelé communément du
Mississipi, établi depuis plusieurs mois dans la rue Quincampoix, de
laquelle chevaux et carrosses furent bannis, augmenta tellement qu'on
s'y portait toute la journée, et qu'il fallut placer des gardes aux deux
bouts de cette rue, y mettre des tambours et des cloches pour avertir à
sept heures du matin de l'ouverture de ce commerce et de la retraite à
la nuit, enfin redoubler les défenses d'y aller les dimanches et les
fêtes. Jamais on n'avait ouï parler de folie ni de fureur qui approchât
de celle-là. Aussi M. le duc d'Orléans fit-il une large distribution de
ces actions à tous les officiers généraux et particuliers, par grades,
employés en la guerre contre l'Espagne. Un mois après on commença à
diminuer les espèces à trois reprises de mois en mois, puis une refonte
générale de toutes. M. le prince de Conti retira forcément le duché de
Mercoeur, que Lassai avait acheté huit cent mille livres. Lassai fut au
désespoir, et la chose se passa de manière qu'elle ne fit pas honneur à
M. le prince de Conti.

La cour de Vienne eut ses orages. Le prince Eugène y était envié\,; son
mérite l'y avait mis à la tête du conseil de guerre, qui est la première
place et de la plus grande autorité. Tout ce qui avait été attaché au
feu prince Herman de Bade et au feu prince Louis son neveu, qui n'avait
pas été sans jalousie de l'éclat naissant du prince Eugène, et qui
malgré ses grandes actions s'en était trouvé obscurci, et tout ce qui
avait tenu au feu duc de Lorraine, était contraire au prince Eugène. Il
se forma donc une cabale puissante, mais qui fut découverte et dissipée
avant que d'avoir pu lui nuire efficacement. En ce même temps le comte
de Koenigseck, ambassadeur de l'empereur ici, fut rappelé pour aller
exercer sa charge de grand maître de la princesse électorale de Saxe, et
Penterrieder vint ici prendre soin des affaires de l'empereur, avec le
simple titre de ministre plénipotentiaire. Il n'était pas d'étoffe à
être élevé même jusque-là, mais sa capacité était fort reconnue.
Koenigseck emporta la réputation d'un homme sage et poli, et qui servait
bien son maître, sans avoir ce rebut de fierté et de roguerie de presque
tous les Impériaux.

M. le duc d'Orléans ne fut pas plus sévère pour le prince Emmanuel,
frère du duc d'Elboeuf, qu'il l'avait été pour Bonneval. La maison
d'Autriche a toujours eu de grands attraits pour la maison de Lorraine.
Sans remonter à la Ligue et aux temps qui en sont voisins, on a vu sous
le feu roi la désertion du prince de Commercy et des fils du prince
d'Harcourt. Le prince d'Elboeuf, traité par le roi avec toute sorte de
bonté, crut faire ailleurs plus de fortune et déserta. Il fut
juridiquement pendu en effigie à la Grève, comme on l'a rapporté ici en
son temps. C'était une manière de brigand, mais à langue dorée, avec
beaucoup d'esprit, qui fit tant de frasques qu'il perdit les emplois
qu'il avait obtenus. Il avait été général de la cavalerie impériale au
royaume de Naples, où il avait épousé, en 1713, Marie-Thérèse, fille
unique de Jean-Vincent Stramboni, duc de Salza, avec qui il vécut fort
mal et n'en eut point d'enfants. Ne sachant plus que devenir ni de quoi
subsister, il obtint des lettres d'abolition et revint. Il mena en
France sa vie accoutumée, et peu à peu s'introduisit à Lunéville, où il
suça le duc de Lorraine tant qu'il put, et il en tira fort gros et même
des terres. Le duc d'Elboeuf le méprisait et le souffrait avec peine, et
ceux de sa maison établis ici n'en faisaient pas plus de cas.

M. le duc d'Orléans fit une distribution de bénéfices qui mérite d'avoir
place ici. Beauvau, d'abord évêque de Bayonne, après de Tournay, puis
archevêque de Toulouse, comme on l'a vu ici en son temps, eut Narbonne.
Son nom et sa conduite méritaient bien ce grand siège\,; mais sa tête
n'était pas assez forte pour être à la tête des états de Languedoc et de
toutes les affaires de ce pays-là. Nesmond, archevêque d'Alby, passa à
Toulouse, et Castries, archevêque de Tours, à Alby. L'abbé de Thesut,
qui avait la feuille des bénéfices depuis la cessation du conseil de
conscience, procura l'archevêché d'Embrun à son parent et son ami
l'évêque d'Alais, qui était Hennin-Liétard, et homme de bien, de savoir
et de mérite. Tours fut donné à l'abbé d'Auvergne. À ce nom, l'abbé de
Thesut s'écria. M. le duc d'Orléans lui dit qu'il avait raison, qu'il ne
voulait pas le lui donner, en déclama autant que l'abbé de Thesut, qui
insista sur le scandale et l'indignité de ce choix. M. le duc d'Orléans
répondit qu'il y avait quatre jours que les Bouillon ne le quittaient
point de vue\,; qu'ils se relayaient\,; qu'ils le persécutaient\,; qu'il
voulait enfin acheter repos.

Un autre sujet aussi bon, mais drôle d'esprit et de manège, eut
Sisteron. Ce fut Lafitau, ce fripon de jésuite qui fit cette course
légère dans la chaise du cardinal de La Trémoille, de Rome à Paris et de
Paris à Rome, pour faire échouer le voyage que le régent avait fait
faire à Rome à l'abbé Chevalier sur la constitution, et qui, par sa
conduite droite, patiente, mais ferme, avait forcé toutes les barricades
qu'on avait multipliées contre lui. Lafitau était aussi chargé de la
secrète négociation personnelle de l'abbé Dubois pour son chapeau, aux
dépens duquel ce bon père entretenait une fille en chambre, en pleine
Rome, et y donnait de fort bons soupers sans s'en cacher beaucoup, à ce
que m'a conté à moi-même le cardinal de Rohan, et que les jésuites, dont
ce compère était parvenu par ses intrigues à s'en faire craindre et
ménager, n'osaient souffler. Ce que j'ai admiré, c'est que, depuis que
le cardinal de Rohan m'eut fait ce récit et que Lafitau fut évêque, il
le fit prêcher un carême devant le roi, qui alors était à Versailles.
L'abbé Dubois découvrit que Lafitau le trahissait au lieu de le servir.
Il n'osa éclater, dans l'état douteux où il était encore, contre un
homme à tout faire et qui avait son secret\,; mais il songea à
l'éloigner de Rome sans le rapprocher de Paris, et le tenir ainsi à
l'écart. C'est ce qui lui fit donner l'évêché de Sisteron, à son extrême
déplaisir. Il se plaignit amèrement. Il lui fâchait beaucoup de cesser
d'être personnage et libertin à son gré pour un aussi petit morceau et
si reculé. Aussi voulut-il refuser\,; mais il fut apaisé à force
d'espérances, et quand il fut à Sisteron on l'y laissa. Les jésuites,
dont la politique ne veut point d'évêques de leur compagnie, firent
aussi les fâchés, mais dans le fond bien aises d'être défaits d'un drôle
qui avait su gagner l'indépendance et leur forcer la main. Avranches fut
donné à un frère de Le blanc, secrétaire d'État, qui était moine et curé
de Dammartin.

\hypertarget{chapitre-xiv.}{%
\chapter{CHAPITRE XIV.}\label{chapitre-xiv.}}

1719

~

{\textsc{Mississipi tourne les têtes.}} {\textsc{- Law se veut pousser,
et pour cela se faire catholique.}} {\textsc{- L'abbé Tencin l'instruit
et reçoit sans bruit son abjuration.}} {\textsc{- Digression sur cet
abbé et sa soeur la religieuse.}} {\textsc{- Caractère de celle-ci.}}
{\textsc{- Elle devient maîtresse de l'abbé Dubois.}} {\textsc{-
Caractère de l'abbé Tencin.}} {\textsc{- Il va à Rome pour le chapeau de
l'abbé Dubois\,; est admonesté en plein parlement en partant.}}
{\textsc{- Law achète l'hôtel Mazarin et y établit sa banque.}}
{\textsc{- Mort de Conflans\,; du célèbre P. Quesnel\,; de Blécourt dont
Louville obtient le gouvernement de Navarreins.}} {\textsc{- Mort de la
princesse de Guéméné.}} {\textsc{- Retour du maréchal de Berwick.}}
{\textsc{- Porteurs de lettres en Espagne arrêtés.}} {\textsc{-
Vaisseaux espagnols aux côtes de Bretagne.}} {\textsc{- Bretons en
fuite\,; d'autres arrêtés.}} {\textsc{- Profusions du régent.}}
{\textsc{- Prince d'Auvergne épouse une aventurière anglaise.}}
{\textsc{- Law se fait garder chez lui.}} {\textsc{- Caractère et
fortune de Nangis et de Pezé, qui obtient le régiment du roi
d'infanterie, et Nangis force grâces.}} {\textsc{- Ma situation avec
Fleury, évêque de Fréjus, avant et depuis qu'il fut précepteur.}}
{\textsc{- Caractère de M\textsuperscript{me} de Lévi.}} {\textsc{- Je
propose à M. de Fréjus une manière singulière, aisée, agréable et utile
d'instruction pour le roi, et je reconnais tôt qu'il ne lui en veut
donner aucune.}} {\textsc{- Je m'engage à faire Fréjus cardinal.}}
{\textsc{- Grâces pécuniaires au duc de Brancas.}} {\textsc{- Six mille
livres de pension à Béthune, chef d'escadre.}} {\textsc{- Torcy obtient
l'abbaye de Maubuisson pour sa soeur.}} {\textsc{- Madame de Bourbon,
depuis abbesse de Saint-Antoine\,; quelle.}} {\textsc{- Mort et état de
l'abbé Morel.}}

~

La banque de Law et son Mississipi étaient lors au plus haut point. La
confiance y était entière. On se précipitait à changer terres et maisons
en papier, et ce papier faisait que les moindres choses étaient devenues
hors de prix. Toutes les têtes étaient tournées. Les étrangers enviaient
notre bonheur, et n'oubliaient rien pour y avoir part. Les Anglais même,
si habiles et si consommés en banques, en compagnies, en commerce, s'y
laissèrent prendre, et s'en repentirent bien depuis. Law, quoique froid
et sage, sentit broncher sa modestie. Il se lassa d'être subalterne. Il
visa au grand parmi cette splendeur, et plus que lui, l'abbé Dubois pour
lui, et M. le duc d'Orléans\,; néanmoins il n'y avait aucun moyen pour
cela qu'on n'eût rangé deux obstacles la qualité d'étranger et celle
d'hérétique, et la première ne pouvait se changer par la naturalisation
sans une abjuration préalable. Pour cela il fallut un convertisseur qui
n'y prît pas garde de si près, et duquel on fût bien assuré avant de s'y
commettre. L'abbé Dubois l'avait tout trouvé, pour ainsi dire, dans sa
poche. C'était l'abbé Tencin que le diable a poussé depuis à une si
étonnante fortune (tant il est vrai qu'il sort quelquefois de ses règles
ordinaires pour bien récompenser les siens, et par ces exemples
éclatants en éblouir d'autres et se les acquérir), que je ne puis me
refuser de m'y étendre.

Cet abbé Tencin était prêtre et gueux, arrière-petit-fils d'un orfèvre,
fils et frère de présidents au parlement de Grenoble. Guérin était son
nom et Tencin celui d'une petite terre qui servait à toute la famille.
Il avait deux soeurs l'une qui a passé sa vie à Paris dans les
meilleures compagnies, femme d'un Ferriol assez ignoré, frère de Ferriol
qui a été ambassadeur à Constantinople, qui n'a point été marié\,;
l'autre soeur religieuse professe pendant bien des années dans les
Augustines de Montfleury aux environs de Grenoble, toutes deux belles et
fort aimables\,; M\textsuperscript{me} Ferriol avec plus de douceur et
de galanterie, l'autre avec infiniment plus d'esprit, d'intrigue et de
débauche. Elle attira bientôt la meilleure compagnie de Grenoble à son
couvent, dont la facilité de l'entrée et de la conduite ne put jamais
être réprimée par tous les soins du cardinal Le Camus. Rien n'y
contribuait davantage que l'agrément et la commodité de trouver au bout
de la plus belle promenade d'autour de Grenoble un lieu de soi-même
charmant, où toutes les meilleures familles de la ville avaient des
religieuses. Tant de commodités, dont M\textsuperscript{me} Tencin abusa
largement, ne firent que lui appesantir le peu de chaînes qu'elle
portait. On la venait trouver avec tout le succès qu'on eût pu désirer
ailleurs. Mais un habit de religieuse, une ombre de régularité quoique
peu contrainte, une clôture bien qu'accessible à toutes les visites des
deux sexes, mais d'où elle ne pouvait sortir que de temps en temps,
était une gêne insupportable à qui voulait nager en grande eau, et qui
se sentait des talents pour faire un personnage par l'intrigue. Quelques
raisons pressantes de dérober la suite de ses plaisirs à une communauté
qui ne peut s'empêcher de se montrer scandalisée des éclats du désordre
et d'agir en conséquence, hâtèrent la Tencin de sortir de son couvent
sous quelque prétexte, avec ferme résolution de n'y plus retourner.

L'abbé Tencin et elle ne furent jamais qu'un coeur et qu'une âme par la
conformité des leurs, si tant est que cela se puisse dire en avoir. Il
fut son confident toute sa vie\,; elle de lui. Il sut la servir si bien
par son esprit et ses intrigues qu'il la soutint bien des années au
milieu de la vie, du monde, des plaisirs et des désordres, dont il
prenait bien sa part, dans la province, et jusqu'au milieu de Paris,
sans avoir changé d'état\,; elle fit même beaucoup de bruit par son
esprit et par ses aventures sous le nom de la religieuse Tencin. Le
frère et la soeur, qui vécurent toujours ensemble, eurent l'art que
personne ne l'entreprît sur cette vie vagabonde et débauchée d'une
religieuse professe, qui en avait même quitté l'habit de sa seule
autorité. On ferait un livre de ce couple honnête, qui ne laissèrent pas
de se faire des amis par leur agrément extérieur et par les artifices de
leur esprit. Vers la fin de la vie du roi ils trouvèrent enfin moyen
d'obtenir de Rome un changement d'état, et de religieuse la faire
chanoinesse, je ne sais d'où et où elle n'alla jamais. Cette solution
demeura imperceptible en nom, en habit, en conduite, et ne fit ni bruit
ni changement. C'est l'état où elle se trouva à la mort du roi. Bientôt
après elle devint maîtresse de l'abbé Dubois, et ne tarda guère à
devenir sa confidente, puis la directrice de la plupart de ses desseins
et de ses secrets. Cela demeura assez longtemps caché, et tant que la
fortune de l'abbé Dubois eut besoin de quelques mesures\,; mais depuis
qu'il fut archevêque, encore plus lorsqu'il fut cardinal, elle devint
maîtresse publique, dominant chez lui à découvert, et tenant une cour
chez elle, comme étant le véritable canal des grâces et de la fortune.
Ce fut donc elle qui commença celle de son frère bien-aimé\,; elle le
fit connaître à son amant secret, qui ne tarda pas à le goûter comme un
homme si fait exprès pour le seconder en toutes choses, et lui être
singulièrement utile.

L'abbé Tencin avait un esprit entreprenant et hardi qui le fit prendre
pour un esprit vaste et mâle. Sa patience était celle de plusieurs vies
et toujours agissante vers le but qu'il se proposait, sans s'en
détourner jamais, et surtout incapable d'être rebutée par aucune
difficulté\,; un esprit si fertile en ressorts et en ressources qu'il en
acquit faussement la réputation d'une grande capacité\,; infiniment
souple, fin, discret, doux ou âpre selon le besoin, capable sans effort
de toutes sortes de formes, maître signalé en artifices, retenu par
rien, contempteur souverain de tout honneur et de toute religion, en
gardant soigneusement les dehors de l'un et de l'autre\,; fier et abject
selon les gens et les conjonctures, et toujours avec esprit et
discernement\,; jamais d'humeur, jamais de goût qui le détournât le
moins du monde, mais d'une ambition démesurée\,; surtout altéré d'or,
non par avarice ni par désir de dépenser et de paraître, mais comme voie
de parvenir à tout dans le sentiment de son néant. Il joignait quelque
légère écorce de savoir à la politesse, et aux agréments de la
conversation des manières et du commerce, une singulière accortise et un
grand art de cacher ce qu'il ne voulait pas être aperçu, et à distinguer
avec jugement entre la diversité des moyens et des routes. Ce ne fut
donc pas merveilles si, produit et secondé par une soeur maîtresse du
ministre effectivement déjà dominant, il fut admis par ce ministre avec
lequel il avait de si naturels rapports, et en même temps si essentiels.
Tel fut l'apôtre d'un prosélyte tel que Law que lui administra l'abbé
Dubois. Leur connaissance était déjà bien faite. La soeur, dont le
crédit n'était pas ignoré de Law dès le commencement de l'amour de
l'abbé Dubois pour elle, n'avait pas négligé de se l'acquérir. Elle
n'était plus débauchée que par intérêt et par ambition avec un reste
d'habitude. Elle avait trop d'esprit pour ne pas sentir qu'à son âge et
à son état, une ambition personnelle ne pouvait la mener bien loin. Son
ambition était donc toute tournée sur ce cher frère, et suivant son
principe, elle le fit gorger par Law, et le gorgé sut de bonne heure
mettre son papier en or. Ils en étaient là quand il fut question de
ramener au giron de l'Église un protestant ou anglican\,; car lui-même
ne savait guère ce qu'il était. On peut juger que l'oeuvre ne fut pas
difficile, mais ils eurent le sens de la faire et de la consommer en
secret, de sorte que ce fut quelque temps un problème, et qu'ils
sauvèrent par ce moyen les bienséances du temps de l'instruction et de
la persuasion, et une partie du scandale et du ridicule d'une telle
conversion opérée par un tel convertisseur.

Quelque habile à se couvrir que fût l'abbé Tencin, ses débauches et ses
diverses aventures l'avaient déshonoré dans le bas étage, parmi lequel
il avait vécu. Sa réputation d'ailleurs avait beaucoup souffert de celle
de sa soeur et de son identité avec elle. Il n'avait pu dérober toutes
leurs aventures au public, il en avait eu d'autres pour des marchés de
bénéfices qui avaient transpiré. On savait aussi, quoique en gros, qu'il
avait tiré immensément de Law. Enfin il lui avait été impossible de
cacher jusqu'alors ses pernicieux talents à tout le monde. Il y passait
aussi pour un scélérat très dangereux que son esprit ployant et ses
grâces rendaient agréable dans un certain commerce général, où il était
souffert par ceux qui le connaissaient, et désiré par ceux qui, n'étant
pas instruits, se prenaient aisément par des dehors flatteurs. Choisi
par l'abbé Dubois pour succéder à Lafitau, et aller à Rome presser sa
pourpre encore fort secrète, il dédaigna d'accommoder un procès qui lui
était intenté en simonie par l'abbé de Vessière, et de plus en
friponnerie pour avoir dérobé une partie du marché qu'il avait fait d'un
prieuré. Dans la faveur où il se trouvait, et à la veille d'aller à Rome
par ordre apparent du régent, mais en effet par celui de l'abbé Dubois
déjà devenu redoutable, il ne put soupçonner que sa partie osât le
pousser, aussi peu que le parlement imaginât de le condamner dans la
brillante position où il était. Ce brillant même l'aveugla, et n'effraya
point sa partie, qui poussa le procès à la grand'chambre. Tencin le
soutint\,; il fit du bruit, le bruit se répandit et devint un objet de
curiosité. La cause était à l'audience du matin à la grand'chambre.
Plusieurs personnes voulurent se divertir de ce qui se passerait à ce
jugement dont le jour fut su. M. le prince de Conti, dont la malice ne
dédaignait aucune occasion de se signaler, y entraîna quelques pairs qui
prirent leurs places en séance avec lui et d'autres gens de qualité qui
remplirent les lanternes et le banc des gens du roi, lesquels étaient
présents en leurs places. Aubry, avocat, qui plaidait contre l'abbé
Tencin, poussa le sien et l'engagea peu à peu en des assertions assez
fortes. Le premier, qui avait son dessein, faiblit, l'autre reprit des
forces, sur quoi le premier avocat l'engagea doucement à des négatives.
Le premier répliqua qu'elles étaient sèches et ne prouvaient rien,
destituées de preuves, à moins que Tencin là présent ne les attestât par
serment. Cette dispute, qui donnait gain de cause à l'abbé en faisant
serment, lui parut une ouverture à saisir pour le gain certain de sa
cause. Il se leva, demanda la permission de parler et l'obtint. Il parla
donc et très bien, s'écria à l'injure et à la calomnie, protesta qu'il
n'avait jamais traité du prieuré dont il s'agissait, négative qui
emportait la friponnerie dont il était accusé, puisqu'elle ne pouvait
porter que sur un marché qu'il protestait être faux, et déclara enfin
qu'il était prêt de lever la main s'il plaisait à la cour, et de
l'affirmer tel, et qu'il n'en avait jamais fait aucun. C'était où
l'attendait sa partie et le piège qu'elle lui avait tendu. L'avocat qui
en avait eu l'adresse le provoqua au serment sur l'offre qu'il en
faisait lui-même\,; il la réitéra, et dit qu'il n'attendait pour le
faire que la permission de la cour. «\,Ce n'est pas la peine, dit alors
ce même avocat, puisque vous y êtes résolu, et que vous l'offrez de si
bonne grâce. Voilà, ajouta-t-il, en secouant sa manche, qui cachait sa
main et un papier qu'elle tenait, voilà une pièce entièrement décisive,
dont je demande à la cour de faire la lecture\,;» et tout de suite il la
fit. C'était le marché original du prieuré, signé de l'abbé Tencin, qui
prouvait la simonie et la friponnerie à n'avoir pas un mot à répliquer.
La pièce passa aussitôt entre les mains des juges, qui furent indignés
de la scélératesse et de la hardiesse de Tencin. L'auditoire en frémit,
qui, excité par M. le prince de Conti, fit une risée et une huée à
plusieurs reprises. Tencin, confondu, perdit toute contenance, fit le
plongeon, et tenta de s'évader\,; mais sa partie, qui s'était flattée de
l'enferrer comme elle fit, s'était à tout événement pourvu de trois ou
quatre gaillards, qui, sans faire semblant de rien, s'étaient mis à
portée de l'abbé, et l'empêchèrent de sortir de sa place. Cependant
Mesmes, premier président, alla aux opinions, qui ne durèrent qu'un
instant, et où M. le prince de Conti ni les pairs qu'il avait menés ne
furent point, parce qu'ils n'avaient pas assisté aux plaidoiries
précédentes. Le premier président remis en place prononça un arrêt
sanglant contre Tencin avec dépens et amende, qui est une flétrissure,
puis fit avancer Tencin, et l'admonesta cruellement sans épargner les
termes les plus fâcheux, et de la voix la plus intelligible. Il la finit
par le condamner à une aumône, qui est une peine infamante. Alors les
huées recommencèrent\,; et, comme il n'y avait plus rien à ajouter,
l'abbé Tencin ne trouva plus d'obstacle pour se couler honteusement dans
la presse et se dérober aux regards des honnêtes gens et aux insultes de
la canaille. Ce jugement se répandit à l'instant par tout Paris avec
l'éclat et le scandale qui en était inséparable.

Tout autre que l'abbé Dubois aurait changé d'agent pour Rome, mais
celui-ci se trouvait tellement à son point et dans ses moeurs, et ses
talents lui semblèrent si difficiles à rassembler dans un autre, qu'il
le fit partir dès le lendemain pour le faire disparaître, et par là
faire cesser plus tôt ce que sa présence eût renouvelé. Dubois eut
raison sans doute. Ce n'était ni du mérite ni de la vertu qu'il
attendait le cardinalat. Son négociateur était supérieur à tout autre
pour faire valoir utilement l'or, l'intrigue et les divers ressorts où
l'abbé Dubois avait établi toutes ses espérances. Les manèges de son
agent à Rome se trouveront en leur lieu. Law fut fort touché d'une
aventure si infâme et si publique arrivée à son convertisseur, qui ne
fit pas honneur à sa conversion, qui avait déjà bien fait parler le
monde. Il acheta un million l'hôtel Mazarin\footnote{Cet hôtel renferme
  maintenant la Bibliothèque impériale. La partie la plus ancienne avait
  été bâtie par Chevery ou Chevry, président à la chambre des Comptes.
  Jacques Tubeuf, président à la même chambre, fit bâtir une autre
  partie sur les desservis de l'architecte Le Muet. Mazarin acheta de
  Tubeuf ces maisons qui répondent aux bâtiments portant les numéros 14,
  12 et 10 de la rue des Petits-Champs. Derrière ces maisons il y avait
  de vastes jardins. Mazarin agrandit considérablement l'hôtel Tubeuf\,;
  il fit bâtir la grande galerie qui longe la rue de Richelieu et
  s'étend de la rue des Petits-Champs à l'arcade Colbert, ainsi que la
  galerie Mazarine, qui contient une des parties les plus précieuses des
  manuscrits de la Bibliothèque impériale. Voy. \emph{l'Histoire du
  Palais Mazarin}, par M. le comte L. de La Borde.} pour y mettre sa
banque qui avait été jusqu'alors dans la maison qu'il louait pour cela
du premier président, et dont il n'avait pas besoin par sa place qui
donne un magnifique logement au palais aux premiers présidents du
parlement. Law acheta en même temps cinq cent cinquante mille livres la
maison du comte de Tessé. Conflans, homme de beaucoup d'esprit et de
savoir, mourut assez jeune. Il exerçait une des deux charges de premier
gentilhomme de la chambre de M. le duc d'Orléans pour le fils encore
enfant d'Armentières son frère qui l'avait, et cet enfant après sa mort.
Le chevalier de Conflans, troisième frère, en eut l'exercice, très
savant aussi, avec beaucoup d'esprit.

Le fameux P. Quesnel mourut à Amsterdam où la persécution l'avait fait
retirer. Si la violence lui avait refusé d'être écouté sur son livre si
singulièrement condamné par la constitution \emph{Unigenitus}, et refusé
plusieurs fois malgré toutes ses instances, ses lettres au pape et toute
la soumission la plus entière, chose qu'on ne refuse pas aux hérétiques
ni aux hérésiarques qu'on presse même de s'expliquer, il eut au moins la
consolation d'avoir vécu et de mourir en bon catholique, et de faire en
mourant une profession de foi qui fut aussitôt rendue publique, et qui
se trouva tellement orthodoxe qu'on ne put jamais y toucher. Ce savant
homme et si éclairé s'est acquis une si grande réputation partout, que
je ne m'y étendrai pas davantage. Il avait plus de quatre-vingts ans et
travaillait toujours dans la solitude, la prière et la pénitence.

Blécourt mourut fort vieux. C'était un ancien officier fort attaché au
maréchal d'Harcourt qui l'avait mené avec lui en Espagne. Il y fut
chargé des affaires du roi pendant les absences d'Harcourt, et il était
seul à Madrid à la mort de Charles II, comme on l'a vu ici en son temps.
Le gouvernement des Navarreins qu'il avait fut donné à Louville.

La princesse de Guéméné qui était Vaucelas, mourut en même temps encore
assez jeune.

Le maréchal de Berwick, qui avait fini sa campagne par la prise d'Urgel
et de Rose, arriva. On arrêta des gens au pied des Pyrénées, qui
cherchaient à se couler en Espagne par des chemins détournés. On les
trouva chargés de beaucoup de lettres\,: c'est tout ce qu'on en a su. La
politique de l'abbé Dubois, qui a été expliquée en son lieu, sur le duc
et la duchesse du Maine, fit un secret et des lettres et de qui elles
étaient. Cela fut étouffé sous un air de mépris. Je ne pris pas la peine
d'en parler à M. le duc d'Orléans. Je crois que je le soulageai, car il
ne m'en parla qu'en ce sens et en passant.

Il résolut pourtant et travailla bientôt après à une grande augmentation
de troupes, dont il ne fut pas longtemps à reconnaître qu'il n'avait pas
besoin. Il avait paru sur les côtes de Bretagne quelques vaisseaux
espagnols. Le maréchal de Montesquiou fit marcher des troupes pour leur
empêcher le débarquement. Sur quoi, après diverses tentatives, ils se
retirèrent. C'étaient des vaisseaux de guerre qu'on sut chargés de
troupes de débarquement et de beaucoup d'armes. Noyan, gentilhomme de
Bretagne qui avait été exilé et rappelé, et qui était à Paris, fut mis à
la Bastille. Peu de jours après les femmes de Bonamour et de Landivy,
dont les maris étaient en fuite, furent arrêtées en Bretagne. Pontcallet
s'en sauva en même temps. On courut inutilement après lui.

M. le duc d'Orléans ne se laissait point de profusions ni de faire des
ingrats. Il donna plus de quatre cent mille livres à la maréchale de
Rochefort, dame d'honneur de M\textsuperscript{me} la duchesse
d'Orléans\,; cent mille livres à Blansac, son gendre\,; autant à la
comtesse de Tonnerre sa petite-fille\,; trois cent mille livres à La
Châtre\,; autant au duc de Tresmes\,; deux cent mille livres à Rouillé
du Coudray, conseiller d'État, qui avait été l'âme des finances sous le
duc de Noailles\,; cent cinquante mille livres au chevalier de
Marcieu\,; enfin à tant d'autres que j'oublie ou que j'ignore que cela
ne peut se nombrer\,; sans ce que ses maîtresses et ses roués lui en
arrachaient, et de plus, lui en prenaient les soirs dans ses poches, car
tous ces présents étaient en billets qui valaient tout courant leur
montant en or, mais qu'on lui préférait.

Cette soif de l'or fit faire un singulier mariage au prince d'Auvergne,
nom que le chevalier de Bouillon avait pris depuis quelque temps. Une
M\textsuperscript{me} Trent, Anglaise, qui se disait demoiselle, et
prétendait être à Paris à cause de la religion, s'était fourrée par là
chez M\textsuperscript{me} d'Aligre, de laquelle j'ai parlé plus d'une
fois. Elle retira chez elle cette fille d'abord par charité, et la garda
longtemps, charmée de son ramage. Elle ne tarda pas à se faire connaître
par ses intrigues et par son esprit souple, liant, entreprenant, hardi,
qui surtout voulait faire fortune. Elle attrapa lestement force
Mississipi de Law, qu'elle sut faire très bien valoir. Ce grand bien
donna dans l'oeil au prince d'Auvergne, qui avait tout fricassé. Il
cherchait à se marier sans pouvoir trouver à qui\,; le décri profond et
public où ses débauches l'avaient fait tomber, et d'autres aventures
fort étranges, ni sa gueuserie n'épouvantèrent point l'aventurière
anglaise. Le mariage se fit au grand déplaisir des Bouillon. Elle mena
toujours depuis son mari par le nez, et acquit avec lui des richesses
immenses par ce même Mississipi. Il est pourtant mort avec peu de bien,
parce qu'il avait été soulagé de presque tout son portefeuille que sa
femme avait eu l'adresse de lui faire prêter, et qu'elle a été fort
accusée d'avoir mis de côté. Quoi qu'il en soit, il a été perdu pour le
mari et pour les siens, sans moyens contre la femme qui en demeura
brouillée avec tous les Bouillon et qui n'a point eu d'enfants qui aient
vécu. Elle chercha, avant et depuis la mort de son mari, à faire un
personnage, mais la défiance la fit rejeter partout. Elle se retrancha
donc sur la dévotion, la philosophie, la chimie qui la tua à la fin, au
bel esprit surtout, dans un très petit cercle de ce qu'elle put à faute
de mieux. Avec tout ce florissant Mississipi, il y eut des avis qu'on
voulait tuer Law, sur quoi on mit seize Suisses du régiment des gardes
chez lui, et huit chez son frère qui était depuis quelque temps à Paris.

J'ai différé à ce temps, où Pezé\footnote{Ce nom s'écrit ordinairement
  Pezay.} eut enfin le régiment du roi infanterie, à parler plus à fond
de lui et de Nangis qui le lui vendit, parce que tous deux ont fait en
leur temps une fortune singulière. Celui-ci, porté haut sur les ailes de
l'amour et de l'intrigue, déchut toujours\,; celui-là avec peu de
secours, mais par de grands talents, monta toujours, et par eux touchait
à la plus haute et à la plus flatteuse fortune, lorsque, arrêté au
milieu de sa course, il mourut au lit d'honneur environné de gloire et
d'honneurs qui, lui promettant les plus élevés et les plus distingués,
lui laissèrent en même temps voir la vanité des fortunes et le néant de
ce monde.

Nangis, avec une aimable figure dans sa jeunesse, le jargon du monde et
des femmes, une famille qui faisait elle-même le grand monde, une valeur
brillante et les propos d'officier niais sans esprit et sans talent pour
la guerre, une ambition de toutes les sortes et de cette espèce de
gloire sotte et envieuse qui se perd en bassesses pour arriver, a
longtemps fait une figure flatteuse et singulière par l'élévation de ses
heureuses galanteries et par le grand vol des femmes, du courtisan, de
l'officier. Ce groupe tout ensemble forma un nuage qui le porta
longtemps avec éclat, mais qui, dissipé par l'âge et par les
changements, laissa voir à plein le tuf et le squelette. Il avait le
régiment d'infanterie du roi, qui sous le feu roi était un emploi de
grande faveur, et qui semblait devoir mener à la fortune par les
distinctions et l'affection particulière qu'il donnait à ce régiment
par-dessus tout autre, et par les privances attachées à l'état du
colonel qui travaillait directement avec le roi sur tous les détails de
ce corps, sur lequel nul inspecteur ni le secrétaire d'État de la guerre
n'avaient rien à voir. Après la mort du roi, l'âge de son successeur et
l'incertitude éloignée du goût et du soin qu'il prendrait de ce régiment
dégoûtèrent Nangis. On a vu ici en son temps qu'il le voulut vendre au
duc de Richelieu, puis à Pezé, et de quelle façon capricieuse et pire il
cessa de le vouloir vendre. Il ne lui avait rien coûté, non plus qu'à
ses prédécesseurs, et le vendre était une grâce que M. le duc d'Orléans
aurait bien pu, pour ne pas dire dû, se passer de lui faire. On a vu
aussi en son lieu comment et pourquoi j'y étais fort entré pour Pezé,
auquel il faut venir maintenant, aux dépens peut-être de quelque
répétition, pour mettre mieux le tout ensemble.

Pezé était du pays du Maine, bien gentilhomme mais tout simple, parent
éloigné du maréchal de Tessé par la généalogie et tout au plus près par
la galanterie\,: il avait une mère que le maréchal avait trouvée
aimable. Pezé était un cadet\,; il en prit soin et le mit de fort bonne
heure page de M\textsuperscript{me} la duchesse de Bourgogne dont il
était premier écuyer. Courtalvert, frère aîné de Pezé, avait du bien,
mais pour soi seul, et plantait ses choux chez lui. Leur grand-père
avait épousé la fille aînée d'Artus de Saint-Gelais, seigneur de Lansac
et d'une fille du maréchal de Souvré dont la famille s'était crue
heureuse de se défaire honnêtement de la sorte par la disgrâce de son
corps, et le mari qui la prit s'estima très honoré de faire cette
alliance à quelque prix que ce fût. L'autre fille de M. et de
M\textsuperscript{me} de Lansac épousa Louis de Prie, seigneur de Toucy,
et de ce mariage vint M\textsuperscript{me} de Bullion, grand-mère de
Fervaques, chevalier de l'ordre en 1724, et la maréchale de La Mothe,
laquelle était ainsi cousine germaine du père de Pezé, et lui, par
conséquent, issu de germain des duchesses d'Aumont, mère du duc
d'Humières, de Ventadour et de La Ferté, toutes trois filles de la
maréchale de La Mothe. Cette alliance si proche le tira du régiment des
gardes où il était entré en sortant de page, et le fit gentilhomme de la
manche du roi. C'était un jeune homme de figure commune avec beaucoup
d'esprit et de physionomie, plein de manèges, d'adresses, de finesse, de
ressources dans l'esprit, liant et agréable, le ton du grand monde et de
la bonne compagnie où il était agréable et bien reçu, et d'une ambition
qui lui fit trouver toutes sortes de talents pour arriver à la plus
haute fortune. Il fit si bien qu'il persuada au monde que le roi l'avait
pris en amitié, que cette raison le fit compter, lui acquit des amis
considérables à qui il ne manqua jamais en aucun temps, et lui fraya le
chemin à tout. Je crois avoir reçu la dernière lettre qu'il ait jamais
écrite\,; il m'a vu toujours très soigneusement et m'a toujours parlé de
tout à coeur ouvert. On a vu en son temps que le duc d'Humières fit que
je lui fis obtenir le gouvernement de la Muette dès que le roi eut cette
maison, puis le régiment du roi quand Nangis eut la permission de le
vendre, et Pezé ne l'oublia jamais. Enfin Nangis, lassé de ne point
vendre, chercha à profiter du désir de Pezé et de l'incroyable facilité
de M. le duc d'Orléans, à laquelle je n'eus point de part, mais bien à
l'agrément d'acheter exclusif de tout autre. Pezé donna donc cent vingt
mille livres desquelles Nangis donna soixante-cinq mille livres à
Saint-Abre, qui, moyennant cette somme, lui céda le gouvernement de
Salces, en Languedoc, qu'il avait. Il était de dix mille livres
d'appointements, il fut mis à seize mille livres en même temps pour
Nangis qui, outre sa pension de six mille livres comme colonel du
régiment du roi qui lui fut conservée, en eut une autre pour son frère
le chevalier de Nangis, de quatre mille livres, qui était capitaine de
vaisseau. Saint-Abre eut par le marché une pension du roi de cinq mille
livres, dont deux mille livres furent assurées à une de ses filles après
lui. Ainsi Nangis tira plus de quinze mille livres de rente de ce qui ne
lui avait jamais rien coûté et qu'il désirait de vendre, et avec cela
fut assez sot pour m'en bouder toute sa vie, et fit le mécontent. Aussi
lui et Pezé n'ont jamais été bien ensemble.

Nangis, à force de restes mourants de sa figure passée, devint pour rien
chevalier d'honneur de la reine à son mariage, sans cesser de servir,
fut chevalier de l'ordre, et quoique sans considération, et ayant paru
un très ignorant officier général, son ancienneté parmi les autres
pouliée par sa charge, le fit enfin maréchal de France, pour ne point
servir et achever sa vie sans considération et comme dans la solitude au
milieu de la cour, s'ennuyant et ennuyant les autres, et ne paraissant
guère que pour les fonctions journalières de sa charge. Pezé, au
contraire, passé en Italie avec le régiment du roi, y montra tant de
talents naturels pour la guerre qu'il y saisit d'abord toute la
confiance des généraux des armées, et devint en très peu de temps l'âme
des projets et des exécutions. Il força par sa valeur et par ses
lumières l'envie à lui rendre justice. Il mourut des blessures qu'il
avait reçues à la bataille de Guastalla, avec l'ordre du Saint-Esprit
qui lui fut envoyé en récompense de tout ce qu'il avait fait en Italie,
et il allait rapidement au commandement en chef des armées comme
généralement reconnu le plus capable, à quoi il s'était élevé en fort
peu de temps.

Pezé me fait souvenir, et on verra bientôt pourquoi, que j'ai dépassé le
temps où je devais rapporter la situation où Fleury, évêque de Fréjus,
et moi, étions ensemble. Ses allures, ses sociétés et les miennes du
vivant du feu roi, furent toujours différentes. Quoique nous eussions
des amis communs, il n'y avait nul commerce entre nous, mais sans aucun
éloignement de part et d'autre, et politesse quand nous nous
rencontrions. À la fin de son dernier voyage à la cour, vers la fin de
la vie du feu roi, je le rencontrai assez souvent chez
M\textsuperscript{me} de Saint-Géran\,; il brassait alors bien
sourdement la place de précepteur\,; il sentit apparemment que je
pourrais quelque chose dans la régence que tout le monde voyait
s'approcher de plus en plus par l'état où le roi paraissait. Le prélat
me parut me rechercher, mais avec adresse, et je répondis avec civilité,
mais sans passer les termes de conversations et de plaisanteries
générales et indifférentes et sans nous chercher. Revenu démis de son
évêché et précepteur, nous nous trouvâmes occupés tous deux à des choses
différentes. Vincennes fit encore une séparation de lieu, et il se passa
encore quelques mois après l'arrivée du roi à Paris sans que nous nous
approchassions l'un de l'autre que par des civilités générales et
passagères, quand rarement nous nous rencontrions. J'eus lieu de croire
que cela ne satisfit pas M. de Fréjus.

On a vu ici toute la part qu'eut M\textsuperscript{me} de Lévi à le
faire précepteur. C'était une femme de beaucoup d'esprit, vive à
l'excès, toujours passionnée, et ne voyant ni gens ni choses qu'à
travers la passion, qui en bien ou en mal la possédait sur les choses et
sur les personnes\,; elle s'était donc coiffée de M. de Fréjus, en
vérité jusqu'à la folie, en vérité aussi en tout bien et honneur\,; car
cette femme, avec tous ses transports d'affection ou du contraire, était
foncièrement pétrie d'honneur, de vertu, de religion et de toute
bienséance. Elle était fille du feu duc de Chevreuse, par conséquent
intimement mon amie, et de tout temps dans la plus étroite liaison avec
M\textsuperscript{me} de Saint-Simon. Causant un soir avec elle, elle se
mit sur le propos de M. de Fréjus, et me reprocha que je ne l'aimais
point. Je lui en témoignai ma surprise, parce qu'en effet je n'avais
nulle raison de l'aimer, ni de ne l'aimer pas. Le hasard ne me l'avait
point fait rencontrer chez elle dans les derniers temps du feu roi, où
leur amitié se lia, et elle était presque la seule personne fort de mes
amies qui fut la sienne, et depuis la régence, lui et moi occupés de
choses toutes différentes, n'avions point eu d'occasions de nous voir.
Cela ne la satisfit pas\,; elle revint d'autres fois à la charge. Je
jugeai donc que c'était de concert avec M. de Fréjus, qui de loin
voulait ranger tous obstacles. Je répondis toujours honnêtement pour
lui, parce que je n'avais nulle raison de répondre autrement, tellement
qu'enfin il m'attaqua de politesse, puis de courte conversation chez le
roi, et peu de jours après vint chez moi à l'heure du dîner m'en
demander. De là, il vint assez souvent chez moi, souvent aussi dîner, et
je l'allai voir quelquefois les soirs. Il était, comme on l'a dit
ailleurs, de bonne conversation et de bonne compagnie, et il avait passé
sa vie dans le monde le plus choisi. À force de nous voir, les
raisonnements sur bien des choses entrèrent dans nos conversations.

Un soir assez tard que j'étais chez lui, quelque temps après qu'il eut
commencé ses fonctions de précepteur, on lui apporta un paquet. Comme il
était tard, et lui en robe de chambre et en bonnet de nuit au coin de
son feu, je voulus m'en aller pour lui laisser ouvrir le paquet. Il m'en
empêcha, et me dit que ce n'était rien que les thèmes du roi qu'il
faisait faire aux jésuites qui les lui envoyaient. Il avait raison de
prendre ce secours\,; car il ne savait du tout rien que grand monde,
ruelle et galanterie. Sur ce propos des thèmes du roi, je lui demandai,
comme ne l'approuvant pas, s'il projetait de lui mettre bien du latin
dans la tête. Il me répondit que non, mais seulement pour qu'il en sût
assez pour ne l'ignorer pas entièrement\,; et nous convînmes aisément
que l'histoire, surtout celle de France générale et particulière, était
{[}ce{]} à quoi il le fallait appliquer le plus. Là-dessus il me vint
une pensée que je lui dis tout de suite pour apprendre au roi mille
choses particulières et très instructives pour lui dans tous les temps
de sa vie, et en se divertissant, qui ne pouvaient guère lui être
montrées autrement.

Je lui dis que Gaignières, savant et judicieux curieux, avait passé sa
vie en toutes sortes de recherches historiques, et qu'avec beaucoup de
soins, de frais et de voyages qu'il avait fait exprès, il avait ramassé
un très grand nombre de portraits, de ce qui en tout genre et en hommes
et en femmes, avait figuré en France, surtout à la cour, dans les
affaires et dans les armées, depuis Louis XI\,; et de même, mais en
beaucoup moindre quantité dés pays étrangers, que j'avais souvent vus
chez lui en partie, parce qu'il y en avait tant qu'il n'avait pas pu les
placer, quoique dans une maison fort vaste où il logeait seul vis-à-vis
des Incurables\,; que Gaignières en mourant avait donné au roi tout ce
curieux amas \footnote{La Bibliothèque impériale possède encore
  aujourd'hui une partie des portraits et des manuscrits rassemblés par
  Gaignières. On y trouve beaucoup de renseignements curieux sur les
  anciennes institutions de la France.}. Le cabinet du roi aux Tuileries
avait une porte qui entrait dans une belle et fort longue galerie, mais
toute nue. On avait muré cette porte, on avait fait quelques
retranchements de simples planches dans cette galerie, et on y avait mis
les valets du maréchal de Villeroy. Je proposai donc à M. de Fréjus de
leur faire louer des chambres dans le voisinage, à quoi mille francs
auraient été bien loin, d'ouvrir la porte de communication du roi, et de
tapisser toute cette galerie de ces portraits de Gaignières, qui
pourrissaient peut-être dans quelque garde-meuble\,; de dire aux
précepteurs des petits garçons qui venaient faire leur cour au roi, de
parcourir un peu ces personnages dans les histoires et les mémoires, et
de dresser avec soin leurs pupilles à les connaître assez pour en
pouvoir d'abord dire quelque chose, et ensuite avec plus de détail pour
en causer les uns avec les autres, en suivant le roi dans cette galerie,
en même temps que M. de Fréjus en entretiendrait le roi plus à fond\,;
que de cette manière il apprendrait un crayon de suite d'histoire, et
mille anecdotes importantes à un roi qu'il ne pourrait tirer aisément
d'ailleurs\,; qu'il serait frappé de la singularité des figures et des
habillements qui l'aideraient à retenir les faits et les dates de ces
personnages\,; qu'il y serait aiguisé par l'émulation des enfants de sa
cour, les uns à l'égard des autres, et la sienne à lui-même, de savoir
mieux et plus juste qu'eux\,; que le christianisme ni la politique ne
contraindraient en rien sur la naissance, la fortune, les actions, la
conduite de gens, morts eux et tout ce qui a tenu à eux, et que par là,
peu à peu le roi apprendrait les services et les desservices, les
friponneries, les scélératesses, comment les fortunes se font et se
ruinent, l'art et les détours pour arriver à ses fins, tromper,
gouverner, museler les rois, se faire des partis et des créatures,
écarter le mérite, l'esprit, la capacité, la vertu, en un mot les
manéges des cours dont la vie de ces personnages fournissent des
exemples de toute espèce, conduire cet amusement jusque vers Henri IV,
alors piquer le roi d'honneur en lui faisant entendre que ce qui regarde
les personnages au-dessous de cet âge ne doit plus être que pour lui,
parce qu'il en existe encore des familles et des tenants, et tête à tête
les lui dévoiler\,; mais comme il s'en trouve quantité aussi de ceux-là
dont il ne reste plus rien, les petits garçons y pourraient être admis
comme aux précédents\,; enfin, que cela mettrait historiquement dans la
tête du roi mille choses importantes dont il ne sentirait que les
choses, sans s'apercevoir d'instruction, laquelle serait peut-être une
des plus importantes qu'il pût recevoir pour la suite de sa vie, dont la
vue de ces portraits le ferait souvenir dans tous les temps, et lui
acquerrait de plus une grande facilité pour une étude plus sérieuse,
plus suivie, et plus liée de l'histoire, parce qu'il s'y trouverait
partout avec gens de sa connaissance depuis Louis XI, et cela sans le
dégoût du cabinet et de l'étude, et en se promenant et s'amusant. M. de
Fréjus me témoigna être charmé de cet avis, et le goûter extrêmement.
Toutefois il n'en fit rien, et dès lors je compris ce qui arriverait de
l'éducation du roi, et je ne parlai plus à M. de Fréjus de portraits ni
de galerie, où les valets du maréchal de Villeroy demeurèrent
tranquillement.

Il témoignait à Pezé beaucoup d'amitié. Pezé, qui me voyait fort en
liaison avec lui, me proposa de chercher à le faire cardinal\,; si de
lui-même, ou si le prélat lui en avait laissé sentir quelque chose, je
ne l'ai point démêlé. C'étaient deux hommes extrêmement propres à
s'entendre et à se comprendre sans s'expliquer. Pezé voulait que ce fût
à l'insu de M. le duc d'Orléans\,; car la chose ne pouvant s'acheminer
promptement, l'abbé Dubois pouvait croître en attendant, peut-être
quelque autre qui aurait barré Fréjus. Réflexion faite, je crus pouvoir
tâter le pavé, et me conduire suivant ce que je trouverais. On a vu ici
en son lieu l'étroite liaison où j'avais été avec le nonce Gualterio.
Depuis sa promotion au cardinalat et son départ tout de suite, nous
étions en usage de nous écrire toutes les semaines, et assez souvent en
chiffre. Je le dis à Pezé, et que je sonderais le gué par cette voie,
non que le cardinal Gualterio fût en crédit à Rome bastant pour s'en
servir\,; mais il était fort au fait de tout, et propre à indiquer et à
conduire. Cette menée dura plusieurs mois sans beaucoup de moyens ni
d'apparence, jusqu'à ce que Pezé me pria de la part de Fréjus
d'abandonner l'affaire qu'il avait reconnue impossible à cacher au
régent jusqu'au bout, et qui pourrait lui tourner à mal\,; le rare est
que jamais il ne m'en a parlé qu'une fois unique, qui fut pour me dire
lui-même ce que Pezé m'avait dit de sa part, et me remercier à
merveilles sans jamais m'en avoir parlé ni devant ni après, ni moi à
lui. Cela néanmoins serra la liaison de sorte qu'il me parlait de tout
très librement, et qu'il a continué depuis jusqu'à sa mort la même
ouverture sur les gens, les choses, les affaires à un point qui me
surprenait toujours, d'autant plus que ce n'était jamais que récits ou
dissertations sans me demander mon avis sur rien ni encore moins d'envie
de m'approcher ni des affaires ni de la cour, à quoi je lui donnais beau
jeu par n'en avoir pas plus d'envie que lui. Ce court récit suffit
maintenant. Il servira à éclaircir bien des choses qu'il n'est pas
encore temps de raconter.

Le duc de Brancas eut une pension, de l'argent comptant, un logement à
Luxembourg. Béthune, chef d'escadre, eut une pension de six mille
livres, et Torcy obtint pour sa soeur l'abbesse de Panthemont, à Paris,
celle de Maubuisson que M\textsuperscript{me} de Bourbon avait refusée.
Elle était fille aînée de M. le Duc et de M\textsuperscript{me} la
Duchesse, fort contrefaite, fort méchante, avec de l'esprit. Elle était
religieuse de Fontevrault, dont elle voulait être coadjutrice.
M\textsuperscript{me} de Mortemart, qui en était abbesse et qui la
connaissait bien, s'y opposa toujours. À la fin elle vint au
Val-de-Grâce où elle désola le couvent, et fut enfin abbesse de
Saint-Antoine. Elle en traita cruellement les religieuses, dissipa les
biens, quoique avec une forte pension du roi, et en fit tant qu'à la
prière de M\textsuperscript{me} la Duchesse, de M. le Duc son frère, de
toute sa famille, le roi la fit enlever un matin par le duc de Noailles,
capitaine des gardes du corps, et conduire en une petite abbaye, où elle
est demeurée depuis honnêtement prisonnière.

L'abbé Morel mourut fort vieux. C'était un homme d'esprit et fort
instruit que la débauche avait lié avec Saint-Pouange en leur jeunesse,
et toute leur vie le goût du plaisir. Saint-Pouange qui lui reconnut des
talents le fit connaître à Louvois, qui en essaya pour négocier des
affaires secrètes qu'il soufflait tant qu'il pouvait au ministre des
affaires étrangères. Il s'en trouva si bien qu'il en parla au roi, qui
s'en servit souvent depuis la mort de Louvois, et lui parlait souvent
aussi dans son cabinet, où il le faisait venir par les derrières. Il
disparaissait quelquefois, et j'entendais dire qu'on l'avait envoyé en
commission secrète. Le roi et les ministres en furent toujours contents,
et ses voyages furent toujours impénétrables. Il avait pensions et
abbayes, voyait bonne compagnie, paraissait quelquefois à la cour, et le
roi en public lui parlait souvent et avec un air de bonté\,: en son
genre c'était un personnage et un honnête homme aussi.

\hypertarget{chapitre-xv.}{%
\chapter{CHAPITRE XV.}\label{chapitre-xv.}}

1719

~

{\textsc{Promotion de dix cardinaux.}} {\textsc{- Leur discussion.}}
{\textsc{- Spinola, Althan, Perreira.}} {\textsc{- Gesvres.}} {\textsc{-
Sagesse et dignité des évêques polonais.}} {\textsc{- Bentivoglio.}}
{\textsc{- Bossu, dit Alsace, et comment\,; est malmené par
l'empereur.}} {\textsc{- Belluga\,; sa double et sainte magnanimité.}}
{\textsc{- Salerne.}} {\textsc{- Mailly\,; son ambition\,; sa
conduite.}} {\textsc{- Pourquoi les nonces de France, devenant
cardinaux, n'en reçoivent plus les marques qu'en rentrant en Italie.}}
{\textsc{- Tout commerce étroitement et sagement défendu aux évêques,
etc., de France avec Rome, et comment enfin permis.}} {\textsc{- Haine
de Mailly contre le cardinal de Noailles, et ses causes.}} {\textsc{-
Sentiments de Mailly étranges sur la constitution.}} {\textsc{- Comment
transféré d'Arles à Reims.}} {\textsc{- Sa conduite dans ce nouveau
siège.}}

~

Le pape fit une promotion de dix cardinaux dont un réservé \emph{in
petto}. La France n'en eut point, parce que Bissy avait passé sur son
compte dans les derniers temps de la vie du roi, à la faveur de la
constitution. Les neuf déclarés furent Gesvres, archevêque de Bourges
pour la Pologne\,; Mailly, archevêque de Reims, \emph{proprio motu\,;}
Spinola, nonce à Vienne\,; Bentivoglio, nonce à Paris\,; Bossu,
archevêque de Malines, \emph{proprio motu\,;} Perreira y la Cerda pour
le Portugal\,; Althan pour l'empereur, frère de son favori et évêque de
Vaccia\,; Belluga, évêque de Murcie pour l'Espagne, et le P. Salerne
jésuite. Il n'y a point de remarque à faire sur Spinola, nonce à Vienne,
ni sur Althan et Perreira, nommés par l'empereur et par le roi de
Portugal\,; il y en a sur les six autres. On n'en pervertira\footnote{Saint-Simon
  a employé ce mot dans le sens d'\emph{intervertir}.} le rang que sur
Mailly dont on parlera le dernier.

Gesvres avait plus de soixante ans, il avait été jeune à Rome, il s'y
était initié au Vatican. Innocent XI, Odeschalchi, tout ennemi de la
France qu'il fut toujours, l'avait tellement pris en affection qu'il lui
donna une place de camérier d'honneur. Le nouveau prélat sut lui plaire
et à toute sa cour, dont il prit si bien les manières qu'il ne s'en est
jamais défait depuis, habitude, goût ou politique. Tout lui riait à
Rome\,; il y passait pour un des prélats favoris, et qui touchait de
plus près à la pourpre\,; et personne ne douta ni à Rome ni en France
qu'il ne l'eût obtenue à la première promotion, lorsque les démêlés sur
les franchises entre le pape et le feu roi vinrent au point que le
marquis de Lavardin, son ambassadeur à Rome, ne put jamais obtenir
audience, qu'il fût excommunié, et que tous les François eurent ordre de
sortir de Rome. Gesvres obéit comme les autres, mais à son grand regret
et à celui du pape et de toute sa cour. Phélypeaux, archevêque de
Bourges, frère de Châteauneuf, secrétaire d'État, venait de mourir tout
à propos. Bourges fut donné à Gesvres en arrivant pour prix de son
obéissance et de l'abandon de ses espérances à Rome\,; il fut le premier
abbé qui de ce règne fut fait archevêque tout d'un coup\,; il ne regarda
ce poste que comme une planche après le naufrage, et ne songea qu'à s'en
faire un échelon pour arriver où il tendait, aussitôt que les affaires
seraient accommodées entre la France et Rome. Il perdit son protecteur
en Innocent XI. Ottobon, qui lui succéda sous le nom d'Alexandre VIII,
fit passer le roi par où il voulut, puis se moqua de lui. Son pontificat
fut trop court pour donner lieu à Gesvres de travailler utilement pour
soi. Pignatelli, dit Innocent XII, lui succéda, régna plus longtemps. Il
témoigna de l'estime et de la bonté à Gesvres, mais il n'était plus à
Rome ni dans la prélature. Gesvres sentait qu'il lui fallait une
nomination. Il n'oublia rien pour se lier étroitement avec Pomponne,
Croissy et Torcy, fils du dernier, gendre de l'autre, qui avaient en
commun les affaires étrangères. Il y réussit parfaitement, et il brigua
la nomination du roi Jacques d'Angleterre. Mais elle ne put réussir. Il
se tourna vers celle de M. le prince de Conti, qui venait d'être élu roi
de Pologne et qui partait pour se rendre en ce pays-là. On a vu en son
lieu le peu de goût de ce prince pour cette couronne, et son prompt
retour. Gesvres ne se rebuta point. Les évêques polonais, tous sénateurs
du royaume, ont eu le bon sens de ne céder point aux cardinaux, en sorte
qu'il n'y a guère que l'archevêque de Gnesne qui le puisse être, parce
qu'étant primat du royaume et régent dans l'interrègne il n'y a point de
difficulté avec lui\,; c'est ce qui rend la nomination de Pologne facile
à obtenir aux étrangers. Gesvres sut si bien manéger qu'il eut celle de
l'électeur de Saxe, élu roi de Pologne au lieu de M. le prince de Conti.
Dans la suite le victorieux roi de Suède l'ayant forcé à céder sa
couronne à l'heureux Stanislas Leczinski, Gesvres fit encore si bien
qu'il eut sa nomination\,; et ce nouveau roi ayant été précipité du
trône par un retour de fortune et l'électeur de Saxe y étant remonté,
Gesvres eut encore une nouvelle confirmation de sa précédente
nomination, et tout cela avec le consentement du roi. Il passa donc plus
de trente ans de sa vie à pourchasser le cardinalat et à n'avoir autre
chose dans le coeur et dans la tête.

Archevêque de nom sans presque jamais de résidence, épargnant tout pour
ses agents à Rome et pour ses vues du cardinalat, il avait tout démeublé
ou vendu à Bourges depuis la mort du roi, et déclaré qu'il n'y
retournerait plus. Parvenu enfin à la pourpre si ardemment et si
persévéramment souhaitée, et transporté de joie après tant de soins, de
peines et de travaux, qui eût cru qu'arrivé enfin à l'unique but de
toute sa vie, il n'en eût pas joui pleinement\,? Mais voilà de ces
traits des jugements de Dieu qui confondent les hommes. Gesvres fut
encore moins cardinal qu'il n'avait été archevêque. Idolâtre de sa santé
et de ses écus, il ne pensa qu'à éviter d'aller à Rome, et, pour en
montrer son impossibilité, n'alla presque point à Versailles quand la
cour y fut retournée, et dînait en chemin. Il s'abstint des thèses, des
sacres, de toutes cérémonies, même de celles du Saint-Esprit, après
qu'il eut été admis à l'ordre, du conseil de conscience formé \emph{ad
honores}, et de toutes sortes d'affaires. Il vécut dans sa maison
solitaire où sa pourpre ne lui fut d'aucun usage, que pour la voir dans
ses miroirs et s'entendre donner de l'Éminence par ses valets. Point de
visites\,; en recevait très peu, mangeait seul, très sobrement et
médicinalement, avec une très bonne santé, donnait deux ou trois dîners
l'année avec peu de choix, voyait quelques nouvellistes italiens et
quelques savants obscurs, car il n'était pas sans savoir ni sans lumière
pour les affaires\,; se promenait les matins aux Tuileries pour prendre
l'air avec des gens la plupart inconnus, et se défit enfin de son
archevêché en faveur de l'abbé de Roye, qu'il voulut \emph{mordicus}, et
pas un autre, non pas même de son neveu, quoique fort bien avec lui et
avec le duc de Tresmes, son frère, parce qu'il crut que l'abbé de Roye y
ferait plus de bien et ne tourmenterait personne sur la constitution,
qu'il n'avait jamais honorée que des lèvres, et fit toujours de grandes
aumônes dans l'archevêché de Bourges.

Bentivoglio avait quitté tard un régiment de cavalerie qu'il commandait
au service de l'empereur, pour entrer en prélature. Sa naissance lui
valut en moins de rien la nonciature de France, où il se signala par
toute la débauche, les emportements, les fureurs dont on a parlé ici et
qu'on ne répétera pas. Il ne les signala pas moins à l'unique conclave
où il se trouva, et assez peu après il mourut d'un emportement de colère
qui l'étouffa et en délivra le monde.

Bossu, dont le nom était Hennin-Liétard, était frère du prince de
Chimay, mort mon gendre, que Charles II avait fait tout jeune chevalier
de la Toison, qui servit depuis Philippe V en Espagne\,; qui le fit
lieutenant général et grand d'Espagne. Bossu fut envoyé tout jeune faire
ses études à Rome, et livré aux jésuites pour avoir soin de son
éducation et de sa fortune. Ils suppléèrent à ses talents qui en tout
genre étaient nuls, mais ils en firent un grand dévot et se l'acquirent
sans réserve. Des aveugles-nés de grande naissance qui les peut élever à
tout avec du secours, sont merveilleusement propres à la société qui
n'en laisse guère échapper de ceux dont ils se peuvent saisir, et les
familles, qui espèrent bien y trouver leur compte, les leur offrent
volontiers. Elles mettent ainsi de grands bénéfices et de grandes
dignités dans leur maison, et les jésuites règnent avec autorité par des
sujets grandement établis, qui ne se connaissent pas eux-mêmes. Bossu
revint {[}de{]} Rome parfaitement jésuite\,; c'était toute l'instruction
qu'il y avait acquise, la seule dont son génie pût être susceptible,
l'unique dont l'intérêt de sa famille et celui de ses instituteurs pût
élever sa fortune\,: aussi lui valut-elle, promptement l'archevêché de
Malines et une belle et très riche abbaye dans Malines même, dont les
jésuites furent en effet archevêques et abbés. Ils se trouvèrent si bien
d'un disciple si entièrement abandonné à eux, qu'ils n'oublièrent rien
pour le faire valoir à Rome et le porter à la pourpre dont ils
tireraient encore plus d'éclat et de fruit. Il aurait eu des concurrents
qui lui auraient coupé chemin, si on se fût douté à Vienne qu'il pût
être sur les rangs d'une promotion. Quelque zèle et quelque soumission
que les jésuites aient de tout temps pour la cour impériale, leurs
intérêts leur sont encore plus chers, et le coup frappé ils ne manquent
point de ressources pour le cacher ou le faire oublier. Cette
considération, bien loin de les arrêter, ne fit qu'aiguiser leurs
sourdes intrigues. Ils firent comprendre Bossu dans cette promotion sans
aucune participation de la cour de Vienne, et l'ignorant et dévot Bossu,
transporté de joie de sa promotion, en prit à l'instant toutes les
marques dans Malines, sans en demander, ni encore moins en attendre la
permission de l'empereur. Ce monarque, accoutumé à dominer également et
ses sujets et la cour de Rome, entra en grande colère, menaça Rome,
saisit les revenus du nouveau cardinal et le traita avec toute la
hauteur d'un souverain justement irrité. Les jésuites qui s'y étaient
attendus firent le plongeon comme des serviteurs fidèles qui n'avaient
point de part en ce choix, et firent rendre à leur créature rougie les
plus grandes soumissions à l'empereur et à ses ministres. L'affaire
était faite, il ne s'agissait plus que d'en sortir\,: avec toutes ces
soumissions, Bossu n'en garda pas moins toutes les marques et le rang de
sa nouvelle dignité. Sa conscience ne lui permettait pas de manquer au
pape qui la lui avait conférée, mais en même temps il trahit son
humilité. Il prit le nom de cardinal d'Alsace. Il prétendit le premier
de sa maison sortir par mâles des anciens comtes d'Alsace. On en rit en
Flandre\,; mais partout ailleurs il ne put le faire passer et ne fut
jamais que le cardinal de Bossu. L'empereur eut grand'peine à lui
permettre d'aller à Rome pour le conclave. Il ne lui donna main levée de
ses revenus pour ce voyage qu'à condition de venir à Vienne directement
de Rome, dès que le pape serait élu et couronné, demander pardon de sa
faute. Il y alla donc, y fut retenu six mois, y reçut tous les dégoûts
dont on put s'aviser, qui le poursuivirent toujours depuis en Flandre.
La constitution venue on peut juger avec quelle aveugle fureur cette
créature des jésuites s'y signala.

Belluga arriva à la pourpre par des sentiers plus droits\,; c'était un
bon gentilhomme castillan que sa rare piété avait fait choisir à
Philippe V au commencement de son règne pour l'évêché de Murcie. Il s'y
conduisit comme on s'y était attendu, et y fut en exemple à toute
l'Espagne. Quelques années après, la guerre y fut portée jusque dans ses
entrailles. Le roi et la reine, contraints d'abandonner Madrid sans
argent, sans subsistance pour ce qui leur restait de troupes, sans
espérance d'en pouvoir lever, avec fort peu de sauver aucune pièce de la
monarchie. Dans cette extrémité, qui fit si grandement éclater
l'attachement et la fidélité espagnole à jamais mémorable, l'évêque de
Murcie se signala entre les seigneurs et les prélats. Il fournit seul,
gratuitement, deux mois de subsistance à l'armée, ou du sien qu'il
épuisa et engagea, ou du fonds de ses diocésains qu'il toucha par
l'ardeur de ses prédications, et encore plus par son exemple\,; et il
donna, de plus, de quoi payer aux troupes plusieurs prêts qui leur
étaient dus. Le sort des armes et les efforts de cette héroïque nation
ayant raffermi le trône et rendu la couronne à Philippe V, l'évêque de
Murcie ne crut pas qu'il lui fût rien dû\,; il compta n'avoir fait que
remplir son devoir, ne songea ni à se montrer ni à faire parler de
lui\,; demeura, comme il avait fait auparavant, renfermé dans son
diocèse, uniquement occupé du soin de son salut et de celui de ses
ouailles, sans que la cour aussi parût penser à lui. L'épuisement où
tant et de si cruelles secousses avaient mis les finances fit chercher
les moyens de les réparer un peu. La Crusade parut d'un secours plus
prompt et plus net, on l'augmenta fort d'un trait de plume. C'est une
imposition sur le clergé que les papes, dominant en Espagne ainsi que
dans tous les pays d'obédience\footnote{Les pays d'obédience étaient
  ceux où le pape nommait aux bénéfices et exerçait une juridiction plus
  étendue que dans les autres. L'Allemagne était un pays d'obédience. Il
  y avait aussi dans l'ancienne France plusieurs provinces qui étaient
  pays d'obédience et ne reconnaissaient point le concordat de François
  Ier. Telles étaient la Bretagne, la Provence et la Lorraine. Le pape
  pouvait pendant huit mois de l'année y nommer aux bénéfices vacants.},
et surtout dans ceux d'inquisition, ont accordée souvent aux rois
d'Espagne pour la guerre des Mores, et depuis leur expulsion, souvent
encore sous prétexte de leur faire la guerre en Afrique. Comme l'Espagne
y a toujours eu quelques places, qui ont soutenu des sièges sans fin,
parce que les Mores n'entendent rien à l'attaque des places, cette
imposition, plus ou moins forte, a presque toujours subsisté et comme
passé en ordinaire\,; mais la surtaxe, et de la seule autorité du roi,
émut le clergé et l'évêque de Murcie plus qu'aucun. C'était un grand
homme de bien, mais de peu de lumière\,; il ne crut pas pouvoir en
conscience livrer au roi un bien consacré aux autels et aux pauvres. Il
fit grand bruit\,; il résista avec la plus grande fermeté aux ordres
réitérés du roi, et comme son exemple à lui donner dans sa nécessité
avait été grand et en spectacle à toute l'Espagne, celui de sa
résistance n'eut pas moins de crédit pour le refus. Le roi, embarrassé,
s'écrie et menace\,; Belluga, inébranlable, porta ses plaintes à Rome,
et fut cause que l'affaire devint très considérable et ne put finir que
par un accommodement.

Lors de son plus grand feu la promotion se fit, et Belluga, célèbre à
Rome par son zèle et sa fermeté pour l'autorité du pape et pour
l'immunité du clergé, y fut compris sans qu'il y eût jamais pensé. Il le
montra bien\,; il n'en apprit la nouvelle qu'avec surprise, et tout
aussitôt déclara qu'il n'accepterait jamais la pourpre sans la
permission du roi, qu'il n'espérait pas dans la disgrâce où il se
trouvait. En effet, le roi d'Espagne regarda la promotion de Belluga
comme une injure qui lui était faite, et lui envoya défendre de
l'accepter. Mais le refus de Belluga avait prévenu la défense. Le pape,
piqué à son tour, dépêcha un courrier à Belluga avec un bref impératif
d'accepter en vertu de la sainte obéissance. Mais ce bref ne put tenter
ni ébranler même ce sublime Espagnol. Il répondit modestement au bref,
qu'il n'y allait ni de la religion ni de l'Église qu'il fût cardinal ou
qu'il ne le fût pas, mais qu'il y allait du devoir et de la conscience
d'un sujet d'obéir à son roi, de lui être fidèle et soumis, dont nulle
puissance ne le pouvait délier ni le faire départir. C'est qu'il ne
s'agissait ici que d'une dignité\,; s'il y avait eu de la religion ou de
l'hérésie mêlée, je ne sais si on penserait au delà des Pyrénées comme
on pense en deçà, et comme toute l'antiquité a pensé en tout pays. Quoi
qu'il en soit, telle fut la digne réponse du grand évêque de Murcie,
dans laquelle il persévéra, malgré tout ce que Rome commise y employa de
caresses et de menaces. Ce spectacle plaisait fort à Madrid, qui
laissait faire, sans se remuer, et qui le laissa durer plusieurs mois.
Belluga ne se remua pas davantage\,; il ne fit ni ne laissa faire la
plus petite démarche auprès du roi d'Espagne\,; il ne fut pas moins
tranquille ni moins absorbé dans ses devoirs et dans les occupations de
sa vie accoutumée. Rome aussi dédaignait d'agir auprès du roi d'Espagne,
ou plutôt n'osait se commettre à un refus. Lorsque Belluga n'y songeait
plus et que la longueur du spectacle l'eut fait tomber, le roi d'Espagne
dépêcha deux courriers, l'un à Belluga, avec ordre d'accepter\,; l'autre
au pape, portant sa nomination au cardinalat en faveur de Belluga. Ainsi
l'affaire fut finie avec une gloire sans égale pour Belluga, qui, sans
se hâter ni changer rien à son habit ni à sa calotte, vint présenter sa
barrette au roi d'Espagne, la recevoir de sa main, et l'en remercier
comme ne la tenant que de ses bienfaits. Ce contraste fut un peu fort
pour les cardinaux d'Alsace et de Mailly, et il fut célébré partout.

Dans la suite Belluga, qui avait plus de zèle que de lumière, voulut
entreprendre des réformes que les évêques d'Espagne ne purent souffrir.
Ils s'élevèrent contre avec d'autant plus de succès que leur résidence,
leurs moeurs, leurs aumônes, leur vie pleinement et uniquement
épiscopale est en exemple de tout temps soutenu à tous les évêques du
monde. Belluga, ne pouvant procurer à son pays le bien qu'il s'était
proposé, se dégoûta tellement qu'il fit trouver bon au roi qu'il lui
remît l'évêché de Murcie, et qu'il se retirât à Rome. Il y fut comme à
Murcie, sujet très attaché à son roi, chargé même de ses affaires dans
des entre-temps, et y a eu part dans tous, et sa vertu qui surnagea
toujours aux lumières, surtout politiques, lui acquit une vénération, et
même pendant toute sa longue vie une considération que celles-ci ne
peuvent atteindre, quoique plus dans leur centre en cette capitale du
monde que partout ailleurs.

Salerne était un jésuite italien du royaume de Naples, transporté je ne
sais par quelle aventure en Allemagne, ni par quelle autre fort bien
dans les bonnes grâces de Frédéric-Auguste, électeur de Saxe, en la
conversion duquel il eut beaucoup de part\,; mais je ne sais s'il y eut
plus de peine que le Tencin à celle de Law. L'électeur de Saxe voulait
être roi de Pologne, et il ne pouvait être élu sans être catholique. Nul
sujet du duché de Saxe ne pouvait embrasser la religion catholique sans
perdre à l'instant tous les biens qu'il y possédait. La qualité de chef
et de protecteur né de tous les protestants d'Allemagne est attachée à
la dignité d'électeur de Saxe, qui est chargé de tous leurs griefs, de
les faire redresser, de leur faire maintenir et rétablir tout ce que les
diverses paix et pacifications leur ont accordé. Un titre qui a des
fonctions si continuelles et si importantes, et qui le met à la tête du
corps protestant, et en moyen de le mouvoir, lui donne la première
considération dans l'Empire et dans toute l'Allemagne, et une autorité
et un crédit qui le fait fort ménager par tous les souverains
d'Allemagne et beaucoup par les empereurs. Auguste ne voulait pas perdre
de si grands avantages ni se commettre avec ses propres États passionnés
pour le luthéranisme. Son domestique n'était pas plus aisé sur ce point.
Le détail de cette grande affaire n'appartient point à ces Mémoires. Il
s'y faut contenter de l'exposition du fait, et de dire qu'Auguste fut
assez habile ou assez heureux pour concilier des choses si fort
opposées. Il fut catholique et roi de Pologne\,; il ne se brouilla ni
avec ses sujets ni avec le corps des protestants\,; il demeura toujours
leur chef et leur protecteur, dont il conserva toujours la
considération, le crédit et l'autorité en Allemagne. Sa mère était fille
de Frédéric III, roi de Danemark, qui survécut vingt ans à son
couronnement à Cracovie, et qui ne le voulut jamais voir depuis. Il
avait épousé en 1693 Christine-Évérardine, fille de Christian-Ernest de
Brandebourg, marquis de Bareith\footnote{Cenom s'écrit quelquefois
  Bareuth ou Bayreuth. Aujourd'hui Bareuth est une ville du royaume de
  Bavière.}, qui se retira dans un château à la campagne dès qu'elle sut
sa conversion, ne prit jamais les marques de reine ni n'en voulut
admettre les traitements, fut plusieurs années sans pouvoir se résoudre
à le voir quand il venait en Saxe, et ne le vit enfin que comme en
visites très courtes et très froides, sans avoir jamais voulu approcher
des frontières de Pologne. L'électeur s'en consola aisément, mais il
avait encore un autre dessein à exécuter. C'était de convertir son fils
aîné et de lui assurer la couronne de Pologne, sans perdre après lui la
précieuse qualité de chef et protecteur né des protestants. Pour arriver
à ce but, il fallait séparer doucement le jeune prince d'une mère si
entêtée de sa religion, sans montrer ses desseins sur lui, et le confier
à des personnes assez sûres et assez intelligentes pour tourner le
prince électoral suivant ses vues. C'est à quoi il eut encore le bonheur
de réussir, et ce qui le détermina à le dépayser de Saxe par de longs
voyages. Le P. Salerne eut l'honneur de la conversion du fils comme il
avait eu celle du père. Il accompagna le jeune prince dans tous ses
voyages, déguisé en cavalier\,; il le confessait et le dirigeait, et
comme il n'était pas encore temps que sa conversion parût, il lui disait
la messe avant que la suite du prince le sût éveillé, dont il avait une
permission du pape. Au retour de ses voyages, la conversion, comme on
l'a vu ici, fut déclarée, et presque en même temps son mariage avec une
archiduchesse. Salerne en porta la nouvelle au pape qui le récompensa du
chapeau. C'était, comme on le voit, un homme d'esprit et d'intrigue,
doux, honnête, insinuant et dont les moeurs et la conduite n'ont point
reçu de blâme. Il mourut à Rome chez les jésuites où il voulut toujours
loger, neuf ans après sa promotion, toujours fort considéré et chargé
des affaires de ses prosélytes.

Mailly, sans ailes comme en avait eu Gesvres, ne visa pas moins haut et
n'y travailla pas moins que lui. Mis dans l'Église malgré lui par un
père et une mère violents et absolus dans leur famille, il fit de
nécessité vertu à travers les plus cuisants regrets, et ne prit
d'ecclésiastique que ce qu'il n'en put laisser\,; ni étude ni savoir
d'aucune espèce ni aptitude ni volonté d'en acquérir, ni piété ni moeurs
que ce qu'il en fallait à l'extérieur pour ne pas ruiner les espérances
de l'état forcé qu'on lui avait fait embrasser. Il vécut longtemps les
coudes percés dans un recoin de Saint-Victor, parce qu'il en coûtait
moins à son père, et que cette demeure l'écartait davantage du monde, et
donnait une écorce plus régulière. Le mariage du comte de Mailly son
frère avec une nièce à la mode de Bretagne de M\textsuperscript{me} de
Maintenon, mais dont elle prenait soin comme de sa véritable nièce, et
qu'elle fit dame d'atours de M\textsuperscript{me} la duchesse
d'Orléans, puis de M\textsuperscript{me} la duchesse de Bourgogne, valut
enfin une légère abbaye à ce malheureux reclus, et quelque liberté
ensuite par une place d'aumônier du roi. Nos maisons, du même pays,
étaient anciennement et plusieurs fois alliées\,; l'amitié et les
liaisons s'étaient toujours conservées entre elles.

J'étais fort des amis du comte de Mailly et de sa femme. Je le devins de
l'abbé de Mailly dès qu'il parut à la cour. Il parvint à force de bras à
l'archevêché d'Arles, à la mort du dernier Grignan. À peine y fut-il
nommé qu'il songea à mettre à profit le voisinage d'Avignon et la
facilité de la mer pour le commerce avec Rome. Il fit toutes sortes
d'avances à Gualterio, vice-légat d'Avignon, qui y répondit en homme de
beaucoup d'esprit et fort liant, qui n'ignorait pas ce qu'était
l'archevêque d'Arles et la comtesse de Mailly, sa belle-soeur. Le grand
but de ces vice-légats, et qui leur fait souhaiter cette vice-légation,
est d'en sortir par la nonciature de France qui leur assure le
cardinalat. Pour cela il faut s'y rendre agréable, parce qu'une des
distinctions des trois grandes couronnes, l'Empire, la France et
l'Espagne, est l'exclusion pour leur nonciature de tout sujet qui leur
déplaît, et le choix pour la remplir entre trois ou quatre sujets que
Rome leur propose. La liaison fut donc bientôt formée entre les deux
prélats par leurs vues et leurs besoins respectifs, qui se tourna dans
la suite en amitié intime qui ne finit qu'avec leur vie, on l'a vu ici
ailleurs, et que ce fut leur amitié qui forma la mienne avec Gualterio,
qui a duré jusqu'à la mort. Il vint bientôt nonce en France. Il y plut
extrêmement, et sut gagner si bien les bonnes grâces du roi, que, devenu
cardinal, il lui donna l'abbaye de Saint-Victor à Paris. On a vu ici en
son temps qu'il s'était noyé à Rome, par la visite qu'il fit en partant
de France aux bâtards\,; ce qui a fait que depuis lui aucun nonce n'a
reçu la calotte rouge à Paris, et que sur le point de leur promotion,
ils ont toujours, été rappelés et ne l'ont reçue qu'à l'entrée de
l'Italie. Quelques années après sa promotion, Gualterio revint de Rome
tout exprès pour voir le roi, et on a vu en son lieu ici avec quelle
distinction il y fut reçu, jusqu'à donner de la jalousie par l'exemple
du cardinal Mazarin. Il retourna à Rome avec parole du roi de l'ordre du
Saint-Esprit à la première promotion. Le roi mourut sans la faire. M. le
Duc en acquitta la promesse en 1724. Mailly, pendant ces années, tâchait
de les employer sourdement par le commerce caché qu'il entretenait à
Rome, où il se faisait des amis tant qu'il pouvait. Il trouva moyen de
se procurer des occasions d'écrire au pape et de s'en attirer des brefs,
mais tout cela dans le plus ténébreux secret. Depuis la fin de la Ligue,
et la force du règne de Henri IV, il était aussi sagement qu'étroitement
défendu à tous évêques, bénéficiers et ecclésiastiques d'avoir aucun
commerce avec Rome, sans une permission expresse qui passait par celui
des secrétaires d'État qui avait les affaires étrangères, qui
l'accordait difficilement, qui limitait le temps, et qui ne s'étendait
jamais au delà de l'affaire pour laquelle elle était accordée. C'était
un crime et sévèrement châtié, qu'y écrire même une seule fois sans en
avoir obtenu permission, parce que toutes les affaires ordinaires comme
bulles, dispenses, etc., s'y faisaient par la seule entremise des
banquiers en cour de Rome. Le roi était fort jaloux sur ce point. Ce n'a
été que tout à la fin de son règne que l'affaire de la constitution, qui
fit tant de fripons, d'ambitieux et de fortunes, et le crédit et
l'intérêt du P. Tellier énervèrent cette loi si salutaire, puis
l'anéantirent, dont la France sent encore tout le poids et le malheur.
On a vu ailleurs ici combien il y eut de peine et de travail à sauver M.
d'Arles, surpris en cette faute à l'occasion des reliques de saint
Trophime, dont il avait envoyé un présent au pape qu'il s'était fait
demander, dont il fut sur le point d'être perdu. Cet orage, que
M\textsuperscript{me} de Maintenon eut grande peine à calmer, et qui fit
grand bruit à la cour, rendit l'archevêque d'Arles plus timide, mais
sans lâcher prise, et lui servit à Rome. On peut juger qu'un homme
d'ambition si suivie n'avait pas négligé de se dévouer aux jésuites et
de se les acquérir. Une haine commune les unissait.

La comtesse de Mailly, et les Mailly leurrés et accoutumés à la voir la
nièce favorite de M\textsuperscript{me} de Maintenon, n'avaient pu
digérer la fortune si supérieure de la nièce véritable, et ce que les
Noailles avaient tiré de ce mariage. N'osant s'en prendre à
M\textsuperscript{me} de Maintenon, ils s'en prenaient aux Noailles
qu'ils haïssaient parfaitement\,; l'archevêque d'Arles en était irrité
plus qu'aucun d'eux. Il ne pouvait supporter l'éclat du cardinal de
Noailles, dont les avances et la douceur ne le purent jamais ramener, en
sorte que, se trouvant d'une assemblée du clergé où le cardinal de
Noailles, lors en pleine faveur, présidait, il prit à tâche, sourdement
étayé des jésuites, de lui faire contre en toute occasion, sans que la
patience et tout ce que le cardinal put faire pour le rendre plus
traitable, y put réussir, tellement que l'archevêque leva le masque et
lui rompit publiquement en visière. Le cardinal, tout modéré qu'il
était, ne crut pas devoir souffrir cette insulte. Il la repoussa avec
sagesse, mais avec la hauteur qui convenait à sa place, et comme au fond
il avait raison, et qu'il sut bien l'expliquer et le démontrer, il
confondit l'archevêque, qui ne sut que balbutier, et qui fut blâmé
publiquement de toute l'assemblée. Cet éclat obligea le cardinal d'en
rendre compte au roi. Le roi lava doucement la tête à l'archevêque, et
l'obligea d'aller faire des excuses au cardinal, sans que les jésuites
osassent dire un mot en sa faveur, ni que lui eût pu gagner
M\textsuperscript{me} de Maintenon qui le tança fortement. Voilà ce
qu'il ne pardonna jamais aux Noailles, et qui le rendit l'ennemi ardent
et irréconciliable du cardinal de Noailles tout le reste de sa vie,
jusqu'à m'avoir dit à moi-même dans le feu de l'affaire de la
constitution, et lui cardinal, sur laquelle nous n'étions pas d'accord,
qu'il ne se souciait de la constitution comme telle en façon du monde\,;
qu'il ne l'avait jamais soutenue avec ardeur, comme il ferait toujours,
que parce que le cardinal de Noailles était contre, et qu'il aurait été
contre avec la même violence, si le cardinal de Noailles avait été pour.
Il ne me dissimula pas aussi que la vue prochaine du chapeau lui avait
fait faire les fortes démarches qu'il avait crues utiles pour se
l'assurer et se l'accélérer.

Le Tellier, fils du chancelier de ce nom, et frère de Louvois, étant
mort en 1710, archevêque de Reims depuis longues années, et toute sa vie
peu ami des jésuites, le P. Tellier se fit un capital de le remplacer
d'un homme à tout faire pour les jésuites, et à réparer dans ce diocèse
les longues pertes qu'ils y avaient faites. Il y voulut aussi avec
autant de choix un ennemi du cardinal de Noailles, qui, par l'éminence
de ce grand siège, devînt un personnage nécessaire, sûr en même temps
pour eux et propre à lui opposer. D'autres qualités, il ne s'en
embarrassa guère, l'autorité et la violence suppléant aisément à tout.
Dès qu'il ne s'agissait que des deux premières il ne lui fallut pas
chercher beaucoup pour trouver son fait. La naissance, les entours de
Mailly, le siège d'Arles qu'il occupait depuis longtemps, et où il avait
presque toujours résidé, rendirent facile sa translation à Reims. Mailly
gagna tout à ce changement, et n'y perdit pas même la facilité qu'il
avait à Arles pour son commence et ses intrigues à Rome, sur lequel la
rigueur de la cour était peu à peu tombée par les manéges du P. Tellier,
aux vues duquel cette liberté était devenue nécessaire. Ainsi Mailly,
devenu plus considérable à Rome par l'éclat de son nouveau siège et par
sa proximité de Paris et de la cour, redoubla d'efforts à Rome, et
n'oublia rien ici, pour en mériter l'objet de ses désirs. L'affaire de
la constitution lui en présenta tous les moyens qu'il en saisit avec
avidité, et qui lui fournit ceux d'exercer sa haine contre le cardinal
de Noailles. L'orgueil souffrait toutefois de se voir avec son siège,
son zèle, son affinité avec M\textsuperscript{me} de Maintenon, si loin
derrière les cardinaux de Rohan et de Bissy, et confondu avec d'autres
évêques\,; mais ce fut une épreuve qu'il fallut essuyer dans l'espérance
du chemin qu'elle lui ferait faire. Ainsi s'écoulèrent les restes du
règne du roi et les premiers temps de la régence. La constitution y
ayant enfin pris le dessus, Mailly s'unit étroitement à Bentivoglio,
tous deux dévorés du désir de la pourpre, et tous deux persuadés qu'ils
ne se la pouvaient accélérer qu'en mettant tout en feu. Mailly donc
n'aspira plus qu'à se faire le martyr de Rome, ne garda plus de mesures,
abandonna Rohan, Bissy et les plus violents évêques, comme de tièdes
politiques, qui abandonnaient le saint-siège et la cause de l'Église. De
là ses lettres et ses mandements multipliés, le double mérite qu'il
recueillit à Rome d'avoir osé les faire et les publier, et de n'avoir pu
être arrêté par tous les ménagements que le régent avait eus pour lui.
Ce n'était pas des ménagements qu'il souhaitait, c'était tout le
contraire, pour acquérir à Rome la qualité de martyr et en recueillir le
fruit. Aussi en fit-il tant que l'emportement d'une de ses lettres la
fit brûler par arrêt du parlement\,; aussi en fit-il éclater sa joie et
son mépris un peu sacrilègement. Il fonda une messe à perpétuité dans
son église, à pareil jour, pour remercier Dieu d'avoir été trouvé digne
de participer aux opprobres de son fils unique pour la justice\,; il
espérait sans doute engager à quelque violence d'éclat, par cette
étrange fondation, qui le conduirait plus tôt à son but\,: il y fut
trompé.

Le châtiment alors ne pouvait tomber que sur sa personne, et on ne peut
agir contre la personne d'un pair qu'au parlement, toutes les chambres
assemblées et les pairs convoqués. Outre l'embarras d'une affaire de
cette qualité, la constitution et ses suites étaient détestées, et on ne
craignait rien tant là-dessus que l'assemblée du parlement. On laissa
donc tomber l'éclat où l'archevêque voulait engager. Sa conduite, qui
scandalisa jusqu'aux plus emportés constitutionnaires, le décrédita même
dans leur parti\,; mais les prélats ne donnaient pas les chapeaux\,; ce
n'était qu'à Rome qu'ils se distribuaient, et ce n'était que vers Rome
que toutes ses démarches se dirigeaient. Enfin il fut content par la
promotion dont il s'agit ici\,; lui et son ami Bentivoglio y furent
compris tous deux. Ces violents procédés ne le servirent peut-être pas
mieux que ses flatteries. Le pape se piquait singulièrement de bien
parler et de bien écrire en latin\,; il voulait s'approcher de saint
Léon et de saint Grégoire, ses très illustres prédécesseurs\,; il
s'était mis à faire des homélies\,; il les prononçait, puis les montrait
avec complaisance\,; pour l'ordinaire, on les trouvait pitoyables, mais
on l'assurait qu'elles effaçaient celles des pères de l'Église les plus
savants, les plus élégants et les plus solides. Mailly s'empressa d'en
avoir, et encore plus de se distiller en remercîments et en éloges. Ils
achevèrent de gagner et de déterminer le pape, qui le fit cardinal, sans
participation de la France ni de pas un de ses parents ou amis de ce
pays-ci.

\hypertarget{chapitre-xvi.}{%
\chapter{CHAPITRE XVI.}\label{chapitre-xvi.}}

1719

~

{\textsc{M. le duc d'Orléans, fort irrité de la promotion de
l'archevêque de Reims, me mande, me l'apprend et dispute cette affaire
avec Le Blanc et moi, où La Vrillière, gendre du frère de l'archevêque,
survient.}} {\textsc{- Velleron dépêché à l'archevêque avec défense de
porter aucune marque de cardinal et de sortir de son diocèse.}}
{\textsc{- Ridicule aventure et dépit de Languet, évêque de Soissons.}}
{\textsc{- Son état, son ambition, ses écrits, sa conduite.}} {\textsc{-
Conduite de l'archevêque de Reims.}} {\textsc{- Il obéit aux ordres que
Velleron lui porte.}} {\textsc{- Quel était Velleron.}} {\textsc{- Ma
conduite avec le régent sur l'archevêque de Reims.}} {\textsc{- Rare et
insigne friponnerie des abbés Dubois et de La Fare-Lopis à l'égard l'un
de l'autre.}} {\textsc{- L'archevêque de Reims clandestinement à
Paris.}} {\textsc{- Mystère très singulier de ce retour.}} {\textsc{-
Faiblesse et ambition de l'archevêque de Reims.}} {\textsc{- Son premier
succès et ma duperie.}} {\textsc{- Manége de Dubois à l'égard de
l'archevêque de Reims, dont je suis encore parfaitement la dupe.}}
{\textsc{- Comment Mailly, archevêque de Reims, obtint enfin de recevoir
des mains du roi sa calotte rouge, où je le conduisis.}}

~

M. le duc d'Orléans m'envoya chercher un peu après midi\,; il n'y avait
pas une heure qu'il avait reçu la nouvelle de la promotion\,; l'abbé
Dubois qui la lui avait portée n'était déjà plus avec lui. C'était le
dimanche 10 décembre\,; je le trouvai seul avec Le Blanc\,; La Vrillière
y vint une demi-heure après. M. le duc d'Orléans était fort en colère\,;
il m'apprit la promotion, et tout de suite qu'il dépêchait à Reims, où
était l'archevêque, le chevalier de Velleron, enseigne des gardes du
corps, avec un ordre du roi de l'empêcher de sortir de Reims, de l'y
faire retourner s'il le rencontrait en chemin, de lui défendre de porter
la calotte rouge ni aucune marque ni titre de cardinal, et de la lui
ôter de dessus la tête en cas qu'il l'y eût mise. Je sentis tout le
crime d'une ambition désordonnée, qui m'était connue depuis si
longtemps. Je sentis aussi toute la faiblesse du régent après le premier
feu passé, qui le portait lors aux extrémités, et tous les embarras à
l'égard d'une dignité que les couronnes ont mise en possession paisible
de toute indépendance, de toute infidélité et de toute vraie impunité.
Je sentis encore que la chose était à ce point qu'il fallait perdre cet
homme, qui était mon parent, et, tel qu'il fût, mon ami depuis si
longtemps, ou le laisser en possession de son larcin. Je me conduisis
donc en conséquence\,; je montrai autant de colère que M. le duc
d'Orléans, je ne le contredis en rien, je discutai avec lui tous les
plus violents partis sans en exclure ni en inclure pas un. Je donnai à
sa colère tout le jeu et tout l'essor qu'elle voulut prendre, et
j'applaudis à tout. J'aurais tout gâté à faire autrement\,; il n'était
pas temps de chercher à diminuer ce feu, je l'aurais embrasé davantage,
et j'aurais ôté la force à ce que je me proposais bien de lui
représenter peu après. Ces délibérations d'extrémités fort en l'air et
peu digérées durèrent jusqu'à près de trois heures. Je ne voulus rien
abréger pour laisser évaporer tout le feu, et parus être aussi fâché que
lui. Je l'étais en effet, parce que rien n'est plus préjudiciable à
l'État ni plus directement opposé au droit des rois sur leurs sujets
qu'une telle porte ouverte à l'ambition des ecclésiastiques, qui, au
mépris du souverain, de son autorité, de ses intérêts, se livrent à une
puissance étrangère, souvent ennemie, pour en obtenir une dignité
amphibie qui les élève à un rang monstrueux, les met à la tête du
clergé, les soustrait à tout châtiment et à toute poursuite, quelque
félonie qu'ils puissent commettre, leur donne un crédit, une
considération, une autorité infinie, avec le droit certain d'avoir pour
deux et trois cent mille livres de rente en bénéfices, et d'obtenir tout
ce qui leur convient à leur famille, sans rendre le plus léger service à
l'État ni à l'Église, séduit une infinité d'autres par l'espérance, et
rend le pape plus maître du clergé que le roi\,; mais Mailly, de plus ou
de moins, n'augmentait guère cette plaie\,; il était mon parent et mon
ami\,; je ne voulais pas laisser casser la corde sur lui\,; et
d'ailleurs je connaissais trop le régent pour le sentir capable de lui
tenir la même rigueur qu'en pareil et même moindre cas le roi tint au
cardinal Le Camus. À la fin le régent se souvint que nous n'avions pas
dîné, et nous congédia.

Le Blanc, que M. le duc d'Orléans employait pour le moins {[}autant{]}
en espionnages et en choses secrètes qu'à son fait de secrétaire d'État
de la guerre, était fort souvent au Palais-Royal. Il avait accoutumé sa
femme à faire mettre à table la compagnie chez lui sans lui, quand il
n'était pas rentré à deux heures, et comme il en était près de trois
quand il arriva ce jour-là, il trouva le dîner avancé, et la compagnie
en peine de ce qui pouvait l'avoir tant retardé. Le hasard le fit placer
à table vis-à-vis Languet, évêque de Soissons. Le Blanc fit ses excuses,
et dit qu'il ne cacherait point ce qui l'avait retenu si tard au
Palais-Royal, parce que la chose allait être publique\,: chacun dressa
les oreilles et demanda de quoi il s'agissait. Le Blanc répondit que
c'était de la promotion que le pape venait de faire. À ce mot, Languet
se met presque en pied et s'écrie les yeux allumés\,: «\,Et qui, et
qui\,?» Le Blanc nomme les nouveaux cardinaux\,; Mailly fut nommé le
second, comme il l'était dans la liste. À ce nom, Languet tombe sur sa
chaise, la tête sur son assiette, se la prend à deux mains, et s'écrie
tout haut\,: «\,Ah\,! il m'a pris mon chapeau.\,» Un éclat de rire de la
compagnie, mal étouffé et surpris, après quelques moments de silence,
réveilla le désintéressé prélat. Il demeura déconcerté, laissa raisonner
sur la promotion, balbutia tard, courtement, rarement, tortilla quelques
bouchées lentement, et de loin à loin, pour faire quelque chose, devint
le spectacle de la compagnie, et la quitta lorsqu'on fut hors de table
tout le plus tôt qu'il put. Cette aventure fut bientôt publique, et me
fut contée le lendemain par le chevalier de Tourouvre, qui vint dîner
chez moi, et qui s'était trouvé la veille à table chez Le Blanc, à côté
de Languet. Qui eût dit du plat abbé Languet, bourgeois de Dijon,
languissant dans les antichambres de Versailles, où je l'ai vu cent fois
entrant chez le maître ou la maîtresse de l'appartement, et le
retrouvant en sortant sur le même coffre de l'antichambre\,; qui
croyait, avec raison, avoir fait fortune par une place pécuniaire
d'aumônier de M\textsuperscript{me} la duchesse de Bourgogne, et une de
grand vicaire d'Autun\,; qui croirait, dis-je, que, non content d'être
arrivé à se voir évêque, et évêque de Soissons, il ne se serait pas
trouvé au comble, et eût osé lever les yeux jusqu'à la pourpre et en
approcher en effet de fort près\,? Saint-Sulpice d'abord, dont
l'illustre curé était son frère, bien différent de lui, et la
constitution après qui le fit évêque, en se livrant corps et âme au P.
Tellier, lui tournèrent la tète d'ambition. Peu de gens osèrent se
déshonorer au commencement de cette affaire par un abandon à découvert.
Il fut des premiers, et bientôt après il se signala par ces fameux
avertissements ou tocsins, qui firent tant de bruit et de scandale, dont
il se donna constamment pour l'auteur tout aussitôt qu'ils parurent sous
son nom.

Mailly, archevêque de Reims, me vint conter, mourant de rire, que
Tourneli, docteur de Sorbonne, qui les avait faits, mais qui, pour leur
donner du poids, les voulait donner sous le nom d'un évêque, était allé
les lui porter, et le prier, jusqu'à l'importunité, de les adopter et
d'y laisser mettre son nom pour les publier comme son ouvrage\,; qu'il
ne voulut tâter ni de l'ouvrage, ni du mensonge, ni de se revêtir du
travail d'autrui, et que sa surprise avait été sans égale, lorsque peu
après il les voyait imprimés sous le nom de Languet, évêque de Soissons,
qui s'en déclarait publiquement l'auteur. Tant que Tourneli vécut, ce
prélat s'illustra de sa plume parmi les siens\,; mais quand la mort la
lui eut enlevée, le tuf parut à plein dans les compositions de Languet.
Il était très vrai qu'il briguait sourdement la pourpre\,; mais on ne
laissa pas à la fin de le savoir, et on l'en crut même fort proche.
Rome, suivant sa politique, l'entretenait d'espérances, sans la vouloir
prostituer à un sujet aussi infime, et duquel, à beaucoup moins, elle
était bien sûre de tirer toutes les folies et toutes les fureurs qu'elle
voudrait\,; aussi ne s'y est-elle pas trompée, et la suite en a donné la
pleine démonstration même fort au delà des intentions de Rome. En effet,
il se trouvera bien peu d'auteurs et encore moins d'évêques aussi hardis
à citer faux, à tronquer les passages, à en tirer le contraire précis de
ce qu'on y lit lorsqu'on y joint ce qui précède et ce qui suit, à
présenter effrontément des sophismes avec une fécondité surprenante, à
offrir en thèse la proposition réfutée\,; à supposer des faits et des
mensonges clairs avec la dernière audace, à remettre en principe certain
le faux dont il a été convaincu. C'est trop en dire pour n'en pas citer
au moins un {[}exemple{]} d'une si grande foule.

Transféré à l'archevêché de Sens par des voies peu correctes, il y
trouva des suffragants d'un autre aloi que lui. Caylus, évêque
d'Auxerre, dont la vie si épiscopale, et les savants écrits et la
conduite sur l'affaire de la constitution, ont si avantageusement réparé
une légère et courte complaisance pour la cour et pour
M\textsuperscript{me} de Maintenon qui l'avait placé, et qui lui ont
fait un si grand nom, était depuis longtemps exilé de son diocèse et en
butte à tous les opprobres des jésuites et des tenants de la
constitution. Cet état le fit choisir entre les autres suffragants de
Sens par l'intègre métropolitain, pour hasarder un éclat dont il ne
présumait pas que l'opprimé prélat osât former la moindre plainte.
Languet publia donc un mandement plein de charité et de zèle, par lequel
supposant qu'il avait reçu des plaintes et des requêtes de tous les
curés et chanoines du diocèse d'Auxerre, contre la doctrine de leur
évêque, et pour lui demander protection contre la violence qu'il faisait
à leur foi et à leur obéissance à celle de l'Église, il avait résisté
longtemps pour donner lieu par sa patience à la résipiscence de son
suffragant\,; mais qu'enfin, ne pouvant plus être sourd à tant
d'instances et de cris redoublés de tous les pasteurs et chanoines du
diocèse d'Auxerre, il était forcé de rompre le silence pour aller à leur
secours, etc. Qui est l'homme assez hardi pour oser douter de la vérité
d'un fait de cette nature si nettement et si expressément exposé par un
mandement imprimé et répandu partout, dont ce fait si bien énoncé est
l'unique matière\,? Toutefois une si raisonnable confiance ne dura pas
longtemps. Trois semaines après que ce mandement fut répandu, il en
parut un de l'évêque d'Auxerre, par lequel il témoigne à ses diocésains
l'extrême surprise où il est du roman dont son métropolitain abuse le
public, sous la forme d'un mandement, et joint, pour en démontrer la
calomnie et l'imposture, une lettre à lui évêque d'Auxerre, écrite et
signée par tous les curés et chanoines de son diocèse, à l'exception de
quatre, par laquelle ils se plaignent amèrement de la fiction de
Languet, protestent que pas un d'eux ne lui a fait de plainte ni adressé
de requête, déclarent à leur évêque qu'ils ont la même foi que lui, et
qu'ils ont toujours adhéré, adhèrent et adhéreront toujours à ses
sentiments qu'il a si doctement et si clairement manifestés par ses
instructions pastorales, mandements et autres ouvrages consentent et
demandent que cette présente lettre soit rendue publique, comme
contenant la plus pure vérité et leurs véritables sentiments. Cette
lettre, imprimée à la suite du mandement de l'évêque d'Auxerre, fit le
bruit qui se peut imaginer, avec une surprise inexprimable.

L'archevêque de Sens, confondu et hors d'état de la moindre réplique, se
tut à la vérité et se tint quelque temps en silence et assez retiré,
mais bientôt il reprit vigueur avec son impudence accoutumée, sans
toutefois oser remettre sur le tapis rien qui pût avoir trait, au
démenti si public qui l'avait déshonoré si à plein. Cette prudence ne
lui était pas ordinaire\,: convaincu cent fois de passages tronqués, de
citations fausses et frauduleuses, et de tout ce qui en est dit plus
haut, il avait très ordinairement osé, après quelque intervalle,
remettre en preuves décisives ce sur quoi il avait été convaincu de
faux, avec un front d'airain qui ne cherchait qu'à surprendre et qui ne
rougissait jamais. Mais c'est assez s'arrêter sur un prélat qui, tout
vil qu'il est en tout genre, doit pourtant être montré tel qu'il est par
les personnages qu'il a faits et qu'il n'a cessé, quoique vainement, de
vouloir faire\,; car sa misérable \emph{Marie Alacoque}, faite par un
jésuite, et si longtemps depuis imprimée sous son nom, n'a jamais été
adoptée par Languet comme son ouvrage, que pour revenir à la pourpre par
des détours qu'il a crus sûrs et qui le paraissaient, mais qui sont tout
à fait hors et au delà des matières de ces Mémoires qu'il faut
maintenant reprendre.

Dans le moment que La Vrillière sut la commission résolue pour le
chevalier de Velleron, dont j'ai parlé ci-dessus, il dépêcha un courrier
à Reims pour en avertir l'archevêque, et qu'il se perdrait sans
ressource si cet officier le trouvait avec la calotte rouge, qu'il avait
ordre en ce cas de lui ôter de gré ou de force, l'exhorta à obéir aux
ordres qu'il lui portait, et lui manda qu'il n'y avait que ce moyen de
calmer l'orage et de parvenir ensuite par degrés au consentement de son
cardinalat. La Vrillière était gendre du feu comte de Mailly, frère de
l'archevêque, qui me conta l'après-dînée du même jour la précaution
qu'il avait prise, et raisonna avec moi des mesures de conduite auprès
du régent et à l'égard de la tête opiniâtre et enivrée de la pourpre,
qu'il fallait tâcher d'empêcher de se jeter dans des précipices. L'avis
réussit et arriva à temps\,; l'archevêque avait déjà fait quelque chose
de bien et quelque chose de mal. Il avait reçu la calotte par le
courrier du pape, au lieu de l'envoyer tout de suite au régent. Mais il
n'avait voulu recevoir à Reims aucun compliment de personne, il avait
fermé sa porte et il était parti pour Paris. Velleron le trouva en deçà
de Soissons, sans calotte rouge ni aucune marque de cardinal. Velleron,
content de n'avoir point à le faire dépouiller, se contenta de lui
déclarer la défense dont il était chargé en lui montrant ses ordres. Ils
disputèrent un peu de temps dans le chemin tous deux pied à terre,
l'archevêque voulant continuer sa route pour remettre lui-même sa
calotte au régent, Velleron insistant sur l'ordre de retourner à Reims
et d'y demeurer jusqu'à nouvel ordre. Enfin il l'emporta et il fit
retourner l'archevêque à Soissons, où il l'accompagna et où ils
couchèrent. L'archevêque écrivit de là au régent, pour lui rendre compte
de sa conduite et de son obéissance, et l'assurer qu'il s'en
retournerait à Reims, où il attendrait ses ordres. Velleron le crut de
bonne foi. C'était un cadet de Provence, d'une médiocre naissance, fils
pourtant d'une soeur du feu cardinal de Janson. Il avait du monde, de la
politesse, de la figure, de l'honneur et de la valeur, mais rien du tout
au delà\,; les dames le portèrent, il fit fortune et il est mort
ambassadeur en Angleterre, chevalier de l'ordre, sous le nom de comte de
Cambis. Il partit donc de Soissons pour Paris en même temps que
l'archevêque pour Reims, quoiqu'il eût ordre de rester auprès de lui.
L'archevêque, qui avait son dessein, sut s'en défaire. Il fut tancé
d'être revenu, mais on ne le renvoya ni lui ni aucun autre à Reims. Ils
avaient séjourné un jour à Soissons, qui s'était passé en disputes et en
représentations qui avaient enfin abouti à ce qui vient d'être expliqué,
tellement que Velleron arriva le 14 décembre, le cinquième jour après
que le régent eut su la promotion.

Je n'avais pas perdu ce temps-là. J'avais vu souvent M. le duc
d'Orléans, et agité avec lui plus à tête reposée, la diversité des
extrémités où on pouvait se porter et les inconvénients de chacune, et
comme j'étais fort incertain de ce qui arriverait du voyage de Velleron,
je me contentai de me servir de tous les embarras résultants des partis
extrêmes, pour laisser le régent dans celui du choix sans lui montrer
aucune affection pour l'archevêque, pour profiter avec plus de force de
ce que ce prélat pouvait faire de satisfaisant et de la faiblesse du
régent à prendre sérieusement, beaucoup plus à soutenir un parti extrême
de longue haleine. Le succès du voyage de Velleron me mit en état
d'entamer un autre langage. Je fis valoir le respect de l'archevêque,
même avant d'avoir reçu ni pu recevoir aucun ordre qui lui avait fait
refuser de recevoir aucun compliment à Reims, et de n'avoir pris aucune
marque de cardinal, ainsi que Velleron l'avait trouvé avec sa calotte
noire et son habit ordinaire. Je convins de la sottise d'avoir reçu la
calotte rouge du courrier du pape au lieu de l'avoir envoyée tout de
suite\,; mais je tâchai de la couvrir de la joie, de la surprise, de la
pensée qu'il était peut-être plus respectueux de l'apporter lui-même,
puisqu'il ne l'avait pas mise sur sa tête, ainsi que je le supposais,
puisqu'il en avait refusé les compliments, fermé sa porte à tout le
monde, et que Velleron l'avait rencontré en chemin sans en être paré.
Enfin je fis valoir son obéissance d'être retourné à Reims.

Quelque furieux que fût l'abbé Dubois de la promotion de deux François,
dont l'une était inattendue, qui pourrait porter un grand préjudice à un
troisième, qui était lui-même, sans oser encore le dire tout haut, et
qui, dans cette fougue, animait tant qu'il pouvait M. le duc d'Orléans,
et par lui-même même et par ses émissaires, je m'aperçus incontinent du
bon effet de la conduite de l'archevêque qui ouvrait une porte à M. le
duc d'Orléans pour sortir de cette affaire sans violence\,; mais non
seulement l'archevêque avait contre lui Dubois, les envieux de sa
pourpre, ceux qui raisonnaient bien sur la manière dont il l'obtenait,
et tous ceux qui étaient opposés à la constitution, mais les plus
ardents de ceux qui la favorisaient, les uns dans le dépit de se voir
gagnés de la main, et reculés avec peu d'espérance, les autres piqués de
voir leur égal, leur compersonnier\footnote{Saint-Simon a déjà employé
  le mot \emph{compersonnier} dans le sens d'associé.} dans le maniement
de cette affaire, en devenir un des chefs, et les laisser si loin
derrière\,; les chefs même de se trouver un égal qui voudrait partager
leur autorité en partageant leur rang et leurs distinctions, avec qui ce
même rang les forcerait de compter, avec des égards qu'il saurait bien
se faire rendre\,; qu'ils seraient contraints de ménager même du côté de
Rome, et qui ne se détacherait pas facilement de ses idées particulières
de se faire un parti dans le leur, et qui chercherait sans cesse à
pointer et à primer, ce que la naissance ni le siège du cardinal de
Bissy ne lui avaient pas permis de tenter à l'égard du cardinal de
Rohan. Tant d'obstacles ne me rebutèrent point. Tous ceux-là avaient à
combattre une chose faite, l'engagement solennel de la cour de Rome, la
faiblesse du régent qui était la meilleure pièce en faveur de
l'archevêque\,; je m'en servis utilement pour lui faire sentir que Rome
ne reculerait pas, et qu'à chose faite, et qui malheureusement n'était
pas sans exemple, il était de la prudence de se prendre à tout ce qui
pouvait sauver l'honneur et les apparences, et d'éviter une longue suite
des plus épineux embarras dont on ne pouvait prévoir ni le terme, ni la
fin, ni tout ce qu'ils en pouvaient faire naître de plus fâcheux encore.
Ces représentations étaient tellement conformes au naturel de M. le duc
d'Orléans qu'elles firent plus de progrès et plus prompts que je ne
l'avais espéré.

Les choses en étaient là quand le mercredi matin du 20 décembre, La
Vrillière me vint dire que l'archevêque de Reims était arrivé la veille
fort tard à Paris. Ce voyage sans aucun concert avec nous, et fait à
l'insu de tout ce qui lui appartenait, nous parut une équipée qui
romprait toutes nos mesures et rejetterait M. le duc d'Orléans dans sa
première colère, pour être venu du lieu de son exil sans sa permission.
Nous nous trompions tous\,: l'abbé de La Fare-Lopis, son grand vicaire
et son homme à tout faire, était un fripon du premier ordre, plein
d'esprit et de ressources, qui jusqu'alors s'était présenté à tout
vainement, parce qu'il s'était tellement décrié par son abandon au P.
Tellier et aux jésuites, que jusqu'aux chefs de la constitution en
avaient en même temps peur et mépris, et l'avaient écarté de tout. La
promotion admise de Mailly lui parut une planche après le naufrage, si
elle pouvait l'être par son industrie. Il s'était affronté là-dessus à
l'abbé Dubois avec toute la hardiesse et la délicatesse possible, et
avait eu l'art d'en essuyer les plus énormes pouilles en face, sans se
fâcher qu'à propos et par mesure. Il eut celui de lui faire revenir
qu'il se méprenait beaucoup sur ses vues du côté de Rome, de s'élever si
fortement contre ce qu'elle venait de faire en faveur de Mailly, au lieu
de s'y faire un mérite de l'y servir, de l'aider à la tirer de
l'embarras de l'engagement si public où elle venait de se jeter, et à
Mailly de s'acquérir sur lui le service de lui faciliter le prompt
consentement du régent, au lieu d'irriter ce prélat par ses fougues,
duquel il voyait avec évidence quel était son crédit et sa considération
à Rome qui hasardait sciemment tout pour lui, et qui pouvait lui nuire
ou le servir si puissamment pour son chapeau. Ce funeste chapeau était
la boussole de Dubois, et plus funestement encore Dubois était devenu la
boussole du régent. Réflexion faite, le chapeau séducteur, quoique
encore vu de si loin, changea subitement Dubois. Il manda l'abbé de La
Fare, lui fit cent amitiés, et à force de prolonger des verbiages,
chercha à le faire parler pour profiter du ton qu'il prendrait.

La Fare plus fin que lui encore parce que, sans fougue et maître de
lui-même, rien ne le détournait des moyens de son but, se mit à rire, et
lui dit qu'il n'avait jamais été un moment la dupe des emportements
qu'il lui avait témoignés\,; qu'il avait senti tout d'abord que ces
mêmes emportements étaient le ton et le langage indispensable d'un
ministre en tel cas\,; qu'il n'en avait donc rien du tout sur le coeur,
ni pour soi ni pour Mailly, et tout de suite ajouta qu'il avait encore
soupçonné que ce grand appareil d'éclat, qui était bon pour le monde,
pouvait n'être pas inutile au désir qu'il ne croyait pas impossible
qu'eut Dubois de servir Mailly auprès du régent par des réflexions qu'il
lui ferait naître, et d'autant moins suspectes que la colère de lui
Dubois n'avait pas été moindre, et avait encore paru avec beaucoup moins
de mesures que celle du régent. À cette ouverture, Dubois, transporté de
croire avoir trompé qui le trompait en effet, embrasse l'abbé de La
Fare, avoue qu'il l'a deviné, s'écrie qu'un génie supérieur tel que le
sien mériterait le ministère, l'accable de louanges et de protestations
pour Mailly, et, plein de ses désirs qu'il ne peut cacher, lui montre à
découvert tout ce qu'il attend à Rome de la reconnaissance de Mailly, et
le plus profond secret en l'une et l'autre cour. La Fare, ravi de tenir
l'abbé Dubois pris dans le filet qu'il lui avait tendu, lui promet tout,
exagère le crédit de Mailly à Rome, ce que Dubois peut tirer de sa
reconnaissance, mais en même temps demande tout. Bref ils ne se
quittèrent point sans paroles réciproques, dont le gage fut de la part
de La Fare des propos en l'air qui ne coûtaient rien, tandis que Dubois
lui dit de mander à Mailly de venir secrètement sans en avertir aucun
des siens, de se tenir caché dans sa maison sans y voir que trois ou
quatre personnes au plus de ses plus proches ou de ses plus intimes, et
qu'il se chargeait lui Dubois de le renvoyer bientôt à peu près content,
et en chemin de l'être dans peu tout à fait, parce que cette affaire ne
se pouvait conduire à bien que par degrés. Ce mystère demeura
religieusement renfermé entre l'abbé Dubois, l'abbé de La Fare et
Mailly, archevêque de Reims, qui laissa pleinement croire à La
Vrillière, à moi, qui le vîmes tous les jours, et au peu de ce qui le
vit, qu'il était venu à l'aventure et au hasard de tout ce qui pourrait
en arriver. Cependant, quoique venu de la sorte, nous ne crûmes pas
prudent, quelque caché qu'il se tînt chez lui, de laisser apprendre à M.
le duc d'Orléans son arrivée par d'autres qui la pourraient découvrir,
et qui en la lui disant n'iraient pas à la parade de la colère qui en
serait l'effet. Mailly qui avait ses raisons qu'il ne nous disait pas,
approuva fort que nous révélassions son arrivée. La Vrillière n'osa s'en
charger, le paquet en tomba sur moi. Mailly était en calotte noire\,;
mais il avait la rouge dans sa poche\,; il l'en tirait de fois à autre
devant moi, la considérait, avec ravissement, par-ci, par-là la baisait,
puis me disait les yeux enflammés qu'il {[}ne{]} se la laisserait pas du
moins arracher de ses mains\,; en vérité je crois qu'il couchait avec
elle, comme font les enfants avec une poupée qu'on vient de leur donner.
Je parlai donc dès le lendemain à M. le duc d'Orléans, de l'arrivée
subite et clandestine de l'archevêque.

Ma surprise fut grande de le voir sourire et me dire d'un air affable\,:
«\,Il a bien envie de porter sa calotte.\,» Je cherchai à lui faire un
mérite de ce qu'il ne l'avait que dans sa poche, et nulle autre marque
de cardinal\,; puis voyant le régent en si belle humeur, j'en profitai
pour m'étendre sur le respect, l'obéissance, l'attachement de
l'archevêque, dont il pouvait profiter en le traitant avec bonté, pour
éviter des embarras infinis avec Rome sur sa promotion\,; pour y faire
sûrement passer et valoir tout ce qu'il voudrait sans la connaissance
des cardinaux de Rohan et de Bissy, lequel l'avait si traîtreusement
trompé, comme lui-même l'avait vu, le lui avait reproché, et me l'avait
dit, par ses lettres prises au courrier de Rome, toutes contraires, et
avec fureur, à celles qu'il lui avait donné sa parole formelle d'écrire.
Enfin je flattai le régent par son goût d'opposer, dans le même parti,
des chefs les uns aux autres. À mesure que je sentais que mes raisons
prenaient, je m'applaudissais de mon bien-dire, tandis que mes discours
n'avaient pas la moindre part à leur succès. J'ignorais pleinement
l'abbé Dubois gagné et auteur du voyage, qu'il avait tout aplani en
telle sorte que le régent n'attendait que la première confidence de
l'arrivée de l'archevêque et l'accompagnement de quelques propos
là-dessus, pour en venir à la composition résolue entre l'abbé Dubois et
lui. Ce fut donc sans peine, et avec grand étonnement, que je crus
obtenir que M. le duc d'Orléans verrait l'archevêque, recevrait ses
respects, ses pardons, ses excuses, lui prescrirait ses volontés et les
conditions sous lesquelles, après un délai raisonnable, il lui
permettrait d'être cardinal. Celle que M. le duc d'Orléans mit pour lors
fut que je lui amènerais le lendemain, entre six et sept heures du soir,
l'archevêque par les derrières, que je serais seul en tiers, et que
l'archevêque viendrait et s'en retournerait seul avec moi dans mon
carrosse, et sans flambeaux.

Je crus avoir remporté une incroyable victoire, et j'admirais avec
quelle facilité La Vrillière, à qui je la contai, n'en pouvait revenir,
et trouvait mon crédit suprême. Mailly joua en apparence le même
personnage que La Vrillière faisait tout de bon, et il est vrai que je
m'en applaudissais, quoique j'y sentisse toute la faiblesse de M. le duc
d'Orléans, mais sans me douter le moins du monde de l'influence de
l'abbé Dubois. Je menai donc l'archevêque au régent avec le mystère qui
m'avait été prescrit. Tous deux d'abord parurent embarrassés l'un de
l'autre. Je me mis de la conversation en chancelier de l'archevêque. Ils
se remirent et parlèrent convenablement tous deux. J'avais fort fait le
bec à l'archevêque, dont je craignais la hauteur et l'indiscrète
vivacité autre panneau où je tombai encore. Il avait pris sa leçon de
Dubois même par l'abbé de La Fare que je ne vis ni n'aperçus jamais dans
toute cette affaire, que longtemps après cette présentation. Les propos
finis, M. le duc d'Orléans déclara à l'archevêque les conditions
auxquelles il voulut qu'il se soumît pour arriver au consentement du roi
d'accepter publiquement la pourpre\,: n'en porter ni la qualité, ni
calotte, ni aucune marque sur soi, à ses armes, ni dans ses titres,
jusqu'à ce qu'il eût reçu la calotte des mains du roi, retourner
aussitôt à Reims, et ne point sortir de son diocèse sans être mandé\,;
de n'écrire à personne en France que dans son style ordinaire, et ne
signer que \emph{l'archevêque duc de Reims}. Néanmoins permis à lui
d'écrire aux étrangers hors du royaume en cardinal, et de signer ces
lettres-là\,: \emph{le cardinal de Mailly}. C'était là un si grand pas
que j'en demeurai étourdi. Je me jetai dans les remercîments, et je ne
sortais point d'étonnement d'en trouver si peu dans l'archevêque. Je
l'attribuai à sa vanité, et n'imaginai jamais qu'il eût en entrant la
plus légère idée de ce qui se passerait, tandis qu'intérieurement il se
moquait de ma simplicité, et sûrement M. le duc d'Orléans beaucoup
davantage\,; et je ne sus avoir été joué de la sorte que des années
après que le roi eut donné la calotte au cardinal de Mailly.

Achevons tout de suite ce qui regarde ce cardinal presque éclos jusqu'à
ce qu'il le soit tout à fait, pour n'avoir pas à revenir à une matière
et à un personnage qui n'a guère d'autre part en celles de ces Mémoires
que sa promotion. Dubois, résolu de profiter de sa situation, le laissa
languir cinq mois dans son diocèse dans cet état amphibie, en attendant
une occasion utile de l'en tirer et le préparer cependant par l'ennui et
l'impatience, à se rendre flexible à tout ce qu'il pourrait en exiger.
De temps en temps je pressais le régent de finir sa peine\,; il me
répondait qu'à la façon dont l'archevêque s'était fait cardinal, il
n'avait pas à se plaindre d'un délai et d'un séjour dans son diocèse,
qui le laissait cardinal au dehors du royaume, et qui lui répondait
enfin d'obtenir sûrement sa calotte des mains du roi. Je sentais cette
vérité peut-être plus encore que ne faisait celui qui me la disait. Je
laissais un intervalle, puis je demandais quand cet état finirait\,; à
la fin j'obtins, à ce que je crus, le retour de l'archevêque et qu'en
arrivant, la calotte lui serait donnée, et je me remerciais de ce que
mon éloquence et ma persévérance avait enfin réussi. La Vrillière ne se
laissait point de me remercier, et toute la famille et les amis\,; autre
duperie et tout aussi lourde que la première. Je n'eus pas plus de part
à la conclusion que je n'en avais eue à l'ébauche, et le rare est que
sur toutes les deux La Vrillière soit mort dans l'erreur et qu'il y a
fort peu de gens qui n'y soient encore. Voici donc ce qui mit enfin
publiquement la calotte rouge sur la tête du cardinal.

J'ai fait mention plus haut, par anticipation, du corps de doctrine du
cardinal de Noailles, approuvé par les cardinaux de Rohan et de Bissy,
et par une assemblée d'évêques, tenue par eux à Paris. Sur quoi je dois
avouer que j'ai confondu une autre affaire de même genre, sur laquelle
le cardinal de Bissy écrivit à Rome avec fureur, tout le contraire de ce
qu'il avait formellement promis à M. le duc d'Orléans, duquel la
défiance fit arrêter le courrier un peu en deçà de Lyon, et prendre les
lettres de Bissy que M. le duc d'Orléans montra à ce cardinal, avec les
reproches que méritait sa perfidie. Ce corps de doctrine ainsi approuvé,
et que la même perfidie redoublée des cardinaux de Rohan et de Bissy fit
aussi échouer, il fut question de le faire approuver par tous les autres
évêques absents, avant de l'envoyer à Rome. Pour y parvenir, on choisit
plusieurs du second ordre bien dévoués à la constitution et à faire
fortune par elle, qu'on endoctrina et qu'on chargea de porter ce corps
de doctrine chacun à un nombre d'évêques qu'on leur assigna. L'abbé de
La Fare-Lopis n'avait garde de n'être pas du nombre de ces courriers, et
il était naturel qu'étant grand vicaire et l'homme de confiance de
l'archevêque de Reims, il eût la commission de lui porter le corps de
doctrine à signer. On craignait qu'il ne se rendît plus difficile
qu'aucun, par sa haine personnelle contre le cardinal de Noailles et par
ses ménagements pour Rome dans la conjoncture où il se trouvait, à
laquelle on n'avait point encore fait part d'un ouvrage qui touchait ses
prétentions de si près. L'abbé de La Fare, à qui le voyage de Reims fut
destiné, saisit en habile compagnon la difficulté qu'on craignait, la
grossit tant qu'il put, effraya l'abbé Dubois de l'effet du refus du
prélat, de la vigueur et du peu de ménagement de l'archevêque, assis sur
un siège tel que celui de Reims, que le pape venait de faire cardinal et
qui était sans doute de fort mauvaise humeur du hoquet qu'on faisait
durer si longtemps, à lui en laisser prendre les marques, la qualité et
le rang.

La Fare n'oublia rien pour augmenter l'embarras de l'abbé Dubois, et le
laissa quelques jours dans cette peine. Dubois le mandait sans cesse
pour chercher quelque expédient. Quand La Fare le jugea à son point, il
lui dit qu'après bien des réflexions, il croyait lui en pouvoir proposer
un\,; mais qu'il était unique, et à son avis \emph{causa sine qua non}.
Il verbiagea un peu avant de s'en ouvrir, pour exciter le désir de
Dubois\,; puis, l'ayant amené à ne rien refuser, il lui dit que,
puisqu'il regardait comme si essentiel d'amener l'archevêque à signer
l'approbation d'un corps de doctrine fait par son ennemi et inconnu
encore à Rome, il fallait flatter sa vanité dans la manière et à la fin
le satisfaire\,; que, pour cela, il fallait le distinguer des autres
prélats, à qui on envoyait des gens du second ordre, et lui députer à
lui l'évêque de Soissons\,; que cela était tout naturel, parce qu'il
était son premier suffragant, ardent constitutionnaire, d'ailleurs son
voisin, dont le voyage serait imperceptible d'ailleurs, Soissons étant
sur le chemin de Paris à Reims\,; que cela aurait un tout autre poids
auprès de l'archevêque, que non pas lui La Fare, son grand vicaire,
quoique son ami\,; mais que cela ne suffisait pas encore\,; qu'il
fallait toucher l'archevêque par son intérêt le plus vif et le plus
pressant, profiter de l'occasion de mettre fin à un état de souffrance
qui ne pouvait pas toujours durer\,; que pour cela il fallait encore s'y
prendre avec la délicatesse que demandait la vanité\,; qu'après avoir
bien tout pesé et balancé, il croyait qu'il fallait charger Languet de
deux lettres de M. le duc d'Orléans pour l'archevêque\,: par l'une le
presser de signer en termes qui flattassent son orgueil, y ajouter que
ce n'était point comme condition que la signature lui était demandée, et
que, signant ou refusant, il pouvait venir, quand il voudrait, recevoir
sa calotte des mains du roi\,; par l'autre lettre lui mander qu'il
fallait signer nettement et sur-le-champ ou compter qu'il demeurerait
exilé et sans calotte pour toujours\,; l'une pour lui faire un
sauve-l'honneur qu'il pût montrer, et donner en même temps plus de poids
ici et à Rome à sa signature\,; l'autre pour lui parler François et lui
serrer le bouton par son plus sensible et à découvert. L'abbé Dubois
goûta l'expédient, le fit approuver par M. le duc d'Orléans, qui écrivit
les deux lettres. Languet, évêque de Soissons, si outré que l'archevêque
lui eût pris son chapeau, eut le goupillon de le lui aller assurer\,; il
porta les deux lettres à l'archevêque, qui empocha l'une et se para de
l'autre. Il signa tout de suite, et se hâta d'accourir jouir en plein de
son cardinalat.

Toute difficulté étant ainsi levée, je menai le cardinal, mais encore en
calotte noire, à M. le duc d'Orléans. L'accueil fut très gracieux\,; le
régent lui dit qu'il prendrait le lendemain les ordres du roi pour le
jour et l'heure de lui donner la calotte. Je ne vis jamais homme si
transporté de joie de se voir enfin au bout de ses longs et persévérants
travaux. Ce fut donc le surlendemain que j'allai prendre l'archevêque
chez lui sur les dix heures du matin\,; je le menai dans mon carrosse
aux Tuileries. Comme il était archevêque de Reims, cardinal ou non, je
n'avais point d'embarras avec lui\,: nous fûmes aussitôt introduits dans
le cabinet du roi, qui y était seul avec M. le duc d'Orléans, le
maréchal de Villeroy, M. de Fréjus et deux ou trois autres. M, le duc
d'Orléans le présenta au roi, ne le nommant qu'archevêque, mais ajoutant
ce qui l'amenait avec quelques propos obligeants. Aussitôt l'archevêque
qui avait à la main sa calotte rouge, la présenta au roi, ôta la noire
qu'il avait sur la tête, se baissa tout le plus bas qu'il lui fut
possible, et reçut sur sa tête la rouge des mains du roi, après quoi il
lui fit une profonde révérence, et quelques mots de remercîment. Alors
M. le duc d'Orléans l'appela M. le cardinal, lui fit son compliment, et
ce qui était dans la chambre. Tout cela fut extrêmement court\,: nous
fîmes tous deux la révérence, et nous nous en allâmes. Le cardinal se
contint tant qu'il put\,; mais il ne touchait pas à terre. Je le remenai
chez lui au bout du Pont-Royal. Ainsi finit cette longue et mystérieuse
affaire.

\hypertarget{chapitre-xvii.}{%
\chapter{CHAPITRE XVII.}\label{chapitre-xvii.}}

1719

~

{\textsc{Sécheresse où ces Mémoires vont tomber, et ses causes.}}
{\textsc{- Chute du cardinal Albéroni qui se retire en Italie.}}
{\textsc{- Dona Laura Piscatori nourrice et assafeta de la reine
d'Espagne.}} {\textsc{- Son caractère.}} {\textsc{- Albéroni arrêté en
chemin, emportant le testament original de Charles II et quelques autres
papiers importants, qu'il ne rend qu'à force de menaces.}} {\textsc{-
Joie publique en Espagne de sa chute, et dans toute l'Europe.}}
{\textsc{- Marcieu garde honnêtement à vue le cardinal Albéroni jusqu'à
son embarquement à Marseille, qui ne reçoit nulle part ni honneur ni
civilité.}} {\textsc{- Sa conduite en ce voyage.}} {\textsc{- Folles
lettres d'Albéroni au régent sans réponse.}} {\textsc{- Aveuglement
étrange de souffrir dans le gouvernement aucun ecclésiastique, encore
pis des cardinaux.}} {\textsc{- Cause de la rage d'Albéroni.}}
{\textsc{- But de tout ministre d'État ecclésiastique ou qui parvient à
se mêler d'affaires.}} {\textsc{- Disposition du roi très différente, et
sa cause, pour M. le duc d'Orléans et pour l'abbé Dubois, également haïs
du maréchal de Villeroy et de l'évêque de Fréjus.}} {\textsc{- Conduite
de tout cet intérieur.}} {\textsc{- M. le duc d'Orléans résolu de
chasser le maréchal de Villeroy et de me faire gouverneur du roi.}}
{\textsc{- Il me le dit.}} {\textsc{- Je l'en détourne.}}

~

Nous voici arrivés à une époque bien curieuse\,; mais quel dommage que
Torcy n'ait pas poussé plus loin qu'il n'a fait le recueil des extraits
des lettres que le secret de la poste lui ouvrait, et quel déplaisir de
ce que le crédit imposant et toujours augmentant de l'abbé Dubois sur M.
le duc d'Orléans ne lui permettait plus sa confiance accoutumée pour
ceux qui lui étaient le plus fidèlement attachés\,! Ce double malheur
privera désormais ces Mémoires des plus curieuses connaissances. Je n'y
veux et n'y puis écrire que ce qui a passé sous mes yeux ou ce que j'ai
appris de ceux-là mêmes par qui ont passé les affaires. J'aime mieux
avouer franchement mon ignorance que de hasarder des conjectures qui
sont souvent peu différentes des romans\,; c'est où j'en serai souvent
réduit désormais\,; mais je préfère la honte de l'avouer et d'en avertir
pour le reste de ces Mémoires, à me faire de déplorables illusions, et
tromper ainsi mes lecteurs, si tant est que ces Mémoires voient jamais
le jour.

Les tyrans et les scélérats ont leur terme, ils ne peuvent outrepasser
celui que leur a prescrit l'arbitre éternel de toutes choses. On a si
amplement vu qu'Albéroni était l'un et l'autre par tout ce qui d'après
Torcy a été ici rapporté de lui, qu'il n'y a plus rien à ajouter sur ce
monstrueux personnage. L'Europe entière, victime de ses forfaits par un
endroit ou par un autre, détestait un maître absolu de l'Espagne, dont
la perfidie, l'ambition, l'intérêt personnel, les vues toujours
obliques, souvent les caprices, quelquefois même la folie, étaient les
guides, et dont l'unique intérêt continuellement varié et diversifié
selon que la fantaisie le lui montrait, se cachait sous des projets
toujours incertains, et dont la plupart étaient d'exécution impossible.
Accoutumé à tenir le roi et la reine d'Espagne dans ses fers et dans la
prison la plus étroite et la plus obscure, où il avait su les renfermer
sans communication avec personne, à ne voir, à ne sentir, à ne respirer
que par lui, et à revêtir toutes ses volontés en aveugles, il faisait
trembler toute l'Espagne, et avait anéanti tout ce qu'elle avait de plus
grand par ses violences. Accoutumé à n'y garder aucune sorte de mesure,
méprisant son maître et sa maîtresse, dont il avait absorbé toutes les
volontés et tout le pouvoir, il brava successivement toutes les
puissances de l'Europe, et ne se proposa rien moins que de les tromper
toutes, puis de les dominer, de les faire servir à tout ce qu'il
imagina, et se voyant enfin à bout de toutes ses ruses, à exécuter seul
et sans alliés le plan qu'il s'était formé. Ce plan n'était rien moins
que d'enlever à l'empereur tout ce que la paix d'Utrecht lui avait
laissé en Italie, de ce que la maison d'Autriche espagnole y avait
possédé, d'y dominer le pape, le roi de Sicile, auquel il voulait ôter
cette île comme arrachée à l'Espagne par la même paix, dépouiller
l'empereur du secours de la France et de l'Angleterre en soulevant la
première contre le régent par les menées de l'ambassadeur Cellamare et
du duc du Maine, et jetant le roi Jacques en Angleterre par le secours
du Nord, occuper le roi Georges par une guerre civile\,; enfin de
profiter pour soi de ces désordres pour transporter sûrement en Italie,
que son cardinalat lui faisait regarder comme un asile assuré contre
tous les revers, l'argent immense qu'il avait pillé et ramassé en
Espagne, sous prétexte d'y faire passer les sommes nécessaires au roi
d'Espagne pour y soutenir la guerre et les conquêtes qu'il y ferait, et
cet objet d'Albéroni était peut-être le moteur en lui de ses vastes
projets. Leur folie ne put être comprise\,; ce ne fut qu'avec le temps
qu'on découvrit enfin avec le plus grand étonnement que son obstination
dans son plan, et à rejeter toutes les propositions les plus
raisonnables n'avait point d'autre fondement que sa folie, ni d'autres
ressources que les seules forces de l'Espagne contre celles de
l'empereur, de la France, de l'Angleterre et de la Hollande, que cette
dernière couronne entraîna après soi. Pour comble d'extravagance, la
découverte de la conspiration brassée en France, et le bon ordre qui y
fut mis aussitôt, ni les contretemps arrivés dans le Nord, qui ne
laissèrent plus d'espérance à Albéroni d'occuper ces deux couronnes chez
elles assez puissamment pour leur faire quitter prise au dehors, ne le
purent déprendre de pousser la guerre et ses projets, dont les
prodigieux préparatifs avaient entièrement achevé d'épuiser l'Espagne
sans l'avoir pu mettre en état de tenir un moment contre toute l'Europe,
neutre ou alliée pour soutenir l'empereur en Italie, qui à la fin y
gagna Naples, la Sicile et quelques restes de la Lombardie qu'il n'y
possédait pas.

Albéroni abhorré en Espagne en tyran cruel de la monarchie qu'il
s'appropriait uniquement, en France, en Angleterre, à Rome, et par
l'empereur comme un ennemi implacable et personnel, semblait n'avoir pas
la moindre inquiétude. Il était pourtant impossible que le roi et la
reine d'Espagne ignorassent les malheurs de leurs troupes et de leur
flotte en Sicile, le danger prochain de la révolution de Naples,
l'impossibilité de réparer tant de pertes, et de soutenir avec les
seules forces de l'Espagne, qui n'en avait plus aucune, toutes celles de
l'empereur, de la France et de l'Angleterre, même la Hollande, unies, et
les cris du pape et de toute l'Italie. Le régent et l'abbé Dubois, qui
n'avaient que trop de raisons de regarder depuis longtemps Albéroni
comme leur ennemi personnel à chacun d'eux, étaient sans cesse
sourdement occupés des moyens de sa chute\,; ils crurent ce moment
favorable, ils surent en profiter. Le comment, c'est le curieux détail
qui n'est pas venu jusqu'à moi, et qui mérite d'être bien regretté. M.
le duc d'Orléans a survécu Dubois de trop peu de mois pour que j'aie pu
ressasser avec lui beaucoup de choses, et celle-ci est une de celles que
je n'ai point mises sur le tapis depuis que sa confiance me fut
rouverte, entraîné par le courant et par d'autres choses, et comptant
toujours d'avoir le temps d'y revenir. Tout ce que j'ai su avec
connaissance par M. le duc d'Orléans dans le temps même, mais en deux
mots, et depuis en Espagne, sans y avoir trouvé plus d'éclaircissement
et de détails, c'est ce qu'on a vu dans ce qui a été rapporté ici de
Torcy, qu'Albéroni avait toujours redouté, {[}et qui{]} lui arriva. Il
tremblait du moindre Parmesan qui arrivait à Madrid\,; il n'omit rien
par le duc de Parme et par tous les autres moyens qu'il put imaginer
pour les empêcher d'y venir\,; il regarda sans cesse avec tremblement le
peu de ceux dont il n'avait pu rompre le voyage ni procurer le renvoi.

Parmi ceux-ci, il ne craignit rien tant que la nourrice de la reine, à
laquelle, parmi ses ménagements, il lâchait quelquefois des coups de
caveçon pour la contenir, où le raisonnement politique avait peut-être
moins de part que l'humeur. Cette nourrice qui était une grosse paysanne
du pays de Parme, s'appelait Dona Laura Piscatori\,; elle n'était venue
en Espagne que quelques années après la reine qui l'avait toujours
aimée, et qui la fit peu après son \emph{assafeta}, c'est-à-dire sa
première femme de chambre, mais qui en Espagne est tout autrement
considérable qu'ici. Laura avait amené son mari, paysan de tous points,
que personne ne voyait et ne connaissait\,; mais Laura avait de
l'esprit, de la ruse, du tour, des vues à travers la grossièreté
extérieure de ses manières, qu'elle avait conservées ou par habitude,
peut-être aussi par politique pour se faire moins soupçonner, et comme
les personnes de cette extraction, parfaitement intéressée. Elle
n'ignorait pas combien impatiemment Albéroni souffrait sa présence et
craignait sa faveur auprès de la reine, qu'il voulait posséder seul\,;
et plus sensible aux coups de patte qu'elle recevait de lui de temps en
temps qu'à ses ménagements ordinaires, elle ne le regardait que comme un
ennemi très redoutable, qui la retenait dans d'étroites bornes, qui
l'empêchait de profiter de sa faveur en contenant là-dessus la reine
elle-même, et duquel le dessein était de la faire renvoyer à Parme, et
de n'oublier rien pour y réussir. Voilà tout ce que j'ai pu apprendre
sans autre détail, sinon que voyant la conjoncture favorable, par ce qui
vient d'être représenté de la situation des affaires d'Espagne, où la
tyrannie d'Albéroni était généralement abhorrée, elle fut aisément
gagnée par l'argent du régent, et l'intrigue de l'abbé Dubois pour
hasarder d'attaquer Albéroni auprès de la reine, et par elle auprès du
roi, comme un ministre qui avait ruiné l'Espagne, qui était l'unique
obstacle de la paix pour ses vues personnelles, auxquelles il avait
sacrifié sans cesse Leurs Majestés Catholiques et les avait commises
seules contre toutes les puissances de l'Europe. Comme je ne raconte que
ce que je sais, je serai bien court sur un événement si intéressant.

Laura réussit. Albéroni, au moment le moins attendu, reçut un billet du
roi d'Espagne, par lequel il lui ordonnait de se retirer à l'instant
sans voir ni écrire à lui ni à la reine, et de partir dans deux fois
vingt-quatre heures pour sortir d'Espagne\,; et cependant un officier
des gardes du corps fut envoyé auprès de lui jusqu'à son départ. Comment
cet ordre accablant fut reçu, ce que fit et ce que devint le cardinal,
je l'ignore\,; je sais seulement qu'il obéit et qu'il prit son chemin
par l'Aragon. On eut si peu de précaution à l'égard de ses papiers et
des choses qu'il emportait qui furent immenses en argent et en
pierreries, que ce ne fut qu'après les premières journées que le roi
d'Espagne fut averti que le testament original de Charles II ne se
trouvait plus. On jugea aussitôt qu'Albéroni avait emporté ce titre si
précieux par lequel Charles II nommait Philippe V roi d'Espagne, et lui
léguait tous ses vastes États, pour s'en servir peut-être à gagner les
bonnes grâces et la protection de l'empereur, en lui faisant un
sacrifice. On envoya arrêter Albéroni. Ce ne fut pas sans peine et sans
les plus terribles menaces qu'il rendit enfin le testament, en jetant
les plus hauts cris, et quelques autres papiers importants qu'on s'était
aperçu en même temps qui manquaient. La terreur qu'il avait imprimée
l'était si profondément, que jusqu'à ce moment personne n'osa parler ni
montrer sa joie, quoique parti. Mais cet événement rassurant contre le
retour, ce fut un débordement sans exemple d'allégresse universelle,
d'imprécations et de rapports contre lui au roi et à la reine, tant des
choses les plus publiques qu'eux seuls ignoraient, que d'une infinité de
forfaits particuliers qui ne sont plus bons qu'à passer sous silence.

M. le duc d'Orléans ne contraignit point sa joie, moins encore l'abbé
Dubois\,: c'était leur ouvrage qui renversait leur ennemi personnel, et
avec lui le mur de séparation si fortement élevé par Albéroni entre le
régent et le roi d'Espagne, et du même coup l'obstacle unique de la
paix. Cette dernière raison fit éclater la même joie en Italie, à
Vienne, à Londres\,; les puissances alliées s'en félicitèrent\,;
jusqu'aux Hollandais furent ravis d'être délivrés d'un ministère si
double, si impétueux, si puissant, et on espéra à Turin trouver des
ressources de politique et de ruses qu'Albéroni avait tant contribué à
rendre suspectes ou inutiles. M. le duc d'Orléans dépêcha le chevalier
de Marcieu, homme fort adroit, fort intelligent, et fort dans la main de
l'abbé Dubois aux derniers confins de la frontière pour y attendre
Albéroni, l'accompagner jusqu'au moment de son embarquement en Provence
pour l'Italie, ne le pas perdre de vue, lui faire éviter les grandes
villes et même les gros lieux autant qu'il serait possible, ne pas
souffrir qu'il lui fût rendu aucune sorte d'honneur, surtout empêcher
quelque communication que ce pût être avec lui sans exception de
personne, en un mot, le conduire civilement comme un prisonnier gardé à
vue. Marcieu exécuta à la lettre cette commission désagréable, mais
d'autant plus nécessaire que, tout disgracié qu'était Albéroni, on en
craignait encore les dangereuses pratiques, traversant une grande partie
de la France, où tout ce qui était contraire au régent, avait eu recours
à lui, et où l'affaire de Bretagne n'était pas encore finie, et ce ne
fut pas sans grande raison que toute sorte de liberté, d'accès, de
curiosité même lui fut soigneusement retranchée.

On peut juger ce qu'en souffrit un homme si impétueux et si accoutumé à
tout pouvoir et à tout faire\,; mais il sut s'accommoder à un si grand
et si prompt changement d'état, se posséder, ne se hasarder à aucun
refus, être sage et mesuré en toutes ses manières, très réservé en ses
paroles, avoir l'air de ne prendre garde à rien, à s'accommoder de tout
singulièrement, sans questions, sans prétentions, sans plaintes,
dissimulant tout, et montrant, sans s'en lasser, de prendre Marcieu
comme un accompagnement d'honneur. Il ne reçut donc aucune civilité de
la part du régent, de Dubois, ni de personne, et fit, sans s'arrêter,
avec presque nulle suite, les journées marquées par Marcieu, jusqu'au
bord de la Méditerranée, où il s'embarqua en arrivant, et passa à la
côte de Gênes. Ce fut dans ce voyage où Marcieu apprit de lui l'anecdote
si curieuse touchant la disgrâce de la princesse des Ursins, convenue
entre les deux rois, dont la nouvelle reine d'Espagne fut chargée pour
la manière de l'exécution, qui a été ici racontée au temps de cette
disgrâce, et que je sus du marquis, depuis maréchal de Brancas, à qui
Marcieu l'avait depuis racontée. Albéroni, délivré de son Argus et
arrivé en Italie, s'y trouva aussitôt en d'autres embarras par la colère
de l'empereur, qui ne l'y voulut souffrir nulle part, et par
l'indignation de la cour de Rome, qui se trouva l'emporter, en cette
occasion, sur sa jalousie du respect de sa pourpre. Il fut réduit à se
tenir longtemps errant et caché, et il ne put approcher de Rome que par
la mort du pape. Le surplus de la vie de cet homme si extraordinaire
n'est plus matière de ces Mémoires. Mais ce qui n'y doit pas être oublié
est la dernière marque de rage, de désespoir et de folie, qu'il donna en
traversant la France. Il écrivit de Montpellier, à M. le duc d'Orléans,
des offres de lui donner les moyens de faire la plus dangereuse guerre à
l'Espagne\,; et de Marseille, prêt à s'embarquer, il lui écrivit de
nouveau pour lui réitérer et le presser sur les mêmes offres. Il garda
peu de décence sur le roi et la reine d'Espagne, et ne put s'empêcher
d'ajouter que le pape, l'empereur et Leurs Majestés Catholiques
rendraient compte à Dieu de l'avoir empêché d'avoir les bulles de
l'archevêché de Séville.

On ne peut s'empêcher de s'arrêter ici une dernière fois sur Albéroni et
sur l'aveuglement de souffrir des ecclésiastiques dans les affaires,
surtout des cardinaux, dont le privilège le plus spécial est l'impunité
de tout ce qui est de plus infamant et de plus criminel en tout genre.
Ingratitude, infidélité, révolte, félonie, indépendance, sans qu'il en
soit rien, pas même le plus souvent dans la conduite de personne à
l'égard de ces éminents coupables, même assez peu perceptiblement dans
l'opinion commune qui s'y est accoutumée par les exemples de tous les
temps. Il fallait qu'Albéroni eût la tête bien étrangement tournée par
la rage et le désespoir, pour faire cette plainte si fort inutile sur
Séville. Il avait voulu soulever l'Europe entière contre l'empereur pour
lui arracher l'Italie, sans s'être jamais rendu à aucune sorte de
composition pour l'Espagne, ni de raison\,; devait-il s'étonner que
l'empereur, qui le regardait comme son ennemi personnel, s'opposât à ce
qui augmentait son pouvoir et sa grandeur\,? Il avait traité vingt fois
le pape avec la dernière indignité\,; était-il surprenant qu'il ne le
trouvât pas favorable pour les bulles de Séville\,? Que ne devait-il pas
à Leurs Majestés Catholiques, de quelle poussière ne l'avaient-ils pas
tiré, à quel degré de puissance et de grandeur ne l'avaient-ils pas
élevé, et à quoi et combien de fois ne s'étaient-ils pas commis avec la
plus extrême persévérance pour lui obtenir le chapeau\,? Et il en parle
avec le dernier mépris, et s'offre à faire servir à leur ruine la
connaissance intime que leur aveugle bonté lui a donnée de toutes leurs
affaires, en le faisant régner absolument et si longtemps en Espagne. À
qui fait-il des offres si abominables\,? À un prince qu'il a forcé à
devenir leur ennemi, dont lui-même a fait tout ce qui a été en lui pour
renverser la régence par les plus indignes pratiques, et qu'il ne peut
douter qu'il n'ait contribué à sa chute, à tout le moins qu'il ne la
regarde comme un des plus grands bonheurs qui pussent lui arriver. Voilà
donc tout à la fois le comble du crime et de la folie. Aussi M. le duc
d'Orléans ne lui fit aucune réponse. Mais il faut dévoiler ici le grand
motif de cette rage et de ce désespoir à qui il ne put refuser de
s'exhaler par ces deux lettres.

Tout ecclésiastique qui arrive, de quelque bassesse que ce puisse être,
à mettre le pied dans les affaires, a pour but d'être cardinal et d'y
sacrifier tout sans réserve. Cette vérité est si certaine, et tellement
fortifiée d'exemples de tous les temps jusqu'aux nôtres, qu'elle ne peut
être considérée que comme un axiome le plus évident et le plus certain.
On a vu dans ce qu'on a donné ici d'après Torcy, les ressorts sans
nombre et sans mesure qu'Albéroni inventa et fit jouer pour arracher du
pape le cardinalat, et s'acquérir ainsi tout droit d'impunité la plus
étendue, quoi qu'il commît, de la plus sûre et de la plus ferme
considération, et les moyens de revenir toujours à figurer où que ce
fût. Mais ce n'était qu'un degré\,: ses vues étaient plus vastes, il
voulait Tolède, et pour y arriver il se fit donner le riche évêché de
Malaga et se fit sacrer. Tolède ne vacant point, il saisit l'instant de
la mort de l'illustre cardinal Arias, archevêque de Séville, et en
attendant Tolède, il se fit nommer à se second archevêché d'Espagne. De
là à Tolède, il n'y avait plus qu'un pas\,; mais demeurant même
archevêque de Séville avec sa pourpre, il était à la tête du clergé
d'Espagne. La puissance où il s'était établi lui donnait tous les moyens
nécessaires à le pratiquer sans bruit et se l'attacher. Cardinal et
archevêque, rien ne le pouvait plus tirer d'Espagne\,; ce nouveau titre
l'affermissait dans la place de premier et de tout-puissant ministre.
Appuyé de la sorte il arrivait au but qu'il s'était proposé de se faire
redouter par le roi et la reine, et de devenir même à découvert le tyran
de l'Espagne\,; et si, par impossible à ses yeux, il tombait enfin du
premier ministère, inviolable par sa pourpre, et à la tête du clergé
qu'il se serait attaché, quel odieux personnage, mais quel puissant ne
fût-il pas demeuré en un pays où le clergé a une autorité si grande,
qu'il oblige le roi de compter avec lui sur les levées et sur toutes
autres choses à tous moments\,! C'est ce dessein, bien qu'avorté par
l'opiniâtre et heureux refus des bulles de Séville, suivi de si près par
sa chute, qui le rendit si longtemps inflexible à la démission de
Malaga, que le pape et le roi d'Espagne lui demandèrent\,; c'était tenir
encore par un filet ce projet qui lui était si cher, qui tout chimérique
qu'il fût par n'avoir pas eu le temps de le laisser mûrir et de le faire
éclore, était toujours le plus avant dans son coeur\,; et c'est, pour le
dire en passant, le danger extrême du gouvernement des ecclésiastiques
qui se rendent si facilement indépendants de leur roi, et qui, ce grand
pas fait, ont des moyens de se maintenir par une force, contre laquelle
toute la temporelle a la honte de lutter ou de souffrir tout,
quelquefois d'étranges inconvénients à subir, et toujours en plein
spectacle. Sans remonter pour la France aux cardinaux Balue, Lorraine,
Guise et autres encore, les cardinaux de Retz, Bouillon, et celui-ci en
rafraîchissent l'importante leçon que le cardinal Dubois, s'il eût vécu,
eût certainement renouvelée aux dépens de M. le duc d'Orléans, s'il
l'avait pu. Ce n'est pas idée, imagination, mais réalité effective, dont
il prenait déjà sourdement toutes les mesures et les dimensions. Mais le
roi ne le put jamais aimer, de quoi son gouverneur et son précepteur, en
cela parfaitement de concert, surent parfaitement le garder et
l'éloigner, et M. le duc d'Orléans, qui gémissait sur les fins sous
l'empire de sa créature, tout faible à l'excès qu'il fût, ne lui aurait
pas laissé le temps de l'expulser, connaissant surtout les dispositions
du roi qui l'aimait et le montrait à demi, malgré les deux mêmes et sa
disposition contraire à l'égard de Dubois.

Si on s'étonne de cette différence à l'égard de deux hommes si
principaux, qui étaient également l'objet de la haine du maréchal de
Villeroy et de l'évêque de Fréjus, un mot d'éclaircissement ne peut être
que curieux. Rien de si désagréable que l'énonciation, le forcé et faux
palpable de toutes les manières et de tout l'extérieur de l'abbé Dubois,
même en voulant plaire. Rien de plus gracieux ni de plus agréable que
l'énonciation, l'extérieur et toutes les manières de M. le duc
d'Orléans, même sans penser à plaire\,; cette différence qui fait une
impression naturelle sur tout le monde, frappe et affecte encore plus un
roi de dix ans. Rien encore de si naturellement glorieux que les
enfants, combien plus un enfant couronné et gâté\,! Le roi était en
effet très glorieux, très sensible, très susceptible là-dessus, où rien
ne lui échappait sans le montrer. Dubois ne travaillait point avec lui,
mais il le voyait et lui parlait avec un air de familiarité et de
liberté qui le choquait et qui découvrait aisément le dessein de
s'emparer de lui peu à peu, ce que le maréchal de Villeroy et Fréjus
encore plus redoutaient comme la mort.

Tous deux faisaient remarquer au roi et lui exagéraient les airs peu
respectueux et indécents de l'abbé Dubois à son égard, et l'éloignaient
de lui, pour ainsi dire à la tâche, en lui en inspirant de la crainte.
Ils n'étaient pas en de meilleures dispositions pour M. le duc
d'Orléans. Le maréchal de Villeroy entre le roi et lui, ou le seul
Fréjus en tiers, donnaient carrière à sa haine. Mais le roi le craignait
et ne l'aimait point. L'autorité seule lui donnait quelque créance, mais
faiblement. Fréjus qu'il aimait et qui avait captivé et obtenu toute sa
confiance, aurait été dangereux s'il avait aidé le maréchal contre le
régent, comme il le secondait contre Dubois. Mais il se contentait
d'éviter d'être suspect au maréchal, se reposait sur son bien-dire,
sentait par l'événement du duc du Maine le danger de s'exposer. Il
n'imaginait pas lors qu'une mort si prématurée le porterait au pouvoir
le plus suprême, le plus arbitraire, le plus long, le moins contredit\,;
mais il ne voulait pas nuire à ses vues de grandes places et de grand
crédit, sous M. le duc d'Orléans, par l'affection du roi, et par elle
peu à peu de le faire compter avec lui\,; enfin si l'art et la fortune
le pouvaient porter jusque-là, à chasser M. le duc d'Orléans et à
s'emparer de toutes les affaires. Pour arriver là, il fallait donc deux
choses\,: la première ne se pas faire chasser avant le temps, et se
trouver perdu sans retour avant d'avoir pu commencer à être\,; la
seconde, se conduire de façon à ne pas étranger de lui M. le duc
d'Orléans le moins du monde, pour en pouvoir espérer facilité à ses
desseins d'être\,; devenir en effet sous ses auspices, sans lesquels le
roi quoique majeur ne l'aurait pas mis dans le conseil, encore moins en
influence et en autorité, et pour cela ménager le régent avec un extrême
soin, mais sans rien, non seulement d'affecté, mais encore d'apparent,
et se reposer contre lui sur le maréchal de Villeroy, avec une
approbation la plus tacite qu'il pourrait, en attendant un âge fait du
roi, un progrès plus solide dans sa confiance, une place dans son
conseil, qui lui donnât moyen et caractère de profiter, même de faire
naître des conjonctures, qui lui donnassent ouverture à devenir le
maître et à renvoyer M. le duc d'Orléans à ses plaisirs. Moins plein de
soi et plus clairvoyant que le maréchal de Villeroy, il sentait le goût
intérieur du roi pour M. le duc d'Orléans.

Ce prince n'approchait jamais de lui en public et en quelque particulier
qu'ils fussent, qu'avec le même air de respect qu'il se présentait
devant le feu roi. Jamais la moindre liberté, bien moins de familiarité,
mais avec grâce, sans rien d'imposant par l'âge et la place,
conversation à sa portée, et à lui et devant lui, avec quelque gaieté,
mais très mesurée et qui ne faisait que bannir les rides du sérieux et
doucement apprivoiser l'enfant. Travaillant avec lui, il le faisait
légèrement, pour lui marquer que rien {[}ne se faisait{]} sans lui en
rendre compte, ce qu'il proportionnait et courtement à la portée de
l'âge, et toujours avec l'air du ministre sous le roi. Sur les choses à
donner, gouvernements, places de toutes sortes, bénéfices, pensions, il
les proposait, parcourait brèvement les raisons des demandeurs,
proposait celui qui devait être préféré, ne manquait jamais d'ajouter
qu'il lui disait son avis comme il y était obligé, mais que ce n'était
pas à lui à donner, que le roi était le maître, et qu'il n'avait qu'à
choisir et à décider. Quelquefois même il l'en pressait quand le choix
était peu important\,; et si rarement le roi lui paraissait pencher pour
quelqu'un, car il était trop glorieux et trop timide pour s'en bien
expliquer, et M. le duc d'Orléans y avait toujours grande attention, il
lui disait avec grâce qu'il se doutait de son goût, et tout de suite\,:
«\,Mais n'êtes-vous pas le maître\,? Je ne suis ici que pour vous rendre
compte, vous proposer, recevoir vos ordres et les exécuter.\,» Et à
l'instant la chose était légèrement donnée sans la faire valoir le moins
du monde, et {[}il{]} passait aussitôt à autre chose. Cette conduite en
public et en particulier, surtout cette manière de travailler avec le
roi, charmait le petit monarque\,; il se croyait un homme, il comptait
régner et en sentait tout le gré à celui qui le faisait ainsi régner.

Le régent ni les particuliers n'y couraient pas grand risque\,; le roi
se souciait peu et rarement, et comme il a été remarqué, était trop
glorieux et trop timide pour le montrer souvent, beaucoup moins pour
rien demander. M. le duc d'Orléans était encore fort attentif à bien
traiter tout ce qui environnait le roi de près, avec familiarité, pour
s'en faire un groupe bienveillant, et à chercher à faire des grâces à
ceux pour qui on pouvait croire que le roi avait quelque affection. Cela
servait encore merveilleusement à M. le duc d'Orléans, dans des
occasions de grâces et de places peu importantes, sur lesquelles le roi
aurait montré un goût d'enfant. Comme il était prévenu par l'expérience,
de la façon dont M. le duc d'Orléans en usait toujours là-dessus avec
lui, cela donnait à ce prince la liberté et la facilité de lui
représenter l'importance du poste et les qualités nécessaires pour le
remplir, d'insister, mais en lui disant toujours qu'il était le maître,
qu'il n'avait qu'à prononcer\,; qu'il le suppliait seulement de ne pas
trouver mauvais qu'il lui eût dit ses raisons, parce qu'il était de son
devoir de le faire, et après de lui obéir. Il n'en fallait pas
davantage, le roi se rendait sans chagrin et gaiement\,; mais ces sortes
de cas n'arrivaient presque jamais. Le maréchal de Villeroy était
toujours en tiers à ce travail, par lui ou par le roi\,; il était
difficile que M. de Fréjus ne sût ce qu'il se passait à chaque travail,
de cette conduite du régent, et que le roi qui avait des tête-à-tête
avec son précepteur, que le maréchal de Villeroy qui en enrageait, ne
pouvait empêcher, ne lui témoignât souvent combien il était content de
M. le duc d'Orléans\,; il n'en fallait pas davantage pour le tenir en
bride et laisser au maréchal, qu'il voulait doucement primer et ruiner,
les discours contre le régent, qui ne pouvaient plaire au roi dans la
disposition favorable où M. le duc d'Orléans le tenait continuellement
pour lui.

Ce prince, délivré d'Albéroni, voyait la paix et sa réconciliation
prochaine avec l'Espagne, ce prétexte et les vaines espérances de ce
côté-là ôtées aux brouillons, le duc et la duchesse du Maine hors de
toute mesure d'oser plus branler, leurs adhérents de la cour reconnus
épouvantés et hors d'état et de moyens de plus branler, les autres
atterrés\,; enfin Pontcallet et d'autres nouvellement ou précédemment
arrêtés en Bretagne, prêts à subir un jugement de mort, qui achèverait
de faire rentrer partout chacun en soi-même, et de rétablir la
tranquillité. Il lui restait l'embarras des finances et de
l'administration de Law, et d'achever de vaincre le parlement pour n'y
avoir plus d'entraves, qui tout étourdi qu'il avait été du grand coup
porté sur lui au lit de justice des Tuileries, reprenait peu à peu ses
esprits, et ce caractère si cher, mais si dangereusement usurpé, de
modérateur avec autorité entre le roi et le peuple. Les mêmes seigneurs,
liés secrètement avec M. et M\textsuperscript{me} du Maine, découverts
et déconcertés, et qui l'étaient aussi avec cette compagnie, n'avaient
pas renoncé à chercher de figurer avec elle et par elle. Le maréchal de
Villeroy était comme leur chef, il était tombé dans le dernier
abattement, ainsi que les maréchaux de Villars et d'Huxelles, lorsque M.
et M\textsuperscript{me} du Maine furent arrêtés. Ils y étaient
longtemps demeurés\,; mais la ridicule issue d'un si grand et si juste
éclat, leur avait rendu quelque petit courage, et Villeroy avait repris
tous ses grands airs et ses tons de roi de théâtre, appuyé de sa place
et gâté par les pitoyables ménagements de M. le duc d'Orléans, qui s'en
croyait dédommagé en se moquant de lui en son absence, tandis qu'il en
était dominé en présence avec la plus méprisante hauteur du maréchal,
qui avait l'audace de s'en parer au public, et de s'en faire valoir au
parlement et aux halles où il voulait toujours représenter M. de
Beaufort.

Tout cela pesait à M. le duc d'Orléans\,; il craignait un ralliement
public avec le parlement sur le désordre de Law, qui entraînerait tout
le monde et par l'intérêt particulier et pécuniaire de chacun, et par le
fantôme du bien de l'État qu'ils auraient pour eux, et qui tiendrait M.
le duc d'Orléans en bride. Je crois que Law, qui sentait mieux que
personne l'état où il avait mis les finances et son propre danger, et
{[}mieux{]} que M. le duc d'Orléans même, le lui grossit, et le pressa
de songer à le parer à temps, et qu'il s'y fit aider par M. le Duc et
par ses autres confidents tels que l'abbé Dubois et autres de
l'intérieur. Je dis que je le crois, parce qu'aucun d'eux ne m'en parla,
et que je n'ai pu me persuader que, sans une grande et puissante
impulsion, M. le duc d'Orléans pût prendre la résolution de chasser le
maréchal de Villeroy. C'était dans un temps où l'abbé Dubois, qui était
tout à fait maître, éloignait ce prince de moi, et où je m'éloignais de
lui encore davantage, piqué du retour du duc et de la duchesse du Maine,
et indigné de voir Dubois en pleine possession de son esprit. Ainsi tout
se passait tellement sans moi que je n'eus pas la moindre idée qu'il fût
question de se défaire du maréchal de Villeroy.

Travaillant un jour à mon ordinaire tout à la fin de cette année avec M.
le duc d'Orléans, il m'interrompit un quart d'heure au plus après avoir
commencé, pour me faire ses plaintes du maréchal de Villeroy. Cela lui
arrivait quelquefois\,; mais de là s'échauffant en discours de plus
forts en plus forts, il se leva tout d'un coup, et me dit que cela
n'était plus tenable, car ce fut son expression\,; qu'il voulait et
allait le chasser, et tout de suite, que je fusse gouverneur du roi. Ma
surprise fut extrême, mais je ne perdis pas le jugement. Je me mis à
sourire et répondis doucement qu'il n'y pensait pas. «\, Comment,
reprit-il, j'y pense très bien, et si bien que je veux que cela soit, et
ne pas différer ce qui devrait être fait il y a longtemps. Qu'est-ce
donc que vous trouvez à cela\,?» Se mit à se promener ou plutôt à
toupiller dans ce petit cabinet d'hiver. Alors je lui demandai s'il y
avait bien mûrement pensé. Là-dessus il m'étala toutes ses raisons pour
ôter le maréchal et toutes celles de me mettre en sa place, trop
flatteuses pour les rapporter ici. Je le laissai dire tant qu'il voulut,
puis je parlai à mon tour sans vouloir être interrompu. Je convins de
tout sur le maréchal de Villeroy, parce qu'en effet il n'y avait pas
moyen de disconvenir d'aucune de ses plaintes, de ses raisons et de ses
conséquences\,; mais je m'opposai fortement à l'ôter. Je fis d'abord
souvenir M. le duc d'Orléans de toutes les raisons que je lui avais
alléguées pour le détourner d'ôter à M. du Maine la surintendance de
l'éducation du roi, combien lui-même les avait trouvées sages et bonnes,
combien il en était demeuré persuadé, et qu'il n'avait cédé qu'à la
force et à la constante persécution de M. le Duc. Je lui distinguai bien
les raisons communes avec ce qui regardait M. le Duc d'une part, le
parlement de l'autre, d'avec celles qui ne regardaient que le duc du
Maine et lui-même, le danger d'intervertir la disposition du feu roi à
l'égard d'une personne aussi chère et précieuse que celle de son
successeur. De là, j'entrai en comparaison des personnages\,; je lui fis
sentir la différence d'ôter un homme quelque grand et établi qu'il fût,
mais haï, mais envié, mais abhorré des princes du sang et du gros du
monde, mais toutefois très dangereux à conserver par son esprit, ses
vues, sa cabale, d'avec un autre homme mis pareillement de la main du
roi mort entre ses bras, sans esprit ni mérite, peu dangereux par
conséquent, adoré du peuple et du gros du monde, orné du masque
d'honnête homme et {[}tenu{]} pour incapable de pouvoir et de vouloir
remuer et faire un parti dans l'État, chéri du parlement et de toute la
magistrature par les soins qu'il en avait pris de longue main, toutes
choses, excepté le point du parlement, diamétralement contraires entre
le maréchal de Villeroy et le duc du Maine. Je m'étendis là-dessus, et
je répondis à toutes ses répliques.

Je lui dis que le maréchal de Villeroy n'était à son égard que ce qu'il
le faisait être, et ce que tout autre serait avec autant de vent et de
fatuité, et aussi peu d'esprit et de sens\,; qu'il l'avait gâté et le
gâtait sans cesse, dont le maréchal savait se prévaloir\,; qu'on ne
s'accoutumait ni en public ni en particulier à voir combien il lui
imposait, l'air de supériorité du maréchal avec lui comme s'il eût été
encore au temps de Monsieur, et lui en celui de sa première jeunesse\,;
que pour lui, pour les siens, pour Lyon, pour tous ceux pour qui le
maréchal daignait non pas demander, mais témoigner quelque petit désir,
{[}tout{]} était accordé sur-le-champ et sans mesure, et que résolu de
lui cacher tout, il lui disait une infinité de choses, et l'admettait
continuellement dans le secret de la poste\,; qu'avec cette conduite que
l'affaire du duc du Maine n'avait que légèrement altérée et encore pour
fort peu de temps, il ne devait pas être surpris des avantages que le
maréchal en savait prendre\,; qu'il n'y avait qu'à changer une conduite
aussi étrange et aussi dangereuse, et tenir ferme dans ce changement,
sans se donner la peine d'aller plus loin\,; qu'il verrait tout aussitôt
le maréchal de Villeroy se croire perdu, tremblant, petit et
respectueux, souple, tel enfin qu'il s'était montré à la disgrâce, et
bien plus encore à l'éclat de l'affaire du duc et de la duchesse du
Maine\,; que la durée de ce changement achèverait de le déconcerter, de
le renverser, de le décréditer en lui ôtant l'opinion du monde que le
maréchal lui imposait, et que lui n'osait lui résister\,; que déchu de
la sorte et toujours tremblant pour son sort, il ne pourrait jamais lui
nuire\,; que dépouillé de {[}ce{]} qui le rehaussait, non de sa place,
il y paraîtrait tel qu'il était, par conséquent méprisable et méprisé\,;
que c'était dans cette réduction qui était entre ses mains qu'il fallait
mettre et tenir toujours le maréchal, qui, en cette posture, lui serait
bien meilleur demeurant dans sa place, que destitué, parce qu'il y
serait nu et seul, au lieu que destitué il aurait pour lui l'aboiement
de tout le monde, l'air et l'honneur de martyr du bien public, celui
dont la présence était incompatible avec les derniers excès de Law et la
ruine universelle\,; qu'en laissant le maréchal de Villeroy sans y
toucher, mais en le traitant constamment comme je venais de le proposer,
il l'anéantissait\,; que, le chassant, il en faisait un personnage, une
idole du parlement, du peuple, des provinces, un point de ralliement
sinon dangereux, du moins embarrassant, d'autant plus qu'il avait laissé
passer le moment de l'envelopper avec le duc et la duchesse du Maine\,;
qu'il ne se pouvait donc plus agir ici du bien et de la tranquillité de
l'État ni d'intelligences étrangères et criminelles, comme à l'égard du
duc et de la duchesse du Maine, et du parti qu'ils avaient formé, mais
uniquement de l'intérêt et des soupçons de lui régent, et d'un sacrifice
qu'il se ferait à lui-même du seigneur le plus marqué du royaume, chargé
de toute la confiance du feu roi jusqu'à sa mort, mis uniquement par là
auprès du roi son successeur, de sa main, dont Son Altesse Royale
intervertirait pour la seconde fois les dernières, les plus intimes et
les plus sacrées dispositions.

Ébranlé, mais non dépris encore de sa résolution, il essaya de
m'affaiblir en redoublant la tentation de la place de gouverneur du roi,
et me comblant sur tout ce qu'il me prodigua là-dessus. Je lui témoignai
ma reconnaissance en homme qui sentait très bien le prix de la place et
celui de l'assaisonnement qu'il y mettait, mais qui n'en était pas
ébloui. Tout de suite je le suppliai de se rappeler de ce qui s'était
passé entre lui et moi dès avant qu'on sût que le roi écrivait tant de
sa main, et qu'on en soupçonnât une disposition testamentaire\,; qu'il
se souvînt que je lui avais dit qu'il était à présumer, même à désirer
pour Son Altesse Royale, que le roi disposât des places de l'éducation
du roi son successeur\,; mais que si, contre toute apparence, il vînt à
manquer sans l'avoir fait, jamais lui régent, lui successeur immédiat
par le droit des renonciations, si le jeune monarque mourait sans
postérité masculine, jamais lui, si cruellement, si iniquement, mais si
universellement accusé de toutes les horreurs alors récentes, et dont le
souvenir se renouvelait depuis de temps en temps avec tant d'art et
d'audace, ne devait jamais nommer un gouverneur ni aux autres places de
l'éducation et du service intime, personne qui lui fût particulièrement
attaché\,; que plus un homme le serait ou anciennement ou intimement,
encore pis l'un et l'autre, plus il en devait être exclus, quand il
aurait d'ailleurs pour ce grand emploi un talent unique, et tous les
autres qui s'y pouvaient souhaiter\,; qu'il était entré dans mon
sentiment, et qu'il était convenu avec moi de le suivre\,; que je le
sommais donc maintenant de s'en souvenir et de ne pas s'écarter d'une
résolution qui lui avait paru alors si salutaire, et qui par tout ce qui
s'était passé depuis, surtout par l'expulsion du duc du Maine, l'était
devenue de plus en plus. Enfin que ce raisonnement si vrai et si fort,
résultant de la perverse nature des choses, me rendait par excellence
l'homme de toute la France sur qui le choix devait le moins tomber, et
qui en était le plus radicalement exclus par nature\,; qu'aussi
croirais-je lui rendre le plus mauvais et le plus dangereux office de
l'accepter.

M. le duc d'Orléans qui était l'homme que j'aie connu qui avait les
réponses les plus prêtes à la main, et qui s'embarrassait le moins, même
n'ayant rien qui valût à répondre, fut si surpris ou de la force de mes
raisons, ou de la fermeté de mon refus, qu'il resta court et pensif, se
promenant la tête basse sept ou huit pas en avant et autant en arrière,
parce que ce cabinet était fort petit. Je demeurai debout sans le suivre
et sans parler, pour laisser opérer ses réflexions que je ne voulais pas
troubler par des redites inutiles, puisqu'en effet j'avais tout dit
l'essentiel. Ce silence dura assez longtemps\,: puis il me dit qu'il y
avait bien du bon dans ce que je lui avais exposé, mais que le maréchal
de Villeroy était tellement devenu insupportable, et que j'étais si fait
exprès pour l'emploi en tout sens, sur quoi il s'étendit encore, qu'il
avait bien de la peine à changer d'avis. Les mêmes choses se rebattirent
assez longtemps encore\,; les propos finirent par me dire que nous nous
reverrions là-dessus. Je lui répondis que, pour ce qui me regardait,
cela était tout vu de ma part, et que très certainement je ne serais
point gouverneur du roi\,; qu'à l'égard du maréchal, il prît bien garde
aux impulsions d'autrui, et à la sienne propre à lui-même, et qu'il se
gardât bien de faire un si grand pas de clerc. Nous n'en dîmes pas
davantage. Il m'en reparla près à près deux ou trois autres fois, mais
toujours plus faiblement, moi toujours de même, et gagnant toujours du
terrain sur lui, jusqu'à ce que, la dernière fois, il convint avec moi
qu'il n'y songerait plus, et qu'il en userait avec le maréchal de
Villeroy comme je le lui avais proposé\,; mais il n'en eut pas la force.
Il le traita toujours de même, et le maréchal, par conséquent, toujours
sur le haut ton avec lui. J'en étais dépité, mais je n'osai lui en faire
de reproches, de peur de ranimer l'envie de le chasser. D'ailleurs tout
allait tellement de travers, l'abbé Dubois si fort et si publiquement le
maître absolu, que cela joint à la déplorable issue de l'affaire de M.
et de M\textsuperscript{me} du Maine, mon dégoût allait à ne vouloir
plus me mêler de rien, et à voir M. le duc d'Orléans courtement et
précisément pour le nécessaire, et pour, ne rien marquer au monde si
attentif à tout. Ainsi finit l'année 1719.

\hypertarget{chapitre-xviii.}{%
\chapter{CHAPITRE XVIII.}\label{chapitre-xviii.}}

1720

~

{\textsc{1720.}} {\textsc{- Comédie entre le duc et la duchesse du
Maine, qui ne trompe personne.}} {\textsc{- Changement de dame d'honneur
de M\textsuperscript{me} la Duchesse la jeune\,; pourquoi raconté.}}
{\textsc{- Caractère de M. et de M\textsuperscript{me} de Pons.}}
{\textsc{- Abbé d'Entragues\,; son extraction\,; son singulier
caractère\,; ses aventures.}} {\textsc{- Law, contrôleur général des
finances.}} {\textsc{- Grâces singulières faites aux enfants
d'Argenson.}} {\textsc{- Machaut et Angervilliers conseillers d'État en
expectative.}} {\textsc{- Law maltraité par l'avidité du prince de
Conti, qui en est fortement réprimandé par M. le duc d'Orléans.}}
{\textsc{- Ballet du roi.}} {\textsc{- Force grâces pécuniaires.}}
{\textsc{- J'obtiens douze mille livres d'augmentation d'appointements
sur mon gouvernement de Senlis, qui n'en valait que trois mille.}}
{\textsc{- Je fais les derniers efforts pour un conseil étroit, fort
inutilement.}} {\textsc{- Mariage de Soyecourt avec
M\textsuperscript{lle} de Feuquières.}} {\textsc{- Réflexions sur les
mariages des filles de qualité avec des vilains.}} {\textsc{- Mort du
comte de Vienne\,; son caractère, son extraction.}} {\textsc{- Mort du
prince de Murbach.}} {\textsc{- Mort de l'impératrice mère, veuve de
l'empereur Léopold.}} {\textsc{- Son deuil et son caractère.}}
{\textsc{- Mort du cardinal de La Trémoille.}} {\textsc{- Étrange
friponnerie et bien effrontée de l'abbé d'Auvergne pour lui escroquer
son archevêché de Cambrai.}} {\textsc{- Digression sur les alliances
étrangères du maréchal de Bouillon et de sa postérité.}} {\textsc{- Abbé
d'Auvergne\,; comment fait archevêque de Tours, puis de Vienne.}}

~

Cette année commença par une comédie fort ridicule dont personne ne fut
la dupe, ni le public, ni ceux pour qui elle fut principalement jouée,
ni ceux qui la jouèrent, si ce n'est peut-être la seule
M\textsuperscript{me} la Princesse qui y fit un personnage principal, et
qui était faite pour l'être de tout. Le duc et la duchesse du Maine, qui
par la perfidie de l'abbé Dubois avaient eu, comme on l'a vu ici, tout
le temps nécessaire, et beaucoup au delà pour sauver leurs papiers, et
pour s'arranger ensemble depuis que Cellamare fut arrêté chez lui
jusqu'au jour qu'ils le furent eux-mêmes, avaient très bien pris leur
parti, et chacun d'eux suivant leur caractère. M\textsuperscript{me} du
Maine appuyée de son sexe et de sa naissance, s'affubla de tout dans ses
réponses aux interrogatoires qu'elle subit, et dont on lut ce qu'il plut
à l'abbé Dubois au conseil de régence, accusa fortement Cellamare,
Laval, etc., sauva tant qu'elle put les Malézieu, Davisard, et ses
intimes créatures, son mari surtout, pour qui elle se fit fort, et
stipula tout sans, disait-elle, lui en avoir donné connaissance,
c'est-à-dire, sans lui avoir jamais laissé entrevoir ni intelligence en
Espagne, ni parti, ni rien qui pût aller à brouiller l'État, ni à
attaquer le régent, mais seulement à lui procurer des remontrances assez
fortes et assez nombreuses pour l'engager doucement à réformer lui-même
beaucoup de choses dont on se plaignait de son administration. Quoi
qu'elle avouât, elle ne craignait rien pour sa tête ni même pour une
prison dure et longue. Les exemples des princes de Condé la rassuraient
dans toutes les générations, qui s'étaient trouvés en termes encore plus
forts.

Le duc du Maine, déchu de l'état et de la qualité de prince du sang,
tremblait pour sa vie. Ses crimes contre l'État, contre le sang royal,
contre la personne du régent, si longuement, si artificieusement, si
cruellement offensée, le troublaient d'autant plus qu'il sentait tout ce
que raison, justice, exemple, devoir à l'égard de l'état et du sang
royal, vengeance enfin exigeaient de lui. Il songea donc de bonne heure
à se mettre à couvert sous la jupe de sa femme. Ses réponses et tous ses
propos furent constamment les mêmes d'une parfaite ignorance et dans le
plus grand concert entre eux deux. Il n'avait vu en effet que ses
domestiques les plus affidés, Cellamare presque point, et dans le
dernier secret, dans le cabinet de M\textsuperscript{me} du Maine,
inaccessible à tous autres de leur confidence, à qui il ne parlait que
par la duchesse du Maine\,: ainsi, ni papiers ni dépositions à craindre.
Ainsi, quand elle eut parlé, avoué, raconté, Laval aussi de rage de ce
qu'elle avait dit, et peu d'autres\,; le duc du Maine, à qui cela fut
communiqué à Dourlens, s'exclama contre sa femme, dit rage de sa folie
et de sa félonie, du malheur d'avoir une femme capable de conspirer, et
assez hardie pour le mettre de tout sans lui en avoir jamais parlé, le
faire criminel sans qu'il le fût le moins du monde, et si fort hors de
tout soupçon des menées de sa femme, qu'il était resté hors d'état de
les arrêter, de lui imposer, d'avertir même M. le duc d'Orléans s'il eût
trouvé les choses poussées au point de le devoir faire. Dès lors le duc
du Maine ne voulut plus ouïr parler d'une femme qui à son insu avait
jeté lui et ses enfants dans cet abîme, et quand, à leur sortie de
prison, il leur fut permis de s'écrire et de s'envoyer visiter, il ne
voulut rien recevoir de sa part, ni lui donner aucun signe de vie.
M\textsuperscript{me} du Maine s'affligeait en apparence du traitement
qu'elle en recevait, en avouant toutefois combien elle était coupable
envers lui de l'avoir engagé à son insu et trompé de la sorte. Ils en
étaient là ensemble quand on les rapprocha de Paris. Le duc du Maine
alla demeurer à Clagny, château bâti autrefois tout près de Versailles
pour M\textsuperscript{me} de Montespan, et M\textsuperscript{me} du
Maine à Sceaux. Ils virent ensuite M. le duc d'Orléans séparément sans
coucher à Paris, où ils soutinrent chacun leur personnage, et comme
l'abbé Dubois avait jugé que le temps était venu de se donner auprès
d'eux le mérite de finir leur disgrâce, tout fut bon auprès de M. le duc
d'Orléans qui voulut bien leur paraître persuadé de l'ignorance du duc
du Maine. Pendant leur séjour en ces deux maisons de campagne où ils ne
virent que fort peu de gens, M\textsuperscript{me} du Maine se donna
pour faire diverses tentatives auprès du duc du Maine, et lui pour les
rebuter. Cette farce dura depuis le mois de janvier qu'ils arrivèrent à
Sceaux et à Clagny, jusque tout à la fin de juillet. Alors ils crurent
que le jeu avait assez duré pour y mettre une fin. Ils s'en étaient
trouvés quittes à si bon marché, et comptaient tellement sur l'abbé
Dubois, qu'ils pensaient déjà à se remonter en grande partie, et, pour y
travailler utilement, il fallait être en mesure de se voir et de se
concerter et commencer par pouvoir être à Paris comme ils voudraient, où
ils ne pouvaient pas ne pas loger ensemble.

L'apparente brouillerie avait été portée jusqu'à ce point, que les deux
fils du duc du Maine, revenus d'Eu à Clagny peu de jours après lui,
furent longtemps sans aller voir M\textsuperscript{me} du Maine, et ne
la virent depuis que très rarement et sans coucher à Sceaux. Enfin, le
parti pris de mettre fin à cette comédie, voici comme ils la terminèrent
par une autre. M\textsuperscript{me} la Princesse prit un rendez-vous
avec le duc du Maine, le dernier juillet, à Vaugirard, dans la maison de
Landais, trésorier de l'artillerie\,; elle y arriva un peu après lui
avec la duchesse du Maine qu'elle laissa dans son carrosse. Elle dit à
M. du Maine qu'elle avait amené une dame qui avait grande envie de le
voir. La chose n'était pas difficile à entendre, le concert était pris.
Ils mandèrent la duchesse du Maine. L'apparent raccommodement se passa
entre eux trois. Ils furent longtemps ensemble. Un reste de comédie les
tint encore séparés, mais se voyant et se rapprochant par degrés jusqu'à
ce qu'à la fin le duc du Maine retourna demeurer à Sceaux avec elle.

Pendant ces six mois, on acheva peu à peu de vider la Bastille des
prisonniers de cette affaire, dont quelques-uns furent légèrement et
courtement exilés. Laval fut plus maltraité, ou pour mieux dire le moins
bien traité. Il avait été l'âme au dehors de toute la conspiration et
dans tout le secret du duc et de la duchesse du Maine qui en dit assez
dans ses interrogatoires, c'est-à-dire dans le peu de ceux qui furent
lus au conseil de régence, et sur lesquels l'avis ne fut demandé à
personne et où personne aussi n'opina, pour prouver complètement cela
contre lui. Aussi sortit-il de la Bastille enragé contre elle, et ne le
lui a pas pardonné, dont elle se soucia aussi peu que font tous les
princes et princesses, quand ils n'ont plus besoin des gens, parce
qu'ils se persuadent que tout est fait pour eux, et eux uniquement pour
eux-mêmes. Le courant de la vie dans tous les temps, et les
conspirations de tous les siècles en sont la preuve et la leçon.

On ne s'aviserait pas de faire ici mention du changement des domestiques
de l'hôtel de Condé, si elle ne servait à montrer l'étrange contraste de
la conduite des gens de qualité la plus distinguée, ainsi que de celle
de ceux qui en sont les singes\,: conduite si nouvelle, et en contraste
si grand et si public avec elle-même. On a vu en son lieu à quel point
le duc et la duchesse du Maine les avaient enivrés, et jusqu'à quelles
folies ils les avaient jetés en se moquant d'eux pour arriver à leur but
personnel, avec toute cette gloire dont M. et M\textsuperscript{me} du
Maine avaient fait leur instrument pour les tromper et les conduire en
aveugles. La femme de l'aîné de la maison de Montmorency, de laquelle M.
le Prince, père du héros, était gendre, et dont les dépouilles ont
constitué ses grands biens, était dame d'honneur de
M\textsuperscript{me} la Duchesse la jeune, et y eut tant de dégoûts
qu'elle se retira. Il est vrai que son mari était pauvre en tout genre,
et elle, avec beaucoup de mérite, de très petite étoffe.
M\textsuperscript{me} de Pons lui succéda avec empressement\,; son mari
était l'aîné de cette grande et illustre maison de Pons, mais si pauvre
que M. de La Rochefoucauld, le favori de Louis XIV, prit soin de lui
jusqu'à son logement, son vêtement et sa nourriture. Il avait de la
grâce, une éloquence naturelle, beaucoup d'esprit et fort orné\,;
beaucoup de politesse, mais à travers laquelle transpirait même
grossièrement une extrême gloire et une opinion de soi-même rebutante.
Il eut du roi une charge dans la gendarmerie où il servit comme point,
et ne vit guère plus de cour que de guerre. Il avait un des plus beaux
visages qu'on pût voir. Ce visage, soutenu de son esprit, donna dans les
yeux de M\textsuperscript{me} de La Baume qui l'épousa. Elle était fille
unique de M. de Verdun et riche héritière, parce qu'elle était restée
seule des enfants de son père, qui n'avait point paru à la guerre ni à
la cour, qui était riche, et qui avait beaucoup amassé. Lui et le
maréchal de Tallard étaient fils des deux frères, Verdun de l'aîné, et
avait de grandes prétentions contre Tallard, ce qui les engagea à marier
leurs enfants.

Le mariage ne dura guère. La Baume, fils aîné du maréchal, et qui
promettait beaucoup, mourut sans enfants des blessures qu'il reçut à la
bataille d'Hochstedt, perdue par son père comme on l'a vu en son lieu,
n'ayant été marié que six mois. Sa veuve se remaria en 1710 à M. de
Pons, à qui elle porta de grands biens et force procès et prétentions,
dont ils tourmentèrent tant le maréchal de Tallard, qu'ils en tirèrent à
peu près ce qu'ils voulurent. La femme était aussi dépiteusement laide
que le mari était beau, et aussi riche qu'il était pauvre\,; d'ailleurs
autant de gloire, d'esprit, de débit et d'avarice l'un que l'autre.
Cette avarice et leur procès l'emporta sur leur gloire\,; ils briguèrent
la place que M\textsuperscript{me} de Montmorency-Fosseux quittait, et
l'obtinrent\,: leurs affaires liquidées, M\textsuperscript{me} de Pons
s'en lassa et s'en retira. Elle était très méchante, très difficile à
vivre, maîtresse absolue de son mari, dont l'humeur était pourtant
dominante, et qui régnait tant qu'il pouvait sur tous ceux qu'il
fréquentait. Cette humeur peu compatible avec celle de MM. de La
Rochefoucauld, moins encore avec tous les secours qu'il en avait reçus,
rendit le commerce rare et froid entre eux, dès qu'il n'en eut plus
besoin. Le chevalier de Dampierre, écuyer de M. le Duc, qui était
Cugnac, bonne noblesse, qui a eu un chevalier du Saint-Esprit en 1595,
et lieutenant général d'Orléanais sous Henri IV, présenta la femme de
son frère. Cet écuyer imposait aisément à son maître par l'énormité de
sa prestance, beaucoup d'esprit et fort avantageux, quoique soutenu
d'aucune qualité personnelle, glorieux à l'excès, et qui avait persuadé
M. le Duc qu'il était, comme on dit, de la côte de saint Louis.
Moyennant ce caquet sa belle-soeur eut la place\,; ils en avaient grand
besoin, car ils n'avaient pas de chausses\,; et voilà comme l'excès de
l'orgueil et de la bassesse s'accommodent presque toujours.

La singularité du personnage et d'un événement arrivé en ce même temps,
mérite de n'être pas oubliée. L'abbé d'Entragues était un homme qui
avait été extrêmement du grand monde\,; il n'était rien moins que
Balzac\,; je ne sais d'où ce nom d'Entragues leur était venu, car les
Balzac sont fondus dans les Illiers. Le nom de celui-ci était Crémeaux,
gentilhomme, tout ordinaire, du côté de Lyon\,; ce qui les mit au monde
fut le mariage de son frère avec la soeur utérine de Mine de La
Vallière, maîtresse du roi, du nom de Courtalvel, de la plus petite
noblesse. Le père de cette soeur s'appelait Saint-Remy, premier maître
d'hôtel de Gaston frère de Louis XIII. Il épousa la veuve de La
Vallière, qui s'appelait Le Prévost, et qui n'était rien, veuve en
premières noces de Bernard-Rezay, conseiller au parlement, dont elle
n'avait point eu d'enfants. De La Vallière elle eut la maîtresse du roi,
et le grand-père du duc de La Vallière d'aujourd'hui\,; de son dernier
mari, cette M\textsuperscript{me} d'Entragues, belle-soeur de l'abbé
dont il s'agit.

La différence d'une mère avouée que n'avaient pas les enfants de
M\textsuperscript{me} de Montespan, et l'attachement dont
M\textsuperscript{me} la princesse de Conti se piqua toujours pour sa
mère et pour tous ses parents, les distingua. Ce fut donc la protection
de M\textsuperscript{me} d'Entragues, propre tante de
M\textsuperscript{me} la princesse de Conti qui introduisit chez elle
l'abbé d'Entragues. Elle aima toujours beaucoup M\textsuperscript{me}
d'Entragues, qui était aussi fort aimable par son esprit fait pour le
grand monde dont elle fut toujours. De là, l'abbé d'Entragues se mit
dans les bonnes compagnies dont il avait le ton et le langage, avec une
plaisante singularité, qui le rendait encore plus amusant, qui était son
vrai caractère\,; mais ce caractère n'était pas sûr\,; il était méchant,
se plaisait aux tracasseries et à brouiller les gens, ce qui le fit
chasser de beaucoup de maisons considérables\,; il eut abbayes et
prieurés, mais jamais d'ordres. C'était un grand homme, très bien fait,
d'une pilleur singulière, qu'il entretenait exprès à force de saignées,
qu'il appelait sa friandise\,; dormait les bras attachés en haut pour
avoir de plus belles mains\,; et, quoique vêtu en abbé, il était mis si
singulièrement qu'il se faisait regarder avec surprise. Ses débauches le
firent exiler plus d'une fois. L'étant à Caen, il y vint des Grands
Jours\footnote{C'est-à-dire des commissaires chargés par le roi de tenir
  des assises extraordinaires pour punir les crimes que n'avait pu
  atteindre la justice ordinaire.}, parmi lesquels était Pelletier de
Sousy, qui a eu depuis les fortifications, père de des Forts, qui a été
ministre et contrôleur général des finances. Pelletier, qui avait connu
l'abbé d'Entragues quoique assez médiocrement, crut qu'arrivant au lieu
de son exil, il était honnête de l'aller voir. Il y fut donc sur le
midi\,; il trouva une chambre fort propre, un lit de même, ouvert de
tous côtés, une personne dedans à son séant, galamment mise, qui
travaillait en tapisserie, coiffée en coiffure de nuit de femme, avec
une cornette à dentelle, force fontanges\footnote{Les fontanges étaient
  des noeuds de rubans qu'aux XVIIe et XVIIIe siècles les femmes
  portaient sur le devant de leur coiffure et un peu au-dessus du front.
  On rapporte que M\textsuperscript{lle} de Fontanges s'apercevant à la
  promenade que sa coiffure manquait de solidité, prit une de ses
  jarretières et la noua autour de sa tête. On trouva ce noeud charmant,
  et ce que le hasard avait produit, devint sur-le-champ une mode qui a
  duré jusqu'à la seconde moitié du XVIIIe siècle.}, de la parure, une
échelle de rubans à son corset, un manteau de lit volant et des mouches.
À cet aspect Pelletier recula, se crut chez une femme de peu de vertu,
fit des excuses, et voulut gagner la porte, dont il n'était pas éloigné.
Cette personne l'appela, le pria de s'approcher, se nomma, se mit à
rire\,: c'était l'abbé d'Entragues, qui se couchait très ordinairement
dans cet accoutrement, mais toujours en cornettes de femme plus ou moins
ajustées. Il y aurait tant d'autres contes à faire de lui qu'on ne
finirait pas. Avec cela beaucoup de fonds d'esprit et de conversation,
beaucoup de lecture et de mémoire, du savoir même, de l'élégance
naturelle et de la pureté de langage\,; fort sobre, excepté de fruit et
d'eau.

Dans le temps dont il s'agit, il passait sa vie chez
M\textsuperscript{me} la princesse de Conti, chez Beringhen, premier
écuyer, et dans plusieurs maisons considérables qui lui étaient restées.
On sut, sans que rien eût pu en faire douter, qu'il avait été faire la
cène un dimanche au prêche chez l'ambassadeur de Hollande\,; il s'en
vanta même, et dit qu'il avait eu enfin le bonheur de faire la cène avec
ses frères. On en fut d'autant plus surpris qu'il était de race
catholique, et qu'aucune religion n'avait jusqu'alors paru l'occuper ni
le retenir. L'éclat de cette folie, et le bruit qu'en fit le clergé, ne
permit pas à M. le duc d'Orléans de se contenter d'en rire comme il eût
bien voulu. Il donna donc ordre, au bout de trois ou quatre jours, de
l'arrêter et de le mener à la Bastille\,; mais dans l'intervalle, il
avait pris le large et gagné Anchin pour sortir du royaume\,; de là à
Tournay, rien de plus court ni de plus aisé. La fantaisie le prit
d'aller à Lille et de se nommer chez le commandant. On avait averti aux
frontières, et celle-là, comme la plus proche, l'était déjà. Le
commandant s'assura de lui et en rendit compte à M. le duc d'Orléans,
qui le fit mettre dans la citadelle. L'abbé d'Entragues s'en lassa, et
fit là son abjuration, après laquelle il revint enfin à Paris sans qu'il
en fût autre chose, ni à son égard, ni à celui de ses bénéfices. Comme
on ne pouvait rien imaginer de sérieux d'un homme si frivole, il fut
reçu chez M\textsuperscript{me} la Duchesse, chez M\textsuperscript{me}
la princesse de Conti, chez M\textsuperscript{me} du Maine, et dans
toutes les maisons qu'il avait accoutumé de fréquenter, et où il était
très familier, et reçu comme s'il ne lui était rien arrivé. Il affecta
quelque temps de se montrer à la messe avec un grand bréviaire, puis
revint peu à peu à sa vie et à sa conduite ordinaire. Il ne laissait
pas, avec toute la dépravation de ses moeurs et un jeu qui l'avait
souvent dérangé, de donner toute sa vie considérablement aux pauvres, et
avec tous les fruits et la glace qu'il avalait de passer quatre-vingts
ans sans infirmité. Il soutint avec beaucoup de courage et de piété la
longue maladie dont il mourut, et il finit fort chrétiennement une vie
fort peu chrétienne.

Le désordre des finances augmentait chaque jour, ainsi que les démêlés
d'Argenson et de Law, qui s'en prenaient l'un à l'autre. Celui-ci avait
l'abord gracieux\,; il tenait par son papier un robinet de finance qu'il
laissait couler à propos sur qui le pouvait soutenir. M. le Duc,
M\textsuperscript{me} la Duchesse, Lassai, M\textsuperscript{me} de
Verue y avaient puisé force millions et en tiraient encore. L'abbé
Dubois y en prenait à discrétion. C'étaient de grands appuis, outre le
goût de M. le duc d'Orléans qui ne s'en pouvait déprendre. Les audiences
du garde des sceaux, plus de nuit que de jour, désespéraient ceux qui
travaillaient sous lui et ceux qui y avaient affaire. La difficulté des
finances et ses luttes contre Law lui avaient donné de l'humeur qui se
répandait dans ses refus. Les choses en étaient venues au point qu'il
fallait que l'un des deux cédât à l'autre une administration où leur
concurrence achevait de mettre la confusion. Quelque liaison, même
intime, qui subsistât entre lui et l'abbé Dubois qui avait échoué à les
faire compatir ensemble, la vue du cardinalat et la nécessité de
beaucoup d'argent à y répandre ne permit pas à Dubois de balancer dans
cette extrémité qui ne pouvait plus se soutenir. La conversion de Law
avait un but auquel il était temps qu'il arrivât. Il était pénétré de la
bonté de son système, et il s'en promettait des merveilles de la
meilleure foi du monde, sitôt qu'il ne serait plus traversé.

Argenson voyait l'orage s'approcher\,; il se sentait dans une place non
moins fragile que relevée\,; il voulait la sauver. Il avait trop
d'esprit et trop de connaissance du monde, et de ceux à qui il avait
affaire, pour ne pas sentir que, s'opiniâtrant aux finances, elles
entraîneraient les sceaux. Il céda donc à Law, qui fut enfin déclaré
contrôleur général des finances, et qui, dans cette élévation si
singulière pour lui, continua à venir chez moi tous les mardis matin, me
voulant toujours persuader ses miracles passés et ceux qu'il allait
faire. Argenson demeura garde des sceaux, et se servit habilement du
sacrifice des finances pour faire passer sur la tête de son fils aîné sa
charge de chancelier de l'ordre de Saint-Louis, et le titre effectif sur
son cadet. Sa place de conseiller d'État qu'il avait conservée, il la
fit donner à son aîné avec l'intendance de Maubeuge, et fit son cadet
lieutenant de police. Le murmure fut grand de voir un étranger
contrôleur général, et tout livré en France à un système dont on
commençait beaucoup à se défier. Mais les Français s'accoutument à tout
et la plupart se consolèrent de n'avoir plus affaire aux heures bizarres
et à l'humeur aigre d'Argenson. M. le duc d'Orléans me dit bien d'avance
ce qu'il allait faire, mais sans consultation. L'abbé Dubois avait tout
envahi, et j'évitais au lieu de m'avancer à rien. On verra bientôt quel
fut le succès de ce choix. Les enfants d'Argenson furent les seuls qui
en profitèrent. On n'avait jamais ouï parler d'un conseiller d'État et
intendant de Hainaut de vingt-quatre ans, ni d'un lieutenant de police
encore plus jeune. On changea en même temps la face et les départements
du conseil des finances, dont le duc de La Force déjà entré dans celui
de régence, ne fut plus. On donna une expectative de conseiller d'État à
Machaut, qui quitta volontiers la place de lieutenant de police pour
celle-ci, et pour les cinquante mille écus qu'il avait donnés au garde
des sceaux, qu'il lui rendit. Angervilliers, intendant d'Alsace, puis de
Paris, eut en même temps une pareille expectative. On en fait ici
mention à cause qu'on le vit depuis ministre et secrétaire d'État ayant
le département de la guerre, et que sa capacité le distingua extrêmement
dans tous ses emplois ainsi que sa probité.

La place de contrôleur général que Law occupait si nouvellement ne le
mit pas à l'abri du pistolet sur la gorge, pour ainsi dire, de M. le
prince de Conti. Plus avide que pas un des siens, et que n'est-ce point
dire\,? il avait tiré des monts d'or de la facilité de M. le duc
d'Orléans, et d'autres encore de Law en particulier. Non content encore,
il voulut continuer. M. le duc d'Orléans s'en lassa, il n'était pas
content de lui. Le parlement recommençait sourdement ses menées\,: elles
commençaient même à se montrer, et le prince de Conti s'intriguait à
tâcher d'y faire un personnage indécent à sa naissance, peu convenable à
son âge, honteux après les monstrueuses grâces dont il était sans cesse
comblé. Rebuté par le régent, il espéra mieux de Law\,; il fut trompé en
son attente\,; les prières, les souplesses, les bassesses, car rien ne
lui coûtait pour de l'argent, n'ayant rien opéré, il essaya la vive
force, et n'épargna à Law ni les injures ni les menaces. En effet, il
lui, fit une telle peur\,: le prince de Conti, ne pouvant lui pis faire
pour renverser sa banque, y fut avec trois fourgons qu'il ramena pleins
d'argent pour le papier qu'il avait, que Law n'osa refuser à ses
emportements, et manifester par ce refus la sécheresse de ses fonds
effectifs. Mais craignant d'accoutumer à ces hauteurs et à cette
tyrannie un prince aussi insatiable, il ne le vit pas plutôt parti avec
son convoi, qu'il en fut porter ses plaintes à M. le duc d'Orléans. Le
régent en fut piqué\,; il sentit les dangereuses suites et le pernicieux
exemple d'un procédé si violent à l'égard d'un étranger sans appui qu'il
venait de faire contrôleur général bien légèrement. Il se mit en colère,
envoya chercher le prince de Conti, et contre son naturel lui lava si
bien la tête qu'il n'osa branler, et eut recours aux pardons\,; mais
outré d'avoir échoué, peut-être plus encore que de la plus que très
verte réprimande, il eut recours au soulagement des femmes. Il se
répandit en propos contre Law, qui ne lui firent plus de peur et moins
de mal encore, mais qui firent peu d'honneur à M. le prince de Conti,
parce que la cause en était connue, et qu'on n'ignorait pas en gros tout
ce qu'il avait tiré de Law\,; le blâme fut général et d'autant plus
pesant que Law était fort déchu de la faveur et de l'éblouissement
public qu'une bagatelle tourna en dépit et en indignation.

Le maréchal de Villeroy, incapable d'inspirer rien au roi de solide,
adorateur du feu roi jusqu'au culte, plein de vent et de frivole, et de
la douceur du souvenir de ses jeunes années, de ses grâces aux fêtes et
aux ballets, de ses belles galanteries, voulut qu'à l'imitation du feu
roi, le roi dansât un ballet. C'était s'en aviser trop tôt. Ce plaisir
était trop pénible pour l'âge du roi, et il fallait vaincre sa timidité
peu à peu et l'accoutumer au monde qu'il craignait, avant de l'engager à
représenter en public, et à danser des entrées sur un théâtre. Le feu
roi élevé dans une cour brillante où la règle et la grandeur se voyaient
avec distinction, et où le commerce continuel des dames de la reine mère
et des autres de la cour l'avait enhardi et façonné de bonne heure,
avait primé et goûté ces sortes de fêtes et d'amusements parmi une
troupe de jeunes gens des deux sexes, qui tous portaient avec droit le
nom de seigneurs et de dames, et où il ne se trouvait que bien peu ou
même point de mélange, parce qu'on ne peut appeler ainsi trois ou quatre
peut-être de médiocre étoffe, qui n'y étaient admis visiblement, que
pour être la force et la parure du ballet, par la grâce de leur figure
et l'excellence de leur danse, avec quelques maîtres à danser, pour y
donner la règle et le ton. De ce temps-là à celui d'alors, il y avait
bien loin. L'éducation de ce temps passé formait chacun à la grâce, à
l'adresse, à tous les exercices, au respect, à la politesse
proportionnée et délicate, à la fine et honnête galanterie. On voit d'un
coup d'oeil toutes les étranges différences sans s'arrêter ici à les
marquer. La réflexion n'était pas la vertu principale du maréchal de
Villeroy. Il ne pensa à aucun des obstacles, soit du côté du roi, soit
du côté de la chose, et déclara que le roi danserait un ballet.

Tout fut bientôt prêt pour l'exécution. Il n'en fut pas de même pour
l'action. Il fallut chercher des jeunes gens qui dansassent, bientôt se
contenter qu'ils dansassent bien ou mal\,; enfin prendre qui on put, par
conséquent marchandise fort mêlée\,; plusieurs qui n'étaient pas pour y
être admis le furent si facilement que de l'un à l'autre Law, au point
où il était parvenu, se hasarda de demander à M. le duc d'Orléans que
son fils en pût être, qui dansait bien, et qui était d'âge à y pouvoir
entrer. M. le duc d'Orléans, toujours facile, toujours entêté de Law,
et, pour en dire la vérité, contribuant de dessein à toute confusion
autant qu'il lui était possible, l'accorda tout de plain-pied, et se
chargea de le dire au maréchal de Villeroy. Le maréchal, qui haïssait et
traversait Law de toutes ses forces, rougit de colère, et représenta au
régent ce qu'il y avait en effet à dire là-dessus\,; le régent lui en
nomma qui, quoique d'espèce fort supérieure, n'en étaient pourtant pas à
être du ballet\,; et quoique les réponses fussent aisées à l'égard de
l'exclusion du petit Law, le maréchal n'en trouva que dans de vaines
exclamations. Il ne put donc résister au régent, se trouvant sans
ressources du côté de M. le Duc, surintendant de l'éducation du roi,
grand protecteur de Law et des confusions, tellement que le fils de Law
fut nommé pour être du ballet.

On ne peut exprimer la révolte publique que cette bagatelle excita, dont
chacun se tint offensé. On ne parla d'autre chose pendant quelques
jours, et sans ménagement, non sans quelques éclaboussures sur quelques
autres du ballet. Enfin le public fut content, la petite vérole prit au
fils de Law, et, à cause du ballet dont il ne pouvait plus être, ce fut
une joie publique. Ce ballet fut dansé plusieurs fois, et le succès ne
répondit en rien aux désirs du maréchal de Villeroy. Le roi fut si
ennuyé et si fatigué d'apprendre, de répéter et de danser ce ballet,
qu'il en prit une aversion pour ces fêtes et pour tout ce qui est
spectacle, qui lui a toujours duré depuis, ce qui ne laisse pas de faire
un vide dans une cour, en sorte qu'il cessa plus tôt qu'on ne l'avait
résolu, et que le maréchal de Villeroy n'en osa plus proposer depuis.

M. le duc d'Orléans, par sa facilité ordinaire ou pour adoucir au monde
la nouvelle élévation de Law à la place de contrôleur général, fit
quantité de grâces pécuniaires\,; il donna six cent mille livres à La
Fare, capitaine de ses gardes\,; cent mille livres à Castries, chevalier
d'honneur de M\textsuperscript{me} la duchesse d'Orléans\,; deux cent
mille livres au vieux prince de Courtenay, qui en avait grand besoin\,;
vingt mille livres de pension au prince de Talmont\,; six mille livres à
la marquise de Bellefonds, qui en avait déjà une pareille, et à force de
cris de M. le prince de Conti une de soixante mille livres au comte de
La Marche son fils, âgé à peine de trois ans\,; il en donna encore de
petites à différentes personnes. Voyant tant de déprédation et nulle
vacance à espérer, je demandai à M. le duc d'Orléans d'attacher douze
mille livres en augmentation d'appointements à mon gouvernement de
Senlis, qui ne valait que mille écus, et dont mon second fils avait la
survivance, et je l'obtins sur-le-champ.

Tout ce que je voyais de jour en jour du gouvernement et des
embarquements de M. le duc d'Orléans, au dedans et au dehors,
m'affligeait de plus en plus et me convainquait de plus qu'il n'y avait
de remède que par le conseil étroit que je lui avais proposé, tel qu'on
l'a vu plus haut. Plus j'en sentais la difficulté par la légèreté de M.
le duc d'Orléans et par l'intérêt capital de l'abbé Dubois si fort
devenu son maître, plus j'y insistais souvent, quoique je me retirasse
de tout le plus qu'il m'était possible, et que M. le duc d'Orléans m'y
donnât beau jeu pour complaire à la jalousie de Dubois, qui craignait
tout, et moi sur tous autres. J'allai même jusqu'à presser M. le duc
d'Orléans de mettre dans ce conseil étroit le duc de Noailles, Canillac,
et tout ce qu'il me savait le plus opposé, non pas que j'estimasse leur
probité ni leur capacité, comme je le lui dis, mais pour lui marquer à
quel point je croyais cet établissement important et pressant à faire,
et que, tels que fussent ceux que je lui nommais, j'aimerais mieux les y
voir et que ce conseil fût établi. L'argument était pressant, aussi M.
le duc d'Orléans en fut-il surpris et embarrassé, parce qu'il en sentit
toute la bonne foi de ma part, conséquemment toute l'énergie. Il ne se
défendait point, mais tirait de longue. Je revenais de temps en temps à
la charge.

Une des dernières fois que je le pressais le plus et qu'il ne savait que
répondre, et c'était encore en nous promenant tous deux dans sa petite
galerie, devant son petit cabinet d'hiver, il se tourna tout d'un coup à
moi et me dit avec quelque vivacité\,: «\,Mais vous me pressez toujours
là-dessus\,; vous voulez ce conseil à tel point que vous consentez que
j'y mette qui je voudrai, jusqu'à ceux que vous haïssez le plus, et
vous, vous n'en voulez pas être\,; franchement, n'est-ce point que vous
sentez qu'il sera pour le moins aussi bon et plus sûr de n'en avoir
point été, quand le roi sera devenu grand\,? » À l'instant je lui saisis
le bras, et d'un ton bien ferme, en le regardant entre les deux yeux, je
lui répondis\,: «\,Oh\,! monsieur, puisque cette idée vous entre dans la
tête, je vous demande d'être de ce conseil, et je vous déclare que j'en
veux être. Je vous ai toujours dit que je n'y voulais point entrer,
parce que je vous connais, que vous auriez cru que je ne vous proposais
et pressais d'établir ce conseil étroit, que parce que, tout devant y
passer, je voulais augmenter par là mon autorité, mon crédit et me mêler
avec poids de toutes les affaires à mon sens et à mon gré, et que cette
opinion vous aurait éloigné d'un établissement si nécessaire, dans votre
idée que je ne vous le proposais et vous en pressais que pour mon
intérêt particulier, au lieu que, n'en voulant pas être, je vous ôtais
toute défiance d'intérêt particulier, que par cela même je donnais plus
de poids à ma proposition, et qu'elle devait vous sembler d'autant plus
pure, que ni vous ni moi ne pouvions pas nous dissimuler que faisant ce
conseil et ne m'en mettant pas, c'était pour moi un dégoût public, une
diminution très grande, très marquée, très publique de ma situation
auprès de vous, parce que peu de gens sauraient que je n'en avais pas
voulu être\,; et qu'entre ce peu-là, la plupart seraient persuadés que
c'était un discours, et qu'en effet je n'avais pu y entrer. Mais,
puisque votre défiance se tourne du côté que vous me la montrez, je vous
répète que je veux être de ce conseil, que je vous le demande, et que,
dès que je fais tant que d'insister auprès de vous pour y entrer, vous
ne pouvez me le refuser. Reste donc à nommer les trois autres\,; il y a
longtemps que je vous presse de le composer, toutes vos réflexions sur
le choix doivent être faites, nommez-les donc, et, au nom de Dieu,
finissons ce qui devrait être fini et établi huit jours après que je
vous en ai parlé la première fois.\,» Il demeura atterré et immobile,
honteux je crois de m'avoir montré une défiance si injuste, pour ne dire
pis, et si nettement repoussée\,; plus embarrassé encore entre la
salubrité de ce dont je le pressais, contre laquelle il sentait qu'il
n'avait aucune sorte de raison à opposer, et l'intérêt radicalement
contraire de l'abbé Dubois qui n'oubliait rien pour l'en empêcher, et
qui le tenait très et trop réellement dans ses fers. J'insistai encore
d'autres fois pour cet établissement, et toujours depuis cette
conversation pour en être, et toujours inutilement. À la fin je m'en
lassai et abandonnai la barque aux courants. J'ai rapporté de suite ce
qui se passa là-dessus à diverses reprises pour n'avoir point à revenir
inutilement sur une chose qui n'a point eu d'exécution.

M\textsuperscript{me} la princesse de Conti fit le mariage de la fille
unique de M\textsuperscript{me} de Feuquières, sa dame d'honneur, avec
Boisfranc, du nom de son père, frère de la défunte femme du duc de
Tresmes, qui se faisait appeler Soyecourt, dont était sa mère, qui,
mariée pour rien à ce vilain, hérita, comme on l'a vu ici en son temps,
de tous les biens de sa maison par la mort de ses deux frères sans
alliance, tués tous deux à la bataille de Fleurus. À ces grands biens,
il en venait d'ajouter de plus considérables depuis peu d'années par
l'héritage entier de tous ceux du président de Maisons. Ce Soyecourt en
masque et vilain en effet, était donc extraordinairement riche et avait
de très belles terres. M\textsuperscript{me} de Feuquières, veuve de
celui qui a laissé de si bons Mémoires de guerre, avait des affaires si
délabrées qu'elle avait été réduite à se mettre ainsi en condition pour
vivre, et pour une protection qui lui aidât à débrouiller les biens de
la maison d'Hocquincourt, dont elle était aussi la dernière et
l'héritière, et ceux de la maison de Pas, dont sa fille était aussi la
dernière et l'héritière, le frère de son père étant cadet, qui avait
épousé la fille de Mignard, peintre célèbre, pour sa beauté, qui avait
plus de quatre-vingts ans, et qui n'avait point eu d'enfants. Il y avait
de grands restes et bons dans ces deux successions, mais il fallait du
temps, de la peine, du crédit, de l'argent pour les liquider et en
jouir\,; et c'est ce qui faisait, en attendant, mourir de faim
M\textsuperscript{me} et M\textsuperscript{lle} de Feuquières et la
marier comme elle le fut. Ainsi ce Seiglière, car c'était le nom de la
famille de ce faux Soyecourt, joignit encore les biens de ces deux
maisons à ceux dont il avait déjà hérité. On le marqua encore ici à
dessein de montrer de plus en plus le désastre, l'ignominie, la
déprédation des mésalliances si honteuses des filles de qualité dont on
croit se défaire pour leur noblesse sans leur rien donner, et dont le
sort ordinaire est de porter tous les biens de leurs maisons, dont elles
deviennent héritières, par une punition marquée, à la lie qu'on leur a
fait épouser, en victimes de la conservation de tous ces biens à leurs
frères qui meurent sans postérité. Pour rendre complet le malheur de ce
mariage, Soyecourt avec de l'esprit, de la figure, de l'emploi à la
guerre, se perdit de débauches, de jeu, de toutes sortes d'infamies,
tellement que, de juste frayeur des arrêts qui le pouvaient conduire au
gibet, il sortit de France peu d'années après, se cacha longtemps dans
les pays étrangers, et mourut enfin en Italie au grand soulagement de sa
femme, de ses enfants et de MM\hspace{0pt}. de Gesvres.

Le comte de Vienne mourut assez subitement dans un âge peu avancé.
C'était un fort honnête homme, qui avait de l'esprit et de la grâce, qui
était fort du monde, au contraire de son frère aîné, le marquis de La
Vieuville, dont la femme était dame d'atours de M\textsuperscript{me} la
duchesse de Berry. Leur nom est Cokseart\footnote{Ce nom s'écrit
  ordinairement Coskaër.}\,;ils sont Bretons, et rien moins que des La
Vieuville de Flandre, dont ils ont pris le nom et les armes qu'ils ont
avec raison trouvés meilleurs que les leurs. On en a parlé ailleurs. Le
comte de Vienne n'eut point d'enfants de sa femme dont il portait le
nom, et qu'on {[}a{]} vu, il n'y a pas longtemps ici, qu'il avait perdue
subitement. Le prince de Murbach mourut en même temps vers Cologne\,; il
était frère de M\textsuperscript{me} de Dangeau, bien fait et de bonne
compagnie\,; il avait fait plusieurs séjours à la cour, il avait force
bénéfices et était riche\,: le nom qu'il portait était celui de son
abbaye commendataire de Murbach\footnote{L'abbaye de Murbach était
  située en Alsace\,; on en voit encore aujourd'hui les ruines dans le
  département du Haut-Rhin.}, qui donne titre de prince de l'empire,
mais qui en France n'opère aucun rang.

L'impératrice mère, veuve de l'empereur Léopold, et soeur de l'électeur
palatin, etc., mourut à Vienne d'apoplexie, qui fut un deuil de six
semaines pour le roi. C'était une princesse fort haute et fort absolue
dans sa cour et dans sa famille, qui avait eu un grand crédit sur
l'esprit de l'empereur Léopold, et plus encore sur celui de l'empereur
son fils, ce qui lui avait donné et conservé une grande considération.
Sa prédilection, de tout temps marquée pour ce prince son second fils,
et l'humeur impétueuse de l'empereur Joseph, son fils aîné, l'avait fort
écartée sous son règne. Elle était haute, fière, altière, grossière,
avec de l'esprit\,; elle aimait et protégea tant qu'elle put sa maison,
et fut toujours fort opposée à la France. Sans être du conseil, elle
entra fort dans les affaires, excepté pendant le règne de l'empereur
Joseph, et y donna un grand crédit à l'électeur palatin, même à ses
autres frères.

Le cardinal de La Trémoille mourut à Rome assez méprisé et à peu près
banqueroutier. Il avait pourtant des pensions du roi, et les fortes
rétributions attachées au cardinal chargé des affaires du roi, le riche
archevêché de Cambrai et cinq abbayes, dont deux fort grosses,
Saint-Amand et Saint-Étienne de Caen. Son ignorance, ses moeurs,
l'indécence de sa vie, sa figure étrange, ses facéties déplacées, le
désordre de sa conduite, ne purent être couverts par son nom, sa
dignité, son emploi, la considération de sa fameuse soeur la princesse
des Ursins, quoique raccommodé avec elle par sa promotion qu'elle avait
arrachée. C'était un homme qui ne se souciait de rien, et qui pourtant
craignait tout, tant il était inconséquent, et qui, pour plaire ou de
peur de déplaire, n'avait sur rien d'opinion à lui. On a assez parlé ici
de lui, en d'autres endroits, pour n'avoir rien à en dire davantage. Sa
mort me fait réparer un oubli qui mérite de trouver place ici, et qui, à
l'esprit près, montrera la parfaite ressemblance de l'abbé d'Auvergne au
cardinal de Bouillon.

On se souviendra ici de ce qu'il y a été dit du duc de Noirmoutiers,
aveugle, frère de M\textsuperscript{me} des Ursins et du cardinal de La
Trémoille, de son esprit et de toute la bonne compagnie qui abonda
toujours chez lui\,; qu'il se mêlait d'une infinité de choses et
d'affaires importantes\,; que, quoique souvent fraîchement avec
M\textsuperscript{me} des Ursins, il était toujours par le besoin son
plus intime correspondant, et il l'était pareillement du cardinal de La
Trémoille. Les Bouillon se piquaient fort d'être de ses amis, et le
voyaient tous sur le pied d'amitié particulière de tout temps. L'abbé
d'Auvergne était sur le même pied et tâchait même d'en tirer avantage
dans le monde. Un an à peu près après que Cambrai eut été donné au
cardinal de La Trémoille, M. de Noirmoutiers, dont la maison joignait la
mienne, qui, comme moi, avait une porte dans le jardin des Jacobins de
la rue Saint-Dominique, m'envoya prier de vouloir bien lui donner un
moment chez moi, et, par l'état où il était, de lui marquer un temps où,
s'il se pouvait, il n'y aurait personne. Quoiqu'il vît beaucoup de monde
chez lui, mais choisi, il n'aimait pas à sortir, ni à se montrer à
personne. C'était presque au sortir de dîner\,; je demandai à son valet
de chambre s'il avait du monde chez lui et ce qu'il faisait. Il me dit
qu'il était seul avec la duchesse de Noirmoutiers. C'était une femme
d'esprit, de sens et de mérite, en qui il avait toute confiance, et qui
suppléait en tout à son aveuglement. Je dis au valet de chambre que je
ne voulais pas donner la peine à M. de Noirmoutiers de venir chez moi,
qu'il me fît ouvrir sa porte sur le jardin des Jacobins, et je m'y en
allais par la mienne.

M. de Noirmoutiers fut d'autant plus sensible à cette honnêteté que je
ne le connaissais en façon du monde, et ne lui avais jamais parlé ni été
chez lui. Après les premiers compliments il m'en fit un sur la confiance
que lui donnait ma réputation, sans me connaître, de s'ouvrir à moi de
la chose du monde qui le peinait et l'embarrassait le plus, lui et le
cardinal de La Trémoille, et qu'après avoir bien pensé, cherché et
réfléchi, il n'avait trouvé que moi à qui il pût avoir recours. Si ce
début me surprit, la suite m'étonna bien davantage. Il commença par me
prier de lui parler sans déguisement, et de ne rien donner à la
politesse et aux mesures dans ma réponse à la question qu'il m'allait
faire, et tout de suite me pria de lui dire sans détour comment son
frère était dans l'esprit de M. le duc d'Orléans, et s'il était ou
n'était pas content de lui. Je lui répondis que, pour le faire aussi
correctement qu'il le désirait, il y avait du temps que rien ne s'était
présenté entre M. le duc d'Orléans et moi, où il fût question de lui,
mais qu'il m'en avait toujours paru content. Il insista et me conjura de
lui dire si le cardinal n'avait point eu le malheur de lui déplaire. Sur
ce que je le rassurai fort là-dessus, il me dit que cela augmentait sa
surprise\,; alors il me dit que l'abbé d'Auvergne, qu'il voyait très
souvent, parce qu'il était ami particulier de tout temps de toute sa
famille, et qui se donnait pour être fort le sien et celui du cardinal
de La Trémoille, avait fait proposer à ce cardinal de lui donner la
démission de l'archevêché de Cambrai, et fait entendre que M. le duc
d'Orléans le voulait ainsi\,; mais qu'il aimait mieux n'y pas
paraître\,; que le cardinal, à qui cela avait semblé extraordinaire, n'y
avait pas ajouté grande foi, mais que les instances s'étant redoublées
avec des avertissements qui dénonçaient la menace, il n'avait pu croire
que l'abbé d'Auvergne allât jusque-là de soi-même\,; que, dans cette
inquiétude, il lui en avait écrit, à lui duc, pour savoir ce qu'il
plaisait au régent, à qui il donnerait sa démission pure et simple
toutes les fois qu'il le désirerait, puisqu'il tenait la place du roi,
et que c'était de sa grâce qu'il avait reçu cet archevêché\,; que cette
affaire les affligeait fort l'un et l'autre\,; qu'il avait cherché les
moyens d'être éclairci des volontés du régent sans avoir pu trouver de
voie sûre\,; que, tandis qu'il les cherchait, les instances s'étaient
redoublées avec un équivalent de menaces des conseils de céder, de s'en
faire un mérite, et des protestations de la peine et de la douleur où
cette volonté déterminée du régent le jetait lui-même abbé d'Auvergne,
son ami, son parent, son serviteur de lui et de son frère, de tous les
temps ainsi que de toute sa famille, etc. Que dans cette crise, ne
sachant au monde à qui s'adresser, il avait imaginé la voie qu'il
prenait avec confiance, et le compliment au bout.

Ma surprise fut telle que je me fis répéter la chose deux autres fois,
sur quoi la duchesse de Noirmoutiers alla chercher les lettres du
cardinal, et m'en lut les articles qui regardaient et qui énonçaient ces
faits, et la perplexité où ils le mettaient. Je leur dis que je leur
rendrais confiance pour confiance dès cette première fois, mais sous le
même secret qu'ils m'avaient demandé\,; qu'à la mort de l'abbé
d'Estrées, nommé à Cambrai, M. le duc d'Orléans s'était hâté de donner
cet archevêché au cardinal de La Trémoille pour le bien donner par la
dignité, la naissance et l'actuel service à Rome\,; mais en même temps
pour se délivrer de la demande que la maison de Lorraine aurait pu lui
en faire pour l'abbé de Lorraine, à qui il ne voulait pas donner ce
grand poste si frontière, et de celle aussi des Bouillon pour l'abbé
d'Auvergne, à qui il l'aurait moins donné qu'à qui ce fût, à cause de sa
mère, de sa belle-mère, de sa belle-soeur, de sa nièce toutes des
Pays-Bas et de leurs biens et alliances\,; que j'étais parfaitement sûr
de cette disposition de M. le duc d'Orléans, qui me l'avait dite dans le
temps même, et que je n'avais rien aperçu depuis qui l'eût pu faire
changer de sentiment\,; que de plus c'était un prince si éloigné de
toute violence qu'il serait fort difficile d'imaginer qu'il songeât à en
faire une de telle nature et à un homme de l'état et de la naissance du
cardinal de La Trémoille, et dont je ne l'avais point vu mécontent. M.
de Noirmoutiers se sentit fort soulagé de cette opinion d'un homme aussi
avant que je l'étais dans la confiance de M. le duc d'Orléans\,; mais il
désira davantage, et me demanda si ce ne serait point abuser de moi dès
la première fois, que de me prier d'en parler franchement au régent. J'y
consentis, mais en avertissant Noirmoutiers que je ne le pouvais qu'en
faisant à M. le duc d'Orléans la confidence entière, à quoi il me
répondit qu'il l'entendait bien ainsi, en le suppliant du secret, et lui
offrant la démission du cardinal, dont il avait pouvoir, si elle lui
était agréable. Je lui dis que j'étais fâché de n'avoir pas été averti
deux heures plus tôt, parce que je sortais d'avec M. le duc d'Orléans,
qui en effet m'avait envoyé chercher tout à la fin de la matinée, auquel
j'en aurais parlé. Là-dessus M. de Noirmoutiers se mit aux regrets à
cause de l'ordinaire de Rome \footnote{L'\emph{ordinaire de Rome} était
  le courrier qui partait à époques fixes, chargé des dépêches de la
  France pour Rome.}. Je voulus lui faire le plaisir entier et retournai
sur-le-champ au Palais-Royal.

Le régent, surpris d'un retour si prompt et si peu accoutumé, m'en
demanda la cause\,; je la lui dis, et le voilà à rire aux éclats, et à
se récrier sur l'insigne friponnerie et l'impudence sans pareille. Il me
chargea de dire de sa part au duc de Noirmoutiers que jamais il n'avait
ouï parler de rien d'approchant ni n'en avait eu la moindre pensée\,;
qu'il était très content du cardinal de La Trémoille, et très éloigné de
se repentir de lui avoir donné Cambrai\,; qu'il le priait donc de le
garder sans aucune inquiétude\,; mais qu'il les priait aussi l'un et
l'autre d'être de plus bien persuadés que, quand bien même il serait
possible que la volonté de s'en démettre vînt au cardinal, et qu'on ne
pût l'en empêcher, il n'y avait en France évêque ni abbé à qui il ne
donnât Cambrai plutôt qu'à l'abbé d'Auvergne. Comme l'heure des plaisirs
du soir approchait, je ne fis pas durer la conversation, et je me hâtai
d'aller délivrer M. et M\textsuperscript{me} de Noirmoutiers, qui se
dilatèrent merveilleusement à mon récit. On peut juger ce qu'il fut dit
entre nous trois de leur bon parent et ami l'abbé d'Auvergne, auquel
toutefois ils résolurent de n'en pas faire semblant, mais de lui faire
écrire par le cardinal de La Trémoille une négative si nette et si
sèche, qu'il n'osât plus retourner à la charge, et qui lui fît sentir
qu'il était découvert. Il le sentit en effet si bien qu'il demeura tout
court, mais sans cesser de voir M. de Noirmoutiers, comme si jamais il
n'eût été question de cette affaire, avec une effronterie en vérité
incroyable.

Quelque hardies, quelque peu imaginables, quelque finement ourdies que
fussent les friponneries de ce bon ecclésiastique et de son oncle, elles
ne furent pas heureuses. On a vu ici (t. II, p.~110) la double
friponnerie par laquelle le cardinal de Bouillon, chargé lors des
affaires du roi à Rome, et surtout de s'opposer en son nom à la
promotion du duc de Saxe-Zeits, évêque de Javarin, que l'empereur
voulait absolument porter à la pourpre, la double friponnerie, dis-je,
par laquelle il pensa tromper le pape et le roi, en faisant passer
l'évêque et l'abbé d'Auvergne avec lui, disant au pape que le roi ne
consentirait à l'évêque qu'à cette seule condition en faveur de son
neveu par amitié pour lui, et mandant au roi que, ne pouvant plus
empêcher la promotion de l'évêque, il avait au moins obtenu qu'un
Français fût promu avec l'impérial, à quoi le pape n'avait jamais voulu
consentir que pour l'abbé d'Auvergne, par amitié pour lui, cardinal de
Bouillon. Le pape, depuis si longtemps arrêté sur la promotion de
l'évêque de Javarin par les plus fortes protestations du roi, qui
n'avait jamais voulu écouter nulle condition là-dessus, fut si étonné de
la proposition du cardinal de Bouillon, dont l'ambition était connue et
la probité fort démasquée, que Sa Sainteté prit le parti de mander le
fait au roi par un billet de sa main, pour être éclairci par sa réponse,
et de faire passer ce billet droit à Torcy pour le remettre au roi sans
aucune participation de son nonce ni de ses principaux ministres à Rome.
Le roi lui répondit de sa main par la même voie, le remercia, lui
témoigna toute son indignation, et, insistant également contre la
promotion de l'évêque de Javarin, lui déclara qu'il aimerait mieux qu'il
le fît cardinal seul que de faire avec lui l'abbé d'Auvergne\,; qu'il ne
souffrirait pas qu'il le fût. Ce mot n'est que pour en rappeler ici la
mémoire\,; l'histoire entière se trouve mieux au temps où elle arriva et
où elle a été ici rapportée.

Mais à propos des raisons d'exclusion de l'abbé d'Auvergne sur Cambrai
par rapport à sa famille, je ne puis m'empêcher de remarquer ici,
puisque cela s'y présente naturellement, l'esprit suivi des Bouillon
depuis que Henri IV eut fait la fortune du vicomte de Turenne en lui
faisant épouser l'héritière de Sedan, le fit maréchal de France pour y
atteindre, et le soutint pour en conserver les biens contre l'oncle
paternel et ses enfants, quoique le maréchal n'eût point eu d'enfants de
leur nièce et cousine. Je ne parle point de tout ce qu'il fit contre
Henri IV et contre Louis XIII depuis qu'il se figura être prince, ni de
ce que firent ses enfants. Je me borne ici à dire un mot de leurs
mariages, pour se fortifier au dehors pour leurs félonies dont la vie de
ce maréchal, depuis cette époque, et celle de ses fils n'a été qu'un
tissu, et les mariages de leur postérité, quoique leur faiblesse et la
puissance de Louis XIV depuis la paix des Pyrénées ne leur ait laissé
que la volonté d'imiter leurs pères sans leur en laisser les moyens. Ce
n'est pas leur rien prêter\,: on le prouve par la désertion du prince
d'Auvergne en pleine guerre, en plein camp, sans mécontentement aucun,
et par la seule et folle espérance de devenir stathouder de Hollande en
se signalant comme il fit contre le roi en propos et en service. On le
prouve par la félonie du cardinal de Bouillon. On le prouve par le refus
de se reconnaître sujets du roi, comme le cardinal eut le front de le
lui écrire, et comme son frère aîné aima mieux risquer tout que de
s'avouer tel, comme cela est expliqué ici (t. VI, p.~395) et l'adresse
fort étrange par laquelle d'Aguesseau, lors procureur général, le sauva
sans s'avouer sujet. Mais revenons à leurs mariages.

H. de La Tour, vicomte de Turenne, qui se fit huguenot, à quoi il gagna
tant, et qui servit si bien Henri IV jusqu'à ce que ce prince lui fit
épouser l'héritière de La Marck, dame de Bouillon, Sedan, etc., et qui
lui fut depuis si perfide, si ingrat et si félon, lui et sa postérité
{[}à Henri IV{]} et à celle de ce monarque qui l'avait fait maréchal de
France pour ce mariage, si connu auparavant sous le nom de vicomte de
Turenne, et depuis sous celui de maréchal de Bouillon, n'avait point eu
de mères que de la noblesse française. Veuf sans enfants de cette
héritière qui avait un frère de son père et des cousins germains, il
conserva par force et par la protection de Henri IV qui s'en repentit
bien depuis, comme on le voit par les Mémoires de Sully, et par tous
ceux et les histoires de ce temps-là\,; il conserva, dis-je, toute la
succession de l'héritière qui lui servit à figurer contre son roi et son
bienfaiteur au dedans et au dehors du royaume, en s'appuyant des
huguenots Français et étrangers, et par des mariages étrangers qu'ils
lui facilitèrent. Ainsi il se remaria à la fille puînée du célèbre
Guillaume de Nassau, prince d'Orange, fondateur de la république des
Provinces-Unies, qui, cherchant de son côté à s'assurer des huguenots de
France, pour se faciliter et se continuer l'appui si nécessaire de cette
couronne à sa république naissante et à la continuité de la grandeur et
de la puissance qu'il y avait acquises et la transmettre aux siens, fit
volontiers ce mariage de sa fille et d'une autre encore fort peu après,
avec Charles de La Trémoille, second duc de Thouars, pair de France, qui
étaient les deux plus grands seigneurs huguenots de France. Mais, pour
montrer quelles alliances celle-là leur donna au dehors, il faut voir
ici les enfants que ce célèbre prince d'Orange eut de quatre femmes
qu'il épousa successivement\,: d'Anne d'Egmont, fille du comte de Buren,
il laissa Philippe-Guillaume, qui à sa mort en 1582, par un assassin, à
cinquante un ans était entre les mains des Espagnols, fut catholique et
attaché à eux toute sa vie, et n'eut point d'enfants d'une fille du
prince de Condé, mort à Saint-Jean d'Angély, et de Charlotte de La
Trémoille. Il était mort particulier en 1618, un an avant son épouse. Sa
soeur unique de même lit fut la comtesse d'Hohenlohe. D'Anne, fille de
Maurice, électeur de Saxe, il eut Maurice, prince d'Orange, qui succéda
à ses charges et à sa puissance, dans la république des Provinces-Unies,
et ne s'y rendit pas moins célèbre, mais il ne se maria point, et mourut
en 1625, à cinquante-huit ans\,; Louis, comte de Nassau, mort sans
alliance aux guerres des Pays-Bas\,; et une fille mariée à un bâtard du
bâtard don Antoine, prieur de Crato, qui se prétendit roi de Portugal,
après la mort du cardinal-roi, lorsque Philippe II envahit cette
couronne sur la branche de Bragance, qui y fut depuis rétablie. Ce
gendre du prince d'Orange courut les mers en qualité de vice-roi des
Indes, et n'eut point de postérité. De Charlotte de Bourbon, professe et
abbesse de Jouarre, qui en sauta les murs, se fit huguenote et se sauva
chez l'électeur palatin, fille du premier duc de Montpensier, mariée en
1572, morte en 1582 de la peur qu'elle eut à Anvers du premier
assassinat de son mari, manqué et blessé légèrement d'un coup de
pistolet, à table auprès d'elle, il eut Louise-Julienne, épouse de
Frédéric IV, électeur palatin, qui de luthérien se fit calviniste, et
qui mourut en 1610. Il eut d'elle quantité d'enfants, entre autres
Frédéric V, électeur palatin, qui se perdit en usurpant la couronne de
Bohême, et fut grand-père de M\textsuperscript{me} la Princesse, etc.,
la duchesse des Deux-Ponts, l'électrice de Brandebourg, épouse de
l'électeur J. Guillaume. De ce même lit, le prince d'Orange eut la
maréchale de Bouillon, morte en 1642, la comtesse d'Hanau, la duchesse
de La Trémoille, une abbesse de Sainte-Croix de Poitiers, qui se sauva
de l'hérésie, et une autre fille mariée à un prince palatin de Lensberg.
Enfin de Louise, fille du célèbre amiral de Coligny, veuve sans enfants
du seigneur de Téligny, il laissa Henri-Frédéric, prince d'Orange, qui
succéda à ses charges et à son autorité en Hollande, mort en 1647, et
qui fut grand-père du fameux Guillaume, prince d'Orange, mort dernier de
cette branche, sur le trône d'Angleterre, 19 mars 1702. On voit d'un
coup d'oeil quelles et combien d'alliances étrangères son mariage donna
au maréchal de Bouillon parmi les protestants.

Ceux de ses filles et du célèbre vicomte de Turenne, son second fils,
qui n'eut point d'enfants, ne lui en procurèrent pas moins en leur
genre, parmi ce qu'il y eut de plus considérable parmi les protestants
de France, de tous lesquels le père et les enfants surent tirer de
grands et de continuels avantages au dedans et au dehors\,; c'est ce qui
détermina le cardinal Mazarin, effrayé des dangers qu'il avait courus et
dans lesquels il avait entraîné le royaume, à s'attacher deux hommes
tels que les deux fils du maréchal de Bouillon, mort à Sedan, en mars
1623, à soixante-huit ans, et à ne rien épargner pour s'en faire un
bouclier personnel, en leur donnant par le traité de l'échange de Sedan,
qu'ils avaient perdu et qu'ils ne pouvaient ravoir ni le conserver après
tant et de si étranges félonies, en leur donnant, dis-je, des millions,
des terres qui se peuvent appeler des États, des emplois les plus
importants et un rang inconnu en France, qui en souleva toute la
noblesse, et qui était inouï, même si nouveau pour ceux de maison
effectivement souveraine, composé d'usurpations, de ruses, de violences,
parmi les troubles, les tourbillons et les forfaits de la Ligue.

Le duc de Bouillon, fils aîné du maréchal, épousa en 1634 une fille de
Frédéric, comte de Berg, gouverneur de Frise, qui n'avait pas moins
d'esprit, de courage, d'entreprise et d'intrigues que son mari, ni moins
de capacité à les ourdir et à les conduire\,; avec de la beauté, de la
vertu, un mérite aimable et soutenu et de la grandeur d'âme\,; elle
mourut à quarante-deux ans, en 1657, et M. de Bouillon à Pontoise, où
était la cour, en 1652, à quarante-sept ans.

M. de Turenne son frère prit soin de ses neveux et de ses nièces. On a
vu à quelle fortune il porta ses trois neveux\,; les deux autres furent
tués en duel avant qu'il eût le temps de les agrandir. Dès cinq nièces,
l'une ne daigna pas se marier, et mourut à quarante-trois ans, sans
avoir trouvé parti digne d'elle\,; deux furent religieuses de
Sainte-Marie, les deux autres mariées, l'aînée au duc d'Elboeuf, dont
les deux derniers ducs d'Elboeuf\,; la dernière, en 1668, à Maximilien,
frère de l'électeur de Bavière, père des électeurs de Cologne et de
Bavière, mis au banc de l'empire pour s'être attachés à la France. Ce
duc Maximilien n'en eut point d'enfants\,; il mourut à Turckheim, en
1605, et elle au même lieu, en 1606, à quarante-deux ans\footnote{Le ms.
  porte 1605 et 1606, mais il faut lire 1705 et 1706.}.

M. de Bouillon, frère du cardinal, et ses enfants\,: leurs mariages sont
connus, au moins épousa-t-il une Italienne\footnote{La duchesse de
  Bouillon était Marie-Anne Mancini, nièce du cardinal Mazarin, et soeur
  de Marie Mancini, qui avait épousé don Lorenzo Colonna, connétable du
  royaume de Naples.}, soeur de la connétable Colonne\,; et un de ses
fils, une Irlandaise fort intrigante.

Mais on ne peut s'empêcher d'admirer la profonde réflexion de son fils
qui lui fit dénicher un parti très singulier pour son fils, l'art et la
dépense qu'il sut employer pour l'obtenir, et ce fils mort aussitôt
après la consommation du mariage, tout ce qu'il mit en oeuvre pour
obtenir dispense de la faire épouser à son second fils. On supprime ici
l'étonnement où elle fut de se trouver ici bourgeoise du quai Malaquais,
comme elle l'osa dire, ayant compté d'épouser un souverain, et de tenir
une cour. Aussi le mariage fut-il peu heureux, et après quelques années
{[}elle{]} finit par retourner en Silésie au grand contentement de son
mari et au sien, d'où elle n'est plus revenue.

Le comte d'Auvergne (on a expliqué ici, t. V, p.~313, 320 et 321, ces
noms de comte et de prince d'Auvergne), frère du duc et du cardinal de
Bouillon, fut marié, par M. de Turenne, son oncle, en 1662, à la fille
unique de Frédéric de Hohenzollern et d'Élisabeth héritière de
Berg-op-Zoom, qui lui apporta dès lors cette grande terre, et d'autres
biens en mariage avec les alliances d'Allemagne et des Pays-Bas. Elle
mourut à Berg-op-Zoom, où elle était allée faire un voyage en 1698,
laissant plusieurs enfants. Il se remaria dès 1699, et toujours en
Hollande, et il épousa à la Haye Élisabeth Wassenaer, qui se fit depuis
catholique à Paris, et qui y mourut sans enfants peu d'années après. Le
comte d'Auvergne mourut ensuite à Paris, à la fin de 1707 à
soixante-sept ans. Le seul de ses enfants, fils et filles, qui se soit
marié, est le prince d'Auvergne, dont la désertion et la conduite ont
été rapportées ici, (t. IV, p.~3), en leur temps. Passé sans cause que
de folles espérances de sa maison, fondées sur leurs alliances en
Allemagne et en Hollande, de la tête de son régiment au camp ennemi dès
l'entrée de la campagne, il fut trouver d'abord sa tante en Bavière, et
deux mois après se mit au service des États généraux. Ce fut lui qui, à
la tête d'un gros détachement, alla recevoir le cardinal de Bouillon,
dont la fuite aux ennemis était concertée. Il épousa, en 1707, la soeur
du duc d'Arenberg, et mourut en 1710, à trente-cinq ans\,; c'était un
gros garçon, fort épais de corps et d'esprit, grossier, et qui comptait
sottement devenir stathouder des Provinces-Unies. Il ne laissa point de
garçons\,; sa fille épousa, en 1722, J. Christian, prince palatin de
Sultzbach, morte à Hippolstein, en 1728, à vingt ans, laissant un fils
unique, Charles-Philippe, prince de Sultzbach, par la mort de son père,
en 1733, et devenu électeur palatin à la fin de 1742. C'est de ces
alliances palatines dont le duc de Bouillon d'aujourd'hui cherche à
s'appuyer, en se parant du nouvel ordre de l'électeur palatin.

Tels ont été l'esprit et les vues constantes de cette branche de la
maison de La Tour depuis que par l'usurpation de Sedan elle a tâché sans
cesse de se séparer de son être, de ne vouloir plus faire partie de la
noblesse française, et de démentir son origine et leurs pères qui de
cette origine ont tiré tout leur honneur et leur lustre, qui ont vécu
parmi elle sans prétention, qui se sont toujours glorifiés d'être sujets
de nos rois. Les réflexions sur tout cela se présentent en foule et bien
naturellement d'elles-mêmes.

Encore un mot sur l'abbé d'Auvergne. Lorsque l'abbé de Castries, sacré
archevêque de Tours, passa peu après à l'archevêché d'Albi, l'abbé
d'Auvergne eut celui de Tours. L'abbé de Thesut, secrétaire des
commandements de M. le duc d'Orléans, qui avait alors la
feuille\footnote{Celui qui avait la feuille des bénéfices présentait au
  roi ou au régent les candidats aux bénéfices vacants.}, travaillant
avec ce prince, fit un cri épouvantable quand il entendit cette
nomination, dont il dit son avis par l'horreur qu'elle lui fit. Le
régent convint de tout, y ajouta même le récit d'aventures de laquais
fort étranges et assez nouvelles, et comme cet énorme genre de débauche
n'était pas la sienne, il avoua à Thesut qu'il avait eu toutes les
peines du monde à faire l'abbé d'Auvergne évêque, mais qu'il en était
depuis longtemps si persécuté par les Bouillon, qu'il fallait à la fin
se rédimer de vexation. Thesut insista encore, puis écrivit la
nomination sur la feuille en haussant les épaules\,; c'est lui-même qui
me raconta ce fait deux jours après. Cela n'a pas empêché peu après la
translation de l'abbé d'Auvergne, sacré archevêque de Tours à
l'archevêché de Vienne, qu'il aima mieux. Tel fut le digne choix du
cardinal Fleury pour la pourpre à la nomination du roi, dont le scandale
fut si éclatant et si universel, que le cardinal Fleury n'en put cacher
sa honte. On se contentera ici de ce mot pour achever de présenter la
fortune de l'un et montrer le digne goût de l'autre, parce que cette
promotion dépasse les bornes de ces Mémoires.

\hypertarget{chapitre-xix.}{%
\chapter{CHAPITRE XIX.}\label{chapitre-xix.}}

1720

~

{\textsc{Comte Stanhope à Paris.}} {\textsc{- Paix d'Espagne.}}
{\textsc{- Grimaldo supplée presque en tout aux fonctions de premier
ministre d'Espagne, sous le titre de secrétaire des dépêches
universelles.}} {\textsc{- Sa fortune, son caractère.}} {\textsc{-
Digression déplacée, mais fort curieuse, sur le premier président de
Mesmes.}} {\textsc{- Duchesse de Villars et dames nommées pour conduire
la princesse de Modène jusqu'à Antibes.}} {\textsc{- Remarques sur le
cérémonial, le voyage et l'accompagnement.}} {\textsc{- Fiançailles et
mariage de cette princesse.}} {\textsc{- Désordre du système et de la
banque de Law se manifeste et produit des suites les plus fâcheuses et
infinies.}} {\textsc{- Commencements et fortune des quatre frères
Pâris.}} {\textsc{- Nouveaux prisonniers à Nantes.}} {\textsc{-
Vingt-six présidents ou conseillers remboursés et supprimés, choisis
dans le parlement de Bretagne.}}

~

Le comte Stanhope, ministre d'État fort accrédité du roi d'Angleterre,
dont il a été fait si souvent mention dans ce qui a été rapporté
ci-devant d'après Torcy sur les affaires étrangères, vint de Londres
conférer avec l'abbé Dubois et M. le duc d'Orléans à l'occasion de la
paix où l'Espagne ne tarda pas d'accéder dès qu'Albéroni fut chassé.
Cette grande démarche fut même accompagnée d'une lettre très amiable du
roi d'Espagne au régent, en sorte que la bonne intelligence parut
rétablie. La place de premier ministre d'Espagne ne fut point remplie.
Albéroni en avait dégoûté Leurs Majestés Catholiques, et leurs sujets
exultèrent de n'en avoir plus\,; mais elle fut en quelque sorte
remplacée sans titre et sans puissance personnelle par un homme qui
doucement en fit toutes les fonctions d'une manière plus agréable\,;
c'est-à-dire, qu'il fut comme le seul qui travaillât avec le roi sur
toutes les matières des autres bureaux dont les secrétaires d'État lui
envoyaient les affaires qui se devaient rapporter, à qui il les
renvoyait avec l'ordre du roi sur chacune. Ainsi les autres secrétaires
d'État travaillaient\,; c'était à eux qu'on s'adressait pour les
affaires de leur département\,; la direction et le détail leur en
demeurait\,; mais ils n'allaient au roi presque que par Grimaldo, hors
des occasions fort rares, et c'était toujours à lui à qui il en fallait
dire un mot, et tâcher de l'avoir favorable, après avoir sollicité les
autres secrétaires d'État, chacun selon que l'affaire le regardait, et
qu'elle était envoyée à Grimaldo pour en parler au roi.

Ce Grimaldo était un Biscayen de la plus obscure naissance et d'une
figure tout à fait ridicule et comique, surtout pour un Espagnol\,;
c'était un fort petit homme blond comme un bassin, gros et fort pansu,
avec deux petites mains appliquées sur son ventre, qui, sans s'en
décoller, gesticulaient toujours, avec un parler doucereux, des yeux
bleus, un sourire, un vacillement de tête qui donnaient l'accompagnement
du visage à son ton et à son discours, avec beaucoup d'esprit\,; il
l'avait très fin, très adroit, très insinuant, très politique, bas et
haut à merveilles, suivant ce qui lui convenait et à qui il convenait,
et avait l'art de ne s'y point méprendre. La première fois que le duc de
Berwick qui me l'a conté fut en Espagne, on le lui voulut donner pour
secrétaire espagnol, et il l'aurait pris s'il eût su l'espagnol, dont il
ne savait pas un mot alors, ou si Grimaldo eût entendu tant soit peu le
français. Hors d'espérance de cette condition, il en chercha une autre,
et il entra commis dans les bureaux d'Orry avant qu'Orry fût devenu
homme principal en Espagne. Il goûta Grimaldo par son esprit et sa
douceur, plus encore parce qu'il le trouva net et infatigable au
travail, fécond en ressources, et ne se rebutant jamais de rien. Ces
qualités le portèrent à la tête d'un des bureaux de son maître, et ce
bureau crût en commis sous lui et en affaires à mesure qu'Orry crût en
autorité et en puissance. Orry le fit goûter et connaître à la princesse
des Ursins, et par eux du roi et de la reine. Approché d'eux, et peu à
peu admis à travailler avec eux au lieu d'Orry, quand celui-ci n'en
avait pas le temps ou ne voulait pas le prendre. De là il parvint à être
secrétaire d'État avec le département de la guerre, où il n'avait rien à
faire qu'à recevoir et à exécuter les ordres d'Orry et de
M\textsuperscript{me} des Ursins, auxquels il faut dire à son honneur
qu'il demeura fidèle à tous les deux après leur chute, et à leurs amis
et créatures tant qu'il a vécu. Dans une telle dépendance, on peut juger
qu'il fut un des premiers dont Albéroni se défit, et qu'il ne le laissa
pas approcher tant qu'il fut le maître. Dans cette espèce d'exil,
Grimaldo, toujours titulaire de son emploi, mais dont il n'exerçait
aucune partie, demeura retiré dans sa maison de Madrid, ayant conservé
l'affection publique et beaucoup d'amis par les manières gracieuses et
polies dont il avait usé avec tout le monde, et son caractère obligeant
qui le portait à servir, toutefois presque sans aucun commerce, tant on
craignait Albéroni, et ce peu de commerce avec ses meilleurs amis ne
subsistait qu'avec de grandes mesures.

Le roi d'Espagne, malgré cet éloignement, n'avait point changé pour
lui\,; il le fit même venir deux ou trois fois parler à lui la nuit et
dans le plus profond secret. Don Alonzo Manriquez, de tout temps favori
du roi et ami intime de Grimaldo, était le dépositaire de ce secret et
le conducteur de Grimaldo au palais. C'est cet Alonzo, dont on aura à
parler dans la suite, qui ne ploya jamais devant Albéroni, dont Albéroni
ne put jamais se défaire\,; connu depuis sous le nom de duc del Arco,
grand d'Espagne et grand écuyer, qui est l'une des trois grandes
charges. Grimaldo, demeuré dans cette situation secrète auprès du roi
d'Espagne, fut remis en place à l'instant de la chute d'Albéroni, et de
secrétaire d'État de la guerre, dont le seul titre lui était demeuré,
fut fait secrétaire des dépêches universelles, ce qui le fit travailler
seul avec le roi à l'exclusion de tous les autres secrétaires d'État ou
chefs de ce peu qui restait de conseils, et porter sans eux leurs
affaires au roi, comme il a été expliqué plus haut, ainsi que toutes les
grâces, et en particulier toutes les affaires étrangères qui ne
passaient que par lui et ne se traitaient qu'avec lui. Il revint le même
qu'il avait été. Le crédit et l'autorité supérieure ne le gâtèrent
point, il se fit considérer, respecter et aimer de tout le monde, si on
en excepte un petit nombre d'envieux, car jusqu'aux refus il les savait
assaisonner avec tant de grâce qu'on ne pouvait lui en savoir mauvais
gré. Il faut pourtant dire que dans cette élévation il ne put résister à
la faiblesse de vouloir être homme de qualité. Il joua donc sur le mot,
s'entêta de la proximité de nom de Grimaldo à Grimaldi\,; il voulut être
de cette maison, il en prit les armes pleines, et, quand avec les années
il crut y avoir accoutumé le monde, il osa quoique inutilement aspirer à
la grandesse. C'en est assez sur lui pour à présent. Je le trouvai en
Espagne dans ce grand emploi et dans toute la faveur et la confiance du
roi d'Espagne. Ce fut donc avec lui que j'eus à traiter, et j'aurai
occasion d'en parler davantage lors de mon ambassade. J'ajouterai
seulement ici que la reine qui avait chassé M\textsuperscript{me} des
Ursins, et Orry par conséquent, et qui avait mis Albéroni en leur place,
dont toutes les impressions en mal lui restèrent toujours, n'aima jamais
Grimaldo, mais le traita comme si elle l'aimait, parce qu'elle n'avait
pu l'ébranler auprès du roi d'Espagne, qu'il ne donnait pas la moindre
prise sur lui, qu'il n'était haï de personne, mais aimé et estimé de
tous, et que son estime passa partout au dehors par la manière dont il
se conduisit toujours et dont il mania les affaires.

Comme j'en étais à cet endroit, j'appris de M. Joly de Fleury, procureur
général, une anecdote trop singulière et trop curieuse pour ne la pas
mettre ici, quoique hors de place, et que j'aurais insérée si je l'avais
sue peu de jours après que le duc et la duchesse du Maine furent
arrêtés. Il m'apprit donc, causant ensemble de ces temps passés, que
M\textsuperscript{lle} de Chausseraye, celle dont il a été parlé plus
d'une fois ici, et qui toute sa vie s'est mêlée de tant de choses, que
le premier président de Mesmes, inquiet au dernier point, peu après que
M. et M\textsuperscript{me} du Maine furent arrêtés, la pressa de lui
obtenir une audience de M. le duc d'Orléans, qui fut secrète, et qu'il
n'osait lui-même demander\,; elle la demanda donc, et ne put en venir à
bout qu'avec peine. Au jour et heure marquée, elle se rendit au
Palais-Royal, et M. le duc d'Orléans eut la complaisance de donner à son
valet de chambre, qu'elle avait amené exprès, nommé du Plessis, fort
connu de lui et de tout le monde, sa clef d'une de ses portes secrètes,
car il en avait plusieurs qui, des rues qui environnent le Palais-Royal,
conduisaient droit et secrètement à ses appartements. Ce du Plessis fut
donc ouvrir au premier président, qui pour se mieux cacher était en
manteau et point en robe, et l'amena à M. le duc d'Orléans qui
l'attendait seul et enfermé avec M\textsuperscript{lle} de Chausseraye.
Là le premier président, qui était beau diseur et qui avait fort la
parole en main, fit à M. le duc d'Orléans les protestations les plus
fortes de fidélité et d'attachement, à l'occasion des occurrences alors
présentes, et comme l'esprit ne lui manquait non plus que le langage, il
n'oublia rien pour démêler, dans l'air froid et sérieux qu'il trouva, si
M. le duc d'Orléans était instruit à son égard de quelque chose, sans y
avoir pu réussir, tant le régent sut se contenir, se mesurer et ne lui
pas laisser apercevoir la moindre chose. Il prit même plaisir à lui
donner lieu de redoubler ses protestations, et à tout son bien-dire.
Quand il en eut assez, il tira une lettre de sa poche, et tout à coup\,:
«\,Monsieur, lui dit-il, d'un ton irrité\,; tenez, lisez cela\,; le
connaissez-vous\,?» À l'instant le premier président fondit à deux
genoux, lui embrassant non pas les jambes mais les pieds, et se mit aux
pardons, aux regrets, aux repentirs, et n'eut si belle peur de sa vie.
M. le duc d'Orléans reprit la lettre, se dépêtra les pieds de ses bras,
et sans dire un mot s'en alla dans un autre cabinet. C'était une lettre
de sa main, par laquelle il répondait du parlement à l'Espagne, et
parlait sans ménagements et sur la chose et sur les moyens.

Éperdu et sans parole, il eut peine à se reconnaître et à se relever de
ce prosternement où il était. M\textsuperscript{lle} de Chausseraye,
guère moins éperdue, mais d'étonnement, lui reprocha la folle hardiesse
de l'avoir commise à lui obtenir cette audience, lui se sentant aussi
coupable\,; toute sa réponse fut de la conjurer de le sauver et d'aller
trouver M. le duc d'Orléans. Elle y alla, et le trouva seul dans la
dernière indignation de l'audace, de l'effronterie de l'audience, de la
scélératesse, de la tromperie et des protestations, avec une telle pièce
écrite de la main du premier président, qu'il lui dit qu'il allait faire
arrêter. La Chausseraye qui connaissait bien à qui elle avait affaire,
se prit à sourire\,: «\,Bon, lui dit-elle, le faire arrêter, il le
mérite bien, et pis\,; mais avec cette pièce en main, et l'aveu qu'il
n'a pu dénier, voilà un homme qui ne peut plus qu'être à vous à vendre
et à dépendre, et c'est la meilleure aventure qui vous pût arriver,
parce que désormais vous en ferez tout ce qu'il vous plaira sans qu'il
ose souffler, ni s'exposer à ne pas être à plaît-il maître sans
réserve.\,» Quoique rien ne fût plus selon l'esprit et le goût de M. le
duc d'Orléans qui aimait, sur toutes autres, ces voies obliques, et dans
son caractère encore d'éviter les grands engagements, tels que faire
faire le procès à ce scélérat si fort du premier ordre, mais qui était
premier président, quoique le procès ne pût être douteux, et un procès
qui par ses dépositions aurait embarrassé non seulement le duc et la
duchesse du Maine, mais bien d'autres gens encore du plus haut parage,
elle eut toutes les peines du monde à suspendre la résolution. Le temps
durait cependant au premier président d'une étrange sorte, qui se
trouvait entre la mort et la vie, car, pour le déshonneur et l'infamie,
il y était accoutumé de longue main\,; enfin Chausseraye le vint
trouver, et après lui avoir dit ce qu'elle jugea à propos pour le
rassurer assez pour lui faire retrouver les jambes, et qu'il en pût
faire usage pour s'en retourner, elle alla appeler du Plessis, et le
renvoya par où il était venu. Il fut longtemps encore dans les transes
de la mort, avec la nécessité de paraître aux fonctions de sa charge et
y faire bonne mine, et parmi les gens qu'il voyait, quoiqu'avec M. le
duc d'Orléans, qui avait du temps, {[}il{]} pouvait compter de bien
sortir d'affaire\footnote{Cette phrase a été exactement reproduite
  d'après le manuscrit. Le sens est probablement que le premier
  président espérait, en gagnant du temps, se tirer d'affaire avec M. le
  duc d'Orléans.}, comme il arriva en effet.

L'abbé Dubois, à qui sûrement le régent ne cacha pas une chose si
importante, n'avait garde de le pousser\,; il voulait être maître de
l'affaire en total, par les raisons qui en ont été rapportées\,; et non
seulement il ne l'était plus en poussant le premier président, mais il
ne pouvait douter que ses dépositions apprendraient à M. le duc
d'Orléans tout ce que lui Dubois lui avait caché de toute cette
conspiration pour en demeurer lui seul le maître, et c'en était bien
plus qu'il n'en fallait pour sauver le premier président, parce que ce
n'était pas moins que de se sauver lui-même d'une si perfide et noire
infidélité. Ainsi toute pensée d'agir contre de Mesmes tomba bientôt, et
la chose demeura entièrement secrète\,; c'est la Chausseraye elle-même
qui la conta longtemps depuis au procureur général telle que je la viens
d'écrire, et je l'ai écrite aussitôt qu'il me l'a eu racontée, pour
l'insérer ici dans l'exactitude précise qu'il me l'a rendue bien des
années après la mort de M. le duc d'Orléans, de ce coquin de Mesmes, si
fort scélérat par excellence, et si prodigieusement impudent, qui mourut
avant le régent comme il avait vécu, et de la Chausseraye, qui mourut
longtemps après.

Il n'est pas étrange que M. le duc d'Orléans ne m'ait jamais parlé de
cette terrible aventure, tenu d'aussi court qu'il l'était alors par
l'abbé Dubois qui le détournait avec empire de tous ceux de sa
confiance, et de moi plus que de pas un, parce que la sienne pour moi
était plus entière, plus fondée, plus de tous les temps, surtout qu'il
l'empêchât de s'ouvrir à moi sur une matière dont il s'était rendu seul
maître, et sur laquelle ma haine pour le duc du Maine et pour le premier
président, qui aurait pu augmenter ma force et ma liberté ordinaire de
parler à M. le duc d'Orléans, aurait fait courir à Dubois le risque de
se voir forcer la main, par conséquent celui de sa ruine, par la
manifestation de tout ce qu'il avait caché au régent, et que les
dépositions du premier président et de bien d'autres nécessairement
arrêtés sur les siennes, auraient mis au net et au grand jour\,; mais ce
qui est, on ne sait si plus inconcevable ou plus déplorable, peu de mois
passèrent si bien non pas l'éponge, mais effacèrent si bien les pointes
de l'impression de cette affaire dans M. le duc d'Orléans, qu'il se
servit depuis du premier président, qui le trompa encore, et qu'après en
avoir été servi de la sorte, et conduit par là à la nécessité de faire
l'éclat d'envoyer le parlement à Pontoise, moins de quatre mois après,
le premier président eut le front, et assez de mépris pour soi-même et
pour le régent, pour oser lui demander de l'argent, et en quantité, en
dédommagement de ce qu'il lui en avait coûté à Pontoise à tenir table
ouverte à tout le parlement, à s'y moquer de lui avec cette compagnie de
la manière la plus indécente, et la moins mesurée, comme on le verra en
son lieu, et que l'extrême merveille est qu'il en obtint plus de quatre
cent mille francs à la vérité en cachette, mais non pas telle, que je ne
l'aie su dès lors et bien d'autres gens avec moi. Voilà de ces prodiges
que je comprends qu'on a bien de la peine à croire, quand on ne les a
pas vus, et pour ainsi dire quand on ne les a pas touchés avec la main,
et qui caractérisent le régent d'une façon bien étrange.

La duchesse de Villars fut nommée pour conduire Mille de Valois, avec
deux\footnote{Le manuscrit ne mentionne, en cet endroit, que deux
  dames\,; mais plus loin on voit qu'il y en avait trois\,:
  M\textsuperscript{me}s de Simiane, de Goyon et de Bacqueville.} dames
de qualité qui furent M\textsuperscript{me}s de Simiane, de Goyon et de
Bacqueville dont on parlera après.

M\textsuperscript{me} de Villars, qui voyait tous les jours contester
les choses les plus établies et les plus certaines, ne voulut pas
s'exposer à aucune difficulté et fit décider jusqu'à ce qui n'avait pas
besoin de l'être\,: il le fut donc qu'elle aurait partout le même
traitement que Mille de Valois, à la main près, c'est-à-dire un
fauteuil, un cadenas à table, une soucoupe, un verre couvert, les
cuillers, fourchette et couteau de vermeil, les assiettes de même, le
tout pareil à ceux de la princesse. M\textsuperscript{lle} de Valois en
avait, et le même genre de domestiques qu'elle pour la servir à table,
et rien de tout cela pour aucune des dames de qualité qui mangeaient
avec M\textsuperscript{lle} de Valois et la duchesse de Villars\,; ces
distinctions déplurent à ces dames\,; mais ne les pouvant empêcher,
elles firent en sorte que M\textsuperscript{lle} de Valois, qui
s'arrêtait partout et allongeait tant qu'elle put son voyage jusqu'à un
excès dont on se plaignit de Modène à M. le duc d'Orléans, se mit
souvent à manger seule en public. La duchesse de Villars sentit
l'affectation, mais ne voulut pourtant pas prendre le cadenas et les
autres distinctions en mangeant avec les dames, lorsque
M\textsuperscript{lle} de Valois mangeait seule, quoique les duchesses
les eussent toujours prises dans la vie ordinaire et commune jusque vers
le milieu du règne du feu roi\,; elle se contenta donc de rendre compte
de l'affectation de manger souvent seule en public, sur quoi
M\textsuperscript{lle} de Valois reçut un ordre de M. son père de manger
toujours avec la duchesse de Villars et les dames, ce qui fut toujours
exécuté depuis je dis ceci d'avance, pour n'avoir plus à y revenir,
ainsi que tout ce qui regarde ce mariage.

Les fiançailles se firent à l'ordinaire dans le cabinet du roi, sur les
six heures du soir, le dimanche Il février, par le cardinal de Rohan\,;
la queue de M\textsuperscript{lle} de Valois portée par
M\textsuperscript{lle} de Montpensier sa soeur, depuis reine
d'Espagne\,; M. le duc de Chartres chargé de la procuration du prince de
Modène. Il ne se trouva personne ou comme personne de la cour aux
fiançailles, parce que rien n'est pareil aux fantaisies, aux hauts et
aux bas des François. Il est très certain que les princes et les
princesses du sang ont toujours prié à leurs fiançailles\,; il ne l'est
pas moins que les fils de France n'ont jamais prié aux fiançailles de
leurs enfants. M. le duc d'Orléans était le premier petit-fils de France
qui eût à marier ses enfants. M\textsuperscript{me} la duchesse de Berry
épousant un fils de France n'était pas dans le cas\,; il ne se
présentait qu'ici pour la première fois, et M. le duc d'Orléans,
supérieur en rang aux princes du sang, et régent, ne songea pas à faire
prier personne, de manière que les fiançailles se firent fort
solitairement, et cette foule qui l'environnait, hommes et femmes et de
toutes qualités, jusqu'aux plus grands qui lui prostituaient toutes
sortes de bassesses pour en obtenir et souvent en arracher des grâces,
se tint chacun chez soi comme de concert pour n'avoir pas été conviée.
M\textsuperscript{me} la duchesse d'Orléans le sentit et le régent s'en
moqua. Le roi donna à M\textsuperscript{lle} de Valois un beau collier
de diamants et de perles, et, une heure après les fiançailles, alla lui
dire adieu au Palais-Royal, et voir Madame et M. {[}le duc{]} et
M\textsuperscript{me} la duchesse d'Orléans. Le lendemain à midi le
mariage fut célébré à la messe du roi avec la même assistance que la
veille, et non plus. Au sortir de la messe le roi donna la main à la
mariée et la conduisit à son carrosse, qui était au roi, et dit au
cocher\,: «\,A Modène,\,» suivant l'usage. Le cortège était autour comme
si elle fût partie en effet\,; elle retourna au Palais-Royal, y eut
quelque temps après la rougeole, ne reçut ni devant ni après aucunes
visites de cérémonie, différa tant qu'elle put, partit enfin, abrégea
toutes ses journées, augmenta les séjours et les allongea. Elle reçut
divers avis de M. le duc d'Orléans sur cette conduite qui n'eurent pas
grand effet, jusqu'à ce que, sur les plaintes réitérées du duc de
Modène, le régent envoya des ordres si absolus qu'ils firent doubler le
pas. Elle s'embarqua à Antibes où la duchesse de Villars et les dames
prirent congé d'elle et prirent le chemin du retour.

M\textsuperscript{me} de Simiane, fille du comte de Grignan, chevalier
de l'ordre, et de la fille de M\textsuperscript{me} de Sévigné, si
connue par son esprit et par ses lettres, et veuve de M. de Simiane,
premier gentilhomme de la chambre de M. le duc d'Orléans et lieutenant
général de Provence, après son beau-père, demeura en Provence et n'en
revint plus. M\textsuperscript{me} Goyon était fille de
M\textsuperscript{me} Desbordes, qui avait passé sa vie sous-gouvernante
des enfants et des petits-enfants de Monsieur, quoique femme d'un
huissier de la chambre, mais elle avait un vrai mérite, et quoique le
mari de sa fille ne fût qu'écuyer de la grande écurie, il ne laissait
pas d'être homme de qualité, et de même nom que MM. de Matignon.
D'ailleurs elle avait été élevée auprès des filles de M. le duc
d'Orléans, qui l'aimaient toutes beaucoup. Pour M\textsuperscript{me} de
Bacqueville, il n'y eut personne qui n'en fût scandalisé. À la vérité,
elle était fille de M. de Châtillon, chevalier de l'ordre, premier
gentilhomme de la chambre de Monsieur, etc., mais comme elle n'avait
rien, on l'avait mariée à ce Bacqueville qui était riche, mais le néant.
Son nom est Boyvin. Son père, qui s'appelait Bonnetot, était premier
président de la chambre des comptes de Rouen, d'une avarice sordide,
dont le père était un fermier laboureur en son jeune temps, qui s'était
enrichi au commerce des blés. Ce Bacqueville voulut être homme d'épée\,;
son mariage lui valut un régiment. Il y montra de la valeur, mais tant
d'avarice et de folies qu'il fut cassé. Il se brouilla bientôt avec sa
femme à qui il ne donnait rien, et qu'il accablait d'extravagances\,;
qui les fit séparer. Il n'en a pas moins fait depuis dans l'obscurité où
il est tombé. Sa soeur avait épousé Aligre, président à mortier, dont
elle a été la seconde femme. Je ne sais ce qu'on donna à ces dames pour
leur voyage. La duchesse de Villars eut cent mille francs. Son choix fut
une nouveauté\,; jamais duchesse n'avait conduit de princesse du sang.
Cet honneur jusqu'alors avait été réservé aux filles de France et aux
petites-filles de France depuis qu'il y en eut\,; mais c'était la fille
du régent qui venait de faire duc et pair le beau-père de la duchesse de
Villars et son mari par conséquent, dont on a vu l'histoire ici en son
lieu, et le duc de Brancas, presque tous les soirs des soupers de M. le
duc d'Orléans, et familièrement bien avec lui de toute sa vie.
M\textsuperscript{me} la grande-duchesse {[}de Toscane{]} embrassant la
princesse de Modène pour lui dire adieu\,: «\,Allez, mon enfant, lui
dit-elle, et souvenez-vous de faire comme j'ai fait\,; ayez un enfant ou
deux, et faites si bien que vous reveniez en France\,; il n'y a de bon
parti que celui-là.\,» Leçon étrange, mais dont la princesse de Modène
ne sut que trop bien profiter.

Le système de Law tirait à sa fin. Si on se fût contenté de sa banque,
et de sa banque réduite en de justes bornes et sages, on aurait doublé
tout l'argent du royaume et porté une facilité infinie à son commerce et
à celui des particuliers entre eux, parce que, la banque toujours en
état de faire face partout, des billets continuellement payables de
toute leur valeur auraient été de l'argent comptant et souvent
préférables à l'argent comptant par la facilité du transport. Encore
faut-il convenir, comme je le soutins à M. le duc d'Orléans dans son
cabinet, et comme je le dis hardiment en plein conseil de régence, quand
la banque y passa, comme on l'a vu ici alors, que, tout bon que pût être
cet établissement en soi, il ne pouvait l'être que dans une république,
ou que dans une monarchie telle qu'est l'Angleterre, dont les finances
se gouvernent absolument par ceux-là seuls qui les fournissent et qui
n'en fournissent qu'autant et que comme il leur plaît\,; mais dans un
État léger, changeant, plus qu'absolu, tel qu'est la France, la solidité
y manquait nécessairement, par conséquent la confiance au moins juste et
sage, puisqu'un roi, et sous son nom une maîtresse, un ministre, des
favoris, plus encore d'extrêmes nécessités, comme celles où le feu roi
se trouva dans les années 1707, 1708, 1709 et 1710, cent choses enfin
pouvaient renverser la banque, dont l'appât était trop grand et en même
temps trop facile. Mais d'ajouter comme on fit au réel de cette banque
la chimère du Mississipi, de ses actions, de sa langue toute
particulière, de sa science, c'est-à-dire un tour de passe-passe
continuel pour tirer l'argent des uns et le donner aux autres, il
fallait bien, puisqu'on n'avait ni mines ni pierre philosophale, que ces
actions, à la fin, portassent à faux, et que le petit nombre se trouvât
enrichi de la ruine entière du grand nombre comme il arriva. Ce qui hâta
la culbute de la banque et du système fut l'inconcevable prodigalité de
M. le duc d'Orléans qui, sans bornes et plus s'il se peut sans choix, ne
pouvait résister à l'importunité jusque de ceux qu'il savait à n'en
pouvoir douter lui avoir toujours été, lui être encore les plus
contraires, et en même temps fort à mépriser, donnait à toutes mains,
plus souvent se laissait arracher par des gens qui s'en moquaient et
n'en savaient gré qu'à leur effronterie. On a peine à croire ce qu'on a
vu, et la postérité considérera comme une fable ce que nous-mêmes nous
ne nous remettons que comme un songe. Enfin, tant fut donné à une nation
avide et prodigue, toujours désireuse et nécessiteuse par son luxe, son
désordre, la confusion des états, que le papier manqua et que les
moulins n'en purent assez fournir. On peut juger par là de
l'inimaginable abus de ce qui était établi comme une ressource toujours
prête, et qui ne pouvait subsister telle qu'en ajustant ensemble les
deux bouts et de préférence à tout, se conservant toujours de quoi
répondre sur-le-champ à tous venants. C'est ce dont je m'informais à Law
tous les mardis matin qu'il venait toujours chez moi\,; il m'amusa
longtemps avant de m'avouer son embarras, et de se plaindre modestement
et timidement à moi que le régent jetait tout par les fenêtres. J'en
savais par le dehors plus qu'il ne pensait, et c'était ce qui me faisait
insister et le presser sur son bilan. En m'avouant enfin, quoique
légèrement, ce qu'il ne pouvait plus me cacher, il m'assurait qu'il ne
manquait pas de ressources, pourvu que M. le duc d'Orléans le laissât
faire. Cela ne me persuada pas. Alors les billets commencèrent à perdre,
un moment après à se décrier, et le décri à devenir public. De là,
nécessité de les soutenir par la force, puisqu'on ne le pouvait plus par
industrie, et, dès que la force se fut montrée, chacun désespéra de son
salut. On vint à vouloir d'autorité coactive, à supprimer tout usage
d'or, d'argent et de pierreries, je dis d'argent monnayé, à prétendre
persuader que depuis Abraham, qui paya argent complant la sépulture de
Sara, jusqu'à nos temps, on avait été dans l'illusion et dans l'erreur
la plus grossière dans toutes les nations policées du monde, sur la
monnaie et les métaux dont on la fait\,; que le papier était le seul
utile et le seul nécessaire\,; qu'on ne pouvait faire un plus grand mal
à nos voisins, jaloux de notre grandeur et de nos avantages, que de
verser et faire passer chez eux tout notre argent et toutes nos
pierreries\,; mais comme à ceci il n'y avait point d'enveloppe, et qu'il
fut permis à la compagnie de Indes de faire visiter dans toutes les
maisons, même royales, d'y confisquer tous les louis d'or et tous les
écus qui s'y trouveraient, et de n'y laisser que des pièces de vingt
sous et au-dessous, et encore jusqu'à deux cents francs pour les
appoints des billets et pour acheter le nécessaire des moindres denrées,
avec défenses et de fortes punitions d'en garder davantage, en sorte
qu'il fallut porter tout ce qu'on avait à la banque de peur d'être
décelé par un valet, personne ne se laissa persuader, et de là recours à
l'autorité de plus en plus, qui ouvrit toutes les maisons des
particuliers aux visites et aux délations pour n'y laisser aucun argent,
et pour punir très sévèrement quiconque en réserverait de caché. Jamais
souveraine puissance ne s'était si violemment essayée et n'avait attaqué
rien de si sensible ni de si indispensablement nécessaire pour le
temporel. Aussi fut-ce un prodige plutôt qu'un effort de gouvernement et
de conduite, que des ordonnances si terriblement nouvelles n'aient pas
produit non seulement les révolutions les plus tristes et les plus
entières, mais qu'il n'en ait pas seulement été question, et que, de
tant de millions de gens, ou absolument ruinés ou mourant de faim et des
derniers besoins auprès de leur bien, et sans moyens aucuns pour leur
subsistance et leur vie journalière, il ne soit sorti que des plaintes
et des gémissements. La violence toutefois était trop excessive et en
tous genres trop insoutenable pour pouvoir subsister longtemps, il en
fallut donc revenir à de nouveaux papiers et à de nouveaux tours de
passe-passe\,; on les connut tels, on les sentit, mais on les subit
plutôt que de n'avoir pas vingt écus en sûreté chez soi, et une violence
plus grande en fit souffrir volontiers une moindre. De là tant de
manèges, tant de faces différentes en finance, et toutes tendantes à
fondre un genre de papier par un autre, c'est-à-dire faire toujours
perdre les porteurs de ces différents papiers, et ces porteurs l'étaient
par force, et la multitude universelle. C'est ce qui en finance occupa
tout le reste du gouvernement et de la vie de M. le duc d'Orléans, ce
qui chassa Law du royaume, ce qui sextupla toute marchandise, toute
denrée, jusqu'aux plus viles, ce qui fit une augmentation ruineuse de
toute espèce de salaire, ce qui ruina le commerce général et le
particulier, ce qui fit, aux dépens du public, la subite richesse de
quelques seigneurs qui les dissipèrent, et n'en furent que plus pauvres,
en fort peu de temps, et ce qui fit les énormes fortunes de toute espèce
d'employés en divers degrés en cette confusion, et qui valut des
millions à une multitude de gens de la plus basse lie du peuple, du
métier de traitants et de commis ou employés de financiers, qui surent
profiter promptement et habilement du Mississipi et de ses suites\,;
c'est ce qui occupa encore le gouvernement plusieurs années après la
mort de M. le duc d'Orléans\,; c'est enfin ce dont la France ne se
relèvera jamais, quoiqu'il soit vrai que les terres en soient
considérablement augmentées. Pour dernière plaie les gens tout
puissants, princes et princesses du sang surtout, qui ne s'étaient fait
faute du Mississipi, et qui ont mis toute leur autorité à s'en sauver
sans rien perdre, l'ont rétabli sur ce qu'ils ont appelé la compagnie
d'Occident qui, avec les mêmes tours de passe-passe particuliers, et un
commerce exclusif aux Indes, achève d'anéantir celui du royaume,
sacrifié à l'énorme intérêt d'un petit nombre de particuliers dont le
gouvernement n'a osé s'attirer la haine et la vengeance en attaquant un
article si délicat.

Il se fit cependant plusieurs exécutions violentes et des confiscations
de sommes considérables trouvées dans les maisons visitées. Un nommé
Adine, employé à la banque, en fut pour dix mille écus confisqués, dix
mille francs d'amende, et son emploi ôté. Beaucoup de gens cachèrent
leur argent avec tant de secret, qu'étant morts sans avoir pu dire où
ils l'avaient mis, ces petits trésors sont demeurés enfouis et perdus
pour les héritiers. On ôta les emplois qu'on avait donnés aux quatre
frères Pâris depuis quelque temps, et on les éloigna de Paris,
soupçonnés de cabaler contre Law parmi les gens de finance. Ils étaient
fils d'un hôtelier qui tenait un cabaret au pied des Alpes, qui était
seul et sans village ni hameau, dont l'enseigne était \emph{à la
Montagne\,;} ses fils lui servaient, et aux passants, de garçons de
cabaret, pansaient leurs chevaux et servaient dans les chambres, tous
quatre fort grands et bien faits\,; l'un d'eux se fit soldat aux gardes,
et l'a été assez longtemps\,: une aventure singulière les fit connaître.
Bouchu intendant de Grenoble, dont il a été parlé ici quelquefois, était
aussi intendant de l'armée d'Italie, lorsque, après la capture du
maréchal de Villeroy à Crémone, le duc de Vendôme lui succéda dans le
commandement de l'armée. Bouchu, quoique âgé et fort goutteux, mais qui
avait été beau et bien fait, n'avait pas perdu le goût de la
galanterie\,; il se trouva que le principal commis des munitionnaires
chargé de tout ce détail, et de faire tout passer à l'armée, était
galant aussi, et qu'il eut la hardiesse de s'adresser à celle que M.
l'intendant aimait, et qu'il lui coupa l'herbe sous le pied, parce qu'il
était plus jeune et plus aimable. Bouchu, outré contre lui, résolut de
s'en venger, et, pour cela, retarda tant et si bien le transport de
toutes choses par toutes les remises et toutes les difficultés qu'il fit
naître, quelque chose que pût dire et faire ce commis pour le presser,
que le duc de Vendôme ne trouva rien en arrivant à l'armée, ou plutôt
dès qu'il la voulut mouvoir. Le commis, qui se vit perdu et qui ne douta
point de la cause, courut le long des Alpes chercher quelques moyens de
faire passer ce qu'il pourrait en attendant le reste. Heureusement pour
lui et pour l'armée, il passa à ce cabaret esseulé \emph{de la
Montagne}, et s'informa là comme il faisait partout. Le maître hôtelier
lui parut {[}avoir{]} de l'esprit, et lui fit espérer qu'au retour de
ses fils qui étaient aux champs, ils pourraient lui trouver quelque
passage. Vers la fin du jour, ils revinrent à la maison. Conseil tenu,
le commis leur trouva de l'intelligence et des ressources, tellement
qu'il se livra à eux, et eux se chargèrent du transport qu'il désirait.
Il manda son convoi de mulets au plus vite, et il passa avec eux
conduits par les frères Pâris, qui prirent des chemins qu'eux seuls et
leurs voisins connaissaient, à la vérité fort difficiles, mais courts,
en sorte que sans perdre une seule charge le convoi joignit M. de
Vendôme arrêté tout court faute de pain, et qui jurait et pestait
étrangement contre les munitionnaires, sur qui Bouchu avait rejeté toute
la faute. Après les premiers emportements, le duc de Vendôme, ravi
d'avoir des vivres et de pouvoir marcher et exécuter ce qu'il avait
projeté, se trouva plus traitable. Il voulut bien écouter ce commis, qui
lui fit valoir sa vigilance, son industrie et sa diligence à traverser
des lieux inconnus et affreux, et qui lui prouva par plusieurs réponses
de M. Bouchu, qu'il avait gardées et portées, combien il l'avait pressé
de faire passer les munitions et les farines à temps\,; que c'était la
faute unique de l'intendant à cet égard qui avait mis l'armée dans la
détresse où elle s'était trouvée\,; et fit en même temps confidence au
général de la haine de Bouchu, jusqu'à hasarder l'armée pour le perdre,
et la cause ridicule de cette haine\,; en même temps se loua beaucoup de
l'intelligence et de la volonté de l'hôtelier et de ses fils, auxquels
il devait l'invention et le bonheur du passage de son convoi. Le duc de
Vendôme alors tourna toute sa colère contre Bouchu, l'envoya chercher,
lui reprocha devant tout le monde ce qu'il venait d'apprendre, conclut
par lui dire qu'il ne savait à quoi il tenait qu'il ne le fît pendre
pour avoir joué à perdre l'armée du roi. Ce fut le commencement de la
disgrâce de Bouchu, qui ne se soutint plus qu'à force de bassesses, et
qui au bout de deux ans se vit forcé de se retirer\,; ce fut aussi le
premier commencement de la fortune de ces frères Pâris. Les
munitionnaires en chef les récompensèrent, leur donnèrent de l'emploi,
et, par la façon dont ils s'en acquittèrent, les avancèrent promptement,
leur donnèrent leur confiance, et leur valurent de gros profits\,; enfin
ils devinrent munitionnaires eux-mêmes, s'enrichirent, vinrent à Paris
chercher une plus grande fortune, et l'y trouvèrent. Elle devint telle
dans les suites, qu'ils gouvernèrent en plein et à découvert sous M. le
Duc, et qu'après de courtes éclipses, ils sont redevenus les maîtres des
finances et des contrôleurs généraux, et ont acquis des biens immenses,
fait et défait des ministres et d'autres fortunes, et ont vu la cour à
leurs pieds, la ville et les provinces.

Le roi vint pour la première fois au conseil de régence, le dimanche 18
février. Il ne dit rien en y entrant ni pendant le conseil, ni en
sortant, sinon que M. le duc d'Orléans, lui ayant proposé d'en sortir,
de peur qu'il ne s'y ennuyât, il voulut y demeurer jusqu'à la fin.
Depuis il ne vint pas à tous, mais assez souvent, toujours jusqu'au
bout, et sans remuer ni parler. Sa présence ne changea rien à la séance,
parce que son fauteuil y était toujours seul au bout de la table, et que
M. le duc d'Orléans, le roi présent ou non, n'avait qu'un tabouret
pareil à ceux de tout ce qui y assistait. Le maréchal de Villeroy ne
changea point sa séance accoutumée. Peu de jours après le duc de Berwick
y entra aussi\,; on en murmura dans le monde, parce qu'il était
étranger\,; mais cet étranger se trouvait nécessairement proscrit,
expatrié, naturalisé François, en France depuis trente-deux ans, dans un
continuel service, duc, pair, maréchal de France, grand d'Espagne,
général des armées des deux couronnes, et une fidélité plus
qu'éprouvée\,; de plus, pour ce qui se passait alors au conseil de
régence, n'importait plus qui en fût\,; nous étions déjà quinze, il fit
le seizième. Une fois que le roi y vint, un petit chat qu'il avait le
suivit, et quelque temps après, sauta sur lui, et de là sur la table, où
il se mit à se promener, et aussitôt le duc de Noailles à crier, parce
qu'il craignait les chats. M. le duc d'Orléans se mit aussitôt en peine
pour l'ôter, et moi à sourire, et à lui dire\,: «\,Eh, monsieur, laissez
ce petit chat, il fera le dix-septième\,!» M. le duc d'Orléans se mit à
rire de tout son coeur, et à regarder la compagnie qui en rit, et le roi
aussi, qui m'en parla le lendemain à son petit lever, comme en ayant
senti la plaisanterie, mais en deux mots, ce qui courut Paris aussitôt.

Il y eut beaucoup de nouveaux prisonniers à Nantes, et on supprima
vingt-six présidents ou conseillers du parlement de Bretagne, qu'on
remboursa avec du papier. Ce ne furent point les vingt-six charges des
dernières augmentations\,; ce furent les personnes en jardinant (comme
on dit des coupes de futaies), choisies dans cette compagnie desquelles
on était mécontent. Cela n'y causa pas le plus petit mouvement, la
commission du conseil se rendait redoutable à Nantes, et il y avait des
troupes répandues dans la province.

\hypertarget{chapitre-xx.}{%
\chapter{CHAPITRE XX.}\label{chapitre-xx.}}

1720

~

{\textsc{Abbé Dubois obtient l'archevêché de Cambrai.}} {\textsc{-
L'abbé Dubois refusé d'un dimissoire par le cardinal de Noailles, en
obtient un de Besons, archevêque de Rouen, et va dans un village de son
diocèse, près de Pontoise, recevoir tous les ordres à la fois de
Tressan, évêque de Nantes\,; se compare là-dessus à saint Ambroise.}}
{\textsc{- Mot du duc Mazarin.}} {\textsc{- Singulière anecdote sur le
pouvoir de l'abbé Dubois sur M. le duc d'Orléans, à l'occasion du sacre
de cet abbé.}} {\textsc{- Sacre de l'abbé Dubois par le cardinal de
Rohan.}} {\textsc{- Les Anglais opposés au roi Georges, ou jacobites,
chassés de France à son de trompe.}} {\textsc{- Politique terrible de la
cour de Rome sur le cardinalat.}} {\textsc{- Mort de
M\textsuperscript{me} de Lislebonne.}} {\textsc{- Douze mille livres de
pension, qu'elle avait, {[}sont{]} données à M\textsuperscript{me} de
Remiremont, sa fille.}} {\textsc{- Mort et successeur du grand maître de
Malte.}} {\textsc{- Mort et caractère du P. Cloche, général de l'ordre
de Saint-Dominique.}} {\textsc{- Mort de Fourille\,; sa pension donnée à
sa veuve.}} {\textsc{- Mort et caractère de M\textsuperscript{me} de La
Hoguette.}} {\textsc{- Mort de Mortagne, chevalier d'honneur de
Madame.}} {\textsc{- Mort de M\textsuperscript{me} la Duchesse,
brusquement enterrée.}} {\textsc{- Visites et manteaux chez M. le Duc.}}
{\textsc{- Testament, etc.}}

~

Cambrai vaquait, comme on l'a vu naguère, par la mort à Rome du cardinal
de La Trémoille, c'est-à-dire le plus riche archevêché et un des plus
grands postes de l'Église. L'abbé Dubois n'était que tonsuré\,; cent
cinquante mille livres de rente le tentèrent, et peut-être bien autant
ce degré pour s'élever moins difficilement au cardinalat. Quelque
impudent qu'il fût, quel que fût l'empire qu'il avait pris sur son
maître, il se trouva fort embarrassé et masqua son effronterie de ruse,
il dit à M. le duc d'Orléans qu'il avait fait un plaisant rêve, et lui
conta qu'il avait rêvé qu'il était archevêque de Cambrai. Le régent qui
sentit où cela allait fit la pirouette et ne répondit rien. Dubois, de
plus en plus embarrassé, bégaya et paraphrasa son rêve\,; puis, se
rassurant d'effort, demanda brusquement pourquoi il ne l'obtiendrait
pas, Son Altesse Royale de sa seule volonté pouvant ainsi faire sa
fortune. M. le duc d'Orléans fut indigné, même effrayé, quelque peu
scrupuleux qu'il fût au choix des évêques, et d'un ton de mépris, lui
répondit\,: «\,Qui\,! toi, archevêque de Cambrai\,!» en lui faisant
sentir sa bassesse et plus encore le débordement et le scandale de sa
vie. Dubois s'était trop avancé pour demeurer en si beau chemin\,; lui
cita des exemples. Malheureusement il n'y en avait que trop, et en
bassesse et en étranges moeurs, grâce comme on l'a vu ailleurs à Godet,
évêque de Chartres, avec ses séminaristes de néant et ignorants dont il
remplit les évêchés, au P. Tellier et à la constitution, pour bassesse,
ignorance, et mauvaises moeurs tout à la fois, et à ceux qui l'ont
suivi.

M. le duc d'Orléans, moins touché de raisons si mauvaises qu'embarrassé
de résister à l'ardeur de la poursuite d'un homme qu'il n'avait plus
accoutumé d'oser contredire sur rien, chercha à se tirer d'affaire, et
lui dit\,: «\,Mais tu es un sacre, et qui est l'autre sacre qui voudra
te sacrer\,? --- Ah\,! s'il ne tient qu'à cela, reprit vivement l'abbé,
l'affaire est faite\,; je sais bien qui me sacrera, il n'est pas loin
d'ici. --- Et qui diable est celui-là, répondit le régent, qui osera te
sacrer\,? --- Voulez-vous le savoir\,? répliqua l'abbé\,; et ne tient-il
qu'à cela encore une fois\,? --- Eh bien\,! qui\,? dit le régent. ---
Votre premier aumônier, reprit Dubois, qui est là dehors\,; il ne
demandera pas mieux\,; je m'en vais le lui dire\,;» embrasse les jambes
de M. le duc d'Orléans, qui demeure court et pris sans avoir la force du
refus, sort, tire l'évêque de Nantes à part, lui dit qu'il a Cambrai, le
prie de le sacrer, qui le lui promet à l'instant\,; rentre, caracole,
dit à M. le duc d'Orléans qu'il vient de parler à son premier aumônier,
qui lui a promis de le sacrer, remercie, loue, admire, scelle de plus en
plus son affaire, en la comptant faite et en persuadant le régent qui
n'osa jamais dire que non c'est de la sorte que Dubois se fit archevêque
de Cambrai.

L'extrême scandale de cette nomination fit un étrange bruit. Tout
impudent que fût Dubois, il en fut extrêmement embarrassé, et M. le duc
d'Orléans si honteux qu'on remarqua bientôt qu'on lui faisait peine de
lui en parler. Question fut bientôt de prendre les ordres. Dubois se
flatta que, dans la posture où il se trouvait et le besoin que le
cardinal {[}de Noailles{]} avait et aurait continuellement de lui dans
la situation si, pénible où l'affaire de la constitution, menée comme
elle l'était, le mettait, lui ferait faire envers lui toutes les
avances, avec d'autant plus d'empressement que le cardinal avait lieu
d'être fort mal content de lui et de toute la protection qu'il donnait à
ses ennemis, qu'il ménageait de loin pour son cardinalat\,; et que le
cardinal, dans l'espérance de se le ramener, au moins de l'adoucir, s'en
ferait un mérite auprès de M. le duc d'Orléans et de lui, et envers le
public d'un si bon procédé à l'égard d'un homme qui l'avait si peu
mérité de lui. Il se trompa\,; la chair et le sang n'eurent jamais de
part à la conduite du cardinal de Noailles. Les vices d'esprit et de
coeur et les moeurs si publiques de l'abbé Dubois lui étaient connus. Il
eut horreur de contribuer en rien à le faire entrer dans les ordres
sacrés. Il sentit toute la pesanteur du nouveau poids dont son refus
l'allait charger de la part d'un homme devenu tout-puissant sur son
maître, qui sentirait dans toute étendue l'insigne affront qu'il
recevrait, et quelles en seraient les suites pour le reste de leur vie.
Rien ne l'arrêta, il refusa le dimissoire\footnote{On appelait
  \emph{dimissoire} la lettre par laquelle un évêque permettait qu'un de
  ses diocésains fût promu aux ordres ou à l'épiscopat par un autre
  évêque.} pour les ordres avec un air de douleur et de modestie, sans
que rien le pût ébranler, et garda là-dessus un parfait silence, content
d'avoir rempli son devoir, et y voulant mettre tout ce que ce même
devoir y pouvait accorder à la charité, à la simplicité, à la modestie.
On peut juger des fureurs où cet affront fit entrer Dubois, qui de sa
vie ne le pardonna au cardinal de Noailles, lequel en fut
universellement applaudi, et d'autant plus loué et admiré qu'il ne le
voulut point être. Il fallut donc se tourner ailleurs.

Besons, frère du maréchal, tous deux si attachés et si bien traités et
récompensés de M. le duc d'Orléans, tous deux sous leur air rustre,
lourd et grossier, si bons courtisans, avait été transféré de
l'archevêché de Bordeaux à celui de Rouen, et Pontoise est de ce dernier
diocèse, qui touche ainsi celui de Paris, et s'approche de cette ville à
peu de lieues en deçà de Pontoise même. L'abbé Dubois voulait gagner le
temps et s'éviter la honte d'un voyage marqué. Les Besons lui parurent
devoir être de meilleure composition que le cardinal de Noailles\,; ils
en furent en effet. L'archevêque de Rouen donna le dimissoire. Dubois,
sous prétexte des affaires dont il était chargé, obtint un bref pour
recevoir à la fois tous les ordres, et se dispensa lui-même de toute
retraite pour s'y préparer. Il alla donc un matin à quatre ou cinq
lieues de Paris, où dans une église paroissiale du diocèse de Rouen, du
grand vicariat de Pontoise, Tressan, évêque de Nantes, premier aumônier
de M. le duc d'Orléans, donna dans la même messe basse, qu'il célébra
\emph{extra tempora}, le sous-diaconat, le diaconat et la prêtrise à
l'abbé Dubois, et en fut après récompensé de l'archevêché de Rouen et
des économats\footnote{Les personnes chargées des \emph{économats}
  avaient l'administration des revenus d'un évêché, d'une abbaye et en
  un mot de tous les bénéfices pendant la vacance. Le roi nommait à ces
  économats.} à la mort de Besons qui avait l'un et l'autre, et qui ne
le fit pas longtemps attendre. On cria fort contre les deux prélats, et
l'archevêque, qui était estimé et considéré avec raison, y eut à perdre.
Pour l'autre, il n'y fit que gagner.

Le même jour que l'abbé Dubois prit ainsi tous les ordres à la fois, il
y eut conseil de régence l'après-dînée au vieux Louvre, parce que toutes
les rougeoles qui couraient, même dans le Palais-Royal, empêchaient
qu'il se tînt à l'ordinaire aux Tuileries. On fut surpris d'un conseil
de régence sans l'abbé Dubois, qui y rapportait ce qu'il lui plaisait
des affaires étrangères, mais on le fut bien davantage de l'y voir
arriver. Il n'avait pas perdu de temps en actions de grâces de tout ce
qu'il venait de recevoir. Ce fut un nouveau scandale qui réveilla et qui
aggrava le premier. Il venait, à ce que dit plaisamment le duc Mazarin,
de faire sa première communion. Tout le monde était déjà arrivé dans le
cabinet du conseil, et M. le duc d'Orléans aussi, et on y était debout
et épars. J'étais dans un coin du bas bout, qui causais avec M. le
prince de Conti, le maréchal de Tallard et un autre qui m'échappe,
lorsque je vis entrer l'abbé Dubois en habit court, avec son maintien
ordinaire. Nous ne l'attendions point en tel jour, ce qui fit que
naturellement nous nous écriâmes. Cela lui fit tourner la tête, et
voyant M. le prince de Conti venir à lui, qui de son côté, avec ce
ricanement de M. son père, mais qui, assurément était bien éloigné d'en
avoir les grâces, et au contraire était cynique, s'avança deux pas à
lui, lui parla de tous les ordres si brusquement reçus le matin même
tous à la fois, de sa prompte arrivée au conseil si peu de moments après
cette cérémonie, quoique faite au loin de Paris, de son sacre qui allait
suivre de si près, de sa surprise et de celle de tout le monde, et tout
de suite lui fit un pathos avec tout l'esprit et la malignité possible
qui tenait d'un assez plaisant sermon, et qui aurait plus que démonté
tout autre. Dubois, qui n'avait pas eu l'instant de placer une seule
parole, le laissa dire, puis répondit froidement que, s'il était un peu
plus instruit de l'antiquité, il trouverait ce qui l'étonnait fort peu
étrange, puisque lui abbé ne faisait que suivre l'exemple de saint
Ambroise, dont il se mit à raconter l'ordination qu'il étala. Je n'en
entendis pas le récit, car dans le moment que j'ouïs saint Ambroise, je
m'enfuis brusquement à l'autre bout du cabinet, de l'horreur de la
comparaison et de la peur de ne pouvoir m'empêcher de lui dire
d'achever, car je sentais que cela me prenait à la gorge, et de dire
combien peu saint Ambroise se pouvait défier d'être ainsi saisi et
ordonné, quelle résistance il y fit, et avec combien d'éloignement et de
frayeur, enfin toute la violence qui lui fut unanimement faite. Cette
impie citation de saint Ambroise courut bientôt le monde avec l'effet
qu'on peut penser. La nomination et cette ordination se firent dans la
fin de février.

J'achèverai tout de suite ce qui regarde cette matière pour ne la pas
séparer, et n'avoir pas à y revenir. On y trouvera une anecdote curieuse
sur l'autorité de l'abbé Dubois, sur son maître et sur la frayeur et le
danger de lui déplaire. Il eut ses bulles au commencement de mai, et fut
sacré le dimanche 9 juin. Tout Paris et toute la cour y fut conviée. Je
ne le fus point\,; j'étais lors mal avec lui, parce que je ne le
ménageais guère avec M. le duc d'Orléans, sur ses vues du cardinalat et
sur son abandon dans les affaires à ce qui convenait aux Anglais et à
l'empereur, par lesquels il comptait d'arriver à la pourpre romaine.
Comme il redoutait ma liberté, ma franchise, ma façon de parler à M. le
duc d'Orléans qui lui faisait de fréquentes impressions, quoique je m'en
donnasse assez rarement la peine, et qu'il avait celle de les effacer,
il revenait à moi de temps en temps, me ménageait, me courtisait,
toujours pourtant détournant tant qu'il pouvait la confiance de M. le
duc d'Orléans en moi, qu'il resserrait sans cesse, mais qu'il ne pouvait
arrêter totalement ni même longtemps, quoique, comme je l'ai dit, je me
retirasse beaucoup par le dégoût de tout ce que je voyais. Ainsi nous
étions bien en apparence quelquefois, et souvent mal.

Ce sacre devait être magnifique, et M. le duc d'Orléans y devait
assister. J'en dirai quelques mots dans la suite. Plus la nomination et
l'ordination de l'abbé Dubois avait fait de bruit, de scandale et
d'horreur, plus les préparatifs superbes de son sacre les augmentaient,
et plus l'indignation en éclatait contre M. le duc d'Orléans. Je fus
donc le trouver la veille de cet étrange sacre, et d'abordée je lui dis
ce qui m'amenait. Je le fis souvenir que je ne lui avais jamais parlé de
la nomination de l'abbé Dubois à Cambrai, parce qu'il savait bien que je
ne lui parlais jamais des choses faites\,; que je ne lui en parlerais
pas encore, si je n'avais appris qu'il devait aller le lendemain à son
sacre\,; que je me tairais avec lui de la façon dont il se faisait,
telle qu'il ne pourrait mieux, si l'usage était encore de faire des
princes du sang évêques, et qu'il fût question de son second fils, parce
que je regardais cela comme chose déjà faite, mais que mon attachement
pour lui ne me permettait pas de lui cacher l'épouvantable effet que
faisait universellement une nomination de tous points si scandaleuse,
une ordination si sacrilège, des préparatifs de sacre si inouïs pour un
homme de l'extraction, de l'état, des moeurs et de la vie de l'abbé
Dubois, non pour lui reprocher ce qui n'était plus réparable, mais pour
qu'il sût à quel point en était la générale indignation contre lui, et
que de là il conclût ce que ce serait pour lui d'y mettre le comble en
allant lui-même à ce sacre\,; je le conjurai de sentir quel serait ce
contraste avec l'usage, non seulement des fils de France, mais des
princes du sang, de n'aller jamais à aucun sacre, parce que je
n'appelais pas y aller la curiosité d'en voir un une fois en leur vie,
que les rois et les personnes royales avaient eue quelquefois\,;
j'ajoutai qu'à l'opinion que sa vie et ses discours ne donnaient que
trop continuellement de son défaut de toute religion, on ne manquerait
pas de dire, de croire et de répandre qu'il allait à ce sacre pour se
moquer de Dieu et insulter son Église\,; que l'effet de cela était
horrible et toujours fort à craindre, et qu'on y ajouterait avec raison
que l'orgueil de l'abbé Dubois abusait de lui en tout, et que ce trait
public de dépendance, par une démarche si étrangement nouvelle et
déplacée, lui attirerait une haine, un mépris, une honte dont les suites
étaient à redouter\,; que je ne lui en parlais qu'en serviteur
entièrement désintéressé\,; que son absence ou sa présence à ce sacre ne
changerait rien à la fortune de l'abbé Dubois, qui ne serait ni plus ni
moins archevêque de Cambrai, et n'obscurcirait en rien la splendeur
préparée pour ce sacre, telle qu'elle ne pourrait être plus grande, si
on avait un fils de France à sacrer\,; qu'en vérité c'en était bien
assez pour un Dubois, sans prostituer son maître aux yeux de toute la
France, et bientôt après de toute l'Europe, par la bassesse inouïe d'une
démarche, où on verrait bien que l'extrême pouvoir de Dubois sur lui
l'aurait entraîné de force. Je finis par le conjurer de n'y point aller,
et par lui dire qu'il savait en quels termes actuels l'abbé Dubois et
moi étions ensemble\,; que j'étais le seul homme de marque qu'il n'eût
point convié\,; que nonobstant tout cela, s'il me voulait promettre et
rie tenir sa parole de n'aller point à ce sacre, je lui donnais la
mienne d'y aller, moi, et d'y demeurer tout du long, quelque horreur que
j'en eusse et quelque blessé que je fusse de ce que cela ferait sûrement
débiter que ce trait de courtisan était pour me raccommoder avec lui,
moi si éloigné d'une pareille misère et qui osai me vanter, puisqu'il le
fallait aujourd'hui, d'avoir jusqu'à ce moment conservé chèrement toute
ma vie mon pucelage entier sur les bassesses.

Ce propos, vivement prononcé et encore plus librement et plus
énergiquement étendu, fut écouté d'un bout à l'autre. Je fus surpris
qu'il me dit que j'avais raison, que je lui ouvrais les yeux, plus
encore qu'il m'embrassa, me dit que je lui parlais en véritable ami, et
qu'il me donnait sa parole et me la tiendrait de n'y point aller. Nous
nous séparâmes là-dessus, moi le confirmant encore, lui promettant de
nouveau que j'irais, et lui me remerciant de cet effort. Il n'eut nulle
impatience, nulle envie que je m'en allasse, car je le connaissais bien,
et je l'examinais jusqu'au fond de l'âme, et ce fut moi qui le quittai,
bien content de l'avoir détourné d'une si honteuse démarche et si
extraordinaire. Qui n'eût dit qu'il ne m'eût tenu parole\,? car on va
voir qu'il le voulait\,; mais voici ce qui arriva.

Quoique je me crusse bien assuré là-dessus, néanmoins la facilité et
l'extrême faiblesse du prince, et l'empire sur lui et l'orgueil de
l'abbé Dubois, m'engagèrent à prendre le plus sûr avant d'aller au
sacre. J'envoyai aux nouvelles le lendemain matin au Palais-Royal, et
cependant je fis tenir mon carrosse tout prêt pour tenir ma parole. Mais
je fus bien confus, quelque accoutumé que je fusse aux misères de M. le
duc d'Orléans, quand celui que j'avais envoyé voir ce qui se passait
revint et me rapporta qu'il venait de voir M. le duc d'Orléans monter
dans son carrosse et environné de toute la pompe des rares jours de
cérémonie, partir pour aller au sacre. Je fis ôter mes chevaux, et
m'enfonçai dans mon cabinet.

Le surlendemain j'appris par un coucheur favori de M\textsuperscript{me}
de Parabère, qui était lors la régnante, mais qui n'était pas fidèle,
qu'étant couchée la nuit qui précéda le sacre avec M. le duc d'Orléans,
au Palais-Royal, entre deux draps, ce qui n'arrivait guère ainsi dans la
chambre et le lit de M. le duc d'Orléans, mais presque toujours chez
elle, il s'était avisé de lui parler de moi avec éloge, que je ne
rapporterai pas, et avec sentiment sur mon amitié pour lui, et que,
plein de ce que je lui venais de représenter, il n'irait point au sacre,
dont il me savait le meilleur gré du monde. La Parabère me loua, convint
que j'avais raison, mais sa conclusion fut qu'il irait. M. le duc
d'Orléans, surpris, lui dit qu'elle était donc folle. «\,Folle, soit,
répondit-elle, mais vous irez. --- Et moi, reprit-il, je vous dis que je
n'irai pas. --- Si, vous dis-je, dit-elle, et vous irez. --- Mais,
reprit-il, cela est admirable, tu dis que M. de Saint-Simon a raison, et
au bout, pourquoi donc irais-je\,? --- Parce que je le veux, dit-elle.
--- En voici d'une autre, répliqua-t-il, et pourquoi veux-tu que j'y
aille, quelle folie est cela\,? --- Pourquoi, dit-elle, parce que. ---
Oh\,! parce que, répondit-il, parce que, ce n'est pas là parler\,; dis
donc pourquoi si tu peux.\,» Après quelque dispute\,: «\, Voulez-vous
donc absolument le savoir\,? c'est que vous n'ignorez pas que l'abbé
Dubois et moi avons eu, il n'y a pas quatre jours, maille à partir
ensemble, et qui n'est pas encore bien finie. C'est un diable qui
furette tout\,; il saura que nous avons couché ici cette nuit ensemble.
Si demain vous n'allez pas à son sacre, il ne manquera pas de croire que
c'est moi qui vous en ai empêché\,; rien ne le lui pourra ôter de la
tête, il ne me le pardonnera pas\,; il me fera cent tracasseries et cent
noirceurs auprès de vous, et finira promptement par nous brouiller\,;
or, c'est ce que je ne veux pas, et c'est pour cela que je veux que vous
alliez à son sacre, quoique M. de Saint-Simon ait raison.\,» Là-dessus,
débat assez faible, puis résolution et promesse d'aller au sacre, qui
fut bien fidèlement exécutée.

La nuit suivante la Parabère coucha chez elle avec son
greluchon\footnote{Mot familier et libre, dit l'ancien
  \emph{Dictionnaire de l'Académie}. Ildésigne l'amant aimé et favorisé
  secrètement par une femme qui se fait payer par d'autres amants.}, à
qui elle raconta cette histoire tant elle la trouvait plaisante. Par
cette même raison le greluchon la rendit à Biron, qui le soir même me la
conta. Je déplorai avec lui les chaînes du régent, à qui je n'ai jamais
parlé depuis de ce sacre, ni lui à moi\,; mais il fut après bien honteux
et bien embarrassé avec moi. Je n'ai point su s'il poussa la faiblesse
jusqu'à conter à l'abbé Dubois ce que je lui avais dit pour l'empêcher
d'aller à son sacre, ou s'il en fut informé par la Parabère, pour se
faire un mérite auprès de lui d'avoir fait changer M. le duc d'Orléans
là-dessus et faire montre de son crédit. Mais il en fut très
parfaitement informé et ne me l'a jamais pardonné, et j'ai su depuis par
Belle-Ile qu'il avait dit à M. Le Blanc et à lui que, de toutes les
contradictions que je lui avais fait essuyer, même du danger pressant où
je l'avais mis quelquefois, rien ne l'avait si profondément touché et
blessé, et jusqu'au fond de l'âme, que d'avoir voulu empêcher M. le duc
d'Orléans d'assister à son sacre, duquel il est maintenant temps de
parler.

Tout y parut également superbe et choisi pour faire éclater la faveur
démesurée d'un ministre éperdu d'orgueil et d'ambition sans bornes, la
servitude la plus publique et la plus démesurée où il avait réduit son
maître, et l'audace effrénée de s'en parer en la manifestant aux yeux de
toute la France avec le plus grand éclat, et de là ceux de toute
l'Europe, à qui il voulait apprendre de la manière la plus éclatante que
lui était entièrement le maître de la France, soit pour le dedans, soit
pour le dehors, sous un nom qui n'était qu'une vaine écorce, et qu'à lui
seul il fallait s'adresser pour quelque grâce et pour quelque affaire
que ce fût, comme à l'unique dispensateur et au seul véritable arbitre
de toutes choses en France.

Le Val-de-Grâce fut choisi pour y faire le sacre comme étant un
monastère royal, le plus magnifique de Paris et l'église la plus
singulière. Le cardinal de Rohan, ravi de faire contre en tout au
cardinal de Noailles et de profiter du refus qu'il avait fait à l'abbé
Dubois de lui permettre d'être ordonné dans son diocèse, saisit un si
précieux moment de faire bien sa cour au régent et de s'attacher son
ministre, en s'empressant pour faire la cérémonie. En effet un cardinal
de sa naissance, évêque de Strasbourg, et brillant de toutes sortes
d'avantages, était un consécrateur fort au-dessus de tous ceux que
l'abbé Dubois aurait pu désirer. Il n'y a guère en fait d'honneur que la
première démarche de chère\,; Rohan avait franchi le saut quand, à la
persuasion intéressée du maréchal de Tallard, comme on l'a vu ici en son
lieu, il subit la loi que lui fit le P. Tellier, pour le faire grand
aumônier, et se livra, contre le cardinal de Noailles, ses propres
lumières et la vérité à lui parfaitement connue et reconnue, à toutes
les scélératesses et à toutes les violences dont ce terrible jésuite le
rendit son ministre, et que l'intérêt et l'orgueil d'être chef de parti
et de n'en abandonner pas l'honneur et le profit au cardinal de Bissy,
lui fit continuer depuis en premier. Avec le revêtement constant d'un
tel personnage, il ne fallait pas s'attendre qu'aucune considération de
honte ni d'infamie retînt le cardinal de Rohan d'une si étrange
prostitution, moins encore que sa conscience l'arrêtait un moment sur le
sacrilège dont il allait se rendre le ministre. L'abbé Dubois fut donc
comblé de l'honneur qu'il lui voulut bien faire\,; M. le duc d'Orléans
témoigna au cardinal toute la part qu'il y prenait, et Rohan, charmé des
espérances qu'il conçut de ce grand trait de politique, plus sensibles
pour sa maison que pour sa cause, laquelle ne fut jamais que pour servir
aux avantages de l'autre, se rit de tous les discours, du bruit de
l'improbation générale et nullement retenue que cette fonction excita,
et qu'il ne regarda que comme des raisons de plus et des fondements
d'augmentation à ses espérances pour tout ce qu'il pouvait désirer d'un
homme tout-puissant, pour l'amour duquel il {[}se{]} livrait à tant
d'opprobres.

À l'égard des deux évêques assistants, Nantes y avait un tel droit par
l'ordination qu'il avait osé donner à l'abbé Dubois, qu'il n'y avait pas
moyen de lui préférer personne. Pour l'autre assistant, Dubois crut en
devoir chercher un dont la vie et la conduite pût être en contre-poids.
Il voulut Massillon, célèbre prêtre de l'Oratoire, que sa vertu, son
savoir, ses grands talents pour la chaire, avaient fait évêque de
Clermont, parce qu'il en passait quelquefois, quoique rarement, quelque
bon parmi le grand nombre des autres qu'on faisait évêques. Massillon au
pied du mur, étourdi, sans ressources étrangères, sentit l'indignité de
ce qui lui était proposé, balbutia, n'osa refuser. Mais qu'eût pu faire
un homme aussi mince, selon le siècle, vis-à-vis d'un régent, de son
ministre et du cardinal de Rohan\,? Il fut blâmé néanmoins et beaucoup
dans le monde, surtout des gens de bien de tout parti, car en ce point
l'excès du scandale les avait réunis. Les plus raisonnables, qui ne
laissèrent pas de se trouver en nombre, se contentèrent de le plaindre,
et on convint enfin assez généralement d'une sorte d'impossibilité de
s'en dispenser et de refuser.

L'église fut superbement parée, toute la France invitée\,; personne
n'osa hasarder de ne s'y pas montrer, et tout ce qui le put pendant
toute la cérémonie. Il y eut des tribunes à jalousies préparées pour les
ambassadeurs et autres ministres protestants. Il y en eut une autre plus
magnifique pour M. le duc d'Orléans et M. le duc de Chartres qu'il y
mena. Il y en eut pour les dames, et comme M. le duc d'Orléans entra par
le monastère, et que sa tribune se trouva au dedans, il fut ouvert à
tous venants, tellement que le dehors et le dedans fut rempli de
rafraîchissements de toutes les sortes et d'officiers qui les faisaient
et distribuaient avec profusion. Ce désordre continua tout le reste du
jour par le grand nombre de tables qui furent servies dehors et dedans
pour tout le subalterne de la fête et pour tout ce qui s'y voulut
fourrer. Les premiers gentilshommes de la chambre de M. le duc d'Orléans
et ses premiers officiers firent les honneurs de la cérémonie, placèrent
les gens distingués, les reçurent, les conduisirent, et d'autres de ses
officiers prirent les mêmes soins à l'égard des gens moins
considérables, tandis que tout le guet et toute la police était occupée
à faire aborder, ranger, sortir les carrosses sans nombre avec tout
l'ordre et la commodité possible. Pendant le sacre qui fut peu décent de
la part du consacré et des spectateurs, surtout en sortant de la
cérémonie, M. le duc d'Orléans témoigna sa satisfaction à ce qu'il
trouva sous sa main de gens considérables de la peine qu'ils avaient
prise, et s'en alla dîner à Asnières avec M\textsuperscript{me} de
Parabère, bien contente de l'avoir fait aller au sacre qu'il vit, et à
ce qu'on lui imposa\footnote{Le verbe \emph{imposer} est ici pris dans
  le sens d'\emph{imputer}.} peut-être trop véritablement, qu'il vit,
dis-je, peu décemment depuis le commencement jusqu'à la fin. Tous les
prélats, les abbés distingués, et quantité de laïques considérables
furent invités pendant la cérémonie par les premiers officiers de M. le
duc d'Orléans à dîner au Palais-Royal. Les mêmes firent les honneurs du
festin qui fut servi avec la plus splendide abondance et délicatesse, et
apprêté et servi par les officiers de M. le duc d'Orléans et à ses
dépens. Il eut deux tables de trente couverts chacune dans une grande
pièce du grand appartement, qui furent remplies de ce qu'il y avait de
plus considérable à Paris, et plusieurs autres tables également bien
servies en d'autres pièces voisines pour des gens moins distingués. M.
le duc d'Orléans donna au nouvel archevêque un diamant de grand prix
pour lui servir d'anneau. Toute cette journée fut livrée à cette sorte
de triomphe qui n'attira pas l'approbation des hommes ni la bénédiction
de Dieu. Je n'en vis pas la moindre chose, et jamais M. le duc d'Orléans
et moi ne nous en sommes parlé.

Dans le même temps que Dubois fut nommé à l'archevêché de Cambrai, on
publia à son de trompe une ordonnance pour faire sortir en huit jours de
toutes les terres de l'obéissance du roi tous les étrangers rebelles,
qui, en conséquence, furent recherchés et punis avec la dernière
rigueur. Ces étrangers rebelles n'étaient autres que des Anglais, et ce
fut un des effets du voyage à Paris du comte Stanhope\,; ce ne fut que
l'exécution jusqu'alors tacitement suspendue d'une clause infâme du
traité fait par Dubois avec l'Angleterre qui y gagnait tout, et la
France rien, rien que la plus dangereuse ignominie. Les Français, depuis
la révocation de l'édit de Nantes réfugiés en Angleterre, ne pouvaient
donner la plus légère inquiétude en France, où personne n'avait droit à
la couronne que celui qui la portait, et sa maison d'aîné mâle en aîné,
et le réciproque stipulé par ce même traité ne pouvait avoir
d'application aux François, dont pas un n'était rebelle, ni opposé à la
maison régnante. Ce réciproque n'était donc qu'un voile, ou plutôt une
toile d'araignée pour faire passer, non l'intérêt des Anglais, mais
celui du roi d'Angleterre et de ses ministres qui craignaient jusqu'à
l'ombre du véritable et légitime roi, bien que confiné à Rome, et des
Anglais de son parti, ou qui par mécontentement favorisaient ce parti
sans se soucier du parti même. La cour sentait que quelque éloignement
qu'eût toute la nation anglaise de revoir sur le trône le fils d'un roi
catholique qu'elle avait chassé, d'un roi qui avait attaqué tous leurs
privilèges, un roi élevé en France qui avait pris les leçons du roi son
père, qui y avait été nourri au milieu de l'exercice le plus constant et
le moins contredit du pouvoir plus qu'absolu, la nation toutefois ne
désirait pas l'extinction de sa famille, sentait la justice de son
droit, voulait y trouver un appui, et de quoi montrer sans cesse à la
maison d'Hanovre que son élévation sur le trône n'était que l'ouvrage de
sa volonté qui également la pouvait chasser, et bien plus justement
qu'elle n'avait ôté la couronne aux Stuarts, et tenir ainsi en bride
perpétuelle le roi Georges, sa famille et ses ministres. La position de
la France à l'égard de l'Angleterre les inquiétait sans cesse sur les
jacobites qui s'y étoient réfugiés par la facilité de leurs commerces et
de leurs intelligences en Angleterre, et par la facilité d'y passer
promptement.

Quelque honteuses preuves qu'eût le gouvernement d'Angleterre de
l'abandon de celui de France à ses volontés, depuis que Dubois en était
devenu l'arbitre unique, ces habiles ministres sentaient combien cette
conduite était personnelle\,; qu'elle ne tenait qu'au désir de la
pourpre que Dubois espérait du crédit du roi Georges auprès de
l'empereur qui, en effet, pouvait tout à Rome\,; que cette conduite
était essentiellement contraire à l'intérêt de la France et
singulièrement odieuse à toute la nation française, grands et petits\,;
conséquemment qu'elle pouvait facilement changer, et qu'il était de
l'intérêt le plus pressant de la maison d'Hanovre et de ses ministres de
profiter de leur situation présente avec la France pour la mettre à
jamais, autant qu'il était possible, hors de moyens de troubler
l'Angleterre, d'y favoriser utilement les jacobites, encore plus d'y
faire des partis et quelque invasion en faveur des Stuarts. Pour arriver
à ce point, il fallait deux choses, s'ôter toute inquiétude à l'égard de
la France en la dépouillant de tous ceux qui leur en pouvaient donner,
et ruiner en Angleterre tout crédit et toute confiance en la France, par
la rendre conjointement avec eux la persécutrice publique et déclarée du
ministère de la reine Anne, et de tout ce parti qui seul avait sauvé la
France des plus profonds malheurs par la paix particulière de Londres,
la séparation de l'Angleterre d'avec ses alliés, enfin par la paix
d'Utrecht, dont la reine Anne s'était rendue la dictatrice et la
maîtresse, et qui avait sauvé la France au moment qu'elle allait être
envahie, et la couronne d'Espagne à Philippe V, à l'instant qu'il
l'allait perdre sans la pouvoir sauver.

Le ministère du roi Georges avait voulu faire sauter les têtes de ce
ministère précédent, précisément pour avoir fait la paix de Londres et
forcé les alliés aux conditions de celle d'Utrecht, et n'avait cessé
depuis de persécuter ce parti avec la dernière fureur. Mettre la France
de moitié de cette persécution effective d'un parti à qui elle devait si
publiquement et si récemment son salut et la conservation de la couronne
d'Espagne à Philippe, par complaisance pour le parti opposé, qui ne
respira jamais que sa ruine radicale, et qui était parvenu à y toucher,
c'était couvrir la France d'une infamie éternelle à tous égards, et la
perdre tellement d'honneur, de réputation, de confiance en Angleterre,
vis-à-vis le parti qu'elle contribuait à y accabler en reconnaissance
d'en avoir été sauvée elle-même, qu'une démarche si contraire à tout
honneur, pudeur et intérêt, lui aliénerait à jamais ce parti, qui
l'avait sauvée, avec plus de rage que n'en pouvait avoir le parti
régnant qui l'avait voulu perdre, qui pour trouver la France si
déplorablement complaisante, ne l'en haïssait pas moins, et qui par là
trouvait le moyen de la mettre hors d'état d'en recevoir aucune
inquiétude, sans toutefois avoir acheté une démarche si destructive de
tout intérêt et de tout honneur, par le plus léger service, par la plus
légère apparence de refroidissement avec ses alliés que la France devait
toujours regarder comme véritables ennemis, par la plus petite justice à
l'égard de l'Espagne, par la moindre reconnaissance de la servitude par
laquelle nous avions pour leur complaire laissé volontairement et si
préjudiciable ment éteindre et anéantir notre marine, en un mot, rien
autre que d'avoir reconnu le pouvoir sans bornes de l'abbé Dubois sur
son maître, et d'en savoir profiter pour en tirer tout, en lui faisant
espérer le chapeau.

Je n'avais rien cédé de tout cela à M. le duc d'Orléans, dès le premier
traité où cette infamie fut stipulée. On a vu en son lieu combien je m'y
opposai dans son cabinet, et depuis au conseil de régence\,; je
n'oubliai aucune des raisons qu'on vient de voir, je les paraphrasai
plus fortement encore. Le maréchal d'Huxelles, maréchal d'Estrées,
plusieurs autres, qui n'osèrent traiter la matière qu'en tremblant, ne
laissèrent pas de laisser voir ce qu'ils en pensaient\,; Torcy même,
dont ces deux paix de Londres et d'Utrecht étaient l'ouvrage, s'éleva
plus que sa douceur et sa timidité naturelles ne le lui permettaient\,;
tout cela ne changea point l'article du traité, mais en suspendit
l'effet. Le gouvernement d'Angleterre y consentit, peut-être tacitement
informé de la révolte des esprits et du murmure général\,; mais les
temps étaient venus de ne plus rien ménager. L'affaire du parlement,
puis la conspiration du duc du Maine découverte et finie, la paix
d'Espagne faite, l'abbé Dubois plus maître que jamais, ses amis les
Anglais le sommèrent de sa parole\,; il fallut bien la tenir dans la vue
plus prochaine de la pourpre\,; la proscription effective fut accordée
et publiée sans qu'il fût possible à personne de l'empêcher. Les cris
publics et l'horreur qui en fut généralement marquée n'en causa aucun
repentir\,; ce ne fut qu'un sacrifice de plus que Dubois eut à présenter
à la cour de Londres pour accélérer sa pourpre, qui ne fut pas plus
goûté par tous les Anglais de tous partis, hors celui des ministres,
qu'il le fut en France, et on peut ajouter dans tout le reste de
l'Europe, qui nous en méprisa, tandis que tout le gros de l'Angleterre
nous en détesta ouvertement, et que le parti de son ministère se moqua
de notre misérable facilité.

Le roi d'Espagne, qui avait tant fait et laissé faire de choses en son
nom, et avec tant de persévérance pour élever Albéroni à la pourpre, en
fit de plus étranges pour l'en faire priver. Il n'y eut point
d'instances qu'il n'en fît faire au pape, qu'il ne lui en fît de sa
main, et pour l'engager encore de l'enfermer au château Saint-Ange, s'il
entrait dans l'État ecclésiastique. Peu content du succès de tant de
démarches, et si empressées, il profita de la paix qu'il venait de faire
avec le roi et avec l'empereur, pour les presser de joindre leurs plus
fortes démarches et leurs offices les plus vifs aux siens, auprès du
pape, pour en obtenir cette privation du chapeau\,; mais cela fut éludé
à Rome, où on obtiendrait plutôt une douzaine de chapeaux à la fois,
quelque chère et difficile que soit cette marchandise, car c'en est une
en effet, que la privation d'un seul. Cette cour qui a élevé si haut
cette dignité si vide de sa nature, et qui, à force de la revêtir et de
la décorer des dépouilles des plus hautes dignités sacrées et profanes,
sans être elle-même d'aucun de ces deux genres, est parvenue avec tout
l'art de sa politique à en faire l'appui de sa grandeur, en fascinant le
monde de chimères, qui à la fin sont devenues l'objet de l'ambition de
toutes les nations, par les richesses, les honneurs, les rangs et le
solide, dont elles se sont réalisées\,; et de là, montant toujours,
cette pourpre est arrivée à rendre inviolables les crimes les plus
atroces, et les félonies les plus horribles de ceux qui en sont revêtus.
C'est le point le plus cher et le plus appuyé des usurpations de leurs
privilèges, parce que c'est lui qui est le plus important à l'orgueil et
à l'intérêt de Rome qui se sert de l'espérance du chapeau pour dominer
toutes les cours catholiques, qui, par ce chapeau, soustrait les sujets
à leur roi, à tous juges pour quoi que ce puisse être, qui domine tous
les clergés, qui est seule juge et la souveraine de ces chapeaux rouges,
qui leur fait tout entreprendre et brasser impunément, et qui se trouve
par là si intéressée à soutenir leur impunité, qu'elle ne peut se
résoudre à y faire la moindre brèche en choses dont le fond ne
l'intéresse point, comme les crimes qui lui sont étrangers, même ceux
qui ont offensé les papes, comme Albéroni avait fait avec si peu de
ménagement, tant de fois, de peur que la privation du chapeau devint et
pût passer en exemple, et privât les papes des pernicieux usages qu'ils
ont si souvent faits des cardinaux, que la vue de pouvoir être
dépouillés de la pourpre arrêterait en beaucoup d'occasions.

Ce raisonnement est tellement celui de la cour de Rome, qu'on a vu des
papes faire tuer, noyer, empoisonner des cardinaux, plutôt que leur ôter
le chapeau. Les Caraffe, les Colonne et bien d'autres en sont des
exemples dont l'histoire n'est point à contester\,; on n'en voit point
de privation du chapeau, car on ne peut pas compter pour telle les temps
de schismes, et ce que les papes et les antipapes faisaient contre les
cardinaux les uns des autres. Ainsi le roi d'Espagne, heurtant ainsi la
partie la plus sensible et la plus essentielle de l'intérêt des papes et
de la cour de Rome, se donna vainement en spectacle de lutte et
d'impuissance, contre un homme de la lie du peuple, pour l'élévation
duquel il avait tout épuisé, et qu'il ne put détruire. Tout ce que ses
instances purent obtenir, encore aidées de la haine personnelle du pape
et de la cour de Rome contre Albéroni, fut de le réduire à errer,
souvent inconnu, jusqu'à la mort du pape\,; alors l'intérêt des
cardinaux l'appela au conclave où il entra comme triomphant, et est
depuis demeuré en splendeur, ou à Rome, ou dans les différentes
légations qu'il a obtenues. Ces leçons sont grandes, elles sont
fréquentes, elles sont bien importantes\,; elles n'en demeureront pas
moins inutiles par l'ambition des plus accrédités auprès des rois, et la
faiblesse des rois à leur procurer cette pourpre si fatale aux États,
aux rois et à l'Église.

Plusieurs personnes moururent à peu près en ce même temps\,: la comtesse
de Lislebonne, qui avait pris depuis plusieurs années le nom de
princesse de Lislebonne, mourut à quatre-vingt-deux ans\,; elle était
bâtarde de Charles IV, duc de Lorraine, si connu par ses innombrables
perfidies, et de la comtesse de Cantecroix, et veuve du frère cadet du
duc d'Elboeuf. Il y a eu occasion de parler ici d'elle quelquefois, et
de la faire assez connaître pour n'avoir plus besoin de s'y étendre\,;
avec beaucoup de vertu, de dignité, de toute bienséance, et non moins
d'esprit et de manége, elle ne céda à aucun des Guise en cette ambition
et cet esprit qui leur a été si terriblement propre, et eût été admise
utilement pour eux aux plus profonds conseils de la Ligue. Aussi
M\textsuperscript{lle} de Guise, le chevalier de Lorraine et elle
n'avaient-ils été qu'un\,; aussi donna-t-elle ce même esprit à
M\textsuperscript{me} de Remiremont, sa fille aînée, et
M\textsuperscript{me} d'Espinoy sa cadette y tourna, et y mit tout ce
qu'elle en avait. Cette perte fut infiniment sensible à ses deux filles,
à Vaudemont, son frère de même amour, encore plus dangereusement
Guisard, si faire se pouvait. Aussi logeaient-ils tous ensemble à Paris,
dans l'hôtel de Mayenne, ce temple de la Ligue, où ils ont conservé ce
cabinet appelé \emph{de la Ligue}, sans y avoir rien changé, par la
vénération, pour ne pas dire le culte d'un lieu où s'étaient tenus les
plus secrets et les plus intimes conseils de la Ligue, dont la vue
continuelle entretenait leurs regrets et en ranimait l'esprit, ce que
prouvent les faits divers qui ont été rapportés d'eux en tant d'endroits
de ces Mémoires, et tout le tissu de leur conduite\,; ainsi on ne leur
prête rien. Mais comme toute impunité, et au contraire toute
considération, était devenue de si longue main leur plus constant
apanage, la pension de douze mille livres qu'avait M\textsuperscript{me}
de Lislebonne, fut donnée à M\textsuperscript{me} de Remiremont\,;

Le grand maître de Malte, Perellos y Roccafull, Espagnol de beaucoup de
mérite, qui eut le frère du cardinal Zandodari pour successeur\,;

Le père Cloche, depuis quarante ans général de l'ordre de
Saint-Dominique, avec la plus grande réputation et la considération à
Rome la plus distinguée et la plus soutenue, et beaucoup d'autorité dans
toutes les affaires\,; aimé, respecté, estimé et consulté par tous les
papes et les cardinaux. Il aurait été cent fois cardinal, s'il n'avait
pas été François et très bon François\,; il avait été confesseur de mon
père jusqu'à son départ pour l'Italie\,;

Fourille, aveugle, qui avait beaucoup d'esprit et fort orné, et
longtemps capitaine aux gardes, estimé et fort dans la bonne compagnie.
Sa pension fut donnée à sa veuve, qui demeurait pauvre avec des enfants,
à l'un desquels on a vu ici que j'avais fait donner une abbaye sans les
connaître\,;

M\textsuperscript{me} de La Hoguette, veuve d'un lieutenant général
sous-lieutenant des mousquetaires, mort aux précédentes guerres du feu
roi en Italie, qui était un fort galant homme et très estimé. Cette
femme était fort riche, avare, dévote pharisaïque, toute merveilleuse,
du plus prude maintien, et qui sentait la profession de ce métier de
fort loin avec de l'esprit et de la vertu, si elle eût bien voulu
n'imposer pas tant au monde\,; elle était très peu de chose, et
toutefois merveilleusement glorieuse. Son mari était neveu de La
Hoguette, archevêque de Sens, si estimé et si considéré sans le
rechercher, et qui refusa l'ordre du Saint-Esprit avec une humilité si
modeste, comme on l'a vu en son lieu ici. La fille unique de
M\textsuperscript{me} de La Hoguette, qui avait épousé Nangis, fut sa
seule héritière, et avec beaucoup de patience et de vertu n'en fut pas
plus heureuse\,;

Mortagne, officier général, qui s'était fait estimer dans la gendarmerie
et dans le monde. Il en a été parlé sur ses deux mariages, l'un et
l'autre assez singuliers. Il s'était fait chevalier d'honneur de Madame.
C'était un fort honnête homme, mais de fort obscure naissance. Son père
était un riche maître de forges devers Liée, qui laissa à son fils un
nom qui n'était pas à lui. Il laissa une fille unique et une veuve assez
digne du duc de Montbazon, mort enfermé à Liège, père de son père, dont
la plupart de la postérité s'est sentie peu ou beaucoup.

M\textsuperscript{me} la Duchesse, soeur de M. le prince de Conti et de
M\textsuperscript{lle} de La Roche-sur-Yon, mourut le 21 mars à Paris,
dans l'hôtel de Condé, après une fort longue maladie, à trente et un
ans, au bout de sept ans de mariage, dont il a été parlé ici en son
temps, pendant lequel elle ne s'était pas contrainte\,: elle fut plainte
sans être regrettée. Les princes du sang rebutés de leurs tentatives
inutiles de faire garder le corps de ces princesses, l'usage de brusquer
l'enterrement, pris depuis ce peu de succès, fut continué en cette
occasion. Le surlendemain de sa mort, sans qu'il y eût eu aucune
cérémonie à l'hôtel de Condé que le pur nécessaire, elle fut portée aux
Carmélites de la rue Saint-Jacques où elle fut enterrée. Le convoi fut
très magnifique. M\textsuperscript{lle} de Clermont accompagna le corps
avec la duchesse de Sully et de Tallard, que M. le Duc et
M\textsuperscript{me} sa mère en avaient priées. Quelques jours après,
M. le Duc reçut les visites de tout le monde, avec la précaution
ordinaire d'un magasin de manteaux dans son antichambre, et l'indécence
ordinaire et affectée contre cette nouvelle pratique, qui a été marquée
ici à son commencement. M\textsuperscript{me} la Duchesse, qui ne laissa
point d'enfants, fit un testament et M\textsuperscript{me} de La
Roche-surYon sa légatrice universelle. Il y avait beaucoup à rendre et
force pierreries, parce que feu M. le prince de Conti avait fort
avantagé cette princesse qui était sa fille aînée.
M\textsuperscript{lle} de La Roche-sur-Yon ne se trouva pas la plus
forte. M. le Duc s'en tira lestement, mais peu d'années avant sa mort il
pensa sérieusement, et fit pleine justice à M\textsuperscript{lle} de La
Roche-sur-Yon qui n'avait osé le plaider, et qui ne pensait plus depuis
longtemps à cette affaire. Le deuil du roi ne fut que de cinq jours pour
M\textsuperscript{me} la Duchesse.

\hypertarget{chapitre-xxi.}{%
\chapter{CHAPITRE XXI.}\label{chapitre-xxi.}}

1720

~

{\textsc{Maison de Horn ou Hornes.}} {\textsc{- Catastrophe du comte de
Horn à Paris.}} {\textsc{- Jugement et exécutions à Nantes.}} {\textsc{-
Mort, famille et extraction du prince de Berghes.}} {\textsc{- Mort du
duc de Perth.}} {\textsc{- Mariage du comte de Grammont avec une fille
de Biron.}} {\textsc{- Mariage de Mailly avec une soeur de la duchesse
de Duras, {[}M\textsuperscript{lle} de{]} Bournonville.}} {\textsc{-
Mariage du duc de Fitz-James avec M\textsuperscript{lle} de Duras.}}
{\textsc{- Mariage de Chalmazel avec M\textsuperscript{lle} de
Bonneval.}} {\textsc{- Mariage du prince d'Isenghien avec la seconde
fille du prince de Monaco.}} {\textsc{- Mariage du marquis de Matignon
avec M\textsuperscript{lle} de Brenne, et de sa soeur à lui avec
Basleroy.}} {\textsc{- Naissance de l'infant don Philippe.}} {\textsc{-
Maulevrier-Langeron, envoyé en Espagne, lui porte le cordon bleu.}}
{\textsc{- Affaire et caractère de l'abbé de Gamaches, auditeur de
rote.}} {\textsc{- Sa conduite à Rome, où il mourut dans cet emploi.}}
{\textsc{- Ce que c'est que la rote.}}

~

Le comte de Horn était à Paris depuis environ deux mois, menant une vie
obscure de jeu et de débauche. C'était un homme de vingt-deux ans, grand
et fort bien fait, de cette ancienne et grande maison de Horn, connue
dès le XIe siècle parmi ces petits dynastes des Pays-Bas, et depuis par
une longue suite de générations illustres. La petite ville et la
seigneurie de Horn en Brabant, près de Ruremonde, a donné l'origine et
le nom à cette maison. Elle est du territoire de Liége, et relevait de
l'ancien comté de Looss. Des trois branches de cette maison J., second
fils de Jacques, fait comte de Horn par l'empereur Frédéric III, et
frère puîné d'autre Jacques qui eut des enfants, sans postérité,
recueillit la succession de son frère et de ses neveux. Il quitta la
prévôté de Liège pour épouser Anne d'Egmont, fille de Floris, comte de
Buren, chevalier de la Toison d'or, et veuve avec des enfants de Joseph
de Montmorency, seigneur de Nivelle. Elle captiva si bien son second
mari que, se voyant sans enfants, et le dernier de la branche aînée de
Horn, il adopta les deux enfants de sa femme, Philippe et Floris de
Montmorency, qui furent tous deux illustres par leurs grands emplois,
tous deux chevaliers de l'ordre de la Toison d'or, tous deux victimes
des cruautés exercées dans les Pays-Bas, tous deux sans avoir laissé de
postérité. Philippe prit le nom de comte de Horn. C'est lui à qui le duc
d'Albe, gouverneur des Pays-Bas, fit couper la tête avec le comte
d'Egmont, et qui furent exécutés ensemble à Bruxelles, le 5 juin 1568.
Floris, son frère, porta le nom de baron de Montigny, député pour la
seconde fois en Espagne, pour supplier Philippe II de ne point établir
l'inquisition aux Pays-Bas, fut arrêté en septembre 1567, puis transféré
du château de Ségovie en celui de Simancas, où il eut la tête tranchée
en octobre 1570. Leurs deux soeurs furent mariées toutes deux dans la
maison de Lalaing.

Thierry de Horn, frère puîné du trisaïeul du dernier de la branche
aînée, fit la seconde branche qui finit à sa dixième génération.

J. de Horn fut chef de la troisième et dernière branche, et portait le
nom de seigneur de Baussignie. Il était second fils de Philippe,
seigneur de Gaësbeck, arrière-petit-fils de Thierry, chef de la seconde
branche. Eugène Max, sa cinquième génération directe, fut fait prince de
Horn. Son fils unique, Philippe-Emmanuel, prince de Horn, eut les
charges, les emplois et les distinctions les plus considérables, civiles
et militaires, sous Charles II, roi d'Espagne, dont il reconnut le
testament, servit de lieutenant général aux sièges de Brisach sous Mgr
le duc de Bourgogne, de Landau, sous le maréchal de Tallard, se
distingua fort sous le même à la bataille de Spire, puis sous le
maréchal de Villeroy, fut blessé de sept coups et prisonnier à la
bataille de Ramilies. D'Antoinette, fille du prince de Ligne, chevalier
de la Toison d'or et grand d'Espagne, il a laissé deux fils\,:
Maximilien-Emmanuel qui a suivi la révolution des Pays-Bas, où tous ses
biens sont situés, et où il porte le nom de prince de Horn, et
Antoine-Joseph portant le nom de comte de Horn dont il s'agit ici, et
qui n'était encore que capitaine réformé dans les troupes autrichiennes,
moins par sa jeunesse que par être fort mauvais sujet, et fort
embarrassant pour sa mère et pour son frère. Ils apprirent tant de
choses fâcheuses de sa conduite à Paris depuis le peu de temps qu'il y
était arrivé, qu'ils y envoyèrent un gentilhomme de confiance avec de
l'argent pour y payer ses dettes, lui persuader de s'en retourner en
Flandre, et, s'il n'en pouvait venir à bout, implorer l'autorité du
régent, à qui ils avaient l'honneur d'appartenir par Madame, pour leur
être renvoyé. Le malheur voulut que ce gentilhomme arriva le lendemain
qu'il eut commis le crime qui va être raconté.

Le comte de Horn alla le vendredi de la Passion, 22 mars, dans la rue
Quincampoix, voulant, disait-il, acheter cent mille écus d'actions, et y
donna pour cela rendez-vous à un agioteur dans un cabaret. L'agioteur
s'y trouva avec son portefeuille et des actions, et le comte de Horn
accompagné, lui dit-il, de deux de ses amis\,; un moment après ils se
jetèrent tous trois sur ce malheureux agioteur\,; le comte de Horn lui
donna plusieurs coups de poignard, et prit son portefeuille\,; un de ses
deux prétendus amis qui était Piémontais, nommé Mille, voyant que
l'agioteur n'était pas mort, acheva de le tuer. Au bruit qu'ils firent,
les gens du cabaret accoururent, non assez prestement pour ne pas
trouver le meurtre fait, mais assez tôt pour se rendre maîtres des
assassins et les arrêter. Parmi cette bagarre, l'autre coupe-jarret se
sauva\,; mais le comte de Horn et Mille ne purent s'échapper. Les gens
du cabaret envoyèrent chercher la justice, aux officiers de laquelle ils
les remirent, qui les conduisirent à la Conciergerie. Cet horrible
crime, commis ainsi en plein jour, fit aussitôt grand bruit, et aussitôt
plusieurs personnes considérables, parents de cette illustre maison,
allèrent crier miséricorde à M. le duc d'Orléans, qui évita tant qu'il
put de leur parler, et qui avec raison ordonna qu'il en fût fait bonne
et prompte justice. Enfin les parents percèrent jusqu'au régent\,; ils
tâchèrent de faire passer le comte de Horn pour fou, disant même qu'il
avait un oncle enfermé, et demandèrent qu'il fût enfermé aux
Petites-Maisons, ou chez les pères de la Charité, à Charenton, chez qui
on met aussi des fous\,; mais la réponse fut qu'on ne pouvait se défaire
trop tôt des fous qui portent la folie jusqu'à la fureur. Éconduits de
leur demande, ils représentèrent quelle infamie ce serait que
l'instruction du procès et ses suites pour une maison illustre, qui
appartenait à tout ce qu'il y avait de plus grand, et à presque tous les
souverains de l'Europe. Mais M. le duc d'Orléans leur répondit que
l'infamie était dans le crime et non dans le supplice. Ils le pressèrent
sur l'honneur que cette maison avait de lui appartenir à lui-même. «\,Eh
bien, messieurs, leur dit-il, fort bien\,; j'en partagerai la honte avec
vous.\,»

Le procès n'était ni long ni difficile. Law et l'abbé Dubois, si
intéressés à la sûreté des agioteurs, sans laquelle le papier tombait
tout court et sans ressource, prirent fait et cause auprès de M. le duc
d'Orléans, pour le rendre inexorable\,; et lui pour éviter la
persécution qu'il essuyait sans cesse pour faire grâce, eux dans la
crainte qu'il ne s'y laissât enfin aller, n'oublièrent rien pour presser
le parlement de juger\,; l'affaire allait grand train, et n'allait à
rien moins qu'à la roue. Les parents, hors d'espoir de sauver le
criminel, ne pensèrent plus qu'à obtenir une commutation de peine.
Quelques-uns d'eux me vinrent trouver, pour m'engager de les y servir,
quoique je n'aie point de parenté avec la maison de Horn\,; ils
m'expliquèrent que la roue mettrait au désespoir toute cette maison, et
tout ce qui tenait à elle, dans les Pays-Bas et en Allemagne, parce
qu'il y avait en ces pays-là une grande et très importante différence
entre les supplices des personnes de qualité qui avaient commis des
crimes\,; que la tête tranchée n'influait rien sur la famille de
l'exécuté, mais que la roue y infligeait une telle infamie, que les
oncles, les tantes, les frères et soeurs, et les trois premières
générations suivantes, étaient exclus d'entrer dans aucun noble
chapitre, {[}ce{]} qui, outre la honte, était une privation très
dommageable, et qui empêchait la décharge, l'établissement et les
espérances de la famille, pour parvenir aux abbayes de chanoinesses, et
aux évêchés souverains\,; cette raison me toucha, et je leur promis de
la représenter de mon mieux à M. le duc d'Orléans, mais sans m'engager
en rien au delà pour la grâce.

J'allais partir pour la Ferté, y profiter du loisir de la semaine
sainte. J'allai donc trouver M. le duc d'Orléans, à qui j'expliquai ce
que je venais d'apprendre. Je lui dis ensuite que quiconque lui
demanderait la vie du comte de Horn, après un crime si détestable en
tous ses points, ne se soucierait que de la maison de Horn, et ne serait
pas son serviteur\,; que je croyais aussi que ne serait pas son
serviteur quiconque s'acharnerait à l'exécution de la roue, à quoi le
comte de Horn ne pouvait manquer d'être condamné\,; que je croyais qu'il
y avait un \emph{mezzo-termine} àprendre, lui qui les aimait tant, qui
remplirait toute justice et toute raisonnable attente, du public\,; qui
éviterait le honteux et si dommageable rejaillissement de l'infamie sur
une maison si illustre et grandement alliée, et qui lui dévouerait cette
maison et tous ceux à qui elle tenait, qui au fond sentaient bien que la
grâce de la vie était impraticable, au lieu du désespoir et de la rage
où tous entreraient contre lui, et qui se perpétuerait et s'aigrirait
même à chaque occasion perdue d'entrer dans les chapitres où la soeur du
comte de Horn était sur le point d'être reçue. Je lui représentai que ce
moyen était bien simple. C'était de laisser rendre et prononcer l'arrêt
de mort sur la roue, de tenir toute prête la commutation de peine toute
signée et scellée pour n'avoir que la date à y mettre à l'instant de
l'arrêt, et sur-le-champ l'envoyer à qui il appartient, puis le jour
même faire couper la tête au comte de Horn. Par là toute justice est
accomplie, et l'arrêt de roue prononcé, le public est satisfait, puisque
le comte de Horn est en effet puni de mort, auquel public, l'arrêt
rendu, il n'importe plus du supplice, pourvu qu'il soit à mort, et la
maison de Horn et tout ce qui y tient, trop raisonnables pour avoir
espéré une grâce de la vie qu'eux-mêmes en la place du régent n'auraient
pas accordée, lui seraient à jamais redevables d'avoir sauvé leur
honneur et les moyens de l'établissement des filles et des cadets. M. le
duc d'Orléans trouva que j'avais raison, la goûta, sentit son intérêt de
ne pas jeter dans le désespoir contre lui tant de gens si considérables
en accomplissant toutefois toute justice et l'attente du public, et me
promit qu'il le ferait ainsi. Je lui dis que je partais le lendemain\,;
que Law et l'abbé Dubois, acharnés à la roue, la lui arracheraient\,; il
me promit de nouveau de tenir ferme à la commutation de peine, m'en dit
là-dessus autant que je lui en aurais pu dire\,; en m'étendant là-dessus
je lui déclarai que je n'étais ni parent ni en la moindre connaissance
avec la maison de Horn, ni en liaison avec aucun de ceux qui se
remuaient pour elle\,; que c'était uniquement raison et attachement à sa
personne et à son intérêt qui me faisait insister, et que je le
conjurais de demeurer ferme dans la résolution qu'il me témoignait,
puisqu'il en sentait tout le bon et toutes les tristes suites du
contraire, et de ne se point laisser entraîner aux raisonnements faux et
intéressés de Law et de l'abbé Dubois, qui se relayeraient pour arracher
de lui ce qu'ils voulaient. Il me le promit de nouveau, et comme je le
connaissais bien, je vis que c'était de bonne foi. Je pris congé et
partis le lendemain.

Ce que j'avais prévu ne manqua pas. Dubois et Law l'assiégèrent, et le
retournèrent si bien que la première nouvelle que j'appris à la Ferté
fut que le comte de Horn et son scélérat de Mille avaient été roués en
Grève, vifs, et avaient expiré sur la roue le mardi saint, 26 mars, sur
les quatre heures après midi, sur le même échafaud, après avoir été
appliqués à la question. Le succès en fut tel aussi que je l'avais
représenté à M. le duc d'Orléans. La maison de Horn et toute la grande
noblesse des Pays-Bas, même d'Allemagne, furent outrées, et ne se
continrent ni de paroles ni par écrit. Il y eut même par mieux
d'étranges partis de vengeance, pourpensés, et, longtemps depuis la mort
de M. le duc d'Orléans, j'ai trouvé de ces messieurs-là, qui n'ont pu se
tenir de m'en parler ni se contenir de répandre le venin qu'ils en
conservaient dans le coeur.

Le même jour, mardi 26 mars, que le comte de Horn fut exécuté à Paris,
plusieurs Bretons le furent à Nantes par arrêt de la commission du
conseil. Les sieurs de Pontcallet, de Talhouet, Montlouis et Coëdic
\footnote{Ce dernier est appelé, dans Lemontey(\emph{Hist. de la
  Régence}, I, 246), du Courdic, capitaine réformé des dragons de
  Bellabre.}, capitaine de dragons, y eurent la tête coupée. Il y en eut
seize autres qu'on ne tenait pas qui l'eurent en même temps en effigie,
qui furent les deux frères Rohan du Poulduc, les deux frères du
Groesker\footnote{Dugroesquar, d'après Lemontey(\emph{ibid}). L'un était
  d'épée, et l'autre d'église, comme on disait alors. L'abbé Dugroesquar
  était un des chefs du mouvement et cherchait à lui donner de l'unité.},
les sieurs de Rosconan, Bourgneuf-Trevelec fils, Talhouet de Boisoran et
Talhouet de Bonamour, La Boissière, Kerpedron de Villeglé, La Beraye, La
Houssaye père, Croser, Kerentré de Goëllo, Melac-Hervieux et Lambilly,
conseiller au parlement de Rennes. Les prisonniers avaient avoué la
conspiration et les mesures prises pour livrer les ports de la Bretagne
à l'Espagne, et y en recevoir les troupes, marcher en armes en France,
etc., le tout juridiquement avoué et prouvé. On les avait éblouis de les
remettre comme au temps de leur duchesse héritière Anne, et de trouver
la plupart de la noblesse de France prête à se joindre à eux pour la
réformation du royaume sous l'autorité du roi d'Espagne, représentée en
France par le duc du Maine. La bouche fut soigneusement fermée aux
commissaires les plus instruits, et l'abbé Dubois sut mettre bon ordre à
la conservation du secret, des détails sur le duc et duchesse du Maine
qu'il avait eu grand soin de faire élargir, et revenir avant d'achever
les procès criminels de Nantes. Il se trouva tant de gens arrêtés et à
arrêter sur les dépositions des prisonniers qu'après l'exécution réelle
de ces quatre, et en effigie de ces seize, on envoya une amnistie pour
tous les prisonniers et accusés non arrêtés, les uns et les autres non
encore jugés, dont dix seulement furent exceptés, qui sont les deux
frères Lescoët, les sieurs de Roscoët, Kersoson, Salarieuc l'aîné,
Karanguen-Hiroët, Coargan, Boissy-Bec-de-Lièvre, Kervasi l'aîné, et les
frères Fontainepers. Noyau, qui était prisonnier, fut mis en liberté par
l'amnistie. Rochefort, président à mortier, et La Bédoyère, procureur
général, et quelques autres du même parlement de Bretagne, eurent ordre
de se défaire de leurs charges, et l'arrêt de la commission du conseil à
Nantes fut rendu public. Plusieurs de ces Bretons coupables, qui se
sauvèrent à temps, se retirèrent par mer en Espagne, où tous eurent des
emplois ou des pensions. Peu y firent quelque petite fortune qui ne les
consola pas de leur pays ni du peu qu'ils y avaient quitté. Beaucoup y
vécurent misérables et méprisés par la plus que médiocrité, à quoi se
réduisit bientôt ce qu'on leur avait donné. Quelques-uns revinrent en
France après la mort de M. le duc d'Orléans et le changement de toutes
choses, mais fort obscurément chez eux\,; la plupart sont morts en terre
étrangère. Telle est presque toujours l'issue des conspirations et le
sort de tant de gens qui, en celle-ci, perdirent la tête ou leur état,
leurs biens, leur famille, pour errer en terre étrangère, et y demander
leur pain, et le recevoir bien court pour l'intérêt, les vues,
l'ambition du duc et de la duchesse du Maine qui les avaient si bien
ensorcelés, et qui n'en perdirent pas un cheveu de leur tête. Il fut
même remarqué que, peu de jours après, le duc du Maine vit pour la
première fois M. le duc d'Orléans à Saint-Cloud.

Le prince de Berghes mourut chez lui en Flandre. Il n'était point de
l'ancienne maison de ce nom, mais des bâtards de Berghes et frère de
M\textsuperscript{lle} de Montigny, cette maîtresse si longtemps aimée
et publiquement par l'électeur de Bavière, qu'il fit enfin épouser au
comte d'Albert, comme on l'a vu ici en son lieu. Elle avait fait en
sorte que l'électeur avait obtenu la grandesse d'Espagne et la Toison
d'Or de Philippe V, pour son frère qui était aussi petit et vilain
qu'elle était belle et bien faite. Il avait épousé une fille du duc de
Rohan qui ne voulait pas lui donner grand'chose, dont il n'eut point
d'enfants, et qui a été une femme de mérite et d'une belle figure. Le
père de ce prince de Berghes était gouverneur de Mons, qu'il défendit
quand le roi le prit, et il est mort chevalier de la Toison d'or et
gouverneur de Bruxelles.

Le duc de Perth mourut presque en même temps dans le château de
Saint-Germain où il était demeuré. C'était un seigneur qui avait quitté
de grands établissements en Écosse, par fidélité pour le roi Jacques qui
le fit gouverneur du prince de Galles. Sa femme était morte à
Saint-Germain, dame d'honneur de la reine d'Angleterre, dont il était
grand écuyer. C'était un homme d'honneur et de beaucoup de piété, qui
valait bien mieux que le duc de Melford son frère. Le roi Jacques les
fit ducs tous deux, le dernier en mourant, comme on l'a vu en son lieu,
et leur donna à tous deux la Jarretière.

Il se fit aussi plusieurs mariages. M\textsuperscript{me} de Biron, qui
ne négligeait rien, avait su profiter de la place de son mari auprès de
M. le duc d'Orléans, et captiver Law pour avoir gros, comme auparavant
elle avait su sucer plusieurs financiers, et quelques-uns jusqu'au sec
pour sa protection. Le duc de Guiche, moyennant le besoin que le régent
crut toujours avoir du régiment des gardes avait tiré des monts d'or de
Law. Il avait déjà marié sa fille aînée au fils aîné de Biron. Ils
firent encore un mariage d'une fille de Biron avec le second fils du duc
de Guiche qu'on appelait le comte de Grammont. En faveur de cette
affaire M. le duc d'Orléans donna huit mille livres de pension à la
nouvelle épouse.

M\textsuperscript{lle} de Bournonville, soeur de la duchesse de Duras,
mais qui ne lui ressemblait en rien, épousa l'aîné de la maison de
Mailly, duquel la mère était soeur du cardinal de Mailly\,; ni l'un ni
l'autre n'étaient pas faits pour la fortune, aussi pour des gens comme
eux sont-ils demeurés dans l'obscurité.

La même duchesse de Duras et son mari marièrent leur fille aînée, qui
n'avait que quatorze {[}ans{]}, au fils aîné du duc et de la duchesse de
Berwick qu'on appela duc de Fitz-James, qui était aussi fort jeune, qui
eut en se mariant dix mille livres de pension. Il mourut peu d'années
après sans enfants. Sa veuve s'est depuis remariée au duc d'Aumont dont
elle a des enfants.

Peu après, Chalmazel épousa M\textsuperscript{lle} de Bonneval, fille du
frère aîné de celui qui a passé en Turquie, tous deux de bonne maison.
Chalmazel était fils d'une soeur de Chamarande, goutteux, veuf et sans
enfants, qui était riche\,; mais lui était Talaru qui est une fort
ancienne maison devers le Lyonnais, alliée à toutes les meilleures des
provinces voisines.

Le prince d'Isenghien, qui n'avait point d'enfants de ses deux femmes,
épousa M\textsuperscript{lle} de Monaco, soeur de la duchesse de
Valentinois, qui en fit la noce chez le comte de Matignon, son
beau-père, avec qui elle demeurait. M. de Monaco était à Monaco et n'en
sortait plus.

Parlant des Matignon, la seconde fille du maréchal de Matignon qui
n'était plus jeune, et s'ennuyait de n'être point mariée, épousa
Basleroy, colonel de dragons. Son nom était La Cour, et si peu de chose,
que son père, qui était riche, épousa pour rien la soeur de Caumartin,
conseiller d'État, et se fit maître des requêtes\,; il n'alla pas plus
loin. Les Matignon outrés furent fort longtemps sans vouloir ouïr parler
de Basleroy et de sa femme, et à la fin les virent et leur pardonnèrent.
Le second fils du maréchal de Matignon épousa aussi
M\textsuperscript{lle} de Brenne, fille d'une soeur de la duchesse de
Noirmoutiers, qui en la mariant la fit son héritière.

La reine d'Espagne accoucha d'un prince qui fut appelé don Philippe, à
qui on envoya le cordon bleu à l'exemple du feu roi qui en avait usé
ainsi envers les infants aînés de celui-ci, et les avait ainsi comme
fils de roi traités en fils de France, quoique, à le prendre en rigueur
de naissance, ils ne fussent que fils d'un fils de France cadet, et par
conséquent petits-fils de France. Maulevrier-Langeron, dont le nom est
Andrault, neveu de l'abbé de Maulevrier, aumônier du roi, duquel on a
parlé ici quelquefois, fut destiné à porter ce cordon bleu, et à être
envoyé du roi en Espagne. Ce fut son oncle qui lui procura cet emploi.
Il venait d'être fait lieutenant général dans une promotion de dix-sept,
dont fut aussi le duc de Duras. Ces Andrault étaient de Bourbonnais,
attachés, mais fort en sous-ordre, à la maison de Condé. On a vu en son
lieu que Langeron, lieutenant général des armées navales, l'était fort
au duc du Maine. On verra que M. le duc d'Orléans aurait pu faire un
meilleur choix, si Dieu me donne le temps d'écrire ici mon ambassade en
Espagne.

L'abbé de Gamaches était à Rome depuis assez longtemps, qu'il y avait
été envoyé succéder au cardinal de Polignac, à la place d'auditeur de
rote pour la France. Il était fils de Gamaches qui avait été mis auprès
de Mgr le duc de Bourgogne avec Cheverny, d'O et Saumery, en qualité de
menins. Le frère de cet abbé avait épousé une fille de Pomponne, frère
de M\textsuperscript{me} de Torcy, et Torcy ministre et secrétaire
d'État des affaires étrangères lui avait valu cet emploi. Le père de
Gamaches était chevalier de l'ordre de 1661, et tous deux avaient épousé
les soeurs de MM. de Loménie et de Brienne, père et fils, et secrétaires
d'État des affaires étrangères, que le fils quitta parce que sa tête se
dérangea, et a vécu longtemps et est mort enfermé. Le nom de l'abbé de
Gamaches est Rouault. Il était fort glorieux, encore plus ambitieux et
fort plein de lui-même\,; il faut dire aussi qu'il n'était pas sans
mérite, et qu'il avait du savoir et de l'esprit pour toute sa race\,;
mais il ne souffrait pas aisément de supérieur, ne démordait point de ce
qu'il avait entrepris, et savait parfaitement être ami et ennemi. Avec
ces qualités il s'appliqua fort à la rote, et y acquit la réputation
d'un des plus capables de ce tribunal. Quand il s'y fut ancré et qu'il
eut acquis des amis et de la considération dans Rome, son génie et son
humeur se déployèrent, et son ambition se développa. Il ne songea qu'à
plaire à la cour de Rome et à ceux qui la gouvernaient ou qui pourraient
la gouverner à leur tour, et se mit en tête de se faire cardinal par
cette voie. Dans ce plan de conduite il ne craignit pas de se lier
étroitement avec les personnages principaux et autres qu'il se crut
utiles, quoique déclarés contre la France, et de marcher ainsi tête
levée dans toutes les routes qui pouvaient favoriser son projet.

L'abbé Dubois avait des agents secrets à Rome pour son chapeau. Gamaches
les découvrit, les suivit, chercha inutilement à avoir par eux quelque
part en leurs menées. Il fut pique du mystère qu'ils lui en firent, se
brouilla avec eux, se mit à les traverser de dépit, et aussi pour faire
sentir à l'abbé Dubois qu'il avait besoin de lui. Dubois en fut bientôt
averti\,; la fureur le saisit contre l'abbé de Gamaches, qu'il trouva
plus court de rappeler, dans la puissance où il se trouvait de tout
faire. Un autre que Gamaches aurait été accablé, mais il l'avait prévu
et s'était préparé à en soutenir le choc. Il commença par s'excuser,
continua par se plaindre\,; mais comme il s'aperçut que cette conduite
n'opérait pas de changement à son rappel, il chaussa le cothurne et osa
se déclarer\,; il déclara donc à l'abbé Dubois que ce rappel n'était
point en sa puissance, pour couler doucement qu'elle n'était pas en
celle du régent, par conséquent en celle du roi même. Il avança
nettement que le feu roi, en le nommant à l'auditorat de rote pour la
France, avait consommé son pouvoir\,; que du moment qu'il était pourvu,
agréé à Rome et en possession, il était devenu magistrat d'un des
premiers tribunaux du monde\,; que dès là il ne dépendait plus du roi,
ni pour sa place, ni pour ses fonctions, ni pour sa personne\,; que si
on pouvait juridiquement prouver des crimes, un auditeur de rote comme
tout autre magistrat en subissait la punition, mais instruite devant le
pape et prononcée par lui, lequel était le souverain de Rome et de la
rote, sous l'autorité et la protection duquel elle faisait ses
fonctions\,; que de crimes ni même de mauvaise conduite, il ne craignait
point qu'on lui en pût imputer, encore moins prouver\,; qu'il s'en
tenait là avec d'autant plus d'assurance qu'il n'avait à répondre que
devant le pape, de l'intégrité et de la bonté duquel il ne pouvait
prendre de défiance. À cette dépêche Dubois sauta en l'air\,; mais quand
il eut bien tempêté, il craignit de se commettre avec une cour dont il
espérait tout et de s'y rendre odieux. Il écouta donc volontiers ce
qu'on lui voulut dire en faveur de l'abbé de Gamaches. Mais comme il
désirait passionnément aussi de tirer de Rome un homme qui lui pouvait
beaucoup nuire, et qui était sur les pistes de tous ses agents, car il
en entretenait trois ou quatre à Rome inconnus les uns aux autres, il
lui offrit l'archevêché d'Embrun, vacant par la mort de Brûlart-Genlis,
le plus ancien prélat de France, et un des plus saints et des plus
résidents évêques. Gamaches, incapable d'abandonner ses vues, le refusa
tout net, et déclara qu'il ne voulait quitter ni Rome ni la rote\,; mais
profitant avec esprit de cet adoucissement, il fit le reconnaissant,
offrit ses services à Dubois, et lui en rendit en effet pour le gagner
et de fort bons. Avec tous ces manéges, il demeura auditeur de rote\,;
mais il en résulta un véritable scandale.

Jamais auditeur de rote n'avait encore imaginé ne pouvoir être rappelé.
C'est un tribunal où, non sans abus, il se porte des affaires, et
souvent très considérables, de toutes les parties de la catholicité\,;
c'est pour cela qu'il est composé de juges de toutes les nations
catholiques, et que chaque roi, ou république, même quelques villes qui
l'ont été autrefois, ont la nomination du juge de sa nation. Ce juge est
son sujet\,; il cesse si peu de l'être par sa nomination, qu'il n'en
fait les fonctions qu'à ce titre, et à titre de sujet, par conséquent
révocable, par le pouvoir d'un souverain sur son sujet. Cet exemple de
prétention de ne pouvoir l'être était donc monstrueux et très
punissable\,; mais la punir n'était pas l'intérêt du maître des affaires
de France, qui les tournait toutes, et les sacrifiait pour avoir un
chapeau. Cette affaire fit donc grand bruit et peu d'honneur à
l'autorité du roi, à laquelle elle a porté une blessure qui doit bien
faire prendre garde à l'avenir au choix des auditeurs de rote. Quoique
toutes les puissances qui en nomment aient le même intérêt, on n'a vu
autre chose que Rome s'avantager de tout, et l'emporter sur choses bien
plus essentielles, et s'il se peut encore moins fondées contre l'intérêt
général, et quelquefois le plus important et le plus sensible de toutes
les puissances de sa communion.

Gamaches, enflé d'un succès qu'il devait à sa hardiesse, et aux
conjonctures qui viennent d'être expliquées, ne se contint plus. Il
avait toujours devant les yeux les exemples de MM. Séraphin, La
Trémoille et Polignac, qui d'auditeurs de rote pour la France étaient
devenus cardinaux\,; mais c'en était trois seuls, et en plus d'un
siècle. Il se brouilla dans la suite avec le cardinal de Polignac,
chargé des affaires du roi à Rome, dont les défauts n'étaient pas de
manquer de douceur, d'agréments, et de tout mettre de sa part dans le
commerce d'affaires, et de société. La brouillerie s'augmenta avec tant
d'éclat, que Gamaches perdit tout respect et toute mesure en discours
publics et en conduite à son égard, ne le vit plus, et cessa de lui
rendre tous les devoirs auxquels il était obligé envers lui comme
cardinal, et comme ministre public du roi\,; il ne vécut pas mieux avec
d'autres cardinaux attachés à la France, pour avoir pris le parti du
cardinal de Polignac\,; tout cela fut su et souffert, parce qu'on avait
laissé gagner ce terrain à Gamaches, et dans les fins aussi, parce
qu'ici on se plut à mortifier le cardinal de Polignac. Ce n'était pas
que depuis quelques années Gamaches n'eût donné de fortes prises sur
soi, et même une qui dura longtemps, et qui fit du bruit à Rome, mais
dont il ne fut autre chose. Gamaches, que rien n'arrêtait pour aller à
son but, avait quantité d'amis dans le sacré collège, dans la prélature,
dans la principale noblesse, dans l'intérieur de la maison du pape, dans
le subalterne important et accrédité\,; quoiqu'il ne fût pas sans
ennemis, on pouvait dire que tout riait à ses espérances. C'est la
situation où le duc de Saint-Aignan le trouva en arrivant à Rome, avec
le caractère d'ambassadeur de France. Ils n'eurent guère le temps de
savoir comment ils s'accommoderaient l'un de l'autre, l'abbé de Gamaches
étant mort peu de temps après d'une maladie ordinaire, mais qui fut fort
courte, et qui mit fin à tous ses grands projets. Il était riche, et
entre ses bénéfices il avait l'abbaye de Montmajour d'Arles qui est très
considérable.

\hypertarget{chapitre-xxii.}{%
\chapter{CHAPITRE XXII.}\label{chapitre-xxii.}}

1720

~

{\textsc{Débordement de pensions, et pensions fixées au grade d'officier
général.}} {\textsc{- M. le duc d'Orléans m'apprend le mariage du duc de
Lorges avec la fille du premier président.}} {\textsc{- Ma conduite
là-dessus.}} {\textsc{- Édit de réduction des intérêts des rentes.}}
{\textsc{- Mouvements du parlement là-dessus.}} {\textsc{-
Remontrances.}} {\textsc{- Retour de Rion à Paris, où il tombe dans
l'obscurité.}} {\textsc{- Enlèvements pour peupler le pays dit
Mississipi, et leur triste succès.}} {\textsc{- La commission du
conseil, de retour de Nantes, s'assemble encore à l'Arsenal\,; peu après
le maréchal de Montesquiou rappelé de son commandement de Bretagne.}}
{\textsc{- Retour du comte de Charolais de ses voyages.}} {\textsc{- Bon
mot de Turménies.}} {\textsc{- Quel était Turménies.}} {\textsc{-
Retrait}} \footnote{Action en justice pour reprendre l'hôtel de Marsan
  qui avait été vendu.} {\textsc{de l'hôtel de Marsan.}} {\textsc{-
Mariage de La Noue avec M\textsuperscript{me} de Chevry.}} {\textsc{-
Quelles gens c'étaient.}} {\textsc{- Fruits amers du Mississipi.}}
{\textsc{- Rare contrat de mariage du marquis d'Oyse.}} {\textsc{- Dreux
obtient la survivance de sa charge de grand maître des cérémonies pour
son fils, et le marie malheureusement.}} {\textsc{- Mort du prince
Vaïni.}} {\textsc{- Mort et caractère du comte de Peyre.}} {\textsc{- Sa
charge de lieutenant général de Languedoc donnée pour rien à Canillac.}}
{\textsc{- Mort de la comtesse du Rouvre\,; curiosités sur elle.}}
{\textsc{- Mort et singularités de la marquise d'Alluy.}} {\textsc{-
Mort de l'abbé Gautier.}} {\textsc{- Mort et détails du célèbre Valero y
Losa, de curé de campagne devenu, sans s'en être douté, évêque, puis
archevêque de Tolède.}} {\textsc{- Éloge du P. Robinet, confesseur du
roi d'Espagne, et son renvoi.}} {\textsc{- Division entre le roi
d'Angleterre et le prince de Galles\,; sa cause\,; leur apparent
raccommodement.}} {\textsc{- Duc de La Force, choisi pour aller faire
les compliments à Londres, n'y va point, parce que le roi d'Angleterre
ne veut point de cet éclat.}} {\textsc{- Masseï à Paris, depuis nonce en
France\,; sa fortune, son caractère.}} {\textsc{- Les Vénitiens se
raccommodent avec le roi et rétablissent les Ottobon.}} {\textsc{- État,
intrigues, audace des bâtards du prince de Montbéliard, qui veulent être
ses héritiers et légitimes.}}

~

Malgré la situation des finances, il reprit à M. le duc d'Orléans un
nouveau débordement de pensions. Il en donna une de six mille livres, et
une autre de quatre mille livres attachée au grade de lieutenant général
et à celui maréchal de camp, avec cette explication\,: qu'elles seraient
incompatibles avec un gouvernement ou avec une autre pension\,; mais
que, si la pension était moindre, elle serait portée jusqu'à cette
fixation. Cela allait bien loin au grand nombre et n'en obligeait aucun
en particulier. La vieille Montauban, dont il a été quelquefois parlé
ici, en eut une de vingt mille livres, et M. de Montauban, cadet du
prince de Guéméné, une de six mille. La duchesse de Brissac, soeur de
Vertamont, qui était fort pauvre, et que son frère, premier président du
grand conseil, logeait et nourrissait, en eut une aussi de six mille
livres. M\textsuperscript{me} de Coetquen, du Puy-Vauban, Polastron, la
fille de feu Puysieux, veuve de Blanchefort, grand joueur, et son fils,
en eurent chacun une de quatre mille livres\,; et huit ou dix autres
personnes qui trois, qui deux mille francs. J'en obtins une de huit
mille livres pour M\textsuperscript{me} la maréchale de Lorges, et une
de six mille livres pour la maréchale de Chamilly, dont le Mississipi
avait fort dérangé les affaires. M. de Soubise et le marquis de Noailles
eurent chacun deux cent mille livres en présent. Jusqu'à Saint-Geniez,
sortant de la Bastille et relégué à Beauvais, ayant d'abord été destiné
fort loin, eut une pension de mille francs. Tout le monde, en effet,
aurait eu besoin d'une augmentation de revenu, par l'extrême cherté où
les choses les plus communes et les plus indispensables, et toutes
autres natures de choses étaient montées, qui, quoiqu'à la fin peu à peu
diminuées, sont demeurées jusqu'à aujourd'hui bien au-dessus de ce
qu'elles étaient avant ce Mississipi. Le marquis de Châtillon, qui a
fait depuis une si grande fortune, eut aussi six mille livres de pension
en quittant son inspection de cavalerie\,; enfin, La Peyronnie, premier
chirurgien du roi en survivance de Maréchal, eut huit mille livres de
pension.

Un jour de vers la fin d'avril, travaillant avec M. le duc d'Orléans, il
m'apprit le mariage du duc de Lorges avec M\textsuperscript{lle} de
Mesmes, et que le premier président lui en avait demandé son agrément.
Je n'en avais pas ouï dire un mot, et la vérité est que je me mis dans
une étrange colère. On a vu, en différentes occasions, ce que j'ai fait
pour ce beau-frère, et ce qui m'arriva pour l'avoir fait capitaine des
gardes, qu'il était, s'il avait voulu se priver de sa petite maison de
Livry, dont la vente était nécessaire pour parfaire les cinq cent mille
livres à donner au maréchal d'Harcourt, qu'il aima mieux garder. Il
m'était cruel de lui voir épouser la fille d'un homme que je faisais
profession d'abhorrer, et que je ne rencontrais jamais au Palais-Royal
sans le lui témoigner, et quelquefois par les choses les plus fortement
marquées. Je m'en retournai à Meudon où nous étions déjà établis.
J'appris à M\textsuperscript{me} de Saint-Simon cette énormité de son
frère, dont elle ne fut pas moins surprise ni touchée que moi. Je lui
déclarai que de ma vie je ne le verrais ni sa femme, et que je ne
verrais jamais non plus M\textsuperscript{me} la maréchale de Lorges, ni
M. ni M\textsuperscript{me} de Lauzun, s'ils signaient le contrat de
mariage et s'ils se trouvaient à cette noce. Je le dis tout haut
partout, et je m'espaçai sur le beau-père et le gendre sans aucune sorte
de mesure. Cet éclat, qui fut le plus grand qu'il me fut possible, et
qui mit un grand désordre dans une famille jusqu'alors toujours si
intimement unie, et qui vivait sans cesse ensemble, arrêta le mariage
tout court pour un temps\,; mais sans que je visse le duc de Lorges, qui
se flattait de me ramener par ses soeurs, et qui, dans l'embarras à mon
égard de ne vouloir pas rompre ce beau mariage, n'osa se hasarder à me
voir.

M. le duc d'Orléans, persuadé par ceux en qui il avait le plus de
confiance sur les finances, résolut de réduire à deux pour cent toutes
les rentes. Cela soulageait fort les débiteurs\,; mais c'était un grand
retranchement de revenu pour les créanciers qui, sur la foi publique, le
taux approuvé et usité, et la loi des contrats d'emprunts, avaient prêté
à cinq pour cent, et en avaient toujours paisiblement joui. M. le duc
d'Orléans assembla au Palais-Royal plusieurs personnes de divers états
de finance, et résolut enfin avec eux d'en porter l'édit. Il fit du
bruit au parlement, qui résolut des remontrances. Aligre présidait ce
jour-là. Le premier président s'en était allé à sa campagne pour y
faire, disait-il, des remèdes. Il est vrai qu'il avait eu une légère
attaque d'apoplexie pour laquelle il avait été un an auparavant à Vichy.
Il fut bien aise d'éviter de se commettre avec M. le duc d'Orléans après
la cruelle aventure qu'il avait eue avec lui, mais sans quitter prise,
et de laisser agir le parlement, qu'il sentait bien comme tout le monde
que l'imbécillité d'Aligre et le peu de cas qu'en faisait la compagnie
ne serait pas capable de retenir. Mesmes, ravi de voir se préparer de
nouvelles altercations entre le régent et le parlement, {[}leur{]}
voulait laisser la liberté de se reproduire sans y être présent, et ne
revenir qu'ensuite pour y jouer son personnage accoutumé de modérateur
et de compositeur entre sa compagnie et le régent, pour en tirer de
l'argent\,; ce qu'il ne désespérait pas encore de sa facilité, et
souffler le feu sous main. Huit jours après la résolution prise des
remontrances, Aligre, à la tête de la députation du parlement, les porta
par écrit au roi, et les lui laissa, après lui avoir fait un fort plat
compliment\,; c'était le 17 avril. Ces remontrances n'ayant point eu de
succès, le parlement s'assembla le 22 et résolut de ne point enregistrer
l'édit, et de faire de nouvelles remontrances. Au sortir de la séance,
les gens du roi vinrent au Palais-Royal rendre compte de ce qui venait
d'être résolu. M. le duc d'Orléans leur répondit court et sec qu'on ne
changerait rien à la résolution qui avait été prise, et les laissa
aussitôt.

Il permit à Rion de revenir à Paris, dont il avait reçu défense de
s'approcher, étant à l'armée du maréchal de Berwick en Navarre, lors de
la mort de M\textsuperscript{me} la duchesse de Berry. Sa présence au
retour de cette campagne, sitôt après cette mort, aurait réveillé bien
des discours. On crut l'intervalle assez long pour qu'on ne songeât plus
à rien. Sa présence, après tout ce qui s'était passé, ne pouvait pas
être agréable au Palais-Royal, et devait l'embarrasser lui-même. Il ne
fit donc qu'y paraître, se montra peu ailleurs, et mena une vie conforme
à son humeur, c'est-à-dire de plaisir, mais particulière, fort voisine
de l'obscurité. Il était fort à son aise, quoique le Mississipi fût venu
un peu tard pour lui\,; il ne garda guère son régiment et ne songea plus
à servir.

À force de tourner et retourner ce Mississipi de tout sens, pour ne pas
dire à force de jouer des gobelets sous ce nom, on eut envie, à
l'exemple des Anglais, de faire dans ces vastes pays des établissements
effectifs. Ce fut pour les peupler qu'on fit à Paris et dans tout le
royaume des enlèvements de gens sans aveu et des mendiants valides,
hommes et femmes, et de quantité de créatures publiques. Si cela eût été
exécuté avec sagesse, discernement, les mesures et les précautions
nécessaires, cela aurait rempli l'objet qu'on se proposait, et soulagé
Paris et les provinces d'un lourd fardeau inutile et souvent
dangereux\,; mais on s'y prit à Paris et partout ailleurs avec tant de
violence et tant de friponnerie encore pour enlever qui on voulait, que
cela excita de grands murmures. On n'avait pas eu le moindre soin de
pourvoir à la subsistance de tant de malheureux sur les chemins, ni même
dans les lieux destinés à leur embarquement\,; on les enfermait les
nuits dans des granges sans leur donner à manger, et dans les fossés des
lieux où il s'en trouvait, d'où ils ne pussent sortir. Ils faisaient des
cris qui excitaient la pitié et l'indignation\,; mais les aumônes n'y
pouvant suffire, moins encore le peu que les conducteurs leur donnaient,
{[}cela{]} en fit mourir partout un nombre effroyable. Cette inhumanité,
jointe à la barbarie des conducteurs, à une violence d'espèce
jusqu'alors inconnue et à la friponnerie d'enlèvement de gens qui
n'étaient point de la qualité prescrite, mais dont on se voulait
défaire, en disant le mot à l'oreille et mettant de l'argent dans la
main des préposés aux enlèvements, {[}de sorte{]} que les bruits
s'élevèrent avec tant de fracas, et avec des termes et des tons si
imposants qu'on trouva que la chose ne se pouvait plus soutenir. Il s'en
était embarqué quelques troupes, qui ne furent guère mieux traitées dans
la traversée. Ce qui ne l'était pas encore fut lâché et devint ce qu'il
put, et on cessa d'enlever personne. Law, regardé comme l'auteur de ces
enlèvements, devint fort odieux, et M. le duc d'Orléans eut à se
repentir de s'y être laissé entraîner.

Châteauneuf, qui avait présidé à la commission de Nantes, revint en ce
temps-ci avec tous ceux qui l'avaient composée, mais pour subsister
encore, et s'assembler à l'Arsenal pour achever de juger ceux des
exceptés de l'amnistie qui ne l'avaient pas été à Nantes\,; et peu après
le maréchal de Montesquiou fut rappelé du commandement de Bretagne, où
il avait eu le malheur de se barbouiller beaucoup et de ne contenter
personne.

M. le comte de Charolais arriva enfin de ses longs voyages, M. le Duc,
content de ce qu'il avait obtenu pour lui, lui avait mandé de revenir,
et le fut attendre à Chantilly avec les familiers de la maison.
Turménies s'y trouva avec eux, il avait été maître des requêtes et
intendant de province avec réputation, et y aurait fait son chemin au
gré de tout le monde\,; mais à la mort de son père, qui était garde du
trésor royal, il préféra le solide si abondant de cette charge aux
espérances des emplois qu'il avait. C'était un garçon de beaucoup
d'esprit, de lecture et de connaissances, d'un naturel libre et gai,
aimant le plaisir, mais avec mesure et pour la compagnie et pour le
temps, fort mêlé avec la meilleure compagnie de la cour et de la ville,
habile, capable, droit et obligeant dans sa charge, sans se faire
valoir, estimé et accrédité avec les ministres, fort bien avec le
régent, et sur un pied de telle familiarité avec M. le duc et M. le
prince de Conti pères et fils, qu'ils vouvaient tout bon de lui, et ce
qu'ils n'auraient souffert de personne. Le voisinage de l'Ile-Adam, la
chasse, la table, l'avait mis sur ce ton avec les pères\,; il avait su
se le conserver avec les fils. C'était un homme qui sentait très bien la
force de ses paroles, mais qui ne retenait pas aisément un bon mot.
L'impunité avait aiguisé sa hardiesse, qui d'ailleurs n'était que
liberté, sans aucun air d'insolence et sans jamais se déplacer avec
personne. Il était petit, grosset, le cou fort court, la tête dans les
épaules, avec de grands cheveux blonds qui lui donnaient encore l'air
plus engoncé, et qui lui avaient valu le sobriquet de Courtcollet. M. le
Duc, averti que M. son frère arrivait, alla, suivi de toute la
compagnie, le recevoir au débarquer de sa voiture et l'embrasser. Tout
ce qui était là les environna et s'empressa à faire sa révérence\,;
après les premiers mots entre les deux frères, M. le Duc lui présenta la
compagnie, que M. le comte de Charolais se contenta de regarder fort
indifféremment sans dire un seul mot à personne, pendant un assez long
temps que ce cercle demeura autour d'eux, dans la place où il avait mis
pied à terre dans la cour. Turménies, voyant ce qui se passait et s'en
ennuyant, se tourne à la compagnie\,: «\,Messieurs, lui dit-il
froidement, mais tout haut, faites voyager vos enfants, et dépensez-y
bien de l'argent,\,» et tout de suite passa d'un autre côté. Cet
apophtegme fit du bruit, et courut fort. Il ne s'en défendit point, et
M. le Duc et M. le comte de Charolais n'en firent que rire. M. le Duc
devait y être accoutumé.

Au commencement des actions de Law, M. le Duc se vanta chez lui, devant
assez de monde, et avec complaisance d'une quantité considérable qu'il
en avait eue. Chacun se taisait, lorsque Courtcollet, impatienté\,:
«\,Fi, monsieur, répondit-il, votre bisaïeul n'en a jamais eu que cinq
ou six, mais qui valaient bien mieux que toutes les vôtres.\,» Chacun
baissa les yeux, et M. le Duc se prit à rire, sans lui en savoir plus
mauvais gré. Il en a quelquefois lâché de bonnes à des ministres du feu
roi, et depuis la régence à M. le duc d'Orléans lui-même, qui n'en
faisait que rire aussi. Il ne vécut que peu d'années après, quoique
point vieux, et fut fort regretté même pour les affaires de sa gestion.
Il ne laissa point d'enfants. M. de Laval, le même de la conspiration du
duc et de la duchesse du Maine, épousa sa soeur qui était veuve de
Bayez, dont il a eu beaucoup de biens et des enfants. Les apophtegmes de
Turménies n'étaient pas réservés aux princes du sang. Il ne s'en
contraignait guère pour personne et avec cela rien moins
qu'impertinent\,; il avait trop d'esprit et de monde pour l'être.

Une affaire purement particulière fit alors grand bruit dans le monde.
Matignon et M. de Marsan avaient épousé les deux soeurs, filles uniques
et sans frères du frère aîné de Matignon\,: lui l'aînée, M. de Marsan la
cadette, veuve alors avec des enfants de M. de Seignelay, ministre et
secrétaire d'État, fils aîné de M. Colbert. Un intérêt commun les avait
étroitement unis, c'était l'amitié de Chamillart, dont ils avaient tiré
des trésors en toute espèce d'affaires de finance. Le comte de Marsan
fit par son testament M. de Matignon tuteur de ses enfants, avec
l'autorité la plus étendue et les plus grandes marques de confiance\,;
et tout le monde est convenu que le comte de Matignon y répondit sans
cesse par tous les soins, l'application et les tendresses d'un véritable
père, et le succès d'un homme habile et accrédité. Le comte de Marsan,
qui n'avait de soi point de bien, ne s'en était fait que d'industrie, de
grâces et de rapines, avait mangé à l'avenant, et laissé ses affaires en
mauvais état. Matignon estima qu'un effet tel que l'hôtel de Marsan, à
Paris, était trop pesant pour des enfants en bas âge, dont le prix
aiderait fort à liquider les biens, et crut aussi, à la conduite qu'il
avait eue dans leurs affaires, la\footnote{L'auteur a fait \emph{hôtel}
  féminin.} pouvoir acheter quoique tuteur. Il l'acheta donc, y dépensa
beaucoup, y alla loger et céda la sienne au maréchal son frère. M. de
Marsan était mort en 1708, veuf pour la seconde fois depuis près de neuf
ans. Le prince de Pons, son fils aîné, était né en 1696\,; par
conséquent il avait vingt-quatre ans en cette année 1720, et il était
marié en 1714 à la fille cadette du duc de Roquelaure. Il pria le duc
d'Elboeuf d'aller dire à Matignon de sa part qu'il se croyait obligé de
retirer l'hôtel de Matignon, qui était l'hôtel de Marsan que le comte de
Matignon avait achetée et payée, mais qu'il ne voulait point que M. de
Matignon songeât à en sortir, et qu'il l'y laisserait toute sa vie. Le
comte de Matignon, aussi surpris qu'indigné du compliment, répondit tout
court qu'il espérait d'assez bonnes raisons pour ne devoir pas craindre
ce retrait\,; qu'il le remerciait de la manière polie dont il lui avait
parlé\,; mais qu'il l'assurait en même temps qu'il ne profiterait pas de
la grâce que le prince de Pons prétendait lui faire\,; et qu'il pouvait
lui dire que, s'il était assez malheureux pour perdre ce procès, il
quitterait sa maison le lendemain et n'y remettrait jamais le pied. Les
procédures ne tardèrent pas après de la part du prince de Pons, qui en
fut extrêmement blâmé, et universellement de tout le monde. Matignon
soutint le procès\,; tout y était pour lui, hors la lettre de la règle.
Il le perdit donc, uniquement par la qualité de tuteur qui acquiert de
son mineur, et ce fut au grand regret du public et des juges mêmes. Le
jour même de l'arrêt, Matignon retourna loger chez le maréchal son
frère, et de dépit acheta et rebâtit presque la superbe maison que son
fils occupe, et qu'il a si grandement augmentée et ornée. Le comte de
Matignon n'eut pas le temps d'y loger. Elle était tout près de le
pouvoir recevoir lorsqu'il mourut chez le maréchal son frère, en janvier
1725. Ce ne fut qu'à sa mort qu'il revit le prince de Pons et son frère,
avec qui les Matignon sont depuis demeurés fraîchement.

Il y a des choses qui occupent dans leur temps et qui vieillissant
s'anéantissent. Je n'en puis toutefois omettre une de ce genre. Il y
avait une petite nièce par femmes de M. de Fénelon, archevêque de
Cambrai, qui déjà veuve à peine mariée, sans enfants et sans biens,
avait une figure aimable, l'air et le goût du monde, un manége infini et
beaucoup d'intrigue, et qui, sans avoir été religieuse et coureuse comme
la Tencin, eut cette similitude avec elle qu'elle fit pour M. de Cambrai
et son petit troupeau, conséquemment pour M\textsuperscript{me} Guyon et
sa petite église, le même personnage que l'ambition du frère et de la
soeur fit faire à celle-ci pour la constitution. La veuve dont je parle
avait trouvé ainsi le moyen de rassembler chez elle bonne compagnie,
mais elle mourait de faim. Elle persuada à un vieil aveugle qui était
riche et qui s'appelait Chevry de l'épouser pour avoir compagnie et
charmer l'ennui de son état. Il y consentit et lui fit toutes sortes
d'avantages. Il se flatta d'autant plus de mener avec elle une vie
agréable qu'elle aimait le monde, le jeu, la parure, et néanmoins fort
dévote, se disait-elle, et disaient ses amis, et il le fallait bien
puisque en cela consistait toute son existence et sa considération.
Chevry, presque aveugle quand il l'épousa, le devint bientôt après tout
à fait. Il fut doux, bon homme, s'accommoda de tout, et quoique compté
presque pour rien, il avait toute sorte de complaisances, hors celle de
mourir, et il ennuyait fort sa femme et cette troupe d'amis. Il mourut
enfin, et ce fut un grand soulagement dans la maison, et une grande joie
pour les amis qui trouvaient là une bonne maison et opulente, où rien ne
contrariait plus leur conversation. Mais les vapeurs qui avaient gagné
la dame pendant la vie de son aveugle ne s'en allèrent pas avec lui. À
ces vapeurs, qui étaient devenues énormes, se joignit la gravelle, qui,
mêlées, la nettoient dans des états étranges, après quoi, presque en un
instant, il n'y paraissait pas. Une pointe de merveilleux faisait
merveilles parmi ce monde qui abondait chez elle\,; elle était les
délices et la vénération de toute cette petite église et le ralliement
de tout ce qui y tenait. C'était là où se tenait le conseil secret\,; et
comme il s'y joignait souvent d'autre bonne compagnie, sa maison était
devenue un petit tribunal qui ne laissait pas d'être compté dans
Paris\,; tout cela flattait sa vanité, l'amusait et l'occupait
agréablement, avec ce talent de s'attirer du monde avec choix et de
soutenir cet abord par la bonne chère. Mais elle n'avait jamais eu de
mari, et elle s'en donna un dont on ne l'aurait jamais soupçonnée, la
petite église par vénération, les autres commensaux par la croire de
meilleur goût, tous par l'état de sa santé. La Noue, espèce de chevalier
d'industrie, s'était introduit chez elle par hasard, la table l'y attira
souvent. Il était frère de Teligny, que la faim avait fait gouverneur de
M. le comte de Clermont, et d'un lieutenant des gardes du corps.
C'étaient de fort simples gentilshommes et fort pauvres, leur nom est
Cordouan\,; j'en ai parlé ailleurs. Il n'avait d'esprit qu'un simple
usage de médiocre monde, et anciennement de jeu et de galanterie
bourgeoise, et rien plus, avec un peu d'effronterie. Il avait servi
toute sa vie dans le subalterne, avait attrapé une place d'écuyer à
l'hôtel de Conti, puis le régiment de ce prince dont la jalousie lui ôta
l'un et l'autre en le chassant de chez lui. M. le duc d'Orléans en eut
pitié, et lui donna une inspection. Ce fut donc ce vieux belâtre qu'elle
épousa, mais dans le dernier secret, tant elle en fut honteuse. Ce
secret dura quatre ans, après lesquels ce beau mariage se déclara. Ce
fut un étrange vacarme parmi les amis de la maison qui, de ce moment, ne
fut plus, ni depuis, à beaucoup près, si fréquentée, et déchut enfin de
cet état de tribunal où tout ce qui se passait était jugé, et où elle
présidait avec empire. Le mari, déclaré, fut toujours amant soumis et
respectueux, mais cela ne dura guère, elle ne put soutenir une telle
décadence. Elle mourut, et La Noue ne profita de rien.

L'extrême folie d'une part, et l'énorme cupidité de l'autre, firent en
ce temps-ci le plus étrange contrat de mariage qui se soit peut-être
jamais vu. C'est un échantillon de celle que le système de Law alluma en
France, et qui mérite d'avoir place ici. Qui pourrait, et qui en
voudrait raconter les effets, les transmutations de papiers, les marchés
incroyables, les nombreuses fortunes dans leur immensité, et encore dans
leur inconcevable rapidité, la chute prompte de la plupart de ces
enrichis par leur luxe et leur démence, la ruine de tout le reste du
royaume, et les plaies profondes qu'il en a reçues et qui ne guériront
jamais, ferait sans doute la plus curieuse et la plus amusante histoire,
mais la plus horrible en même temps, et la plus monstrueuse qui fût
jamais. Voici donc, entre autres prodiges, le mariage dont il s'agit. Le
contrat en fut dressé et signé entre le marquis d'Oyse, âgé lors de
trente-trois ans, fils et frère cadet des ducs de Villars-Brancas, avec
la fille d'André, fameux Mississipien qui y avait gagné des monts d'or,
laquelle n'avait que trois ans, à condition de célébrer le mariage dès
qu'elle en aurait douze. Les conditions furent cent mille écus,
actuellement payés\,; vingt mille livres par an jusqu'au jour du
mariage\,; un bien immense par millions lors de la consommation\,; et
profusions en attendant aux ducs de Brancas père et fils. Les discours
ne furent pas épargnés sur ce beau mariage. Que ne fait point faire
\emph{auri sacra fames\,?} Mais l'affaire avorta avant la fin de la
bouillie de la future épouse, par la culbute de Law. Les Brancas, qui
s'en étaient doutés, le père et les deux fils, s'étaient bien fait payer
d'avance\,; le comble fut que les suites de cette affaire produisirent
des procès plus de quinze ans après, qui furent soutenus sans honte. Ces
Brancas-là n'y étaient pas sujets.

M. le duc d'Orléans, qui prodiguait tout de plus en plus, accorda à
Dreux la survivance de sa charge pour son fils. Ce n'était pas pour le
mérite du père qui n'était pas imposant, et dont la conduite pleine
d'ignorance, de brutalité, et qui pis est d'infidélité dans cette
charge, n'en méritait pas la conservation, bien loin d'une survivance à
un fils de vingt ans. Ce ne pouvait être le désir de gratifier le
parlement en une de ses bonnes et anciennes familles\,; celle-ci qui
venait de peu y était toute nouvelle, et les services militaires du
père, aussi borné qu'il l'était, n'auraient pu durer longtemps sans
l'appui de Chamillart son beau-père qui le poussa, et par la
considération duquel, même après sa chute, sort gendre continua d'être
employé dans l'état des armées parmi le grand nombre, et où, à la valeur
près, il fut toujours compté pour rien. Ce fut donc à Chamillart encore
que cette survivance fut accordée. Cette charge de grand maître des
cérémonies fut créée par Henri III pour M. de Rhodes, et il est vrai
qu'elle ne convient qu'à des gens de la première qualité. MM. de Rhodes
l'ont conservée jusqu'au dernier, qui, se voyant perclus de goutte et
sans enfants, la vendit à Blainville, frère de Saignelay, ministre et
secrétaire d'État, duquel Chamillart la fit acheter par son gendre pour
le recrépir et pour, à l'abri fictif de cette charge et plus du crédit
du beau-père qui fit tout et qui était lors à l'apogée de sa faveur,
faire entrer sa fille dans les carrosses, manger et aller à Marly. Peu
après cette survivance, Dreux maria son fils à une autre Dreux, fille du
frère aîné de Nancré, mort capitaine des Suisses de M. le duc d'Orléans,
dont il a été fait plus d'une fois mention. Cette fille était
puissamment riche et tenue de si court qu'on ne la voyait presque
jamais, et non sans cause, mais qu'on avait su cacher si bien que
personne n'en eut de soupçon. Elle éclata dès le lendemain des noces par
un accès public d'extrême folie qui, suivi de quantité d'autres,
obligèrent de l'enfermer dans un couvent. Mais le mari par leur parenté
hérita d'elle.

Le prince Vaïni, chevalier de l'ordre par la belle cause qui en a été
rapportée ici en son temps, mourut à Rome. On a suffisamment fait
connaître quel il était pour n'avoir rien à y ajouter. Le merveilleux
est que, ayant été trompé à son titre, à sa naissance, à son mérite, à
sa considération à Rome qui était nulle, le fils y fut fait aussi
chevalier de l'ordre et reçu par le duc de Saint-Aignan pendant son
ambassade, lequel fils n'y brilla pas plus que le père.

Le vieux comte de Peyre mourut enfin chez lui, en Languedoc, où il était
l'un des trois lieutenants généraux de cette province, mais sans
fonction. C'était un grand homme de bonne mine, riche et grand tyran de
province, et avec lequel il ne faisait bon pour personne d'avoir
affaire. Il n'avait point de brevet de retenue. Sa charge, qui est de
vingt mille livres, fut donnée sur-le-champ à Canillac, à qui M. le duc
d'Orléans l'avait déjà accordée une fois sur un faux bruit qui se
répandit de la mort de ce comte de Peyre.

En même temps et en même pays mourut aussi la vieille comtesse du Roure,
qui était fille de Claude Marie du Guast, dit le comte d'Artigny et de
Marie Cottelier\footnote{Nous avons reproduit exactement le texte du
  manuscrit\,; mais il est nécessaire de rectifier les erreurs de noms
  qu'il présente. La comtesse du Roure était Claude-Marie du Gast ou du
  Guast, fille d'Achille du Guast, seigneur d'Artigny et de Montgauger
  en Touraine, et de Marie d'Argouge-le-Coutelier.}. Elle fut fille
d'honneur de Madame, première femme de Monsieur, sous le nom de
M\textsuperscript{lle} d'Artigny, compagne et amie intime de
M\textsuperscript{lle} de La Vallière, dont la faveur lui fit épouser en
1666 Pierre Scipion de Beauvoir de Grimoard, frère de la mère du
cardinal de Polignac et fils aîné du comte du Roure, chevalier de
l'ordre en 1661, ainsi que le vicomte de Polignac, son beau-frère,
duquel le père l'avait été aussi en 1633. Par ce mariage le comte du
Roure fit passer à son fils sa charge de lieutenant général de Languedoc
et son gouvernement du Pont-Saint-Esprit. Il y eut plusieurs enfants de
ce mariage de M\textsuperscript{lle} d'Artigny avec le comte du Roure,
dont l'aîné eut aussi la lieutenance générale de Languedoc et le
gouvernement du Pont-Saint-Esprit en épousant la fille du duc de La
Force dont Monseigneur avait été publiquement amoureux, et le fils de ce
dernier mariage, qui n'a point eu les charges de son père tué à la
bataille de Fleurus, a épousé une fille du maréchal duc de Biron qui est
dame du palais de M\textsuperscript{me} la Dauphine. Cette vieille
comtesse du Roure-Artigny, occasion de cet article, était une intrigante
de beaucoup d'esprit et que la faveur de M\textsuperscript{lle} de La
Vallière avait accoutumée à beaucoup de hauteur. Elle se trouva mêlée
dans beaucoup de choses avec la comtesse de Soissons, qui les firent
chasser de la cour, puis avec la même dans les dépositions de la
Voisin\footnote{La Voysin ou La Voisin, fut brûlée vive le 22 février
  1680. On trouve, dans les \emph{Lettres de M\textsuperscript{me} de
  Sévigné}, les détails les plus curieux sur le procès et le supplice de
  cette célèbre empoisonneuse.} qui firent sortir la comtesse de
Soissons du royaume pour toujours. Cette dernière aventure pensa mener
loin la comtesse du Roure. Elle en fut quitte néanmoins pour l'exil en
Languedoc, où elle a passé le reste de sa vie, excepté un voyage de peu
de mois qu'elle obtint de faire à Paris quelques années avant sa mort.
On la craignait partout. Elle vivait d'ordinaire dans un château, et son
mari dans un autre.

La marquise d'Alluye mourut en même temps au Palais-Royal à Paris. Elle
s'appelait de Meaux du Fouilloux \footnote{Bénigne de Meaux du Fouilloux
  ou de Fouilloux. D'après les documents contemporains,
  M\textsuperscript{lle} du Fouilloux était une des filles d'honneur de
  la reine-mère. On lit dans le recueil de Maurepas (ms. B. I., t. II,
  p.~271) des vers sur les filles de la reine, où il est question de
  M\textsuperscript{lle} du Fouilloux\,: Fouilloux, sans songer à
  plaire, / Plaît pourtant infiniment / Par un air libre et charmant.},
avait été aussi fille d'honneur de Madame, première femme de Monsieur,
et amie de Pille d'Artigny dont on vient de parler, et sa compagne\,;
elle épousa, en 1667, n'étant plus jeune, mais belle, le marquis
d'Alluye, fils et frère de Charles et de François d'Escoubleau, marquis
de Sourdis, chevaliers de l'ordre, l'un en 33, l'autre en 88. D'Alluye,
qui était l'aîné, eut le gouvernement d'Orléanais de son père, fut
encore plus mêlé que sa femme dans l'affaire de la Voisin, furent
longtemps exilés, et le mari, qui mourut sans enfants en 1690, n'eut
jamais permission de voir le roi, quoique revenu à Paris. Sa femme, amie
intime de la comtesse de Soissons et des duchesses de Bouillon et
Mazarin, passa sa vie dans les intrigues de galanterie, et quand son âge
l'en exclut pour elle-même, dans celles d'autrui. Le marquis d'Effiat,
dont il a été si souvent mention ici, avait épousé une soeur de son
mari, dont il n'avait point eu d'enfants, et qu'il perdit de bonne
heure. Il protégea la marquise d'Alluye dans la cour de Monsieur, avec
qui elle fut fort bien, et avec Madame toute sa vie. C'était une femme
qui n'était point méchante\,; qui n'avait d'intrigues que de galanterie,
mais qui les aimait tant que, jusqu'à sa mort, elle était le rendez-vous
et la confidente des galanteries de Paris, dont, tous les matins, les
intéressés lui rendaient compte. Elle aimait le monde et le jeu
passionnément, avait peu de bien et le réservait pour son jeu. Le matin,
tout en discourant avec les galants qui lui contaient les nouvelles de
la ville, ou les leurs, elle envoyait chercher une tranche de pâté ou de
jambon, quel que fois un peu de salé ou des petits pâtés, et les
mangeait. Le soir, elle allait souper et jouer où elle pouvait, rentrait
à quatre heures du matin, et a vécu de la sorte grasse et fraîche, sans
nulle infirmité jusqu'à plus de quatre-vingts ans qu'elle mourut d'une
assez courte maladie, après une aussi longue vie, sans souci, sans
contrainte et uniquement de plaisir. D'estime, elle ne s'en était jamais
mise en peine, sinon d'être sûre et secrète au dernier point\,; avec
cela, tout le monde l'aimait, mais il n'allait guère de femmes chez
elle. La singularité de cette vie m'a fait étendre sur elle.

L'abbé Gautier, dont il est si bien et si souvent parlé dans ce qui a
été donné ici, d'après M. de Torcy, sur les négociations de la paix avec
la reine Anne, et de celle d'Utrecht, mourut dans un appartement que le
feu roi lui avait donné dans le château neuf de Saint-Germain, avec des
pensions et une bonne abbaye. Il s'y était retiré aussitôt après ces
négociations où il avait été si heureusement employé, après en avoir
ouvert lui-même le premier chemin, et rentra en homme de bien modeste et
humble, dans son état naturel, et y vécut comme s'il ne se fût jamais
mêlé de rien, avec une rare simplicité, et qui a peu d'exemples en des
gens de sa sorte, qui, dans le maniement des affaires les plus
importantes et les plus secrètes, dont lui-même avait donné la première
clef, sans s'intriguer, s'était concilié l'estime et l'affection du roi
et de ses ministres, de la reine Anne et des siens, et des
plénipotentiaires qui travaillèrent à ces deux paix.

Le célèbre archevêque de Tolède mourut aussi en ce même temps\,; il
s'appelait don Francisco Valero y Losa, et il était simple curé d'une
petite bourgade. Il y rendit des services si importants pour soutenir
les peuples dans le fort de la guerre et des malheurs, les exciter en
faveur du roi d'Espagne, trouver des expédients pour les marches et les
subsistances, avoir des avis sûrs de ce que faisaient et projetaient les
ennemis, que les généraux et les ministres ne pouvaient assez louer son
zèle, son industrie, sa vigilance et sa sagesse. Rien de tant de soins
ne dérangea sa piété, les devoirs de sa paroisse, sa modestie, son
désintéressement. Ses amis, l'orage passé, le pressèrent vainement
d'aller à la cour représenter ses services. Il ne prit pas seulement la
peine d'en faire souvenir. Dans cette inaction qui relevait si
grandement son mérite, le P. Robinet, lors confesseur du roi d'Espagne,
qui ne l'avait pas oublié, en fit souvenir Sa Majesté Catholique à la
vacance de l'évêché de Badajoz, qui le lui donna. Le bon curé, qui n'y
avait jamais songé, l'accepta, s'y retira, et y vécut en excellent
évêque. Ce fut de ce siège que le même confesseur le fit passer à celui
de Tolède, avec l'applaudissement de toute la cour et l'acclamation de
toute l'Espagne. Le prélat y avait aussi peu songé qu'il avait fait à
celui de Badajoz. Il fut dans ce premier siège de toutes les Espagnes
aussi modeste qu'il avait été dans sa cure, et il y fut l'exemple de
tous les évêques d'Espagne, l'exemple de la cour et celui de tout le
royaume. Sa promotion à Tolède perdit le confesseur.

Le cardinal del Giudice, aussi étroitement uni à la princesse des Ursins
alors, qu'ils devinrent ennemis dans la suite, voulait ce riche et grand
archevêché\,; il le demandait hautement, et M\textsuperscript{me} des
Ursins en fit sa propre affaire. Le roi y consentait, lorsque son
confesseur osa lui représenter avec la plus généreuse fermeté quel
affront il ferait à la nation espagnole, à l'amour et aux prodiges
d'efforts de laquelle il devait sa couronne, s'il la frustrait du
premier et du plus grand archevêché, pour le donner à un étranger, qui
déjà tenait de lui le riche archevêché de Montreal en Sicile, et tant de
pensions et d'autres grâces, et fit si bien valoir le mérite, les
services, la piété, le désintéressement de l'évêque de Badajoz, qu'il
emporta pour lui l'archevêché de Tolède. Ce trait et les louanges qu'il
en reçut outra le cardinal, et plus que lui encore M\textsuperscript{me}
des Ursins qui ne pouvait souffrir de résistance à son pouvoir et à ses
volontés. Ce père ne se mêlait de rien que des bénéfices, ne lui donnait
nul ombrage, vivait avec tout le respect, la modestie, la retenue
possible avec elle, avec le cardinal, avec tous les gens en place\,;
mais, comme il ne tenait point à la sienne, il ne faisait sa cour à
personne. M\textsuperscript{me} des Ursins, qui avait déjà éprouvé
quelque peu de sa droiture et de sa fermeté, qui le voyait estimé et
adoré de tout le monde, craignit tout de ce dernier trait, outre
l'extrême dépit de se voir vaincue après s'être déclarée\,; aussi ne lui
pardonnât-elle pas. Elle sut si bien travailler qu'elle fit renvoyer cet
excellent homme environ un an après, et fit à l'Espagne une double et
profonde plaie par la perte qu'elle lit d'un homme si digne d'une si
importante place, et par donner lieu au choix d'un successeur si
différent, et qu'elle-même avait déjà chassé de cette même place. Ce fut
le P. Daubenton, dont on a suffisamment parlé ici dans ce qui y a été
donné d'après M. de Torcy, pour voir qu'on ne dit rien de trop sur le
choix de ce terrible jésuite, dont j'aurai encore lieu de parler, si
Dieu me donne le temps d'écrire mon ambassade d'Espagne et de conduire
ces Mémoires jusqu'au but que je me suis proposé.

Le P. Robinet, véritablement soulagé de n'être plus dans une cour et
dans les affaires, revint en France, et ne se soucia ni de lieu ni
d'emploi. Il fut envoyé à Strasbourg, où il se fit aimer et estimer
comme il avait fait partout, y vécut dans une grande retraite et dans
une grande tranquillité, et y mourut saintement après plusieurs années.
On le regrettait encore en Espagne lorsque j'y ai été, et j'en ai ouï
souvent faire l'éloge. Il faut dire que ce P. Robinet est le seul
confesseur du roi d'Espagne qui ait mérité de l'être, qui en fût digne à
tous égards, et qui ait été goûté, aimé, estimé et honoré de toute la
cour et de toute l'Espagne sans aucune exception.

Il y avait eu depuis longtemps une espèce de guerre déclarée entre le
roi d'Angleterre et le prince de Galles, qui avait éclaté avec de
fréquents scandales, et qui avait partialisé la cour et fait du bruit
dans le parlement. Georges s'était emporté plus d'une fois contre son
fils avec indécence. Il y avait longtemps qu'il l'avait fait sortir de
son palais et qu'il ne le voyait plus. Il lui avait tellement retranché
ses pensions qu'il avait peine à subsister, tellement que le roi eut le
dégoût que le parlement lui en assigna, même abondamment. Jamais le père
n'avait pu souffrir ce fils, parce qu'il ne le croyait point à lui. Il
avait plus que soupçonné la duchesse sa femme, fille du duc de
Wolfenbuttel, d'être en commerce avec le comte Koenigsmarck. Il le
surprit un matin sortant de sa chambre, le fit jeter sur-le-champ dans
un four chaud, et enferma sa femme dans un château, bien resserrée et
gardée, où elle a passé le reste de sa vie. Le prince de Galles, qui se
sentait maltraité pour une cause dont il était personnellement innocent,
avait toujours porté avec impatience la prison de sa mère et les effets
de l'aversion de son père. La princesse de Galles, qui avait beaucoup de
sens, d'esprit, de tour et de grâces, avait adouci les choses tant
qu'elle avait pu, et le roi n'avait pu lui refuser son estime, ni se
défendre même de l'aimer. Elle s'était concilié toute l'Angleterre, et
sa cour, toujours grosse, l'était aussi en ce qu'il y avait de plus
accrédité et de plus distingué. Le prince de Galles s'en autorisait, ne
ménageait plus son père, s'en prenait à ses ministres avec une hauteur
et des discours qui à la fin les alarmèrent. Ils craignirent le crédit
de la princesse de Galles, et de se voir attaqués par le parlement qui
se donne souvent ce plaisir. Ces considérations devinrent de plus en
plus pressantes par tout ce qu'ils découvrirent qui se brassait contre
eux, et qui aurait nécessairement rejailli sur le roi. Ils lui
communiquèrent leurs craintes, ils les lui donnèrent, et le conduisirent
à se raccommoder avec son fils à certaines conditions, par l'entremise
de la princesse de Galles, qui de son côté sentait tous les embarras de
faire et de soutenir un parti contre le roi, et qui a voit toujours
sincèrement désiré la paix dans la famille royale. Elle profita de la
conjoncture, se servit de l'ascendant qu'elle avait sur son mari, et
l'accommodement fut conclu. Le roi donna gros au prince de Galles, et le
vit\,; les ministres se sauvèrent, et tout parut oublié.

L'excès où les choses avaient été portées entre eux, qui tenait toute la
nation britannique attentive aux désordres intestins prêts à en éclore,
n'avait pas fait moins de bruit en toute l'Europe, où chaque puissance,
attentive à ce qui en résulterait, tâchait de souffler ce feu, ou de
l'apaiser, suivant son intérêt. La réconciliation fut donc une nouvelle
intéressante pour toute l'Europe. L'archevêque de Cambrai, que je
continuerai d'appeler l'abbé Dubois, parce qu'il ne porta pas longtemps
le nom de son église que son cardinalat vint effacer, en était lors dans
la crise, et très sensible à ce qui se passait à Londres, d'où il
attendait son chapeau par le ricochet du crédit alors très grand du roi
d'Angleterre sur l'empereur, et de la toute-puissance de l'empereur sur
la cour de Rome qui tremblait devant lui, et n'osait lui rien refuser.
Dans la joie du raccommodement entre le père et le fils, Dubois la
voulut témoigner d'une façon éclatante pour faire sa cour au roi
d'Angleterre. Le duc de La Force, qui ne se mêlait plus de finance, qui
voulait toujours se mêler de quelque chose, et qui n'en trouvait pas
d'occasion dans le conseil de régence, où il ne se portait plus rien
d'effectif depuis que la faiblesse du régent l'avait rendu peu à peu si
nombreux, le duc de La Force, dis-je, qui était toujours à l'affût, eut
le vent de ce dessein, et se proposa à Dubois pour aller en Angleterre
par le chausse-pied d'y aller voir sa mère qui y était retirée depuis
longues années à cause de la religion, mais qu'il n'avait pas songé
jusqu'alors d'aller voir depuis qu'elle était sortie du royaume avec la
permission du feu roi. Law servit le duc de La Force auprès de Dubois,
et il fut nommé pour aller en Angleterre faire les compliments du roi et
du régent sur cette réconciliation, sans qu'on pensât à l'inconvénient
de montrer à l'église française de Londres un seigneur catholique, né et
élevé leur frère, qui les avait depuis persécutés, et qui en avait su
tirer parti du feu roi. On sut incontinent en Angleterre la
démonstration de joie qui venait d'être résolue en France. Georges,
outré du retentissement que les éclats de son domestique avaient faits
par toute l'Europe, ne s'accommoda pas de les voir prolonger par le
bruit que ferait cet envoi solennel. Il fit donc prier le régent de ne
lui en envoyer aucun. Comme on ne l'avait imaginé que pour lui plaire,
le voyage du duc de La Force fut presque aussitôt rompu que déclaré. Il
en fut pour un commencement assez considérable de dépense, et pour faire
revenir beaucoup d'équipages qu'il avait déjà fait partir, et l'abbé
Dubois en recueillit auprès du roi d'Angleterre le double fruit de cet
éclat de joie, et de l'avoir arrêté également pour lui plaire.

Masseï, qui avait apporté la barrette au cardinal de Bissy un peu avant
la mort du roi, arriva à Paris. Il était fils du trompette de la ville
de Florence, et avait été petit garçon parmi les bas domestiques du
pape, alors simple prélat. Son esprit et sa sagesse percèrent\,; il
s'éleva peu à peu dans la maison, et de degré en degré devint le
secrétaire confident de son maître, et enfin son maître de chambre quand
il fut cardinal. Sa douceur et sa modestie le firent aimer dans la cour
romaine où son emploi le fit connaître. Il le perdit à l'exaltation de
son maître\,; il était de trop bas aloi pour être maître de chambre du
pape, mais il en conserva toute la faveur et la confiance\,; le pape lui
parlait presque de tout, le consultait et se trouva bien de ses avis. Il
le fit archevêque \emph{in partibus}, pour le mettre à portée d'une
grande nonciature. Il l'avait envoyé dans ce dessein porter la barrette
au cardinal de Bissy, dans l'apogée de la faveur de cet ambitieux
brouillon, et s'en était servi pour s'assurer de l'agrément de la France
pour le recevoir nonce, quand le Bentivoglio, qui l'était, laisserait la
place vacante. En effet il lui succéda, et comme il était honnête homme
il ne lui ressembla en rien. Il se conduisit durant le plus grand feu de
la constitution avec beaucoup de modération, d'honneur et de sagesse, et
se fit généralement aimer et estimer. Il languit longtemps nonce parce
qu'il n'y eut point de promotion pour les nonces pendant le reste de ce
pontificat, et que Benoît XIII, qui était si fort singulier, et qui eût
été meilleur sous-prieur de dominicains que pape, ne voulut jamais faire
aucun nonce cardinal, et disait d'eux qu'ils n'étaient que des
nouvellistes.

Masseï ne montrait pas la moindre impatience, mais en attendant il
mourait de faim\,; car les nonces ont fort peu, et, à ce qu'était
celui-ci, son patrimoine ni ses bénéfices n'y suppléaient pas. Il ne
s'endetta pas le moins du monde, supporta son indigence avec dignité,
mais il l'avouait pour être excusé de la frugalité de sa vie, et s'en
alla sans rien devoir, véritablement regretté de tout le monde. Il
s'était tellement accommodé de la vie de ce pays-ci et du commerce des
honnêtes gens et des personnes considérables qu'il avait su s'attirer,
qu'il était outré de sentir que cela finirait. Il disait franchement
que, s'il était assuré de sa nonciature pour toute sa vie, avec de quoi
la soutenir honnêtement, il ne voudrait jamais la quitter pour la
pourpre, et s'en aller. Aussi fut-il très affligé, quoique arrivé au
cardinalat et tout de suite à la légation de la Romagne. Le nouveau
cérémonial des bâtards, dont Gualterio s'était si mal trouvé, car ils
étaient rétablis alors, empêcha que la calotte lui arrivât à Paris. Dès
que la promotion fut sur le point de se faire, il reçut ordre de prendre
congé, de partir, et d'arriver dans un temps marqué et fort court à
Forli, sa patrie, où il trouverait sa calotte rouge, comme il l'y trouva
en effet\,; ce fut en 1730. Il vécut encore plusieurs années, et passa
quatre-vingts ans. C'était un homme très raisonnable, droit, modeste, et
qui toute sa vie avait eu de fort bonnes moeurs.

Les Vénitiens, brouillés depuis longtemps avec le feu roi, par
conséquent avec le roi son successeur, s'en lassèrent à la fin, et se
raccommodèrent en ce temps-ci. Ottoboni, père du pape Alexandre VIII,
était chancelier de Venise qui est une grande charge et fort importante,
mais attachée à l'état de citadin et la plus haute où les citadins
puissent arriver\,; la promotion de son fils au pontificat fit inscrire
les Ottobon au livre d'or\footnote{Livre dans lequel étaient inscrites
  les familles patriciennes de Venise.}, et par conséquent ils devinrent
nobles Vénitiens. Le cardinal Ottoboni, après la mort du pape son oncle,
accepta la protection de France sans en avoir obtenu la permission du
sénat, ce qui est un crime à Venise. De là la colère des Vénitiens, qui
effacèrent lui et tous les Ottobon du livre d'or\,; et le roi, qui s'en
offensa, rompit tout commerce avec eux. On a rapporté cette affaire ici
en son temps et ce que c'est que la protection. On ne fait donc qu'en
rafraîchir la mémoire. La république envoya deux ambassadeurs
extraordinaires en France faire excuse de ce qui s'était passé, et
rentrer dans l'honneur des bonnes grâces du roi en rétablissant
préalablement le cardinal et les Ottobon dans le livre d'or et dans
l'état et le rang de nobles Vénitiens, le cardinal demeurant toujours
également protecteur de France sans aucune interruption de ce titre ni
de ses fonctions.

Le prince de Montbéliard, cadet de la maison de Wurtemberg, vint à Paris
pour demander que ses enfants fussent reconnus légitimes et princes,
quoiqu'il les eût de trois femmes qu'il avait eues à la fois, dont deux
étaient actuellement vivantes et chez lui à Montbéliard, tout contre la
Franche-Comté, où il faisait appeler l'une la douairière et l'autre la
régnante, et prétendait que les lois de l'Empire et les règles du
luthéranisme qu'il professait lui permettaient ces mariages. Le comte de
La Marck, comme versé dans les lois allemandes, fut chargé d'examiner
cette affaire avec Armenonville. Qu'une folie de cette nature ait passé
par la tête de quelqu'un, il y a de quoi s'en étonner, mais de la faire
examiner comme chose susceptible de l'être sérieusement, cela fait voir
à quel point le régent était facile à ce qui n'avait point de
contradicteur. M. de Montbéliard du temps du feu roi s'était contenté de
vouloir faire légitimer ses enfants et en avait été refusé\,; maintenant
il veut qu'ils soient non pas légitimés, mais déclarés légitimes. On se
moqua de lui et il s'en retourna chez lui. Qui ne croirait cette chimère
finie\,? Elle reparut à Vienne avec les mêmes prétentions\,; elle y fut
foudroyée par le conseil aulique qui déclara tous ces enfants bâtards.
Ce ne fut pas tout. Le prince de Montbéliard maria un de ses fils à une
de ses filles, sous prétexte que la mère de cette fille l'avait eue d'un
mari à qui il l'avait enlevée, puis épousée, et longtemps après il fut
vérifié que cette fille était de lui, quoiqu'ils ne l'aient pas avouée
et que le mariage ait subsisté. Après ce sceau de réprobation, M. de
Montbéliard mourut.

Le duc de Wurtemberg, à qui ce partage de cadet de sa maison revenait
par l'extinction de cette branche, voulut s'en mettre en possession\,;
les bâtards se barricadèrent et portèrent leurs prétentions au parlement
de Paris. Ils étaient réunis contre le duc de Wurtemberg, mais divisés
entre eux, ceux de chacune des deux prétendues femmes se traitant
réciproquement de bâtards. Le frère et la soeur mariés vinrent à
Paris\,; le mari n'était qu'un lourdaud, mais sa femme une maîtresse
intrigante. Ces sortes de créatures se sentent de loin les unes les
autres. M\textsuperscript{me} de Mezières, dont il a été parlé
quelquefois ici et qui excellait en intrigues, avait marié une de ses
filles à M. de Montauban, cadet du feu prince de Guéméné, au grand
regret des Rohan, qui pourtant, l'affaire faite, jugèrent à propos de
s'aider d'une si dangereuse créature, pour ne l'avoir pas contraire dans
leur famille, et tirer parti de sa fertilité. Elle et cette bâtarde qui
avait épousé son propre frère firent connaissance\,; la Mezières, bien
avertie que la bâtarde avait mis la main sur le riche magot du prince de
Montbéliard, fit espérer sa protection et celle de ses amis, mais à des
conditions. La princesse de Carignan, quoique d'une espèce bien
différente par le mariage qu'elle avait fait, n'était ni moins
intrigante ni moins intéressée que tentes les deux\,; elle entra de part
avec elles moyennant sa protection. Ces deux femmes et leur suite
donnèrent dans l'oeil de la bâtarde\,; elle sentait bien qu'il lui
fallait un crédit très supérieur pour réussir\,; elle crut l'avoir
trouvé, le marché se conclut. Les conditions furent une grosse somme
comptant dès lors à la Mezières, et une moindre à M\textsuperscript{me}
de Carignan, et le mariage arrêté entre le fils de la bâtarde et une
fille de M\textsuperscript{me} de Montauban, qui n'aurait lieu qu'en cas
de plein succès de l'affaire\,; qu'on ne donnerait rien ou presque rien
pour la dot\,; mais que par le gain du procès, le bâtard, frère et mari
tout à la fois de cette bâtarde, père et mère du gendre futur de
M\textsuperscript{me} de Montauban, étant déclaré légitime et héritier
de la comté de Montbéliard, par conséquent de la maison de Wurtemberg,
la Mezières, tous les Rohan et M\textsuperscript{me} de Carignan lui
feraient obtenir le rang de prince étranger\,; et que, dès ce moment du
marché, ils feraient tous leur propre affaire de la sienne. Ce marché
était excellent pour toutes les parties, dont chacune y trouvait
merveilleusement son compte, mais les deux maîtresses intrigantes
surtout, qui empochaient gros dès lors quoi qu'il pût arriver.

Les choses ainsi réglées, les protectrices du frère et de la soeur, mari
et femme, leur firent prendre effrontément le nom, le titre, les armes
et les livrées du feu prince de Montbéliard, leur père, avec un équipage
sortable à ce nouvel état, qui de leur propre autorité préjugeait le
fond du procès. Tous les Rohan se mirent en pièces,
M\textsuperscript{me} de Carignan remua tous les Luynes et fit agir la
duchesse de Lévi, et M\textsuperscript{me} de Dangeau auprès du
cardinal\,; elle-même travailla auprès du garde des sceaux Chauvelin
avec ses bassesses et ses adresses accoutumées et auprès duquel elle
avait grand crédit. Pour remuer tous les dévots à la mode, c'est-à-dire
les jésuites et toute la constitution, les nouveaux Montbéliard
adjurèrent le luthéranisme, et quoique frère et soeur mariés ensemble,
devinrent une merveille de piété. L'effet répondit aux espérances de
cette belle conversion\,; tout ce côté-là s'intrigua pour eux, et prit
leur parti jusqu'au fanatisme. Mais lorsque le succès paraissait
infaillible par tous les ressorts que l'artifice avait su faire jouer,
l'empereur, excité par le duc de Wurtemberg, se fâcha. Il fit dire au
roi, c'est-à-dire au cardinal Fleury, qu'il trouvait fort étrange qu'on
prétendît juger en France une affaire jugée en son conseil aulique, seul
compétent de connaître de l'état des princes de l'empire et de leurs
successions. Il se trouva qu'on était lors en désir et en termes de
conclure la paix avec lui.

Le cardinal, à qui Chauvelin avait, pour son intérêt particulier, qui
n'est pas de ce sujet, fait entreprendre très légèrement et fort mal à
propos cette guerre, en était fort las, quoiqu'elle n'eût guère duré,
tellement que toutes les intrigues ne purent étouffer les égards qu'on
crut devoir aux plaintes de l'empereur, et l'affaire fut arrêtée.
L'intérêt de ces prétendus Montbéliard et de leurs protecteurs était
trop grand pour quitter prise. Ils espérèrent trouver et profiter
d'autres conjonctures, et, en attendant, continuèrent à porter les nom,
armes, titre et livrées qu'ils avaient arborés, ils se rabattirent à se
faire plaindre et à entretenir leurs amis et leur cabale. Cela dura des
années, qui éclaircirent leur plus puissante protection. Les Rohan,
seuls en vigueur, leur restaient et les manèges de la Mezières\,; mais
tout vieillissait\,; et s'engourdissait. Je ne sais comment le duc de
Wurtemberg consentit à revenir procéder au parlement de Paris. Il est
vrai que le roi avait eu lieu d'être fort content de lui pour empêcher
tant qu'il avait pu, et avec succès, les cercles du Rhin de se déclarer
lors de la guerre que la mort de l'empereur avait fait renaître. Le
procès fut donc repris au parlement, mais les choses étaient trop
changées pour les faux Montbéliard. Cette affaire si singulière avait
fait trop de bruit et avait trop duré\,; elle avait à la fin été
éclaircie de tous les artifices dont elle avait été voilée. L'état de
cette bâtardise était connu, celui de cet incestueux et abominable
mariage ne le fut pas moins. Le monde s'indigna qu'une prétention si
monstrueuse fût soufferte\,; les dévots eurent honte à leur tour de
l'avoir tant protégée\,; tellement qu'il intervint enfin un arrêt
contradictoire en la grand'chambre qui replongea cette canaille infâme
dans le néant, d'où elle n'aurait jamais dû sortir, et cela sans plus
d'espérance ni de ressource. La singularité de la chose et des
personnages m'a engagé de couler cette affaire à fond, quoique sa durée
et sa fin dépassent le but que je me suis proposé de bien des années. Le
rare est que, malgré cet arrêt et son exécution pour le comté de
Montbéliard, dont le duc de Wurtemberg fut mis en possession, cette rare
bâtarde a eu l'impudence de conserver dans Paris son prétendu nom,
titre, armes et livrées, qu'elle va traînant où elle peut, sans être
presque plus reçue de personne. Reprenons maintenant le fil de notre
narration.

\hypertarget{note-i.-taille.}{%
\chapter{NOTE I. TAILLE.}\label{note-i.-taille.}}

On trouve des détails curieux sur la taille et sur la manière de la
lever dans un manuscrit de la bibliothèque de l'Arsenal\footnote{Bibl.
  de l'Arsenal, ms. n° 218, in-fol.~(Histoire).}, qui a été rédigé vers
1725. Le passage suivant pourra servir de commentaire aux Mémoires de
Saint-Simon, qui se borne à mentionner cet impôt.

«\,La taille, dit l'auteur anonyme, est une imposition sur chaque
particulier. Elle se divise en taille personnelle, réelle et mixte,
selon les pays.

«\,La taille personnelle est imposée sur le bien fonds que chacun
possède, selon la quantité d'arpents et la bonté du terrain, dont il a
été fait une estimation qui s'appelle cadastre\footnote{Malheureusement
  le cadastre de la France, commencé par ordre de Colbert, n'avait été
  exécuté que dans un petit nombre de généralités.}, qui la divise en
trois espèces\,: le bon, le moyen et le mauvais. Sur quoi il y a
seulement à observer que les fonds nobles en sont exempts, quoique le
possesseur soit roturier, et que le fonds roturier la paye, quoique le
possesseur soit noble.

«\,La taille mixte est en même temps personnelle et réelle, c'est-à-dire
imposée arbitrairement sur la personne à raison des fonds qu'elle
exploite. L'homme noble a le privilège de pouvoir faire exploiter par
des valets quelques charrues sans payer, mais ses fermiers ou métayers
payent la taille pour tous les autres fonds.

«\,Le conseil détermine, sur les besoins de l'État, la somme qu'il faut
imposer pour l'année suivante\,; c'est ce qui s'appelle \emph{le brevet
de taille}. Il détermine aussi, sur les avis des intendants, la somme
que chaque généralité doit payer, dont il envoie la commission à
l'intendant, qui en fait l'imposition dans chaque élection, dont il doit
connaître l'étendue et la valeur. Il y a des tribunaux établis pour
juger de tout ce qui concerne cette imposition et ceux qui en sont
chargés.

«\,Les parlements étaient si contraires aux intérêts du roi dans cette
partie qu'on a été obligé de créer d'autres juridictions uniquement pour
cela.

«\,Le tribunal supérieur s'appelle la cour des aides\,; les juridictions
inférieures sont l'élection, la chambre de grenier à sel, les juges de
ports et des traites. Elles connaissent des différents droits dont nous
parlerons dans l'occasion, et toutes relèvent en dernier ressort de la
cour des aides.

«\,Il y a un autre tribunal appelé les \emph{trésoriers de France}, qui
originairement faisaient les fonctions d'intendants dans les
provinces\,; ils étaient chargés des finances, des ponts et chaussées et
des chemins. Il ne leur reste plus qu'une très petite ombre de cette
autorité entière dévolue aux intendants.

«\,La commission de la taille s'enregistre dans leur bureau, et
pareillement tous les états de payements assignés sur les tailles.

«\,Nous avons dit que les intendants faisaient faire l'imposition de la
taille dans chaque élection. Les élus en font la distribution par
paroisses de leur ressort, et les habitants de chaque paroisse
choisissent des collecteurs qui font l'imposition sur chaque
particulier. Ils sont chargés personnellement et par corps du
recouvrement, qu'ils remettent aux receveurs des tailles de
l'élection\,; celui-ci les remet au receveur général, qui les porte au
trésorier royal.

«\,Il a été établi depuis peu\footnote{Cette notice sur la taille et sur
  les juridictions financières, a été écrite, comme on l'a déjà dit,
  vers 1725.} un autre bureau par où on fait passer les fonds qu'on
appelle la \emph{caisse commune}.

«\,Il faut observer que ni les pays d'états\footnote{Voy. sur les pays
  d'états, t. XIV, p.~482.}, ni les pays conquis, ne payent point cette
taille, ou ne la payent point de la même manière\,; ils s'imposent
eux-mêmes, selon les dons gratuits que le roi leur a demandés.

«\,A considérer le royaume par rapport à cette imposition, il est divisé
en vingt généralités, en pays d'états et en pays conquis. Les pays
d'états sont la Bourgogne et la Bretagne, le Languedoc, le Béarn et
l'Artois. Quoique la Provence ait perdu son privilège d'état, toutes les
impositions s'y lèvent à peu près de la même manière. Les pays conquis
sont la Flandre, les Trois-Évêchés (Toul, Metz et Verdun), l'Alsace et
la Franche-Comté. Le reste du royaume contient les vingt généralités et
un bureau des finances {[}par généralité{]}. C'est par généralité, et
non par province, que les intendants sont distribués\,; ainsi la
Normandie en a trois\,: Rouen, Caen et Alençon, et il n'y en a qu'un
pour les provinces de Limousin et Angoumois.

«\,Les intendants\footnote{Voy. sur les intendants, t. III, p.~442.}
sont des commissaires tirés du conseil pour rendre compte aux ministres
de tout ce qui se passe dans leur district\,; et quoiqu'ils prennent la
qualité de \emph{commissaire de justice, police et finance}, leur
autorité ne s'étend point sur les contestations ou procès ordinaires\,;
ce n'est qu'autant qu'ils ont rapport aux habitants des lieux où sont
les troupes.

«\,De ce que nous avons dit sur la manière dont s'impose la taille, il
en résulte un arbitraire qui arrête toute industrie, et qui non
seulement empêche la culture des terres, nais encore les fait
abandonner. Un seigneur n'oublie rien pour obtenir de l'intendant une
diminution sur la taille de son village, et il l'obtient à proportion de
son crédit à la cour.

«\,Dans les répartitions particulières, le crédit de l'homme en charge
ou riche épouvante le collecteur, qui est obligé de faire tomber tout le
fardeau sur le pauvre\,; la haine et la vengeance achèvent l'injustice
de cette imposition, et le pauvre laboureur, hors d'état de la payer,
abandonne sa terre et va mendier avec toute sa famille.

«\,Dans un état qui m'a été remis des impositions de la taille dans la
généralité de Paris en 1720, j'ai vu avec étonnement que, dans des
paroisses contiguës, l'une paye jusqu'à quinze sous par livre du bail à
ferme, tandis que l'autre ne paye que trois sous.

«\,Si, à la face de la cour et des ministres, il se commet de pareilles
injustices, qu'est-ce qu'on doit penser des provinces\,? Je sais aussi
qu'il y a environ trois ans qu'un particulier avait établi une
manufacture de savon à Bagnolet, qu'il a été obligé d'abandonner, par la
taille exorbitante où il avait été imposé\,; dommage encore plus grand
pour la paroisse que pour ce particulier, qui portera son industrie
ailleurs, peut-être chez nos voisins.

«\,Dans l'imposition de la taille sont compris le taillon destiné au
payement de l'ordinaire des guerres \footnote{C'est-à-dire des dépenses
  ordinaires de l'armée.}, les fonds pour l'entretien des ponts et
chaussées, et, en, temps de guerre, le quartier d'hiver, dont les
répartitions se font au sou la livre sur les taillables.

«\,La capitation, qui est une imposition par tête sans exception, et qui
a commencé sous le feu roi\footnote{Voy. \emph{Mémoires de
  Saint-Simon}., t. I, p.~227, 228, note.} , s'impose aussi au sou la
livre sur les taillables, et arbitrairement sur tous les autres
particuliers.

«\,Il y a à observer qu'actuellement elle s'impose à Paris uniquement
par le prévôt des marchands\footnote{Voy. sur le prévôt des marchands,
  t. III, p.~442.}, à l'exclusion des échevins, et en cela on augmente
le produit, parce que les échevins, abusant de leur ministère,
favorisaient et leurs parents et presque tous les bourgeois\,; mais on
est tombé dans un inconvénient encore plus pernicieux. Car ceux dont on
se sert pour cette imposition, ayant intérêt à la grossir, exigent au
delà de la faculté de chacun, et pour la faculté des payements, ils ont
obtenu que les rentes, même viagères, ne seraient payées qu'aux porteurs
de quittances de capitation, contre la foi des arrêts qui les exemptent
de toute saisie, même pour les deniers de Sa Majesté.

«\,Ces manques de foi, qui sont la cause du grand discrédit des effets
royaux, ne coûtent rien à la plupart des ministres, et ils le font si
légèrement, qu'on ne peut s'empêcher de les soupçonner ou d'ignorance ou
d'intérêt particulier.

«\,C'est ici le lieu de faire quelques observations sur l'impôt
personnel et arbitraire.

«\,On a vu l'inconvénient de cet impôt dans l'injustice des
répartitions. Il n'est pas moindre dans la difficulté du recouvrement\,:
on n'en donnera pas d'autre exemple que celui de la capitation dont nous
venons de parler. On a de la peine à arracher vingt sous par an de
capitation d'un artisan, tandis qu'il paye sans attention cinquante
livres annuellement pour un minot de sel, et à proportion pour le vin et
la viande. C'est que l'impôt réparti sur la denrée ne paraît qu'une plus
value de denrée enchérie également pour tout le monde, au lieu que, dans
l'impôt personnel, on croit toujours être taxé injustement, et l'on ne
manque point d'objets de comparaison qui le persuadent.\,»

Les faits confirment pleinement ce que l'auteur dit des abus et des
inconvénients de la taille. Les Mémoires du marquis d'Argenson en
fournissent de nombreuses preuves\,; ainsi il parle souvent de la misère
des campagnes et même de famines, qu'il attribue aux impôts excessifs.
Il écrit dans ses Mémoires encore inédits, à la date du 8 juin 1751\,:

«\,Je suis présentement dans mes terres, à quatre-vingts lieues de
Paris. Les apparences de la récolte ne sont que d'une demi-année au
plus, pourvu cependant qu'il fasse du chaud\,; tous les fruits sont
perdus\,; la vigne a quelque apparence. On laisse encore sortir le blé,
qui va par la Loire à Nantes, et de là en Hollande. Sans cette
permission continuée, il n'y aurait pas un sol pour payer les tailles ni
les propriétaires des terres. Le poids de la taille est plus fatigant
que jamais\,; elle est beaucoup plus forte que dans la généralité de
Paris. Les corvées pour les chemins et le sel\footnote{L'impôt sur le
  sel, ou gabelle.} achèvent de les écraser. Les contraintes des
receveurs des tailles sont une autre taille pire que la première\,:
voilà ce que j'entends dire de tous côtés.

~

{\textsc{«\,Mercredi 16 juin.}} {\textsc{- J'ai recueilli dans ma
province ce que j'entends (lire d'impartial sur l'état des habitants\,;
il s'ensuit que la misère augmente et augmentera de plus en plus, par
les mauvais principes et le \emph{faux travail} du ministère et des
intendants. Je dis faux travail\,; car on se donne bien de la peine pour
faire plus mal.}}

~

«\,Si on laissait faire, on ne détournerait point de l'agriculture pour
porter à des arts inutiles\,; on ne ferait pas de la campagne un séjour
affreux comme on fait. Par ce qu'on fait, la campagne se dépeuple\,; ce
qui augmente chaque jour.

«\,Les grands chemins et belles routes sont bonnes, mais ceux qui les
dirigent ont impatience d'avancer, et précipitent ce travail par des
corvées qui achèvent d'écraser les villages voisins à quatre lieues à la
ronde. Je vois ces pauvres gens y périr de misère\,: on leur paye quinze
sols ce qui vaut un écu, pour leurs voitures. Ainsi en a-t-on encore
pour longtemps chez moi à faire des vingt voitures de huit lieues
chacune, qui met les habitants à l'aumône.

«\,On ne voit que villages ruinés et abattus, et nulles maisons qui se
relèvent\,; ce qui augmente.

«\,Les receveurs des tailles et du sel font chaque année des frais pour
la moitié en sus de l'imposition. Les pauvres sont en retard de payer
par impuissance, et supportent ces frais. Les riches n'osent pas payer
les receveurs mieux qu'ils ne font, de peur d'être surimposés\,; toute
la communauté craint le surhaussement l'année suivante, et paye mal
exprès\,; ainsi la misère s'accroît.

«\,Tout l'argent du revenu des terres va à Paris\,; il ne revient au
plat pays (à la campagne) que quelque argent des étrangers pour le blé
qu'on envoie. Mais gare une mauvaise récolte t tout périrait.\,»

Ailleurs, le marquis d'Argenson met en opposition le triste état des
campagnes et le luxe de la cour\,:

«\,On n'a toujours que des choses fâcheuses, et même funestes, à dire du
dedans du royaume. La maladie s'est jetée dans les moutons, à cause de
la grande humidité de la terre\,; il en périt quantité de troupeaux,
surtout dans quelques provinces comme le Berry. On n'a donné encore
aucun ordre sur la cherté des blés, et on laisse subsister la permission
de les sortir du royaume\,; on en donne même des passeports\,; je sais
une dame qui vient d'en avoir un.

«\,Le roi vient d'accorder au duc de Chaulnes un don de deux cent
soixante mille livres, pour indemnité des dépenses qu'il a faites aux
derniers états de Bretagne, outre les revenus et émoluments ordinaires
de cette place.

«\,On a prétendu que l'hôtel de la chancellerie de France serait mieux
avec un appartement de plain-pied\,; l'on y change l'escalier à la porte
d'entrée\,; ce qui coûtera grande dépense, et M. le chancelier va être
une année sans pouvoir habiter cet hôtel. Mais le pire est que cela
coûte à l'État\,; ce qui scandalise le public.

«\,La marquise de Pompadour paraîtra à Marly avec une robe qui est
garnie de dentelles d'Angleterre pour plus de vingt-deux mille cinq
cents livres.

«\,Tous payements sont retardés. M. le duc d'Orléans m'a dit hier que
ses pensions et tout ce qu'il reçoit au trésor royal étaient retardés
présentement de deux années et un quartier, ce qui est de cinq quartiers
plus qu'à l'ordinaire.\,»

Et ailleurs\,:

«\,Les receveurs des tailles font de grosses fortunes en peu de temps
par les frais énormes des recouvrements\,: chaque habitant est à leur
merci et craint l'augmentation de la taille chaque année. Ils sont
surchargés d'impôts, gagnent peu, voient tout l'argent aller à Paris\,;
toute industrie, toute aisance est découragée\,: de là vient cette ruine
générale en France.\,»

\hypertarget{note-ii.-vuxe9nalituxe9-des-charges.}{%
\chapter{NOTE II. VÉNALITÉ DES
CHARGES.}\label{note-ii.-vuxe9nalituxe9-des-charges.}}

Il est souvent question dans les Mémoires de Saint-Simon, et notamment
dans le présent volume, de la vénalité des charges Comme cet abus de
l'ancienne France remontait à une époque reculée, et que les détails
n'en sont pas connus de tous les lecteurs, il est nécessaire d'en
rappeler l'origine et le caractère.

En 1512, Louis XII, manquant de ressources pécuniaires pour soutenir la
guerre contre la maison d'Autriche, commença à vendre des offices de
finances et même quelques charges de judicature, par exemple des offices
de baillis et de conseillers au parlement. Au premier aspect, on
s'indigne d'un trafic qui livrait au plus offrant les charges d'où
dépendent la vie et l'honneur des citoyens, et il faut bien reconnaître
que, dans la suite, la vénalité des offices fut féconde en abus et en
scandales. Cependant on ne doit pas oublier que ce fut une des
principales causes de l'élévation des classes inférieures, qui,
enrichies par le commerce, purent acquérir des charges de magistrature.
Un des contemporains de Louis XII, Claude de Seyssel, était déjà frappé
de cette révolution. Après avoir indiqué que la nation française est
divisée en trois classes, tiers état, magistrature et noblesse, il
ajoute\footnote{*Traité de la**monarchie*, première partie, chap.~XVII.}\,:
«\,Si peut un chacun du dernier état parvenir au second, par vertu et
par diligence, sans autre moyen de grâce ni de privilège.\,» Ce second
état donnait souvent l'avantage sur la noblesse, placée au premier rang.
«\,On voit tous les jours, ajoute le même écrivain\footnote{Claude de
  Seyssel, \emph{Traité de la monarchie}, deuxième partie, chapitre XX.},
les officiers et les ministres de la justice acquérir les héritages et
seigneuries des barons et nobles hommes, et iceux nobles venir à telle
pauvreté et nécessité qu'ils ne peuvent entretenir l'état de
noblesse.\,»

Sous François Ier, les abus de la vénalité des charges commencèrent à se
manifester de la manière la plus scandaleuse. Ce prince créa jusqu'à des
chambres entières du parlement, composées d'un grand nombre de
magistrats. Ainsi, en 1524, la création et la vente de vingt charges de
conseillers au parlement de Paris lui valut soixante-dix mille livres
tournois (monnaie du temps)\footnote{\emph{Journal d'un bourgeois de
  Paris sous François Ier}, p.~123, 124 (Publication de la Société de
  l'histoire de France).}. La création de seize commissaires au
Châtelet, de quarante notaires à Paris, de baillis, etc.,\footnote{\emph{Ibidem},
  p.~124, 125, 126, 127.}, fut encore une mesure fiscale. Plusieurs de
ces juges ne se faisaient pas scrupule de revendre en détail ce qu'ils
avaient acheté en gros. «\,Il y en a, dit l'ambassadeur vénitien, Marino
Cavalli\footnote{\emph{Relations des ambassadeurs vénitiens}, I, 265.},
qui poussent si loin l'envie d'exploiter leur position, qu'ils se font
pendre tout bonnement à Montfaucon\,; ce qui arrive lorsqu'ils ne savent
pas se conduire avec un peu de prudence\,; car, jusqu'à un certain
point, tout est toléré, principalement si les parties ne s'en plaignent
pas.\,»Le même ambassadeur dit que la longueur des procès était souvent
une spéculation des juges\footnote{\emph{Ibidem}, 263.}\,: «\,Une cause
de mille écus en exige deux mille de frais\,; elle dure dix ans.\,»

Ces abus, qui ne firent que s'accroître sous les règnes suivants,
provoquèrent les plaintes les plus vives. Bodin, dans son traité
\emph{de la République}, et Montaigne, dans ses \emph{Essais},
s'élevèrent contre un trafic scandaleux. Mais il fut surtout attaqué par
François Hotman\footnote{\emph{Franco-Gallia}, chap.~XXI.}\,; il ravale
la vénalité des charges par une comparaison ignoble empruntée à la
boucherie. Il assimile le trafic de ces offices, que l'on achetait en
gros et que l'on revendait en détail, au commerce d'un boucher qui,
après avoir acheté un boeuf, le dépèce et en vend les
morceaux\footnote{«\,* Sicuti lanii bovem opimum pretio emptum in
  macello per partes venditaut*.\,»}. Ces attaques amenèrent d'utiles
réformes\,: la vénalité des charges ne fut pas détruite, mais elle fut
soumise à des conditions de moralité et de capacité\footnote{Voy. art.
  12 de l'ordonnance de Moulins (1566).} . Grâce à ces réformes, que
l'on dut surtout au chancelier de L'Hôpital, les inconvénients de la
vénalité des offices de judicature furent atténués. La science et la
vertu se transmirent avec les charges dans les familles
parlementaires\,: les Lamoignon, les de Harlay, les Molé, pour ne citer
que les plus illustres, datent de la fin du XVIe siècle.

Henri IV et son ministre Sully régularisèrent cette propriété des
offices dans les familles parlementaires. Il fut décidé, en 1604, que
les magistrats, pour en devenir propriétaires, payeraient chaque année
un soixantième du prix de leur charge. Le premier fermier de cet impôt
fut le financier Paulet, d'où vint à la taxe le nom de \emph{paulette}.
Le premier bail pour cet impôt fut conclu pour neuf ans, et rapporta au
trésor deux millions deux cent soixante-trois mille livres (monnaie du
temps). Antérieurement, pour que la vente d'un office fût valable, il
fallait que celui qui le résignait survécût quarante jours à la
transaction. Henri IV déclara que, pour les offices dont les titulaires
auraient payé la paulette, la mort n'entraînerait point la déchéance\,;
les héritiers pouvaient disposer de la charge.

Au XVIIe siècle, la vénalité des charges fut plusieurs fois attaquée.
Richelieu songea à la supprimer\,: «\,Il ne faut plus rétablir la
paulette, dit-il dans ses Mémoires \footnote{\emph{Mémoires de
  Richelieu}, liv. XX.} \,; il faut abaisser les compagnies, qui, par
une prétendue souveraineté, s'opposent tous les jours au bien du
royaume.\,» Colbert eut la même pensée, comme le prouve un mémoire qu'il
présenta à Louis XIV en 1665 \footnote{Ce mémoire a été publié dans la
  \emph{Revuerétrospective}, deuxième série, t. IV, p.~251 et suiv.}
\,;mais comme son projet rencontra des résistances insurmontables, il
sut se contenter des réformes qui pouvaient être immédiatement
appliquées\,; il diminua le prix des offices et en limita le nombre
\footnote{\emph{Anciennes lois françaises}, t. XVIII, p. 66.} . Malgré
ces réformes, le prix des charges de judicature était encore très
élevé\,: un office de président à mortier se vendait trois cent
cinquante mille livres\,; les charges de maître des requêtes et d'avocat
général, cent cinquante mille livres\,; de conseiller au parlement,
quatre-vingt-dix à cent mille livres\,; de premier président de la
chambre des comptes\,; quatre cent mille livres\,; de président de la
même chambre, deux cent mille livres\,; de maître des comptes, cent
vingt mille livres \footnote{\emph{Anciennes lois françaises, ibidem}. -
  Henri Martin, \emph{Histoire de France}, (3e édit), t. XIV, p.~574.}.

Pendant la dernière partie du règne de Louis XIV, les abus de la
vénalité des charges se renouvelèrent de la manière la plus scandaleuse,
et Saint-Simon, qui retrace surtout l'histoire de cette période, en
parle souvent. Mais c'est surtout dans le \emph{Journal} inédit de
Foucault\footnote{J'ai déjà plusieursfoiscité ce journal de Foucault,
  qui est conservé dans le dépôt des mss. de la Bib. impériale.} que
l'on trouve la preuve de ces créations d'offices, multipliées par la
fiscalité. Il suffira d'en citer quelques passages\,: «\,En février
1693, j'ai reçu l'édit et l'arrêt du conseil que M. de Pontchartrain m'a
envoyé au sujet des charges de contrôleur commissaire et trésorier de
l'arrière-ban\footnote{\emph{Ibidem}. fol.~82.}. --- Le roi a créé des
charges d'essayeurs d'étain. --- En 1694, il a été créé, par un édit,
des colonels, majors et autres officiers de milices bourgeoises des
villes et bourgs du royaume. J'ai proposé de les faire prendre (ces
charges) par les mieux accommodés des bourgeois\footnote{\emph{Ibidem},
  fol.~87.}. --- Le 9 janvier, M. de Pontchartrain m'a envoyé l'édit
portant création des certificateurs des criées. --- Le roi a créé des
médecins et chirurgiens royaux. --- Il a été créé des offices de
greffiers alternatifs\footnote{C'est-à-dire remplissant alternativement
  l'office de greffiers des rôles.} des rôles des tailles dans les
paroisses. --- Au mois d'octobre 1696, le roi a créé, par un édit, des
offices de gouverneurs héréditaires dans toutes les villes closes du
royaume, à l'exception de celles où il y a des provisions du roi et des
appointements employés dans les États de Sa Majesté. Ces charges ont été
fort recherchées et bien vendues.\,»

Cette nomenclature, qu'il serait facile de prolonger, prouve à quel
excès avait été portée la vénalité des charges. Elle s'étendait à
l'armée, et Saint-Simon a dit avec raison\,: «\,Cette vénalité est une
grande plaie dans le militaire, et arrête bien des gens qui seraient
d'excellents sujets. C'est une gangrène qui ronge depuis longtemps tous
les ordres et toutes les parties de l'État.\,» Malgré ces abus, la
vénalité des charges trouva des apologistes au XVIIIe siècle.
Montesquieu l'a défendue dans le passage suivant de \emph{l'Esprit des
Lois}\footnote{Liv. V, chap.~XIX, éd. de Ch. Lahure, t. I, p.~61.}\emph{\,;}
«\,Cette vénalité est bonne dans les États monarchiques, parce qu'elle
fait faire, comme un métier de famille, ce qu'on ne voudrait pas
entreprendre pour la vertu\,; qu'elle destine chacun à son devoir, et
rend les ordres de l'État plus permanents.\,» La vénalité des charges de
judicature, supprimée en 1771, par le président Maupeou, fut rétablie en
1774, et a duré jusqu'à la révolution française.

\end{document}
